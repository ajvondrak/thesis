\subsection{Low-level Optimizations}\label{sec:compiler:cfg}

The low-level \gls{IR} in \factor|compiler.cfg| takes the more conventional
form of a \gls{CFG}.  A \gls{CFG} (not to be confused with ``context-free
grammar'') is an arrangement of instructions into \term{basic blocks}: maximal
sequences of ``straight-line'' code, where control does not transfer out of or
into the middle of the block.  Directed edges are added between blocks to
represent control flow---either from a branching instruction to its target, or
from the end of a basic block to the start of the next one\todo{cite}.
Construction of the low-level \gls{IR} proceeds by analyzing the control flow
of the high-level \gls{IR} and converting the nodes of \cref{sec:compiler:tree}
into lower-level, more conventional instructions modeled after typical assembly
code.  There are over a hundred of these instructions, but many are simply
different versions of the same operation.  For instance, while instructions are
generally called on \term{virtual registers} (represented in Factor simply by
integers), there are \term{immediate} versions of instructions.  The
\factor|##add| instruction, as an example, represents the sum of the contents
of two registers, but \factor|##add-imm| sums the contents of one register and
an integer literal.  Other instructions are inserted to make stack reads and
writes explicit, as well as to balance the height.  Below is a categorized list
of all the instruction objects (each one is a subclass of the \factor|insn|
tuple).

\todo[inline]{Is the complete list really necessary?}
\begin{itemize}
\item
\begin{flushleft}
Loading constants:
\Verb|##load-integer|,
\Verb|##load-reference|
\end{flushleft}

\item
\begin{flushleft}
Optimized loading of constants, inserted by representation selection:
\Verb|##load-tagged|,
\Verb|##load-float|,
\Verb|##load-double|,
\Verb|##load-vector|
\end{flushleft}

\item
\begin{flushleft}
Stack operations:
\Verb|##peek|,
\Verb|##replace|,
\Verb|##replace-imm|,
\Verb|##inc-d|,
\Verb|##inc-r|
\end{flushleft}

\item
\begin{flushleft}
Subroutine calls:
\Verb|##call|,
\Verb|##jump|,
\Verb|##prologue|,
\Verb|##epilogue|,
\Verb|##return|
\end{flushleft}

\item
\begin{flushleft}
Inhibiting \gls{TCO}:
\Verb|##no-tco|
\end{flushleft}

\item
\begin{flushleft}
Jump tables:
\Verb|##dispatch|
\end{flushleft}

\item
\begin{flushleft}
Slot access:
\Verb|##slot|,
\Verb|##slot-imm|,
\Verb|##set-slot|,
\Verb|##set-slot-imm|
\end{flushleft}

\item
\begin{flushleft}
Register transfers:
\Verb|##copy|,
\Verb|##tagged>integer|
\end{flushleft}

\item
\begin{flushleft}
Integer arithmetic:
\Verb|##add|,
\Verb|##add-imm|,
\Verb|##sub|,
\Verb|##sub-imm|,
\Verb|##mul|,
\Verb|##mul-imm|,
\Verb|##and|,
\Verb|##and-imm|,
\Verb|##or|,
\Verb|##or-imm|,
\Verb|##xor|,
\Verb|##xor-imm|,
\Verb|##shl|,
\Verb|##shl-imm|,
\Verb|##shr|,
\Verb|##shr-imm|,
\Verb|##sar|,
\Verb|##sar-imm|,
\Verb|##min|,
\Verb|##max|,
\Verb|##not|,
\Verb|##neg|,
\Verb|##log2|,
\Verb|##bit-count|
\end{flushleft}

\item
\begin{flushleft}
Float arithmetic:
\Verb|##add-float|,
\Verb|##sub-float|,
\Verb|##mul-float|,
\Verb|##div-float|,
\Verb|##min-float|,
\Verb|##max-float|,
\Verb|##sqrt|
\end{flushleft}

\item
\begin{flushleft}
Single/double float conversion:
\Verb|##single>double-float|,
\Verb|##double>single-float|
\end{flushleft}

\item
\begin{flushleft}
Float/integer conversion:
\Verb|##float>integer|,
\Verb|##integer>float|
\end{flushleft}

\item
\begin{flushleft}
\Gls{SIMD} operations:
\Verb|##zero-vector|,
\Verb|##fill-vector|,
\Verb|##gather-vector-2|,
\Verb|##gather-int-vector-2|,
\Verb|##gather-vector-4|,
\Verb|##gather-int-vector-4|,
\Verb|##select-vector|,
\Verb|##shuffle-vector|,
\Verb|##shuffle-vector-halves-imm|,
\Verb|##shuffle-vector-imm|,
\Verb|##tail>head-vector|,
\Verb|##merge-vector-head|,
\Verb|##merge-vector-tail|,
\Verb|##float-pack-vector|,
\Verb|##signed-pack-vector|,
\Verb|##unsigned-pack-vector|,
\Verb|##unpack-vector-head|,
\Verb|##unpack-vector-tail|,
\Verb|##integer>float-vector|,
\Verb|##float>integer-vector|,
\Verb|##compare-vector|,
\Verb|##test-vector|,
\Verb|##test-vector-branch|,
\Verb|##add-vector|,
\Verb|##saturated-add-vector|,
\Verb|##add-sub-vector|,
\Verb|##sub-vector|,
\Verb|##saturated-sub-vector|,
\Verb|##mul-vector|,
\Verb|##mul-high-vector|,
\Verb|##mul-horizontal-add-vector|,
\Verb|##saturated-mul-vector|,
\Verb|##div-vector|,
\Verb|##min-vector|,
\Verb|##max-vector|,
\Verb|##avg-vector|,
\Verb|##dot-vector|,
\Verb|##sad-vector|,
\Verb|##horizontal-add-vector|,
\Verb|##horizontal-sub-vector|,
\Verb|##horizontal-shl-vector-imm|,
\Verb|##horizontal-shr-vector-imm|,
\Verb|##abs-vector|,
\Verb|##sqrt-vector|,
\Verb|##and-vector|,
\Verb|##andn-vector|,
\Verb|##or-vector|,
\Verb|##xor-vector|,
\Verb|##not-vector|,
\Verb|##shl-vector-imm|,
\Verb|##shr-vector-imm|,
\Verb|##shl-vector|,
\Verb|##shr-vector|
\end{flushleft}

\item
\begin{flushleft}
Scalar/vector conversion:
\Verb|##scalar>integer|,
\Verb|##integer>scalar|,
\Verb|##vector>scalar|,
\Verb|##scalar>vector|
\end{flushleft}

\item
\begin{flushleft}
Boxing and unboxing aliens:
\Verb|##box-alien|,
\Verb|##box-displaced-alien|,
\Verb|##unbox-any-c-ptr|,
\Verb|##unbox-alien|
\end{flushleft}

\item
\begin{flushleft}
Zero-extending and sign-extending integers:
\Verb|##convert-integer|
\end{flushleft}

\item
\begin{flushleft}
Raw memory access:
\Verb|##load-memory|,
\Verb|##load-memory-imm|,
\Verb|##store-memory|,
\Verb|##store-memory-imm|
\end{flushleft}

\item
\begin{flushleft}
Memory allocation:
\Verb|##allot|,
\Verb|##write-barrier|,
\Verb|##write-barrier-imm|,
\Verb|##alien-global|,
\Verb|##vm-field|,
\Verb|##set-vm-field|
\end{flushleft}

\item
\begin{flushleft}
The \gls{FFI}:
\Verb|##unbox|,
\Verb|##unbox-long-long|,
\Verb|##local-allot|,
\Verb|##box|,
\Verb|##box-long-long|,
\Verb|##alien-invoke|,
\Verb|##alien-indirect|,
\Verb|##alien-assembly|,
\Verb|##callback-inputs|,
\Verb|##callback-outputs|
\end{flushleft}

\item
\begin{flushleft}
Control flow:
\Verb|##phi|,
\Verb|##branch|
\end{flushleft}

\item
\begin{flushleft}
Tagged conditionals:
\Verb|##compare-branch|,
\Verb|##compare-imm-branch|,
\Verb|##compare|,
\Verb|##compare-imm|
\end{flushleft}

\item
\begin{flushleft}
Integer conditionals:
\Verb|##compare-integer-branch|,
\Verb|##compare-integer-imm-branch|,
\Verb|##test-branch|,
\Verb|##test-imm-branch|,
\Verb|##compare-integer|,
\Verb|##compare-integer-imm|,
\Verb|##test|,
\Verb|##test-imm|
\end{flushleft}

\item
\begin{flushleft}
Float conditionals:
\Verb|##compare-float-ordered-branch|,
\Verb|##compare-float-unordered-branch|,
\Verb|##compare-float-ordered|,
\Verb|##compare-float-unordered|
\end{flushleft}

\item
\begin{flushleft}
Overflowing arithmetic:
\Verb|##fixnum-add|,
\Verb|##fixnum-sub|,
\Verb|##fixnum-mul|
\end{flushleft}

\item
\begin{flushleft}
\Gls{GC} checks:
\Verb|##save-context|,
\Verb|##check-nursery-branch|,
\Verb|##call-gc|
\end{flushleft}

\item
\begin{flushleft}
Spills and reloads, inserted by the register allocator:
\Verb|##spill|,
\Verb|##reload|
\end{flushleft}
\end{itemize}


% dls.pdf verbatim:

%Low-level IR is built from high-level IR by analyzing control flow and making
%stack reads and writes explicit. During this construction phase and a
%subsequent branch splitting phase, the SSA structure of high-level IR is lost.
%SSA form is reconstructed using the SSA construction algorithm described in
%[8], with the minor variation that we construct pruned SSA form rather than
%semi-pruned SSA, by first computing liveness. To avoid computing iterated
%dominance frontiers, we use the TDMSC algorithm from [13]. 

%The main optimizations performed on low-level IR are local dead store and
%redundant load elimination, local value numbering, global copy propagation,
%representation selection, and instruction scheduling.  The local value
%numbering pass eliminates common subexpressions and folds expressions with
%constant operands [9].

%Following value numbering and copy propagation, a representation selection
%pass decides when to unbox floating point and SIMD values. A form of
%instruction scheduling intended to reduce register pressure is performed on
%low-level IR as the last step before leaving SSA form [39].  We use the
%second-chance binpacking variation of the linear scan register allocation
%algorithm [43, 47]. Our variant does not take φ nodes into account, so SSA
%form is destructed first by eliminating $\phi$ nodes while simultaneously
%performing copy coalescing, using the method described in [6].

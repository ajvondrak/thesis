\section{Low-level Optimizations}\label{sec:compiler:cfg}

\subsection{Representation}

The low-level \gls{IR} in \Verb|compiler.cfg| takes the more conventional form
of a \gls{CFG}.  A \gls{CFG} (not to be confused with ``context-free grammar'')
is an arrangement of instructions into \term{basic blocks}: maximal sequences
of ``straight-line'' code, where control does not transfer out of or into the
middle of the block.  Directed edges are added between blocks to represent
control flow---either from a branching instruction to its target, or from the
end of a basic block to the start of the next one \autocite{DragonBook}.
Construction of the low-level \gls{IR} proceeds by analyzing the control flow
of the high-level \gls{IR} and converting the nodes of \cref{sec:compiler:tree}
into lower-level instructions typical of assembly code.  There are over a
hundred of these instructions, but many are simply different versions of the
same operation.  For instance, while instructions are generally called on
\term{virtual registers} (represented in Factor simply by integers), there are
\term{immediate} versions of instructions.  The \Verb|##add| instruction, as an
example, represents the sum of the contents of two registers, but
\Verb|##add-imm| sums the contents of one register and an integer literal.
Other instructions are inserted to make stack reads and writes explicit, as
well as to balance the height.

For posterity, below is a categorized list of all the instruction objects (each
one is a subclass of the \Verb|insn| tuple).  It's not imperative to know each
of these instructions.  Their names generally indicate their purposes clearly
enough.  If need be, though, you can refer back to this list.

\begin{itemize}
\item
\begin{flushleft}
Loading constants:
\Verb|##load-integer|,
\Verb|##load-reference|
\end{flushleft}

\item
\begin{flushleft}
Optimized loading of constants, inserted by representation selection:
\Verb|##load-tagged|,
\Verb|##load-float|,
\Verb|##load-double|,
\Verb|##load-vector|
\end{flushleft}

\item
\begin{flushleft}
Stack operations:
\Verb|##peek|,
\Verb|##replace|,
\Verb|##replace-imm|,
\Verb|##inc-d|,
\Verb|##inc-r|
\end{flushleft}

\item
\begin{flushleft}
Subroutine calls:
\Verb|##call|,
\Verb|##jump|,
\Verb|##prologue|,
\Verb|##epilogue|,
\Verb|##return|
\end{flushleft}

\item
\begin{flushleft}
Inhibiting \gls{TCO}:
\Verb|##no-tco|
\end{flushleft}

\item
\begin{flushleft}
Jump tables:
\Verb|##dispatch|
\end{flushleft}

\item
\begin{flushleft}
Slot access:
\Verb|##slot|,
\Verb|##slot-imm|,
\Verb|##set-slot|,
\Verb|##set-slot-imm|
\end{flushleft}

\item
\begin{flushleft}
Register transfers:
\Verb|##copy|,
\Verb|##tagged>integer|
\end{flushleft}

\item
\begin{flushleft}
Integer arithmetic:
\Verb|##add|,
\Verb|##add-imm|,
\Verb|##sub|,
\Verb|##sub-imm|,
\Verb|##mul|,
\Verb|##mul-imm|,
\Verb|##and|,
\Verb|##and-imm|,
\Verb|##or|,
\Verb|##or-imm|,
\Verb|##xor|,
\Verb|##xor-imm|,
\Verb|##shl|,
\Verb|##shl-imm|,
\Verb|##shr|,
\Verb|##shr-imm|,
\Verb|##sar|,
\Verb|##sar-imm|,
\Verb|##min|,
\Verb|##max|,
\Verb|##not|,
\Verb|##neg|,
\Verb|##log2|,
\Verb|##bit-count|
\end{flushleft}

\item
\begin{flushleft}
Float arithmetic:
\Verb|##add-float|,
\Verb|##sub-float|,
\Verb|##mul-float|,
\Verb|##div-float|,
\Verb|##min-float|,
\Verb|##max-float|,
\Verb|##sqrt|
\end{flushleft}

\item
\begin{flushleft}
Single/double float conversion:
\Verb|##single>double-float|,
\Verb|##double>single-float|
\end{flushleft}

\item
\begin{flushleft}
Float/integer conversion:
\Verb|##float>integer|,
\Verb|##integer>float|
\end{flushleft}

\item
\begin{flushleft}
\Gls{SIMD} operations:
\Verb|##zero-vector|,
\Verb|##fill-vector|,
\Verb|##gather-vector-2|,
\Verb|##gather-int-vector-2|,
\Verb|##gather-vector-4|,
\Verb|##gather-int-vector-4|,
\Verb|##select-vector|,
\Verb|##shuffle-vector|,
\Verb|##shuffle-vector-halves-imm|,
\Verb|##shuffle-vector-imm|,
\Verb|##tail>head-vector|,
\Verb|##merge-vector-head|,
\Verb|##merge-vector-tail|,
\Verb|##float-pack-vector|,
\Verb|##signed-pack-vector|,
\Verb|##unsigned-pack-vector|,
\Verb|##unpack-vector-head|,
\Verb|##unpack-vector-tail|,
\Verb|##integer>float-vector|,
\Verb|##float>integer-vector|,
\Verb|##compare-vector|,
\Verb|##test-vector|,
\Verb|##test-vector-branch|,
\Verb|##add-vector|,
\Verb|##saturated-add-vector|,
\Verb|##add-sub-vector|,
\Verb|##sub-vector|,
\Verb|##saturated-sub-vector|,
\Verb|##mul-vector|,
\Verb|##mul-high-vector|,
\Verb|##mul-horizontal-add-vector|,
\Verb|##saturated-mul-vector|,
\Verb|##div-vector|,
\Verb|##min-vector|,
\Verb|##max-vector|,
\Verb|##avg-vector|,
\Verb|##dot-vector|,
\Verb|##sad-vector|,
\Verb|##horizontal-add-vector|,
\Verb|##horizontal-sub-vector|,
\Verb|##horizontal-shl-vector-imm|,
\Verb|##horizontal-shr-vector-imm|,
\Verb|##abs-vector|,
\Verb|##sqrt-vector|,
\Verb|##and-vector|,
\Verb|##andn-vector|,
\Verb|##or-vector|,
\Verb|##xor-vector|,
\Verb|##not-vector|,
\Verb|##shl-vector-imm|,
\Verb|##shr-vector-imm|,
\Verb|##shl-vector|,
\Verb|##shr-vector|
\end{flushleft}

\item
\begin{flushleft}
Scalar/vector conversion:
\Verb|##scalar>integer|,
\Verb|##integer>scalar|,
\Verb|##vector>scalar|,
\Verb|##scalar>vector|
\end{flushleft}

\item
\begin{flushleft}
Boxing and unboxing aliens:
\Verb|##box-alien|,
\Verb|##box-displaced-alien|,
\Verb|##unbox-any-c-ptr|,
\Verb|##unbox-alien|
\end{flushleft}

\item
\begin{flushleft}
Zero-extending and sign-extending integers:
\Verb|##convert-integer|
\end{flushleft}

\item
\begin{flushleft}
Raw memory access:
\Verb|##load-memory|,
\Verb|##load-memory-imm|,
\Verb|##store-memory|,
\Verb|##store-memory-imm|
\end{flushleft}

\item
\begin{flushleft}
Memory allocation:
\Verb|##allot|,
\Verb|##write-barrier|,
\Verb|##write-barrier-imm|,
\Verb|##alien-global|,
\Verb|##vm-field|,
\Verb|##set-vm-field|
\end{flushleft}

\item
\begin{flushleft}
The \gls{FFI}:
\Verb|##unbox|,
\Verb|##unbox-long-long|,
\Verb|##local-allot|,
\Verb|##box|,
\Verb|##box-long-long|,
\Verb|##alien-invoke|,
\Verb|##alien-indirect|,
\Verb|##alien-assembly|,
\Verb|##callback-inputs|,
\Verb|##callback-outputs|
\end{flushleft}

\item
\begin{flushleft}
Control flow:
\Verb|##phi|,
\Verb|##branch|
\end{flushleft}

\item
\begin{flushleft}
Tagged conditionals:
\Verb|##compare-branch|,
\Verb|##compare-imm-branch|,
\Verb|##compare|,
\Verb|##compare-imm|
\end{flushleft}

\item
\begin{flushleft}
Integer conditionals:
\Verb|##compare-integer-branch|,
\Verb|##compare-integer-imm-branch|,
\Verb|##test-branch|,
\Verb|##test-imm-branch|,
\Verb|##compare-integer|,
\Verb|##compare-integer-imm|,
\Verb|##test|,
\Verb|##test-imm|
\end{flushleft}

\item
\begin{flushleft}
Float conditionals:
\Verb|##compare-float-ordered-branch|,
\Verb|##compare-float-unordered-branch|,
\Verb|##compare-float-ordered|,
\Verb|##compare-float-unordered|
\end{flushleft}

\item
\begin{flushleft}
Overflowing arithmetic:
\Verb|##fixnum-add|,
\Verb|##fixnum-sub|,
\Verb|##fixnum-mul|
\end{flushleft}

\item
\begin{flushleft}
\Gls{GC} checks:
\Verb|##save-context|,
\Verb|##check-nursery-branch|,
\Verb|##call-gc|
\end{flushleft}

\item
\begin{flushleft}
Spills and reloads, inserted by the register allocator:
\Verb|##spill|,
\Verb|##reload|
\end{flushleft}
\end{itemize}


\subsection{Optimizations}

By translating the high-level \gls{IR} into instructions that manipulate
registers directly, we reveal (and sometimes introduce) further redundancies
that can be optimized away.  The \Verb|optimize-cfg| word in
\vref{lst:optimize-cfg} shows the passes performed in doing this.  The first
word, \Verb|optimize-tail-calls|, performs tail call elimination on the
\gls{CFG}.
%
\term{Tail calls} are those that occur within a procedure and whose results are
immediately returned by that procedure.  Instead of allocating a new call stack
frame, we may convert tail calls into simple jumps, since afterwards the
current procedure's call frame isn't really needed.  In the case of recursive
tail calls, we can convert special cases of recursion into loops in the
\gls{CFG}, so that we won't trigger call stack overflows.  For instance,
consider \vref{fig:optimize-tail-calls}, which shows the effect of
\Verb|optimize-tail-calls| on the following definition:
%
\begin{center}
%
  \factor|: tail-call ( -- ) tail-call ;|
%
\end{center}
%
\noindent Note the recursive call (trivially) occurs at the end of the
definition, just before the return point.  When translated to a \gls{CFG}, this
is a \Verb|##call| instruction, as seen in block $4$ to the left of
\vref{fig:optimize-tail-calls}.  This is also just before the final
\Verb|##epilogue| and \Verb|##return| instructions in block $8$, as blocks
$5$--$7$ are effectively empty (these excessive \Verb|##branch|es will be
eliminated in a later pass).  Because of this, rather than make a whole new
subroutine call, we can convert it into a \Verb|##branch| back to the beginning
of the word, as in the \gls{CFG} to the right.

\inputlst{optimize-cfg}

\inputfig{optimize-tail-calls}

The next pass in \vref{lst:optimize-cfg} is \Verb|delete-useless-conditionals|,
which removes branches that go to the same basic block.  This situation might
occur as a result of optimizations performed on the high-level \gls{IR}.  To
see it in action, \vref{fig:delete-useless-conditionals} shows the
transformation on a purposefully useless conditional,
%
\factor|[ ] [ ] if|.
%
On the left, the \gls{CFG} \Verb|##peek|s at the top of the data stack
%
(\Verb|D 0|),
%
storing the result in the virtual register \Verb|1|.  This value is popped,
so we decrement the stack height
%
in block $2$ using \Verb|##inc-d -1|.
%
Then, \Verb|##compare-imm-branch| compares the value in the virtual register
\Verb|2| (which is a copy of \Verb|1|, the top of the stack) to the immediate
value \factor|f| to see if it's not equal (signified by \Verb|cc/=|)---that is,
to see if we should take the ``then'' branch or the ``else'' one.  However,
both branches jump through several empty blocks and merge at the same
destination.  Thus, we can remove them and replace \Verb|##compare-imm-branch|
with an unconditional \Verb|##branch| to the eventual destination.  We see this
on the right of \vref{fig:delete-useless-conditionals}.

\inputfig{delete-useless-conditionals}

In order to expose more opportunities for optimization, \Verb|split-branches|
will actually duplicate code.  We use the fact that code immediately following
a conditional will be executed along both branches regardless.  If it's
sufficiently short, we copy it up into the branches individually.  That is, we
change
%
\factor|[ A ] [ B ] if C|
%
into
%
\factor|[ A C ] [ B C ] if|,
%
as long as \Verb|C| is small enough.  Later analyses may then, for example,
more readily eliminate one of the branches if it's never taken.
\Vref{fig:split-branches} shows what such a transformation looks like on a
\gls{CFG}.  The example
%
\factor|[ 1 ] [ 2 ] if dup|
%
is essentially changed into
%
\factor|[ 1 dup ] [ 2 dup ] if|
%
thus splitting the block with two predecessors (block $9$) on the left.

\inputfig{split-branches}

The next pass, \Verb|join-blocks|, compacts the \gls{CFG} by joining together
blocks involved in only a single control flow edge.  Mostly, this is to clean
up the myriad of empty or short blocks introduced during construction, like
sequences of a bunch of \Verb|##branch|es.  \Vref{fig:join-blocks} shows this
pass on the \gls{CFG} of
%
\factor|0 100 [ 1 fixnum+fast ] times|,
%
which increments the top of the stack $100$ times.  \Verb|fixnum+fast| is a
specialized version of \factor|+| that suppresses overflow and type checks.  We
use it here to keep the \gls{CFG} simple.  We'll be using this particular code
to illustrate all but one of the remaining optimization passes in
\vref{lst:optimize-cfg}, as it's a motivating example for the work in this
thesis.  None of the passes before \Verb|join-blocks|  change the \gls{CFG}
seen on the left in \vref{fig:join-blocks}, but we get rid of the useless
\Verb|##branch| blocks, giving us the \gls{CFG} on the right.

\inputfig{join-blocks}

\Vref{fig:normalize-height} shows the result of applying
\Verb|normalize-height| to the result of \Verb|join-blocks|.  This phase
combines and canonicalizes the instructions that track the stack height, like
\Verb|##inc-d|.  While the shuffling in this example isn't complex enough to be
interesting, neither is this phase.  It amounts to more cleaning: multiple
height changes are combined into single ones at the beginnings of the basic
blocks.  In \vref{fig:normalize-height}, this means that \Verb|##inc-d| is
moved to the top of block $1$, as compared to the right of
\vref{fig:join-blocks}.

\inputfig{normalize-height}

In converting the high-level \gls{IR} to the low-level, we actually lose the
\gls{SSA} form of \Verb|compiler.tree|.  Not only does the construction do
this, but \Verb|split-branches| also copies basic blocks verbatim, so any value
defined will have a duplicate definition site, violating the \gls{SSA}
property.  \Verb|construct-ssa| recomputes a so-called \term{pruned} \gls{SSA}
form, wherein $\phi$ functions are inserted only if the variables are live
after the insertion point.  This cuts down on useless $\phi$ functions
\autocites{TDMSC,construct-ssa}.  \Vref{fig:construct-ssa} shows the
reconstructed \gls{SSA} form of the \gls{CFG} from \vref{fig:normalize-height},
where \Verb|##phi|s are inserted at the start of block $2$.  Not only is there
one for the element we're incrementing with \Verb|fixnum+fast|, but
\factor|times| introduces an induction variable to track how many iterations
have been done.  As block $2$ is the entry for the loop, it either gets the
values of these variables from the beginning (block $1$) for from the actual
body of the loop (block $3$).  Hence, we need \Verb|##phi|s to distinguish the
values.

\inputfig{construct-ssa}

The next pass, \Verb|alias-analysis|, doesn't change the \gls{CFG} we're
currently working with, so we won't have an accompanying
\lcnamecref{fig:construct-ssa}.  At a high level, \Verb|alias-analysis| is easy
to understand: it eliminates redundant memory loads and stores by rewriting
certain patterns of memory access.  If the same location is loaded immediately
after being stored, we convert the load into a \Verb|##copy| of the value we
wrote.  Two reads of the same location with no intermittent write gets the
second read turned into a \Verb|##copy|.  Similarly, if we see two writes
without a read in the middle, the first write can be removed.

\Verb|value-numbering| is the key focus of this thesis.  It will be detailed
in \cref{sec:vn}.  For now, it does to think of it as a combination of common
subexpression elimination and constant folding.  In \vref{fig:value-numbering}, 
we see several changes from this pass:
%
\begin{itemize}
%
  \item \Verb|##load-integer 23 0| in block $1$ of \vref{fig:construct-ssa}
  (which assigns the value \Verb|0| to the virtual register \Verb|23|) is
  redundant, so is replaced by \Verb|##copy 23 21|.
%
  \item \begin{sloppypar}
  The last instruction in block $2$ is redundant.  The original
  %
  \begin{center} \Verb|##compare-imm-branch 32 f cc/=| \end{center}
  %
  intuitively means ``if $\mem{32} \ne \texttt{\textbf{f}}$, go to \ldots''.
  Its replacement,
  %
  \begin{center} \Verb|##compare-integer-branch 30 26 cc<| \end{center}
  %
  means ``if $\mem{30} < \mem{26}$, go to\ldots''.  This is because the source
  register (\Verb|32|) of the original is a \Verb|##copy| of \Verb|31|, which
  itself is computed by
  %
  \begin{center} \Verb|##compare-integer 31 30 26 cc< 9|\end{center}
  %
  In pseudo-code, this is like $\mem{31} \gets (\mem{30} < \mem{26})$, where
  \Verb|9| is a temporary virtual register used for calculation.  So, the
  \Verb|##compare-imm-branch| is equivalent to a simple
  \Verb|##compare-integer-branch|, because the former is saying ``if $(\mem{30}
  < \mem{26}) \ne \texttt{\textbf{f}}$, go to\ldots'', while the latter simply
  says ``if $\mem{30} < \mem{26}$, go to\ldots''.  Converting it means that the
  calculation of \Verb|31| is unnecessary, so may be pruned away later.
  \end{sloppypar}
%
  \item \begin{sloppypar} The second operands in both \Verb|##add|s of block
  $3$ are just constants stored by \Verb|##load-integer|s.  So, these are
  turned into \Verb|##add-imm|s.
  \end{sloppypar}
%
  \item Also, the second \Verb|##load-integer| in block $3$ just loads
  \Verb|1| like the first instruction in the block.  Therefore, it's replaced
  by a \Verb|##copy|.
\end{itemize}
%
\noindent In \cref{sec:vn}, we'll see how and why this pass fails to identify
other equivalences.

\inputfig{value-numbering}

Following \Verb|value-numbering|, \Verb|copy-propagation| performs a global
pass that eliminates \Verb|##copy| instructions.  Uses of the copies are
replaced by the originals.  So, in \vref{fig:copy-propagation}, we can see that
all of the \Verb|##copy| instructions have been removed.  Also, for instance,
the use of the virtual register \Verb|25| in block $2$ has been replaced by
\Verb|21|, since \Verb|25| was a copy of it.

\inputfig{copy-propagation}

Next, dead code is removed by \Verb|eliminate-dead-code|.
\Vref{fig:eliminate-dead-code} shows that the \Verb|##compare-integer| in
block $2$ and the \Verb|##load-integer| in block $3$ were removed, since they
defined values that were never used.

\inputfig{eliminate-dead-code}

The final pass in \vref{lst:optimize-cfg}, \Verb|finalize-cfg|, itself consists
of several more passes.  We will not get into many details here, but at a high
level, the most important passes figure out how virtual registers should map to
machine registers.  First, we figure out when certain values can be unboxed.
Then, instructions are reordered in order to reduce \term{register pressure}.
That is, we try to schedule instructions around each other so that we don't
need to store more values than we have machine registers.  That way, we avoid
\term{spilling} registers onto the heap, which wastes time on slower memory
accesses.  After removing \Verb|##phi|s and leaving \gls{SSA} form, we perform
a \term{linear scan} register allocation, which replaces virtual registers with
machine registers and inserts \Verb|##spill| and \Verb|##reload| instructions
for the cases we can't avoid.  \Vref{fig:finalize-cfg} shows the output on an
Intel x86 machine, which has enough registers that we needn't spill anything.

\inputfig{finalize-cfg}

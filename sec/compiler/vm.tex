\subsection{Organization}\label{sec:compiler:vm}

At the lowest level, Factor is written atop a C++ \gls{VM} that is responsible
for basic runtime services.  This is where the non-optimizing base compiler is
implemented.  It's the base compiler's job to compile the simplest primitives:
operations that push literals onto the data stack, \factor|call|, \factor|if|,
\factor|dip|, words that access tuple slots as laid out in memory, stack
shufflers, math operators, functions to allocate/deallocate call stack frames,
etc.  The aim of the base compiler is to generate native machine code as fast
as possible.  To this end, these primitives correspond to their own stubs of
assembly code.  Different stubs are generated by Factor depending on the
instruction set supported by the particular machine in use.  Thus, the base
compiler need only make a single pass over the source code, emitting these
assembly instructions as it goes.

This compiled code is saved in an \term{image file}, which contains a complete
snapshot of the current state of the Factor instance, similar to many Smalltalk
and Lisp systems\todo{cite?}.  As code is parsed and compiled, the image is
updated, serving as a cache for compiled code.  This modified image can be
saved so that future Factor instances needn't recompile vocabularies that are
already contained in the image.

The \gls{VM} also handles method dispatch and memory management.  Method
dispatch incorporates a \term{polymorphic inline cache} to speed up generic
words.  Each generic word's call site is associated with a state:
\begin{itemize}
  \item In the \term{cold} state, the call site's instruction computes the
        right method for the class being dispatched upon, which is the
        operation we're trying to avoid.  As it does this, a polymorphic inline
        cache stub is generated, thus transitioning it to the next state.
  \item In the \term{inline cache} state, a stub has been generated that caches
        the locations of methods for classes that have already been seen.  This
        way, if a generic word at a particular call site is invoked often upon
        only a small number of classes (as is often in the case in loops, for
        example), we don't need to waste as much time doing method lookup.  By
        default, if more than three different classes are dispatched upon, we
        transition to the next state.
  \item In the \term{megamorphic} state, the call instruction points to a
        larger cache that is allocated for the specific generic word (i.e., it
        is shared by all call sites).  While not as efficient as an inline
        cache, this can still improve the performance of method dispatch.
\end{itemize}

To manage memory, the Factor \gls{VM} uses a generational \gls{GC}, which
carves out sections of space on the heap for objects of different ages.
Garbage in the oldest generation is collected with a mark-sweep-compact
algorithm, while younger generations rely on a copying collector\todo{cite?}.
This way, the \gls{GC} is specialized for large numbers of short-lived objects
that will stay in the younger generations without being promoted to the older
generation.  In the oldest space, even compiled code can be compacted.  This is
to avoid heap fragmentation in applications that must call the compiler at
runtime, such as Factor's interactive development environment.

Values are referenced by tagged pointers, which use the three least significant
bits of the pointer's address to store type information.  This is possible
because Factor aligns objects on an eight-byte boundary, so the three least
significant bits of an address are always $0$.  These bits give us eight unique
tags, but since Factor has more than eight data types, two tags are reserved to
indicate that the type information is stored elsewhere.  One is for \gls{VM}
types without their own tag, and the other is for user-defined tuples, each of
which has its own type.  Sufficiently small integers (e.g., $29$-bit integers
on a $32$-bit machine, since the other $3$ bits are used for the type tag) are
stored directly in the pointer, so they needn't be heap-allocated.  Larger
integers and floating point numbers are boxed, but the optimizing compiler may
unbox them to store floats in registers.

The \gls{VM} is meant to be minimal, as Factor is mostly \term{self-hosting}.
That is, the real workhorses of the language are written in Factor itself,
including the standard libraries, parser, object system, and the optimizing
compiler.  It's possible for the compiler to be written in Factor because of
the \term{bootstrapping} process that creates a new image from scratch.  First,
a minimal \term{boot image} is created from an existing \term{host} Factor
instance.  When the \gls{VM} runs the boot image, it initiates the
bootstrapping process.  Using the host's parser, the base compiler will compile
the core vocabularies necessary to load the optimizing compiler.  Once the
optimizing compiler can itself be compiled, it is used to recompile (and thus
optimize) all of the words defined so far.  With the basic vocabularies
recompiled, any additional vocabularies can be loaded using the optimized
compiler and saved into a new, working image.

Thus, while the Factor \gls{VM} is important, it is a small part of the code
base.  Since the bootstrapping process allows the optimizing compiler
(hereafter just ``the compiler'') to be written in the same high-level language
it's compiling, we can avoid the fiddly low-level details of the C++ backend.
This is more conducive to writing advanced compiler optimizations, which are
often complicated enough without having a concise, dynamically-typed,
garbage-collected language like Factor to help us.

\subsection{High-level Optimizations}\label{sec:compiler:tree}

To manipulate source code abstractly, we must have at least one \gls{IR}---a
data structure representing the instructions.  It's common to convert between
several \glsplural{IR} during compilation, as each form offers different
properties that facilitate particular analyses.  The Factor compiler optimizes
code in passes across two different \glsplural{IR}: first at a high-level using
the \factor|compiler.tree| vocabulary, then at a low-level with the
\factor|compiler.cfg| vocabulary.

\inputlst{tree}

The high-level \gls{IR} arranges code into a vector of \factor|node| objects,
which may themselves have children consisting of vectors of node---a tree
structure that lends to the name \factor|compiler.tree|.  This ordered sequence
of nodes represents control flow in a way that's effectively simple, annotated
stack code.  \Vref{lst:tree} shows the definitions of the tuples that represent
the ``instruction set'' of this stack code.  Each object inherits (directly or
indirectly) from the \factor|node| class, which itself inherits from
\factor|identity-tuple|.  This is a tuple whose \factor|equal?| method is
defined to always return \factor|f| so that no two instances are equivalent
unless they are the same instance.

Notice that most nodes define some sort of \factor|in-d| and \factor|out-d|
slots, which mark each of them with the input and output data stacks.  This
represents the flow of data through the program.  Here, stack values are
denoted simply by integers, giving each value a unique identifier.  An
\factor|#introduce| instance is inserted wherever the next node requires stack
values that have not yet been named.  Thus, while \factor|#introduce| has no
\factor|in-d|, its \factor|out-d| introduces the necessary stack values.
Similarly, \factor|#return| is inserted at the end of the sequence to indicate
the final state of the data stack with its \factor|in-d| slot.

The most basic operations of a stack language are, of course, pushing literals
and calling functions that pop inputs and push outputs.  The \factor|#push|
node thus has a \factor|literal| slot and an \factor|out-d| slot, giving a name
to the single element it pushes to the data stack.  \factor|#call|, of course,
is used for normal word invocations.  The \factor|in-d| and \factor|out-d|
slots effectively serve as the stack effect declaration.  In later analyses,
data about the word's definition may be stored across the \factor|body|,
\factor|method|, \factor|class|, and \factor|info| slots.

\inputlst{build-tree-1}

The word \factor|build-tree| takes a Factor quotation and constructs the
equivalent high-level \gls{IR} form.  In \vref{lst:build-tree-1}, we see the
output of the simple example
%
\factor|[ 1 + ] build-tree|.
%
Note that
%
\factor|T{ class { slot1 value1 } { slot2 value2 } ... }|
%
is the syntax for tuple literals.  The first node is a \factor|#push| for the
\factor|1| literal.  Since \factor|+| needs two input values, an
\factor|#introduce| pushes a new ``phantom'' value.  \factor|+| gets turned
into a \factor|#call| instance.  Notice the \factor|in-d| slot refers to the
values in the order that they're passed to the word, not necessarily the order
they've been introduced in the \gls{IR}.  The sum is pushed to the data stack,
so the \factor|out-d| slot is a singleton that names this value.  Finally,
\factor|#return| indicates the end of the routine, its \factor|in-d| containing
the value left on the stack (the sum pushed by \factor|#call|).

\inputlst{build-tree-2}

The next tuples in \vref{lst:tree} reassign existing values on the stack to
fresh identifiers.  The \factor|#renaming| superclass has the two subclasses
\factor|#copy| and \factor|#shuffle|.  The former represents the bijection from
elements of \factor|in-d| to elements of \factor|out-d| in the same position;
corresponding values are copies of each other.  The latter represents a more
general mapping.  Stack shufflers are translated to \factor|#shuffle| nodes
with \factor|mapping| slots that dictate how the fresh values in \factor|out-d|
correspond to the input values in \factor|in-d|.  For instance,
\vref{lst:build-tree-2} shows how \factor|swap| takes in the values
\factor|6256132| and \factor|6256133| and outputs \factor|6256134| and
\factor|6256135|, where the former is mapped to the second element
(\factor|6256133|) and the latter to the first (\factor|6256132|).  Thus,
\factor|out-d| swaps the two elements of \factor|in-d|, mapping them to fresh
identifiers.  The \factor|in-r| and \factor|out-r| slots of \factor|#shuffle|
correspond to the \term{retain} stack, which is an implementation detail beyond
the scope of this discussion.

\inputlst{build-tree-3}
\inputlst{build-tree-4}

\factor|#declare| is a miscellaneous node used for the \factor|declare|
primitive.  It simply annotates type information to stack values, as in
\vref{lst:build-tree-3}.  \factor|#terminate| is another one-off node, but a
much more interesting one.  While Factor normally requires a balanced stack,
sometimes we purposefully want to throw an error.  \factor|#terminate| is
introduced where the program halts prematurely.  When checking the stack
height, it gets to be treated specially so that \term{terminated} stack effects
unify with any other effect.  That way, branches will still be balanced even if
one of them unconditionally throws an error.  \vref{lst:build-tree-4} shows
\factor|#terminate| being introduced by the \factor|throw| word.

Next, \vref{lst:tree} defines nodes for branching based off the superclass
\factor|#branch|.  The \factor|children| slot contains vectors of nodes
representing different branches.  \factor|live-branches| is filled in during
later analyses to indicate which branches are alive so that dead ones may be
removed.  For instance, \factor|#if| will have two elements in its
\factor|children| slot representing the true and false branches.  On the other
hand, \factor|#dispatch| has an arbitrary number of children.  It corresponds
to the \factor|dispatch| primitive, which is an implementation detail of the
generic word system used to speed up method dispatch.

You may have noted the emphasis on introducing new values in \factor|out-d|
slots.  Even \factor|#shuffle|s output fresh identifiers, letting their values
be determined by its \factor|mapping|.  The reason for this is that
\factor|compiler.tree| uses \gls{SSA} form, wherein every variable is defined
by exactly one statement.  This simplifies the properties of variables, which
helps optimizations perform faster and with better results.  By giving unique
names to the targets of each assignment, the \gls{SSA} property is guaranteed.
However, \factor|#branch|es introduce ambiguity: after, say, an \factor|#if|,
what will the identifiers in \factor|out-d| be?  It depends on which branch is
taken.  To remedy this problem, after any \factor|#branch| node, Factor will
place a \factor|#phi| node---the classical \gls{SSA} ``phony function''.  The
\factor|phi-in-d| slot seen in \vref{lst:tree} is a sequence of sequences; each
one corresponds to the \factor|out-d| of the child at the same position in the
\factor|children| of the preceding node.  The \factor|#phi|'s \factor|out-d|
gives unique names to the output values, thus ensuring the \gls{SSA} property.
Though it doesn't perform any literal computation, conceptually it select the
``correct'' \factor|out-d| depending on the control flow.

\inputlst{build-tree-5}

For example, the \factor|#phi| in \vref{lst:build-tree-5} will select between
the
%
\factor|6256248|
%
return value of the first child or the 
%
\factor|6256249|
%
output of the second.  Either way, we can refer to the result as
\factor|6256250| afterwards.  The \factor|terminated| slot of the \factor|#phi|
tells us if there was a \factor|#terminate| in any of the branches.

The \factor|#recursive| node encapsulates \term{inline recursive} words.  In
Factor, words may be annotated with simple compiler declarations, which guide
optimizations.  If we follow a standard colon definition with the
\factor|inline| word, we're saying that its definition can be spliced into the
call-site, rather than generating code to jump to a subroutine.  Inline words
that call themselves must additionally be declared \factor|recursive|.  For
example, we could write
%
\factor|: foo ( -- ) foo ; inline recursive|.
%
The nodes \factor|#enter-recursive|, \factor|#call-recursive|, and
\factor|#return-recursive| denote different stages of the recursion---the
beginning, recursive call, and end, respectively.  They carry around a lot of
metadata about the nature of the recursion, but it doesn't serve our purposes
to get into the details.  Similarly, we gloss over the final nodes of
\vref{lst:tree} correspond to Factor's \gls{FFI} vocabulary, called
\factor|alien|.  At a high level, these are used to make calls to C libraries
from within Factor.

% dls.pdf verbatim:

%The optimizing compiler is structured as a series of passes operating on two
%intermediate representations (IRs), referred to as high-level IR and low-level
%IR. High-level IR represents control flow in a similar manner to a
%block-structured programming language\todo{or, y'know, like annotated stack
%code}. Low-level IR represents control flow with a control flow graph of basic
%blocks. Both intermediate forms make use of single static assignment (SSA) form
%to improve the accuracy and efficiency of analysis [12].
%
%High-level IR is constructed by the stack effect checker.  Macro expansion and
%quotation inlining is performed by the stack checker online while high-level IR
%is being con- structed. The front end does not deal with local variables, as
%these have already been eliminated.
%
%When static type information is available, Factor’s compiler can eliminate
%runtime method dispatch and allocation of in- termediate objects, generating
%code specialized to the under- lying data structures. This resembles previous
%work in soft typing [10]. Factor provides several mechanisms to facilitate
%static type propagation:
%
%\begin{itemize}
%
%\item Functions can be annotated as inline, causing the compiler to replace
%calls to the function with the function body.
%
%\item Functions can be hinted, causing the compiler to gener- ate multiple
%specialized versions of the function, each assuming different input types, with
%dispatch at the en- try point to choose the best-fitting specialization for the
%given inputs.  
%
%\item Methods on generic functions propagate the type infor- mation for their
%dispatched-on inputs.  
%
%\item Functions can be declared with static input and output types using the
%typed library.
%
%\end{itemize}
%
%The three major optimizations performed on high-level IR are sparse conditional
%constant propagation (SCCP [45]), escape analysis with scalar replacement, and
%overflow check elimination using modular arithmetic properties.  The major
%features of our SCCP implementation are an extended value lattice, rewrite
%rules, and flow sensitivity.  Our SCCP implementation augments the standard
%single- level constant lattice with information about object types, numeric
%intervals, array lengths and tuple slot types. Type transfer functions are
%permitted to replace nodes in the IR with inline expansions. Type functions are
%defined on many of the core language words.  SCCP is used to statically dispatch
%generic word calls by inlining a specific method body at the call site. This
%inlining generates new type information and new opportunities for constant
%folding, simplification and further inlining. In par- ticular, generic
%arithmetic operations which require dynamic dispatch in the general case can be
%lowered to simpler opera- tions as type information is discovered. Overflow
%checks can be removed from integer operations using numeric interval
%information. The analysis can represent flow-sensitive type information.
%Additionally, calls to closures which combina- tor inlining cannot eliminate
%are eliminated when enough in- formation is available [16].  An escape analysis
%pass is used to discover object alloca- tions which are not stored on the heap
%or returned from the current function. Scalar replacement is performed on such
%allocations, converting tuple slots into SSA values.  The modular arithmetic
%optimization pass identifies in- teger expressions in which the final result is
%taken to be modulo a power of two and removes unnecessary overflow checks from
%any intermediate addition and multiplication operations. This novel
%optimization is global and can operate over loops.

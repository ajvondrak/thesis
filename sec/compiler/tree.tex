\section{High-level Optimizations}\label{sec:compiler:tree}

To manipulate source code abstractly, we must have at least one \gls{IR}---a
data structure representing the instructions.  It's common to convert between
several \glsplural{IR} during compilation, as each form offers different
properties that facilitate particular analyses.  The Factor compiler optimizes
code in passes across two different \glsplural{IR}: first at a high-level using
the \Verb|compiler.tree| vocabulary, then at a low-level with the
\Verb|compiler.cfg| vocabulary.

\inputlst{tree}

\begin{sloppypar}
The high-level \gls{IR} arranges code into a vector of \Verb|node| objects,
which may themselves have children consisting of vectors of node---a tree
structure that lends to the name \Verb|compiler.tree|.  This ordered sequence
of nodes represents control flow in a way that's effectively simple, annotated
stack code.  \Vref{lst:tree} shows the definitions of the tuples that represent
the ``instruction set'' of this stack code.  Each object inherits (directly or
indirectly) from the \Verb|node| class, which itself inherits from
\factor|identity-tuple|.  This is a tuple whose \factor|equal?| method is
defined to always return \factor|f| so that no two instances are equivalent
unless they are the same instance.
\end{sloppypar}

Notice that most nodes define some sort of \Verb|in-d| and \Verb|out-d|
slots, which mark each of them with the input and output data stacks.  This
represents the flow of data through the program.  Here, stack values are
denoted simply by integers, giving each value a unique identifier.  An
\Verb|#introduce| instance is inserted wherever the next node requires stack
values that have not yet been named.  Thus, while \Verb|#introduce| has no
\Verb|in-d|, its \Verb|out-d| introduces the necessary stack values.
Similarly, \Verb|#return| is inserted at the end of the sequence to indicate
the final state of the data stack with its \Verb|in-d| slot.

The most basic operations of a stack language are, of course, pushing literals
and calling functions.  The \Verb|#push| node thus has a \Verb|literal|
slot and an \Verb|out-d| slot, giving a name to the single element it pushes
to the data stack.  \Verb|#call|, of course, is used for normal word
invocations.  The \Verb|in-d| and \Verb|out-d| slots effectively serve as
the stack effect declaration.  In later analyses, data about the word's
definition may be stored across the \Verb|body|, \Verb|method|,
\Verb|class|, and \Verb|info| slots.

\inputlst{build-tree-1}

The word \Verb|build-tree| takes a Factor quotation and constructs the
equivalent high-level \gls{IR} form.  In \vref{lst:build-tree-1}, we see the
output of the simple example
%
\factor|[ 1 + ] build-tree|.
%
Note that
%
\factor|T{ class { slot1 value1 } { slot2 value2 } ... }|
%
is the syntax for tuple literals.  The first node is a \Verb|#push| for the
\Verb|1| literal.  Since \factor|+| needs two input values, an
\Verb|#introduce| pushes a new ``phantom'' value.  \factor|+| gets turned
into a \Verb|#call| instance.  Notice the \Verb|in-d| slot refers to the
values in the order that they're passed to the word, not necessarily the order
they've been introduced in the \gls{IR}.  The sum is pushed to the data stack,
so the \Verb|out-d| slot is a singleton that names this value.  Finally,
\Verb|#return| indicates the end of the routine, its \Verb|in-d| containing
the value left on the stack (the sum pushed by \Verb|#call|).

\inputlst{build-tree-2}

The next tuples in \vref{lst:tree} reassign existing values on the stack to
fresh identifiers.  The \Verb|#renaming| superclass has the two subclasses
\Verb|#copy| and \Verb|#shuffle|.  The former represents the bijection from
elements of \Verb|in-d| to elements of \Verb|out-d| in the same position;
corresponding values are copies of each other.  The latter represents a more
general mapping.  Stack shufflers are translated to \Verb|#shuffle| nodes
with \Verb|mapping| slots that dictate how the fresh values in \Verb|out-d|
correspond to the input values in \Verb|in-d|.  For instance,
\vref{lst:build-tree-2} shows how \factor|swap| takes in the values
\Verb|6256132| and \Verb|6256133| and outputs \Verb|6256134| and
\Verb|6256135|, where the former is mapped to the second element
(\Verb|6256133|) and the latter to the first (\Verb|6256132|).  Thus,
\Verb|out-d| swaps the two elements of \Verb|in-d|, mapping them to fresh
identifiers.  The \Verb|in-r| and \Verb|out-r| slots of \Verb|#shuffle|
correspond to the \term{retain} stack, which is an implementation detail beyond
the scope of this discussion.

\inputlst{build-tree-3}
\inputlst{build-tree-4}

\Verb|#declare| is a miscellaneous node used for the \factor|declare|
primitive.  It simply annotates type information to stack values, as in
\vref{lst:build-tree-3}.  \Verb|#terminate| is another one-off node, but a
much more interesting one.  While Factor normally requires a balanced stack,
sometimes we purposefully want to throw an error.  \Verb|#terminate| is
introduced where the program halts prematurely.  When checking the stack
height, it gets to be treated specially so that \term{terminated} stack effects
unify with any other effect.  That way, branches will still be balanced even if
one of them unconditionally throws an error.  \vref{lst:build-tree-4} shows
\Verb|#terminate| being introduced by the \factor|throw| word.

\begin{sloppypar}
Next, \vref{lst:tree} defines nodes for branching based off the superclass
\Verb|#branch|.  The \Verb|children| slot contains vectors of nodes
representing different branches.  \Verb|live-branches| is filled in during
later analyses to indicate which branches are alive so that dead ones may be
removed.  For instance, \Verb|#if| will have two elements in its
\Verb|children| slot representing the true and false branches.  On the other
hand, \Verb|#dispatch| has an arbitrary number of children.  It corresponds
to the \factor|dispatch| primitive, which is an implementation detail of the
generic word system used to speed up method dispatch.
\end{sloppypar}

You may have noted the emphasis on introducing new values, instead of
reassigning old ones.  Even \Verb|#shuffle|s output fresh identifiers, letting
their values be determined by the \Verb|mapping|.  The reason for this is that
\Verb|compiler.tree| uses \gls{SSA} form, wherein every variable is defined by
exactly one statement.  This simplifies the properties of variables, which
helps optimizations perform faster and with better results \autocite{SSA}.  By
giving unique names to the targets of each assignment, the \gls{SSA} property
is guaranteed.  However, \Verb|#branch|es introduce ambiguity: after, say, an
\Verb|#if|, what will the \Verb|out-d| be?  It depends on which branch is
taken.  To remedy this problem, after any \Verb|#branch| node, Factor will
place a \Verb|#phi| node---the classical \gls{SSA} ``phony function'', $\phi$.
While it doesn't perform any literal computation, conceptually $\phi$ selects
between its inputs, choosing the ``correct'' argument depending on control
flow.  This can then be assigned to a unique value, preserving the \gls{SSA}
property.  In Factor, this is represented by a \Verb|phi-in-d| slot, which is a
sequence of sequences.  Each element corresponds to the \Verb|out-d| of the
child at the same position in the \Verb|children| of the preceding
\Verb|#branch| node.  The \Verb|#phi|'s \Verb|out-d| gives unique names to the
output values.

\inputlst{build-tree-5}

For example, the \Verb|#phi| in \vref{lst:build-tree-5} will select between
the \Verb|6256248| return value of the first child or the \Verb|6256249|
output of the second.  Either way, we can refer to the result as
\Verb|6256250| afterwards.  The \Verb|terminated| slot of the \Verb|#phi|
tells us if there was a \Verb|#terminate| in any of the branches.

The \Verb|#recursive| node encapsulates \term{inline recursive} words.  In
Factor, words may be annotated with simple compiler declarations, which guide
optimizations.  If we follow a standard colon definition with the
\factor|inline| word, we're saying that its definition can be spliced into the
call-site, rather than generating code to jump to a subroutine.  Inline words
that call themselves must additionally be declared \factor|recursive|.  For
example, we could write
%
\factor|: foo ( -- ) foo ; inline recursive|.
%
The nodes \Verb|#enter-recursive|, \Verb|#call-recursive|, and
\Verb|#return-recursive| denote different stages of the recursion---the
beginning, recursive call, and end, respectively.  They carry around a lot of
metadata about the nature of the recursion, but it doesn't serve our purposes
to get into the details.  Similarly, we gloss over the final nodes of
\vref{lst:tree} correspond to Factor's \gls{FFI} vocabulary, called
\Verb|alien|.  At a high level, \Verb|#alien-node|, \Verb|#alien-invoke|,
\Verb|#alien-indirect|, \Verb|#alien-assembly|, and
\Verb|#alien-callback| are used to make calls to C libraries from within
Factor.

\inputlst{optimize-tree}

Now that we're familiar with the structure of the high-level \gls{IR}, we can
turn our attention to optimization.  \Vref{lst:optimize-tree} shows the passes
performed on a sequence of nodes by the word \Verb|optimize-tree|.  Before
optimization can begin, we must gather some information and clean up some
oddities in the output of \Verb|build-tree|.  \Verb|analyze-recursive| is
called first to identify and mark loops in the tree.  Effectively, this means
we detect tail-recursion introduced by \Verb|#recursive| nodes.  Future
passes can then use this information for data flow analysis.  Then,
\Verb|normalize| makes the tree more consistent by doing two things:
%
\begin{itemize}
%
  \item All \Verb|#introduce| nodes are removed and replaced by a single
        \Verb|#introduce| at the beginning of the whole program.  This way,
        further passes needn't handle \Verb|#introduce| nodes.
%
  \item As constructed, the \Verb|in-d| of a \Verb|#call-recursive| will be
        the entire stack at the time of the call.  This assumption happens
        because we don't know how many inputs it needs until the
        \Verb|#return-recursive| is processed, because of row polymorphism.
        So, here we figure out exactly what stack entries are needed, and trim
        the \Verb|in-d| and \Verb|out-d| of each \Verb|#call-recursive|
        accordingly.
%
\end{itemize}

Once these passes have cleaned up the tree, \Verb|propagate| performs probably
the most extensive analysis of all the phases.  In short, it performs an
extended version of \gls{SCCP} \autocite{SCCP}.  The traditional data flow
analysis combines global copy propagation, constant propagation, and constant
folding in a \term{flow-sensitive} way.  That is, it will propagate information
from branches that it knows are definitely taken (e.g., because \Verb|#if| is
always given a true input).  Instead of using the typical single-level
(numeric) constant value lattice, Factor uses a lattice augmented by
information about classes, numeric value ranges, array lengths, and tuple
slots' classes.  Classes can be used in the lattice with the partial-order
protocol described briefly in \cref{sec:primer:oo}.  Additionally, the transfer
functions are allowed to inline certain calls if enough information is present.
This occurs in the transfer function since generic words' inline expansions
into particular methods provide more information, thus giving us more
opportunities for propagation.  This is particularly useful for arithmetic
words.  In Factor, words like \factor|+| and \factor|*| are generics that work
across all sorts of numeric representations, be they \factor|fixnum|s,
\factor|float|s, \factor|bignum|s, etc.  If the operation overflows, the values
are automatically cast up to larger representations.  But iterated refinement
of the inputs' classes can let the compiler select more specific, efficient
methods (e.g., if both arguments are \factor|fixnum|s).

Interval propagation also helps propagate class information.  By refining the
range of possible values a particular item can have, we might discover that,
say, it's small enough to fit in a \factor|fixnum| rather than a
\factor|bignum|.  There are plenty more things that interval propagation can
tell us, too.  For example, it may give us enough information to remove
overflow checks performed by numeric words.  And if the interval has zero
length, we may replace the value with a constant.  This then continues getting
propagated, contributing to constant folding and so forth.

\Verb|propagate| iterates through the nodes collecting all of this data until
reaching a stable point where inferences can no longer be drawn.  Technically,
this information doesn't alter the tree at all; we merely store it so that
speculative decisions may be realized later.  The next word in
\vref{lst:optimize-tree}, \Verb|cleanup|, does just this by inlining words,
folding constants, removing overflow checks, deleting unreachable branches, and
flattening inline-recursive words that don't actually wind up calling
themselves (e.g., because the calls got constant-folded).

\inputlst{escaping}

The next major pass is \Verb|escape-analysis|, whose information is used for
the actual transformation \Verb|unbox-tuples|.  This discovers tuples that
\term{escape} by getting passed outside of a word.  For instance, the inputs to
\Verb|#return| obviously escape, as they are passed to the world outside of
the word in question.  Similarly, inputs to the \Verb|#call| of another word
escape.  So, though the tuples in \Verb|escaping-via-#return| and
\Verb|escaping-via-#call| in \vref{lst:escaping} both escape, we can see the
one in \Verb|non-escaping| does not.  In fact, the last allocation is
unnecessary.  By identifying this, \Verb|unbox-tuples| can then rewrite the
code to avoid allocating a \Verb|data-struct| altogether, instead
manipulating the slots' values directly.  Note that this only happens for
immutable tuples, all of whose slots are \factor|read-only|.  Otherwise, we
would need to perform more advanced pointer analyses to discover aliases.

\Verb|apply-identities| follows to simplify words with known identity
elements.  If, say, an argument to \factor|+| is \Verb|0|, we can simply
return the other argument.  This converts the \Verb|#call| to \factor|+| into
a simple \Verb|#shuffle|.  These identities are defined for most arithmetic
words.

Another simple few passes come next in \vref{lst:optimize-tree}.  True to its
name, \Verb|compute-def-use| computes where \gls{SSA} values are defined and
used.  Values that are never used are eliminated by \Verb|remove-dead-code|.
\Verb|?check| conditionally performs some consistency checks on the tree,
mostly to make sure that no errors were introduced in the stack flow.  If a
global variable isn't toggled on, this part is skipped.  We run
\Verb|compute-def-use| again to update the information after altering the
tree with dead code elimination.

Finally, \Verb|optimize-modular-arithmetic| performs a form of
strength-reduction on artihmetic words that only use the low-order bits of
their inputs/results, which may also remove more unnecessary overflow checks.
\Verb|finalize| cleans up a few random miscellaneous bits of the tree
(removing empty shufflers, deleting \Verb|#copy| nodes, etc.) in preparation
for lower-level optimizations.

\todo[inline]{Double-check zealous syntax-highlighting}

\chapter{Introduction}\label{sec:intro}

Compilers translate programs written in a source language (e.g., Java) into
semantically equivalent programs in some target language (e.g., assembly code).
They let us make our source language arbitrarily abstract so we can write
programs in ways that humans understand while letting the computer execute
programs in ways that machines understand.  In a perfect world, such
translation would be straightforward.  Reality, however, is unforgiving.
Straightforward compilation results in clunky target code that performs a lot
of redundant computations.  To produce efficient code, we must rely on
less-than-straightforward methods.  Typical compilers go through a stage of
\term{optimization}, whereby a number of semantics-preserving transformations
are applied to an \term{\acrlong{IR}} of the source code.  These then
(hopefully) produce a more efficient version of said representation.
Optimizers tend to work in \term{phases}, applying specific transformations
during any given phase.

\Gls{GVN} is such a phase performed by many highly-optimizing compilers.  Its
roots run deep through both the theoretical and the practical.  Using the
results of this analysis, the compiler can identify expressions in the source
code that produce the same value---not just by lexical comparison (i.e.,
comparing variable names), but by proving equivalences between what's actually
computed at runtime.  These expressions can then be simplified by further
algorithms for redundancy elimination.  This is the very essence of most
compiler optimizations: avoid redundant computation, giving us code that runs
as quickly as possible while still following what the programmer originally
wrote.

High-level, dynamic languages tend to suffer from efficiency issues.  They're
often interpreted rather than compiled, and perform no heavy optimization of
the source code.  However, the Factor language (\url{http://factorcode.org})
fills an intriguing design niche, as it's very high-level yet still fully
compiled.  It's still young, though, so its compiler craves all the
improvements it can get.  In particular, while the current Factor version (as
of this writing, $0.94$) has a \term{local} value numbering analysis, it is
inferior to \gls{GVN} in several significant ways.

In this thesis, we explore the implementation and use of \gls{GVN} in improving
the strength of optimizations in Factor.  Because Factor is a young and
relatively unknown language, \cref{sec:primer} provides a short tutorial,
laying a foundation for understanding the changes.  \Cref{sec:compiler}
describes the overall architecture of the Factor compiler, highlighting where
the exact contributions of this thesis fit in.  Finally, \cref{sec:vn} goes
into detail about the existing and new value numbering passes, closing with a
look at the results achieved and directions for future work. 

In the unlikely event that you want to cite this thesis, you may use the
following \textsc{Bib}\TeX~entry:
\begin{Verbatim}[gobble=2,frame=single]
  @mastersthesis{vondrak:11,
    author = {Alex Vondrak},
    title  = {Global Value Numbering in Factor},
    school = {California Polytechnic State University, Pomona},
    month  = sep,
    year   = {2011},
  }
\end{Verbatim}

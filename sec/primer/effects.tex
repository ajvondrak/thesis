\section{Stack Effects}\label{sec:primer:effects}

Everything else about Factor follows from the stack-based structure outlined in
\cref{sec:primer:stack-based}.  Consecutive words transform the stack in
discrete steps, thereby shaping a result.  In a way, words are functions from
stacks to stacks---from ``before'' to ``after''---and whitespace is effectively
function composition.  Even literals (numbers, strings, arrays, quotations,
etc.) can be thought of as functions that take in a stack and return that stack
with an extra element pushed onto it.

With this in mind, Factor requires that the number of elements on the stack
(the \term{stack height}) is known at each point of the program in order to
ensure consistency.  To this end, every word is associated with a \term{stack
effect} declaration using a notation implemented by parsing words.  In general,
a stack effect declaration has the form
%
\begin{center} \Verb|( input1 input2 ... -- output1 output2 ... )| \end{center}
%
\noindent where the parsing word \Verb|(| scans forward for the special token
\Verb|--| to separate the two sides of the declaration, and then for the
\Verb|)| token to end the declaration.  The names of the intermediate tokens
don't technically matter---only how many of them there are.  However, names
should be meaningful for clarity's sake.  The number of tokens on the left side
of the declaration (before the \Verb|--|) indicates the minimum stack height
expected before executing the word.  Given exactly this number of inputs, the
number of tokens on the right side is the stack height after executing the
word.

For instance, the stack effect of the \factor|+| word is
%
\Verb|( x y -- z )|,
%
as it pops two numbers off the stack and pushes one number (their sum) onto the
stack.  This could be written any number of ways, though.
%
\Verb|( x x -- x )|,
%
\Verb|( number1 number2 -- sum )|,
%
and
%
\Verb|( m n -- m+n )|
%
are all equally valid.  Further, while the stack effect
%
\Verb|( junk x y -- junk z )|
%
has the same relative height change, this declaration would be wrong, since
\factor|+| might legitimately be called on only two inputs.

\inputfig{shufflers}

For the purposes of documentation, of course, the names in stack effects do
matter.  They correspond to elements of the stack from bottom-to-top.  So, the
rightmost value on either side of the declaration names the top element of the
stack.  We can see this in \cref{fig:shufflers}, which shows the effects of
standard \term{stack shuffler} words.  These words are used for basic data flow
in Factor programs.  For example, to discard the top element of the stack, we
use the \factor|drop| word, whose effect is simply
%
\Verb|( x -- )|.
%
To discard the element just below the top of the stack, we use \factor|nip|,
whose effect is
%
\Verb|( x y -- y )|.
%
This stack effect indicates that there are at least two elements on the stack
before \factor|nip| is called: the top element is \Verb|y|, and the next
element is \Verb|x|.  After calling the word, \Verb|x| is removed, leaving
the original \Verb|y| still on top of the stack.  Other shuffler words that
remove data from the stack are
%
\factor|2drop| with the effect \Verb|( x y -- )|,
%
\factor|3drop| with the effect \Verb|( x y z -- )|, and
%
\factor|2nip| with the effect \Verb|( x y z -- z )|.

The next stack shufflers duplicate data.  \factor|dup| copies the top element
of the stack, as indicated by its effect \Verb|( x -- x x )|.  \factor|over|
has the effect \Verb|( x y -- x y x )|, which tells us that it expects at
least two inputs: the top of the stack is \Verb|y|, and the next object is
\Verb|x|.  \Verb|x| is copied and pushed on top of the two original
elements, sandwiching \Verb|y| between two \Verb|x|s.  Other shuffler words
that duplicate data on the stack are
%
\factor|2dup| with the effect \Verb|( x y -- x y x y )|,
%
\factor|3dup| with the effect \Verb|( x y z -- x y z x y z )|,
%
\factor|2over| with the effect \Verb|( x y z -- x y z x y )|, and
%
\factor|pick| with the effect \Verb|( x y z -- x y z x )|.

True to the name \factor|swap|, the final shuffler in \cref{fig:shufflers}
permutes the top two elements of the stack, reversing their order.  The stack
effect
%
\Verb|( x y -- y x )|
%
indicates as much.  The left side denotes that two inputs are on the stack (the
top is \Verb|y|, the next is \Verb|x|), and the right side shows the
outputs are swapped (the top element is \Verb|x| and the next is \Verb|y|).
Factor has other words that permute elements deeper into the stack.  However,
their use is discouraged because it's harder for the programmer to mentally
keep track of more than a couple items on the stack.  We'll see how more
complex data flow patterns are handled in \cref{sec:primer:data-flow}.

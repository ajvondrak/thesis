\subsection{Stack-Based Languages}\label{sec:primer:stack-based}

\inputfig{rpn}

Like \gls{RPN} calculators, Factor's evaluation model uses a global stack upon
which operands are pushed before operators are called.  This naturally
facilitates postfix notation, in which operators are written after their
operands.  For example, instead of \texttt{1~+~2}, we write \texttt{1~2~+}.
\vref{fig:rpn} shows how \texttt{1~2~+} works conceptually:
\begin{itemize}
  \item \texttt{1} is pushed onto the stack
  \item \texttt{2} is pushed onto the stack
  \item \texttt{+} is called, so two values are popped from the stack, added,
        and the result (\texttt{3}) is pushed back onto the stack
\end{itemize}
Other stack-based programming languages include Forth\todo{cite},
Joy\todo{cite}, Cat\todo{cite}, and PostScript\todo{cite}.

The strength of this model is its simplicity.  Evaluation essentially goes
left-to-right: literals (like \texttt{1} and \texttt{2}) are pushed onto the
stack, and operators (like \texttt{+}) perform some computation using values
currently on the stack.  This ``flatness'' makes parsing easier, since we don't
need complex grammars with subtle ambiguities and precedence issues.  Rather,
we basically just scan left-to-right for tokens separated by whitespace.  In
the Forth tradition, functions are called \term{words} since they're made up of
any contiguous non-whitespace characters.  This also lends to the term
\term{vocabulary} instead of ``module'' or ``library''.   In Factor, the parser
works as follows.
\begin{itemize}
  \item If the current character is a double-quote (\texttt{"}), try to
        parse ahead for a string literal.
  \item Otherwise, scan ahead for a single token.
        \begin{itemize}
          \item If the token is the name of a \term{parsing word}, that word is
                invoked with the parser's current state.
          \item If the token is the name of an ordinary (i.e., non-parsing)
                word, that word is added to the parse tree.
          \item Otherwise, try to parse the token as a numeric literal.
        \end{itemize}
\end{itemize}

Parsing words serve as hooks into the parser, letting Factor users extend the
syntax dynamically.  For instance, instead of having special knowledge of
comments built into the parser, the parsing word \texttt{!}~scans forward for a
newline and discards any characters read (adding nothing to the parse tree).

Similarly, there are parsing words for what might otherwise be hard-coded
syntax for data structure literals.  Many act as sided delimeters: the parsing
word for the left-delimiter will parse ahead until it reaches the
right-delimiter, using whatever was read in between to add objects to the data
structure.  For example, \factor|{ 1 2 3 }| denotes an array of three numbers.
Note the deliberate spaces in between the tokens, so that the delimeters are
themselves distinct words.  In
%
\Verb[showspaces]|{ 1 2 3 }| (with spaces as marked), the parsing word \Verb|{|
parses objects until it reaches \Verb|}|, collecting the results into an array.
The \verb|{| word would not be called if not for that space, whereas
%
\Verb[showspaces]|{1 2 3}| parses as the word \Verb|{1|, the number \Verb|2|,
and the word \Verb|3}|---not an array.  Further, since the left-delimeter words
parse recursively, such literals can be nested, contain comments, etc.  Other
literals include \vpageref[the following][those in
\vref{lst:literals}~]{lst:literals}.

\inputlst[h]{literals}

\inputfig{quot}

A particularly important set of parsing words in Factor are the square
brackets, \Verb|[| and \Verb|]|.  Any code in between such brackets is
collected up into a special sequence called a \term{quotation}.  Essentially,
it's a snippet of code whose execution is suppressed.  The code inside a
quotation can then be run with the \factor|call| word.  Quotations are like
anonymous functions in other languages, but the stack model makes them
conceptually simpler, since we don't have to worry about variable binding and
the like.  Consider a small example like \factor|1 2 [ + ] call|.  You can
think of \factor|call| working by ``erasing'' the brackets around a quotation,
so this example behaves just like \factor|1 2 +|.  \vref{fig:quot} shows its
evaluation: instead of adding the numbers immediately, \factor|+| is placed in
a quotation, which is pushed to the stack.  The quotation is then invoked by
\factor|call|, so \factor|+| pops and adds the two numbers and pushes the
result onto the stack.  We'll show how quotations are used in
\cref{sec:primer:combinators}.

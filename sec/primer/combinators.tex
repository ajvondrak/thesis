\section{Combinators}\label{sec:primer:combinators}

Quotations, introduced in \cref{sec:primer:stack-based}, form the basis of both
control flow and data flow in Factor.  Not only are they the equivalent of
anonymous functions, but the stack model also makes them syntactically
lightweight enough to serve as blocks akin to the code between curly braces in
C or Java.  Higher-order words that make use of quotations on the stack are
called \term{combinators}.  It's simple to express familiar conditional logic
and loops using combinators, as we'll show first.  In the presence of explicit
data flow via stack operations, even more patterns arise that can be abstracted
away.  The last half of this \lcnamecref{sec:primer:combinators} explores how
we can use combinators to express otherwise convoluted stack-shuffling logic
more succinctly.

\subsection{Control Flow}\label{sec:primer:control-flow}

The most primitive form of control flow in typical programming languages is, of
course, the \mint{java}|if| statement, and the same holds true for Factor.  The
only difference is that Factor's \texttt{\textbf{if}} isn't syntactically
significant---it's just another word, albeit implemented as a primitive.  For
the moment, it will do to think of \texttt{\textbf{if}} as having the stack
effect
%
\Verb|( ? true false -- )|.
%
The third element from the top of the stack is a boolean condition, and it's
followed by two quotations.  The first quotation (\Verb|true|) is called if the
condition is true, and the second quotation (\Verb|false|) is called if the
condition is false.  Specifically, \factor|f| is a special object in Factor for
falsity.  It is a singleton object---the sole instance of the \factor|f|
class---and is the only false value in the entire language.  Any other object
is necessarily boolean true.  For a canonical boolean, there is the \factor|t|
object, but its truth value exists only because it is not \factor|f|.  Basic
\texttt{\textbf{if}} use is shown in \vref{lst:if}.  The first example will
print ``odd'', the second ``empty'', and the third ``isn't f''.  All of them
leave nothing on the stack.

\inputlst{if}

However, the simplified stack effect for \texttt{\textbf{if}} is quite
restrictive.  Because the effect
%
\Verb|( ? true false -- )|
%
has no extra inputs and no outputs at all, it intuitively means that the
\Verb|true| and \Verb|false| quotations both have the effect
%
\Verb|( -- )|.
%
We'd like to loosen this restriction, but per \cref{sec:primer:effects}, Factor
must know the stack height after the \texttt{\textbf{if}} call.  We could give
\texttt{\textbf{if}} the effect
%
\Verb|( x ? true false -- y )|
%
so that the two quotations could each have the stack effect
%
\Verb|( x -- y )|.
%
This would work for the \Verb|example1| word in \vref{lst:if-effects}, yet it's
just as restrictive.  For instance, the \Verb|example2| word would need
\texttt{\textbf{if}} to have the effect
%
\Verb|( x y ? true false -- z )|,
%
since each branch has the effect
%
\Verb|( x y -- z )|.
%
Furthermore, the quotations might even have different effects, but still leave
the overall stack height balanced.  Only one item is left on the stack after a
call to \Verb|example3| regardless, even though the two quotations have
different stack effects: \factor|+| has the effect
%
\Verb|( x y -- z )|,
%
while \factor|drop| has the effect
%
\Verb|( x -- )|.

In reality, there are infinitely many correct stack effects for
\texttt{\textbf{if}}.  Factor has a special notation for such
\term{row-polymorphic} stack effects.  If a token in a stack effect begins with
two dots, like \Verb|..a| or \Verb|..b|, it is a \term{row variable}.  If
either side of a stack effect begins with a row variable, it represents any
number of inputs or outputs.  Thus, we could give \texttt{\textbf{if}} the
stack effect
%
\begin{center} \Verb|( ..a ? true false -- ..b )| \end{center}
%
\noindent to indicate that there may be any number of inputs below the
condition on the stack, and that any number of outputs will be present after
the call to \texttt{\textbf{if}}.  Note that these numbers aren't necessarily
equal, which is why we use distinct row variables (\Verb|..a| and \Verb|..b|)
in this case.  However, this still isn't quite enough to capture the stack
height requirements.  It doesn't communicate that \Verb|true| and \Verb|false|
must affect the stack in the same ways, which has remained tacit to this point.
For this, the notation
%
\Verb|quot: ( stack -- effect )|
%
gives quotations a nested stack effect.  Using the same names for row variables
in both the ``inner'' and ``outer'' stack effects will refer to the same number
of inputs or outputs.  Thus, our final (correct) stack effect for
\texttt{\textbf{if}} is 
%
\begin{center}
%
  \Verb|( ..a ? true: ( ..a -- ..b ) false: ( ..a -- ..b ) -- ..b )|
%
\end{center}
%
\noindent This tells us that the \Verb|true| quotation and the \Verb|false|
quotation will each leave the stack at the same height as \texttt{\textbf{if}}
does overall, and that neither expects any extra inputs.

\inputlst{if-effects}

Though \texttt{\textbf{if}} is necessarily a language primitive, other control
flow constructs are defined in Factor itself.  It's simple to write combinators
for iteration and looping as recursive words that invoke quotations.
\Vref{lst:loops} showcases some common looping patterns.  The most basic yet
versatile word is \factor|each|.  Its stack effect is
%
\begin{center}
%
  \Verb|( ... seq quot: ( ... x -- ... ) -- ... )|
%
\end{center}
%
\noindent Each element \Verb|x| of the sequence \Verb|seq| will be passed to
\Verb|quot|, which may use any of the underlying stack elements.  Here, unlike
\texttt{\textbf{if}}, we enforce that \Verb|quot|'s output stack height is
exactly one less than the input.  Otherwise, depending on the number of
elements in \Verb|seq|, we might dig arbitrarily deep into the stack or flood
it with a varying number of values.  The first use of \factor|each| in
\vref{lst:loops} is balanced, as the quotation has the effect
%
\Verb|( str -- )|
%
and no additional items were on the stack to begin with (i.e., \Verb|...|
stands in for $0$ elements).  Essentially, it's equivalent to
%
\factor|"Lorem" print "ipsum" print "dolor" print|.
%
On the other hand, the quotation in the second example has the stack effect
%
\Verb|( total n -- total+n )|.
%
This is still balanced, since there is one additional item below the sequence
on the stack (namely \Verb|0|), and one element is left by the end (the sum of
the sequence elements).  So, this example is the same as
%
\factor|0 1 + 2 + 3 +|.

\inputlst{loops}

Any instance of the extensive \Verb|sequence| mixin will work with
\factor|each|, making it very flexible.  The third example in \vref{lst:loops}
shows \factor|iota|, which is used here to create a \term{virtual} sequence of
integers from $0$ to $9$ (inclusive).  No actual sequence is allocated, merely
an object that behaves like a sequence.  In Factor, it's common practice to use
\factor|iota| and \factor|each| in favor of repetitive C-like \mint{c}|for|
loops.

Of course, we sometimes don't need the induction variable in loops.  That is,
we just want to execute a body of code a certain number of times.  For these
cases, there's the \factor|times| combinator, with the stack effect
%
\begin{center}
%
  \Verb|( ... n quot: ( ... -- ... ) -- ... )|
%
\end{center}
%
\noindent This is similar to \factor|each|, except that \Verb|n| is a number
(so we needn't use \factor|iota|) and the quotation doesn't expect an extra
argument (i.e., a sequence element).  Therefore, the example in
\vref{lst:loops} is equivalent to
%
\factor|"Ho!" print "Ho!" print "Ho!" print|.

Naturally, Factor also has the \factor|while| combinator, whose stack effect is
%
\begin{center}
%
  \Verb|( ..a pred: ( ..a -- ..b ? ) body: ( ..b -- ..a ) -- ..b )|
%
\end{center}
%
\noindent The row variables are a bit messy, but it works as you'd expect: the
\Verb|pred| quotation is invoked on each iteration to determine whether
\Verb|body| should be called.  The \factor|do| word is a handy modifier for
\factor|while| that simply executes the body of the loop once before leaving
\factor|while| to test the precondition as per usual.  Thus, the last example
in \vref{lst:loops} executes the body once, despite the condition being
immediately false.

In the preceding combinators, quotations were used like blocks of code.  But
really, they're the same as anonymous functions from other languages.  As such,
Factor borrows classic tools from functional languages, like \factor|map| and
\factor|filter|, as shown in \vref{lst:high-order}.  \factor|map| is like
\factor|each|, except that the quotation should produce a single output.  Each
such output is collected up into a new sequence of the same class as the input
sequence.  Here, the example produces
%
\factor|{ 2 3 4 }|.
%
\factor|filter| selects only those elements from the sequence for which the
quotation returns a true value.  Thus, the \factor|filter| in
\vref{lst:high-order} outputs
%
\factor|{ 2 4 }|.
%
Even \factor|reduce| is in Factor, also known as a \term{left fold}.  An
initial element is iteratively updated by pushing a value from the sequence and
invoking the quotation.  In fact, \factor|reduce| is defined as
%
\factor|swapd each|,
%
where \factor|swapd| is a shuffler word with the stack effect
%
\Verb|( x y z -- y x z )|.
%
Thus, the example in \vref{lst:high-order} is the same as
%
\factor|0 { 1 2 3 } [ + ] each|,
%
as in \vref{lst:loops}.

\inputlst{high-order}

These are just some of the control flow combinators defined in Factor.  Several
variants exist that meld stack shuffling with control flow, or can be used to
shorten common patterns such as empty false branches.  An entire list is beyond
the scope of our discussion, but the ones we've studied should give a solid
view of what standard conditional execution, iteration, and looping looks like
in a stack-based language.

\subsection{Data Flow}\label{sec:primer:data-flow}

While avoiding variables makes it easier to refactor code, keeping mental track
of the stack can be taxing.  If we need to manipulate more than the top few
elements of the stack, code gets harder to read and write.  Since the flow of
data is made explicit via stack shufflers, we actually wind up with redundant
patterns of data flow that we otherwise couldn't identify.  In Factor, there
are several combinators that clean up common stack-shuffling logic, making code
easier to understand.

\inputlst{preserve}

The first combinators we'll look at are \factor|dip| and \factor|keep|.  These
are used to preserve elements of the stack.  When working with several values,
sometimes we don't want to use all of them at quite the same time.  Using
\factor|drop| and the like wouldn't help, as we'd lose the data altogether.
Rather, we want to retain certain stack elements, do a computation, then
restore them.  For an uncompelling but illustrative example, suppose we have
two values on the stack, but we want to increment the second element from the
top.  \Verb|without-dip1| in \cref{lst:preserve} shows one strategy, where we
shuffle the top element away with \factor|swap|, perform the computation, then
\factor|swap| the top back to its original place.  A cleaner way is to call
\factor|dip| on a quotation, which will execute that quotation just under the
top of the stack, as in \Verb|with-dip1|.  While the stack shuffling in
\Verb|without-dip1| isn't terribly complicated, it doesn't convey our meaning
very well.  Shuffling the top element out of the way becomes increasingly
difficult with more complex computations.  In \Verb|without-dip2|, we want to
call \factor|-| on the two elements below the top.  For lack of a more robust
stack shuffler, we use \factor|2over| to isolate the two values so we can call
\factor|-|.  The rest of the word consists of shuffling to get rid of excess
values on the stack.  It's also worth noting that \factor|swapd| is a
deprecated word in Factor, since its use starts making code harder to reason
about.  Alternatively, we could dream up a more complex stack shuffler with
exactly the stack effect we wanted in this situation.  But this solution
doesn't scale: what if we had to calculate something that required more inputs
or produced more outputs?  Clearly, \factor|dip| provides a cleaner alternative
in \Verb|with-dip2|.

\factor|keep| provides a way to hold onto the top element of the stack, but
still use it to perform a computation.  In general,
%
\factor|[ ... ] keep|
%
is equivalent to
%
\factor|dup [ ... ] dip|.
%
Thus, the current top of the stack remains on top after the use of
\factor|keep|, but the quotation is still invoked with that value.  In
\Verb|with-keep1| in \cref{lst:preserve}, we want to increment the top, but
stash the result below.  Again, this logic isn't terribly complicated, though
\Verb|with-keep1| does away with the shuffling.  \Verb|without-keep2| shows
a messier example where a simple \factor|dup| will not save us, as we're using
more than just the top element in the call to \factor|-|.  Rather, three of the
four words in the definition are dedicated to rearranging the stack in just the
right way, obscuring the call to \factor|-| that we really want to focus on.
On the other hand, \Verb|with-keep2| places the subtraction word
front-and-center in its own quotation, while \factor|keep| does the work of
retaining the top of the stack.

\inputlst{cleave}

The next set of combinators apply multiple quotations to a single value.  The
most general form of these so-called \term{cleave} combinators is the word
\factor|cleave|, which takes an array of quotations as input, and calls each
one in turn on the top element of the stack.  Of course, for only a couple of
quotations, wrapping them in an array literal becomes cumbersome.  The word
\factor|bi| exists for the two-quotation case, and \factor|tri| for the three
quotations.  Cleave combinators are often used to extract multiple slots from a
tuple.  \cref{lst:cleave} shows such a case in the \Verb|with-bi| word, which
improves upon using just \factor|keep| in the \Verb|without-bi| word.  In
general, a series of \factor|keep|s like
%
\factor|[ a ] keep [ b ] keep c|
%
is the same as
%
\factor|{ [ a ] [ b ] [ c ] } cleave|,
%
%XXX fuck this dumb-shit fucking cunt what the fuck does it fucking want
which is more readable.  We can see this in action in the difference between
\Verb|without-tri| and \Verb|with-tri| in \cref{lst:cleave}.  In cases
where we need to apply multiple quotations to a set of values instead of just a
single one, there are also the variants \factor|2cleave| and \factor|3cleave|
(and the corresponding \factor|2bi|, \factor|2tri|, \factor|3bi|, and
\factor|3tri|), which apply the quotations to the top two and three elements of
the stack, respectively.

\inputlst{spread}

\begin{sloppypar}
To apply multiple quotations to multiple values, Factor has \term{spread}
combinators.  Whereas cleave combinators abstract away repeated instances of
\factor|keep|, spread combinators replace nested calls to \factor|dip|.  The
archetypical combinator, \factor|spread|, takes an array of quotations, like
\factor|cleave|.  However, instead of applying each one to the top element of
the stack, each one corresponds to a separate element of the stack.  Thus,
%
\factor|{ [ a ] [ b ] } spread|
%
invokes \Verb|b| on the top element, and \Verb|a| on the element beneath
the top.  Much like \factor|cleave|, there are shorthand words for the two- and
three-quotation cases.  These are suffixed with asterisks to indicate the
spread variants, so we have \factor|bi*| and \factor|tri*|.  In
\cref{lst:spread}, the \Verb|without-bi*| word shows the simple \factor|dip|
pattern that \factor|bi*| encapsulates.  Here, we're converting the string
\Verb|str1| (the second element from the top) into uppercase and
\Verb|str2| (the top element) to lowercase.  In \Verb|with-bi*|, the
\factor|>upper| and \factor|>lower| words are seen first, uninterrupted by an
extra word, making the code easier to read.  More compelling is the way that
\factor|tri*| replaces the \factor|dip|s that can be seen in
\Verb|without-tri*|.  In comparison, \Verb|with-tri*| is less nested and
easier to comprehend at first glance.  While there are \factor|2bi*| and
\factor|2tri*| variants that spread quotations to two values apiece on the
stack, they are uncommon in practice.
\end{sloppypar}

\inputlst{apply}

Finally, \term{apply} combinators invoke a single quotation on multiple stack
entries in turn.  While there is a generalized word, it's more common to use
the corresponding shorthands.  Here, they are suffixed with at-signs, so
\factor|bi@| applies a quotation to each of the top two stack values, and
\factor|tri@| to each of the top three.  This way, rather than duplicate code
for each time we want to call a word, we need only specify it once.  This is
demonstrated clearly in \cref{lst:apply}.  In \Verb|without-bi@|, we see that
the quotation \factor|[ sq ]| (for squaring numbers) appears twice for the call
to \factor|bi*|.  In general, we can replace spread combinators whose
quotations are all the same with a single quotation and an apply combinator.
Thus, \Verb|with-bi@| cuts down on the duplicated \factor|[ sq ]| in
\Verb|without-bi@|.  Similarly, we can replace the three repeated quotations
passed to \factor|tri*| in \Verb|without-tri@| with a single instance passed
to \factor|tri@| in \Verb|with-tri@|.  Like other data flow combinators, we
have the numbered variants.  \factor|2bi@| has the stack effect
%
\Verb|( w x y z quot -- )|,
%
where \Verb|quot| expects two inputs, and is thus applied to \Verb|w| and
\Verb|x| first, then to \Verb|y| and \Verb|z|.  Similarly, \factor|2tri@|
applies the quotation to the top six objects of the stack in groups of two.
Like their spread counterparts, they are not used very much.

\todo[inline]{Some wrap-up that isn't completely lame.}

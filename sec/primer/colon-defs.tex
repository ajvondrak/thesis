\subsection{Definitions}\label{sec:primer:colon-defs}

\inputlst{hello-world}

\inputlst{norm}

Using the basic syntax of stack effect declarations described in
\cref{sec:primer:effects}, we can now understand how to define words.  Most
words are defined with the parsing word \factor|:|, which scans for a name, a
stack effect, and then any words up until the \factor|;| token, which together
become the body of the definition.  Thus, the classic example in
\vref{lst:hello-world} defines a word named \Verb|hello-world| which expects
no inputs and pushes no outputs onto the stack.  When called, this word will
display the canonical greeting on standard output using the \factor|print|
word.

\inputfig{norm-steps}

A slightly more interesting example is the \Verb|norm| word in
\vref{lst:norm}.  This squares each of the top two numbers on the stack, adds
them, then takes the square root of the sum.  \vref{fig:norm-steps} shows this
in action.  By defining a word to perform these steps, we can replace virtually
any instance of
%
\factor|dup * swap dup * + sqrt|
%
in a program simply with \Verb|norm|.  This is a deceptively important point.
Data flow is made explicit via stack manipulation rather than being hidden in
variable assignments, so repetitive patterns become painfully evident.  This
makes identifying, extracting, and replacing redundant code easy.  Often, you
can just copy a repetitive sequence of words into its own definition verbatim.
This emphasis on ``factoring'' your code is what gives Factor its name.

\inputlst{norm-factored}

As a simple case in point, we see the subexpression \factor|dup *| appears
twice in the definition of \Verb|norm| in \vref{lst:norm}.  We can easily
factor that out into a new word and substitute it for the old expressions, as
in \vref{lst:norm-factored}.  By contrast, programs in more traditional
languages are laden with variables and syntactic noise that require more work
to refactor: identifying free variables, pulling out the right functions
without causing finicky syntax errors, calling a new function with the right
variables, etc.  Though Factor's stack-based paradigm is atypical, it is part
of a design philosophy that aims to facilitate readable code focusing on short,
reusable definitions.

\subsection{Object Orientation}\label{sec:primer:oo}

You may have noticed that the examples in \cref{sec:primer:colon-defs} did not
use type declarations.  While Factor is dynamically typed for the sake of
simplicity, it does not do away with types altogether.  In fact, Factor is
object-oriented.  However, its object system doesn't rely on classes possessing
particular methods, as is common.  Instead, it uses \term{generic words} with
methods implemented for particular classes.  To start, though, we must see how
classes are defined.

\subsubsection{Tuples}

\inputlst{tuples}

The central data type of Factor's object system is called a \term{tuple}, which
is a class composed of named \term{slots}---like instance variables in other
languages.  Tuples are defined with the \factor|TUPLE:| parsing word as shown
in \vref{lst:tuples}.  A class name is specified first; if it is followed by
the \factor|<| token and a superclass name, the tuple inherits the slots of the
superclass.  If no superclass is specified, the default is the \factor|tuple|
class.

Slots can be specified in several ways.  The simplest is to just provide a
single token, which is the name of the slot.  This slot can then hold any type
of object.  Using the syntax
%
\factor|{ name class }|,
%
a slot can be limited to hold only instances of a particular class, like
\factor|integer| or \factor|string|.  There are other forms of slot specifiers,
which we will cover after some examples.

\inputlst{colors}

Consider the two tuples defined in \vref{lst:colors}.  The first,
\factor|color|, has no slots.  With every tuple, a class predicate is defined
with the stack effect
%
\verb|( object -- ? )|
%
whose name is the class suffixed by a question mark.  Here, the word
\factor|color?| is defined, which pushes a boolean (in Factor, either
\factor|t| or \factor|f|) indicating whether the top element of the stack is an
instance of the \factor|color| class.  The second tuple, \factor|rgb|, inherits
from the \factor|color| class.  While this doesn't give \factor|rgb| any
different slots, it does mean that an instance of \factor|rgb| will return
\factor|t| for \factor|color?| due to the ``is-a'' relationship between
subclass and superclass.  The word \factor|rgb?| is similarly defined.

Notice that the \factor|rgb| tuple declares three slots named \factor|red|,
\factor|green|, and \factor|blue|.  Since the slots' classes aren't declared,
any sort of object can be stored in them.  A set of methods are defined to
manipulate an \factor|rgb| instance's slots.  Three \term{reader} words are
defined (one for each slot), analogous to ``getter'' methods in other
languages.  Following the template for naming reader words, this example
defines \factor|red>>|, \factor|green>>|, and \factor|blue>>|.  Each word has
the stack effect
%
\factor|( object -- value )|,
%
and extracts the value corresponding to the eponymous slot.  Similarly, the
\term{writer} words \factor|red<<|, \factor|green<<|, and \factor|blue<<| each
have the stack effect
%
\factor|( value object -- )|,
%
and store values in the corresponding \factor|rgb| slots destructively.  To
leave the modified \factor|rgb| instance on the stack while setting slots, the
\term{setter} words \factor|>>red|, \factor|>>green|, and \factor|>>blue| are
also defined, each with the stack effect
%
\factor|( object value -- object' )|.
%
These words are defined in terms of writers.  For instance, \factor|>>red| is
the same as \factor|over red<<|, since \factor|over| copies a reference to the
tuple (i.e., it doesn't make a ``deep'' copy).

To construct an instance of a tuple, we can use either \factor|new| or
\factor|boa|.  \factor|new| will not initialize any of the slots to a
particular input value---all slots will default to Factor's canonical false
value, \factor|f|.  \factor|new| is used in \vref{lst:colors} to define
\factor|<color>| (by convention, the constructor for \factor|foo| is named
\factor|<foo>|).  First, we push the class \factor|color| onto the stack (this
word is also automatically defined by \factor|TUPLE:|), then just call
\factor|new|, leaving a new instance on the stack.  Since this particular tuple
has no slots, using \factor|new| makes sense.  We might also use it to
initialize a class, then use setter words to only assign a particular subset of
slots' values.

However, we often want to initialize a tuple with values for each of its slots.
For this, we have \factor|boa|, which works similarly to \factor|new|.  This is
used in the definition of \factor|<rgb>| in \vref{lst:colors}.  The difference
here is the additional inputs on the stack---one for each slot, in the order
they're declared.  That is, we're constructing the tuple \textbf{b}y
\textbf{o}rder of \textbf{a}rguments, giving us the fun pun ``\factor|boa|
constructor''.  So, \factor|1 2 3 <rgb>| will construct an \factor|rgb|
instance with the \factor|red| slot set to \factor|1|, the \factor|green| slot
set to \factor|2|, and the \factor|blue| slot set to \factor|3|.

\inputlst{email}

Now that we've seen the various words defined for tuples, we can explore more
complex slot specifiers.  Using the array-like syntax from before, slot
specifiers may be marked with certain \term{attributes}---both with the class
declared (like
%
\factor|{ name class attributes... }|)
%
and without the class declared (as in 
%
\factor|{ name attributes... }|).
%
In particular, Factor recognizes two different attributes.  If a slot marked
\factor|read-only|, the writer (and thus setter) for the slot will not be
defined, so the slot cannot be altered.  A slot may also provide an initial
value using the syntax \factor|initial: some-literal|.  This will be the slot's
value when instantiated with \factor|new|. 

For example, \vref{lst:email} shows a tuple definition from Factor's
\factor|smtp| vocabulary that defines an \factor|email| object.  The
\factor|from| address, \factor|subject|, and \factor|body| must be instances of
\factor|string|, while \factor|to|, \factor|cc|, and \factor|bcc| are
\factor|array|s of destination addresses.  The \factor|content-type| slot must
also be a \factor|string|, but if unspecified, it defaults to
\factor|"text/plain"|.  The \factor|encoding| must be a \factor|word| (in
Factor, even words are first-class objects), which by default is \factor|utf8|,
a word from the \factor|io.encodings.utf8| vocabulary for a Unicode format.

\subsubsection{Generics and Methods}

Unlike more common object systems, we do not define individual methods that
``belong'' to particular tuples.  In Factor, you define a method that
specializes on a class for a particular generic word.  That way, when the
generic word is called, it dispatches on the class of the object, invoking the
most specific method for the object.

Generic words are declared with the syntax
%
\factor|GENERIC: word-name ( stack -- effect )|.
%
Words defined this way will then dispatch on the class of the top element of
the stack (necessarily the rightmost input in the stack effect).  To define a
new method with which to control this dispatch, we use the syntax
%
\factor|M: class word-name definition... ;|.

\inputlst{sets}

An accessible example of a generic word is in Factor's \factor|sets|
vocabulary.  \factor|set| is a \term{mixin} class---a union of other classes
whose members may be extended by the user.  We can see the standard definition
in \vref{lst:sets}.  Note that the \factor|USING:| form specifies vocabularies
being used (like Java's \mint{java}|import|), and \factor|IN:| specifies the
vocabulary in which the definitions appear (like Java's \mint{java}|package|).
We can see here that instances of the \factor|sequence|, \factor|hash-set|, and
\factor|bit-set| classes are all instances of \factor|set|, so will respond
\factor|t| to the predicate \factor|set?|.  Similarly, \factor|sequence| is a
mixin class with many more members, including \factor|array|, \factor|vector|,
and \factor|string|.

\inputlst{cardinality}

\vref{lst:cardinality} shows the \factor|cardinality| generic from Factor's
\factor|sets| vocabulary, along with its methods.  This generic word takes a
\factor|set| instance from the top of the stack and pushes the number of
elements it contains.  For instance, if the top element is a \factor|bit-set|,
we extract its \factor|table| slot and invoke another word, \factor|bit-count|,
on that.  But if the top element is \factor|f| (the canonical false/empty
value), we know the cardinality is $0$.  For any \factor|sequence|, we may
offshore the work to a different generic, \factor|length|, defined in the
\factor|sequences| vocabulary.  The final method gives a default behavior for
any other \factor|set| instance, which simply uses \factor|members| to obtain
an equivalent \factor|sequence| of set members, then calls \factor|length|.

By viewing a class as a set of all objects that respond positively to the class
predicate, we may partially order classes with the subset relationship.  Method
dispatch will use this ordering when \factor|cardinality| is called to select
the most specific method for the object being dispatched upon.  For instance,
while no explicit method for \factor|array| is defined, any instance of
\factor|array| is also an instance of \factor|sequence|.  In turn, every
instance of \factor|sequence| is also an instance of \factor|set|.  We have
methods that dispatch on both \factor|set| and \factor|sequence|, but the
latter is more specific, so that is the method invoked.  If we define our own
class, \factor|foo|, and declare it as an instance of \factor|set| but not as
an instance of \factor|sequence|, then the \factor|set| method of
\factor|cardinality| will be invoked.  Sometimes resolving the precedence gets
more complicated, but these edge-cases are beyond the scope of our discussion.

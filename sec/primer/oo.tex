\section{Object Orientation}\label{sec:primer:oo}

You may have noticed that the examples in \cref{sec:primer:colon-defs} did not
use type declarations.  While Factor is dynamically typed for the sake of
simplicity, it does not do away with types altogether.  In fact, Factor is
object-oriented.  However, its object system doesn't rely on classes possessing
particular methods, as is common.  Instead, it uses \term{generic words} with
methods implemented for particular classes.  To start, though, we must see how
classes are defined.

\subsection{Tuples}

The central data type of Factor's object system is called a \term{tuple}, which
is a class composed of named \term{slots}---like instance variables in other
languages.  Tuples are defined with the \texttt{\textbf{TUPLE:}} parsing word
as shown in \vref{lst:tuples}.  A class name is specified first; if it is
followed by the \factor|<| token and a superclass name, the tuple inherits the
slots of the superclass.  If no superclass is specified, the default is the
\factor|tuple| class.  Any number of slot specifiers follow, and the definition
is terminated by the \factor|;| token.

\inputlst{tuples}

Tuple definitions automatically generate several different words, most of which
depend on how slots are specified.  There are various ways to specify slots,
but we use only two basic forms in later code examples.  We can see both in the
first tuple of \vref{lst:regexp}, which defines an object to represent regular
expressions.  The first three slots have the form
%
\factor|{ name read-only }|,
%
which specifies a slot named \Verb|name| that can't be modified once
initialized, akin to a \mint{java}|final| variable in Java.  The next two
specifiers are simpler, being just the names of the slots.  Such slots can be
modified freely.  The following words are automatically defined for the first
tuple:
\begin{itemize}
  \item The \Verb|regexp| \term{class word} acts like a literal representing
  the class.  This gets used for instantiation and method definitions, which
  we'll see later.
%
  \item The \Verb|regexp?| \term{class predicate} is a word with the stack
  effect \Verb|( object -- ? )|.  That is, it returns a boolean (either
  \factor|t| or \factor|f|, conventionally written in stack effects as a single
  question mark) indicating whether the top of the stack is an instance of the
  \Verb|regexp| class.  This is like a class-specific variant of Java's
  \mint{java}|instanceof|.
%
  \item Each slot has an associated \term{reader} word with the stack effect %
  \Verb|( object -- value )|.  These are analogous  to ``getter'' methods in
  other languages.  Each one is named after the slot whose value is extracted,
  so this example defines \Verb|raw>>|, \Verb|parse-tree>>|, \Verb|options>>|,
  \Verb|dfa>>|, and \Verb|next-match>>|.
%
  \item Similarly, any slot that is not marked \Verb|read-only| has a
  corresponding \term{writer} word with the stack effect                      %
  \Verb|( value object -- )|.  These destructively write the value into the
  eponymous slot of the object.  Here, only two are defined, named \Verb|dfa<<|
  and \Verb|next-match<<|.
%
  \item Extra \term{setter} words are defined in terms of writers.  These will
  have the stack effect \Verb|( object value -- object' )|, leaving the
  modified instance on top of the stack.  The first tuple in \vref{lst:regexp}
  defines \Verb|>>dfa| and \Verb|>>next-match|, which are equivalent to
  \factor|over dfa<<| and \factor|over next-match<<|, respectively.  The
  shuffler duplicates \Verb|object| and pushes it to the top of the stack.
  More accurately, it duplicates a reference to \Verb|object|, as Factor's data
  stack is actually a stack of pointers.  That way, changes to the new top of
  the stack with \Verb|dfa<<| or \Verb|next-match<<| will be reflected in the
  original \Verb|object|, which is left over at the end.
%
  \item \term{Changer} words are also created with the stack effect           %
  \Verb|( object quot -- object' )|.  Here, \Verb|change-dfa| and
  \Verb|change-next-match| are defined.  The quotation is called on the slot's
  current value in \Verb|object|.  The result of calling the quotation is then
  stored in the slot.  For instance, incrementing an integer slot named
  \Verb|foo| could be done with \factor|[ 1 + ] change-foo|.
\end{itemize}

\inputlst{regexp}

The second tuple in \vref{lst:regexp} also defines a class word and predicate.
Since it inherits from \Verb|regexp|, \Verb|reverse-regexp| gets the same five
slots.  If we had any other slot specifiers in the definition, it would have
those in addition to the slots of its parent class.  The reader, writer,
setter, and changer methods will work on instances of \Verb|reverse-regexp|,
since inheritance establishes an ``is-a'' relationship from subclass to
superclass---any instance of \Verb|reverse-regexp| is also an instance of
\Verb|regexp|, though the reverse is not necessarily true.  That is,
\Verb|regexp?| will return \factor|t| on instances of \Verb|reverse-regexp|,
but \Verb|reverse-regexp?| will only return \factor|t| on instances of
\Verb|regexp| that are also \Verb|reverse-regexp|s.  By viewing a class as the
set of all objects that respond positively to the class predicate, we may
partially order classes with the subset relationship.  This fact will be
important later.

To construct an instance of a tuple, we can use either \factor|new| or
\factor|boa|.  \factor|new| will not initialize any of the slots to a
particular input value---all slots will default to Factor's canonical false
value, \factor|f|.  For example, \factor|new| is used in \vref{lst:colors} to
define \Verb|<color>| (by convention, the constructor for \Verb|foo| is named
\Verb|<foo>|).  First, we push the class \Verb|color|, then just call
\factor|new|, leaving a new instance on the stack.  Since this particular tuple
has no slots, using \factor|new| makes sense.  We might also use it to
initialize a class, then use setter words to only assign a particular subset of
slots' values (as long as the slots aren't \Verb|read-only|).

\inputlst{colors}

However, we often want to initialize a tuple with values for each of its slots.
For this, we have \factor|boa|, which works similarly to \factor|new|.  This is
used in the definition of \Verb|<rgb>| in \vref{lst:colors}.  The difference
here is the additional inputs on the stack---one for each slot, in the order
they're declared.  That is, we're constructing the tuple \textbf{b}y
\textbf{o}rder of \textbf{a}rguments, giving us the fun pun ``\factor|boa|
constructor''.  So, \Verb|1 2 3 <rgb>| will construct an \Verb|rgb|
instance with the \Verb|red| slot set to \Verb|1|, the \Verb|green| slot
set to \Verb|2|, and the \Verb|blue| slot set to \Verb|3|.

\subsection{Generics and Methods}

Unlike more common object systems, we do not define individual methods that
``belong'' to particular tuples.  In Factor, for a given generic word you
define a method that specializes on a class.  When the generic word is called
on an object, it selects the method most specific to the object's class.  This
is determined by the aforementioned partial ordering of classes by their
inheritance relationships.

Generic words are declared with the syntax
%
\begin{center} \factor|GENERIC: word-name ( stack -- effect )| \end{center}
%
Words defined this way will then dispatch on the class of the top element of
the stack (necessarily the rightmost input in the stack effect).  To define a
new method with which to control this dispatch, we use the syntax
%
\begin{center} \factor|M: class word-name definition... ;| \end{center}

Factor's \Verb|sets| vocabulary gives us an accessible example of a generic
word.  \Verb|set| is a \term{mixin} class, defined by the \factor|MIXIN:|
parsing word.  That is, the \Verb|set| class is a union of other classes, and
users may extend the members of this union with the \factor|INSTANCE:| word.
We can this in \vref{lst:sets}, which shows the standard members of the
\Verb|set| mixin.  Note that the \factor|USING:| form specifies vocabularies
being used (like Java's \mint{java}|import|) and \texttt{\textbf{IN:}}
specifies the vocabulary in which the definitions appear (like Java's
\mint{java}|package|).  We can see here that instances of the \Verb|sequence|,
\Verb|hash-set|, and \Verb|bit-set| classes are all instances of \Verb|set|, so
will respond \factor|t| to the predicate \Verb|set?|.  Similarly,
\Verb|sequence| is a mixin class with many more members, including
\Verb|array|, \Verb|vector|, and \Verb|string|.

\inputlst{sets}

\Vref{lst:cardinality} shows the \Verb|cardinality| generic from Factor's
\Verb|sets| vocabulary, along with its methods.  This generic word takes a
\Verb|set| instance from the top of the stack and pushes the number of elements
it contains.  For instance, if the top element is a \Verb|bit-set|, we extract
its \Verb|table| slot and invoke another word, \Verb|bit-count|, on that.  But
if the top element is \factor|f| (the canonical false/empty value), we know the
cardinality is $0$.  For any \Verb|sequence|, we may offshore the work to a
different generic, \factor|length|, defined in the \Verb|sequences| vocabulary.
The final method gives a default behavior for any other \Verb|set| instance,
which simply uses \Verb|members| to obtain an equivalent \Verb|sequence| of set
members, then calls \factor|length|.

\inputlst{cardinality}

We can see how the class ordering is used when \Verb|cardinality| selects the
proper method for the object being dispatched upon.  For instance, while no
explicit method for \Verb|array| is defined, any instance of \Verb|array| is
also an instance of \Verb|sequence|.  In turn, every instance of
\Verb|sequence| is also an instance of \Verb|set|.  We have methods that
dispatch on both \Verb|set| and \Verb|sequence|, but the latter is more
specific, so that is the method invoked on an \Verb|array|.  If we define our
own class, \Verb|foo|, and declare it as an instance of \Verb|set| but not as
an instance of \Verb|sequence|, then the \Verb|set| method of
\Verb|cardinality| will be invoked.  Sometimes resolving the precedence gets
more complicated, but these edge-cases are beyond the scope of our discussion.

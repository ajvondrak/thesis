\section{Redundancy Elimination}\label{sec:vn:avail}

Now that we've identified congruences across the entire \gls{CFG}, we must
eliminate any redundancies found.  Since value numbering is now offline, this
entails another pass.  However, replacing instructions is more subtle with
global value numbers than it is with local ones.  Because values come from all
over the \gls{CFG}, we must consider if a definition is \term{available} at the
point where we want to use it.  

\inputfig{not-avail}
\inputfig{is-avail}

\Vref{fig:not-avail,fig:is-avail} show the difference.  In the former, we can
see the \gls{CFG} before value numbering for the code
%
\factor|[ 10 ] [ 20 ] if 10 20 30|.
%
The two extra integers being pushed at the end are there to avoid branch
splitting (see \cref{sec:compiler:cfg}).  In block $4$, there's a
%
\Verb|##load-integer 27 10|,
%
which loads the value \Verb|10|.  In globally numbering values, we associate
the
%
\Verb|##load-integer 22 10|
%
in block $2$ with the value \Verb|10| first, making it the canonical
representative.  However, we can't replace the instruction in block $4$ with
%
\Verb|##copy 27 22|,
%
because control flow doesn't necessarily go through block $2$, so the virtual
register \Verb|22| might not even be defined.  However, in
\cref{fig:is-avail}, we see the \gls{CFG} for the code
%
\factor|10 swap [ 10 ] [ 20 ] if 10 20 30|.
%
In this case, the first definition of the value \Verb|10| comes from block
$1$, which dominates block $4$.  So, the definition is available, and we can
replace the \Verb|##load-integer| in block $4$ with a \Verb|##copy|.

There are several ways to decide if we can use a definition at a certain point.
For instance, we could use dominator information, so that if a definition in a
basic block $B$ can be used by any basic block dominated by $B$
\autocite{Simpson}.  However, here we'll use a data flow analysis called
\term{available expression analysis}, since it was readily implemented.
Mercifully, Factor has a vocabulary that automatically defines data flow
analyses with little more than a single line of code.

\inputlst{avail}

\Vref{lst:avail} shows the vocabulary that defines the available expression
analysis.  It is a forward analysis based on the flow equations below:
\begin{align*}
  \text{\Verb|avail-in|}_i &=
    \begin{cases}
      \varnothing
        & \text{if $i=0$} \\
      \bigcap_{j\in\mathrm{pred}(i)}\text{\Verb|avail-out|}_j
        & \text{if $i>0$}
    \end{cases} \\
  \text{\Verb|avail-out|}_i &= \text{\Verb|avail-in|}_i
                                 \cup 
                                 \text{\Verb|defined|}_i
\end{align*}
%
\noindent Here, the subscripts indicate the basic block number.
$\text{\Verb|defined|}_i$ denotes the result of the \Verb|defined| word
from \cref{lst:avail}.  This returns the set of virtual registers defined in a
basic block.  Since we use virtual registers as value numbers, this is the same
as giving us all the value numbers produced by a basic block.  ``Killed''
definitions are impossible by the \gls{SSA} property, so we needn't track
redefinitions of any virtual register.  Using set intersection as the
confluence operator means that the \Verb|avail-in| set will contain those
values which are available on all paths from the start of the \gls{CFG} to that
block.

\begin{sloppypar}
Using Factor's \Verb|compiler.cfg.dataflow-analysis| vocabulary, the
implementation takes all of two lines of code.  The
%
\factor|FORWARD-ANALYSIS: avail|
%
line automatically defines several objects, variables, words, and methods that
don't warrant full detail here.  One we're immediately concerned with is the
\Verb|transfer-set| generic, which dispatches upon the particular type of
analysis being performed and is invoked on the proper in-set and basic block.
There is no default implementation, as it is the chief difference between
analyses.  So, the next line uses \Verb|defined| and \factor|assoc-union| to
calculate the result of the data flow equation.  Other pieces we'll see used
are the top-level \Verb|compute-avail-sets| word that actually performs the
analysis, the \Verb|avail-ins| hash table that maps basic blocks to their
in-sets, and the \Verb|avail-in| word that is shorthand for looking up a
basic block's in-set.
\end{sloppypar}

We want to use the results of this analysis in the \Verb|rewrite| methods so
that they won't overstep their boundaries, and only make meaningful rewrites.
However, we also want to use \Verb|rewrite| in the
\Verb|determine-value-numbers| pass, where we don't care about availability.
In fact, we want to ignore availability altogether, so that we can discover as
many congruences as possible.  In order to separate these concerns, we need to
have two modes for checking availability.  \cref{lst:avail} defines the
\Verb|available?| word to do just this.  It will only check the actual
availability if \Verb|final-iteration?| is true, otherwise defaulting to
\factor|t|.  Therefore, during the value numbering phase, everything is
considered available.  We further define the utilities \Verb|available-uses?|
and \Verb|with-available-uses?|.  The former checks if all an instruction's
uses are available, and the latter does this only if another quotation first
returns true.  That way, we can guard instruction predicates with a test for
available uses, like
%
\factor|[ ##add-imm? ] with-available-uses?|.

\inputlst{fold-branch}

Finding all the instances where \Verb|rewrite| needed to be altered was
subtle.  Since the old \Verb|value-numbering| was an online optimization, it
didn't need to worry about modifying an instruction in memory.  But by doing
the fixed-point iteration, we cannot permit \Verb|rewrite| to destructively
modify any object instance until the final iteration.  An obvious instance was
in \Verb|compiler.cfg.value-numbering.comparisons| with the word
\Verb|fold-branch|, responsible for modifying the \gls{CFG} to remove an
untaken branch.  We definitely would not want the branch removed while doing
the fixed-point iteration, because the transformation is not necessarily sound.
So, we can protect it with a check for \Verb|final-iteration?|.

\inputlst{self-inverse}

\begin{sloppypar}
More typical were words like \Verb|self-inverse| from
\Verb|compiler.cfg.value-numbering.math| (refer to \cref{lst:self-inverse}).
The idea is to change
%
\begin{center}
  \begin{minipage}{0.2\linewidth}
    \begin{factorcode*}{gobble=6,frame=none}
      ##neg 1 2
      ##neg 3 1
    \end{factorcode*}
  \end{minipage}
\end{center}
%
\noindent into
%
\begin{center}
  \begin{minipage}{0.2\linewidth}
    \begin{factorcode*}{gobble=6,frame=none}
      ##neg 1 2
      ##copy 3 2 any-rep
    \end{factorcode*}
  \end{minipage}
\end{center}
%
\noindent since \Verb|##neg| undoes itself.  But \Verb|rewrite| only has
knowledge of one instruction at a time, so it looks up the redundant
\Verb|##neg|'s source register in the \Verb|vns>insns| table to see if
it's computed by another \Verb|##neg| instruction.  For straight-line code this
is alright, but the source of the originating \Verb|##neg| (in the example, the
virtual register \Verb|2|) isn't necessarily available.  So, we have to use
\Verb|with-available-uses?| to make sure the virtual registers used by the
result of a \Verb|vreg>insn| can themselves be used before we rewrite
anything.
\end{sloppypar}

An even subtler issue that led to infinite loops occured in simplifcations like
the arithmetic distribution in \Verb|compiler.cfg.value-numbering.math|.  The
problem is that the \Verb|rewrite| method would generate instructions that
assigned to entirely brand new registers.  These, of course, would invariably
get value numbered, triggering a change in the \Verb|vregs>vns| table.  A new
iteration would begin, and (since it gets called on the same instructions as
the previous iteration) \Verb|rewrite| would generate new virtual registers
all over again.  Therefore, the \Verb|vregs>vns| table would never stop
changing.  As a stop-gap, distribution had to be disabled altogether until the
final iteration.

\inputlst{gcse}

Armed with the correct rewrite rules, availability information, and global
value numbers, we can perform \gls{GCSE}.  The logic in the \Verb|gcse|
generic in \cref{lst:gcse} is similar to \Verb|process-instruction| from
%XXX
\cref{lst:process-instruction} and \Verb|value-number| from
\cref{lst:gvn-value-number}.  Unlike \Verb|value-number|, we do return an
instruction (or sequence thereof) representing the replacement.  Thus, the
\Verb|array| method uses \factor|map| instead of \factor|each|, to hold onto
the resulting sequence when recursing into several instructions.

\Verb|defs-available| is similar to \Verb|record-defs|, except that value
numbers have already stabilized, so we don't call \Verb|set-vn|.  Instead, we
use the \Verb|make-available| word, which was the last one defined in
\cref{lst:avail}.  As we process the instructions of a block in order, we want
to make sure that local rewrites can still be performed.  We have to ensure
that after processing an instruction, any register it defines is available to
future instructions in the same block.  Thus, we add the virtual register to
that block's \Verb|avail-in| (which acts like a set, even though it's
implemented by a hash table by Factor's data flow analysis framework).
\Verb|alien-call-insn|s, \Verb|##callback-inputs| instructions, and
instances of \Verb|##copy| don't get rewritten any further, so we simply note
that their definitions are available and move on.

The \Verb|?eliminate| word transforms an instruction into a \Verb|##copy|
of the canonical value number that computes it.  If the value number isn't
available, we don't do anything but post-process with \Verb|defs-available|.
If it is, a \Verb|##copy| is produced and its destination is made available.
Thus, \Verb|eliminate-redundancy| works like \Verb|check-redundancy| from
\cref{lst:gvn-value-number}.  We look up the expression computed by the
instruction in the \Verb|exprs>vns| table.  If it's there, we call
\Verb|?eliminate|, but otherwise we leave the instruction alone and make its
definitions available.

If the inputs to a \Verb|##phi| are all congruent, we'll call
\Verb|?eliminate| to transform it into a \Verb|##copy| of its first input
(without loss of generality).  Otherwise, we check for equivalent
\Verb|##phi|s with \Verb|eliminate-redundancy|.

Finally, the \Verb|insn| method will default to calling
\Verb|eliminate-redundancy| on instructions that define only one value, much
like how \Verb|value-number| worked.

\inputlst{new-value-numbering}

The main loop works similarly to \Verb|determine-value-numbers|.
\Verb|final-iteration?| is turned on (set to \factor|t|), and we make sure to
compute the \Verb|avail-in| sets needed to make \Verb|available?| work.
Then, using \Verb|simple-optimization|, we iterate over each basic block.
For each instruction, we first use \Verb|simplify| (refer to
\cref{lst:gvn-simplify}), then call \Verb|gcse| on the rewritten instruction.
Thus, \Verb|rewrite| does the work of propagating available subexpressions
and simplifying instructions, then \Verb|gcse| cleans up redundant
instructions by converting them into \Verb|##copy| instructions if possible.
The new \Verb|value-numbering| word can be seen in
\cref{lst:new-value-numbering}.

\inputfig{gcse-before}
\inputfig{gcse-after}

\begin{sloppypar}
Consider for the last time the example
%
\factor|0 100 [ 1 fixnum+fast ] times|.
%
Before, we had the \gls{CFG} in \cref{fig:gcse-before}.  Making a final pass
with \Verb|eliminate-common-subexpressions| gives us the \gls{CFG} in
\cref{fig:gcse-after}.  Compared to the \gls{CFG} after the old
\Verb|value-numbering| word was called (see \cref{fig:value-numbering}), we
have identified several more redundancies:
\begin{itemize}
  \item The second \Verb|##phi| in block $2$ has been turned into a
  \Verb|##copy| of the first.
%
  \item The \Verb|##compare-integer| of block $2$ has been simplified to
  a \Verb|##compare-integer-imm|, since its operand \Verb|26| is both
  available and known to correspond to the integer value \Verb|100|.
%
  \item Instead of a \Verb|##compare-integer-branch|, we similarly have a
  \Verb|##compare-integer-imm-branch| at the end of block $2$.
%
  \item Because the \Verb|##phi|s have been recognized as copies (i.e., the
  induction variables are congruent), the second \Verb|##add-imm| resulting
  from the old version of \Verb|value-numbering| is turned into a
  \Verb|##copy| of the first.
\end{itemize}
\end{sloppypar}

\inputfig{gcse-copy-prop}
\inputfig{gcse-dce}
\inputfig{gcse-finalize}

Afterwards, the \Verb|copy-propagation| pass cleans up all of these newly
identified copies, as seen in \cref{fig:gcse-copy-prop}.
\Verb|eliminate-dead-code| now gets rid of more instructions than before,
such as the second \Verb|##load-integer| in block $1$, since it has been
propagated to the \Verb|-imm| instructions in block $2$.  See
\cref{fig:gcse-dce}.  At last, after \Verb|finalize-cfg| in
\cref{fig:gcse-finalize}, we see a loop that uses a single register---down from
the three in \cref{fig:finalize-cfg}.

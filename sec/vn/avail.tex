\subsection{Redundancy Elimination}\label{sec:vn:avail}

Now that we've identified congruences across the entire \gls{CFG}, we must
eliminate any redundancies found.  Since value numbering is now offline, this
entails another pass.  However, replacing instructions is more subtle with
global value numbers than it is with local ones.  Because values come from all
over the \gls{CFG}, we must consider if a definition is \term{available} at the
point where we want to use it.  

\inputfig{not-avail}
\inputfig{is-avail}

\Vref{fig:not-avail,fig:is-avail} show the difference.  In the former, we can
see the \gls{CFG} before value numbering for the code
%
\factor|[ 10 ] [ 20 ] if 10 20 30|.
%
The two extra integers being pushed at the end are there to avoid branch
splitting (see \vref{sec:compiler:cfg}).  In block $4$, there's a
%
\Verb|##load-integer 27 10|,
%
which loads the value \factor|10|.  In globally numbering values, we associate
the
%
\Verb|##load-integer 22 10|
%
in block $2$ with the value \factor|10| first, making it the canonical
representative.  However, we can't replace the instruction in block $4$ with
%
\Verb|##copy 27 22|,
%
because control flow doesn't necessarily go through block $2$, so the virtual
register \factor|22| might not even be defined.  However, in
\vref{fig:is-avail}, we see the \gls{CFG} for the code
%
\factor|10 swap [ 10 ] [ 20 ] if 10 20 30|.
%
In this case, the first definition of the value \factor|10| comes from block
$1$, which dominates block $4$.  So, the definition is available, and we can
replace the \Verb|##load-integer| in block $4$ with a \Verb|##copy|.

There are several ways to decide if we can use a definition at a certain point.
For instance, we could use dominator information, so that if a definition in a
basic block $B$ can be used by any basic block dominated by $B$\todo{cite
Simpson}.  However, here we'll use a data flow analysis called \term{available
expression analysis}, since it was readily implemented.  Mercifully, Factor has
a vocabulary that automatically defines data flow analyses with little more
than a single line of code.

\inputlst{avail}

\Vref{lst:avail} shows the vocabulary that defines the available expression
analysis.  It is a forward analysis\todo{cite?}~ based on the flow equations
below:
\begin{align*}
  \text{\factor|avail-in|}_i &=
    \begin{cases}
      \varnothing
        & \text{if $i=0$} \\
      \bigcap_{j\in\mathrm{pred}(i)}\text{\factor|avail-out|}_j
        & \text{if $i>0$}
    \end{cases} \\
  \text{\factor|avail-out|}_i &= \text{\factor|avail-in|}_i
                                 \cup 
                                 \text{\factor|defined|}_i
\end{align*}
%
\noindent Here, the subscripts indicate the basic block number.
$\text{\factor|defined|}_i$ denotes the result of the \factor|defined| word
from \vref{lst:avail}.  This returns the set of virtual registers defined in a
basic block.

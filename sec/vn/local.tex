\subsection{Local Value Numbering}\label{sec:vn:local}

Tracing the exact origins of value numbering is difficult.  It's thought to
have originally been invented in the 1960s by Balke\todo{cite Simpson}.  The
earliest tangible reference to a value numbering (at least, the earliest point
where discussions in the literature seem to start) appears in an oft-cited but
unpublished work of Cocke\todo{cite-like}.  The technique is relatively simple,
but not as powerful as other methods for reasons described hereafter.

The algorithm considers a single basic block.  For each instruction (from top
to bottom) in the block, we essentially let the value number of the assignment
target be a hash of the operator and the value numbers of the operands.  That
is, we hash the \term{expression} being computed by an instruction.  Thus,
assuming a proper hash function, two expressions are \term{congruent} (denoted
$\texttt{x} \cong \texttt{y}$) if
%
\begin{itemize}
%
  \item they have the same operators and
%
  \item their operands are congruent.
%
\end{itemize}
%
\noindent This is our approximation of runtime equivalence.  The first property
is fulfilled by basing the hash, in part, on the operator.  The second property
holds because the hash is based on the value numbers of the statement's
operands---not just the operands as they appear in code (i.e., \term{lexical}
equivalence).  Any information about congruence is propagated through the value
numbers.  We'll have discovered any such equivalences among the operands before
computing the value number of the assignment target because every value in a
basic block is either defined before it's used, or else defined at some point
in a predecessor of the block, which we don't care about when only considering
one basic block.

This is the first shortcoming of the algorithm.  It is \term{local}, focusing
on only one basic block at a time.  Any definitions outside the boundaries of
the basic block won't be reused, even if they reach the block.  This severely
limits the scope of the redundancies we can discover.  We could improve upon
this by considering the algorithm across an entire loop-free \gls{CFG} in any
\term{topological order}.  In such an ordering, a basic block $B$ comes before
any other block $B'$ to which it has an edge.  Thus, any ``outside'' variables
that instructions in $B'$ rely on must have come from $B$ or earlier, which
will have already been computed in a traversal of such an ordering.  However,
\glsplural{CFG} usually contain cycles or loops (at least interesting ones do),
which make such an ordering impossible.  We may still pick a topological order
that ignores back-edges, but we may encounter operands whose values flow along
those back-edges.  We'll later address the issue of encountering instructions
whose operands haven't been processed yet.

In Factor, expressions are basically instructions (the \factor|insn| objects
discussed in \cref{sec:compiler:cfg}) that have had their destination registers
stripped.  Instructions can be converted to expressions with the \factor|>expr|
word defined in the \factor|compiler.cfg.value-numbering.expressions|
vocabulary.  For instance, an \factor|##add| instruction with the destination
register \factor|1| and source registers \factor|2| and \factor|3| will be
converted into an array of three elements:
%
\begin{itemize}
%
  \item The \factor|##add| class word, indicating the expression is derived
        from an \factor|##add| instruction.
%
  \item The value number of the virtual register \factor|2|.
%
  \item The value number of the virtual register \factor|3|.
%
\end{itemize}
%
\noindent Some instructions are not \term{referentially transparent}, meaning
they can't be replaced with the value they compute without changing the
program's behavior.  For example, \factor|##call| and \factor|##branch| cannot
reasonably be converted into expressions.  In these cases, \factor|>expr|
merely returns a unique value.

\inputlst{value-numbering-graph}

The hashing of expressions takes place in the so-called \term{expression graph}
implemented in the vocabulary shown in \vref{lst:value-numbering-graph}.  This
consists of three global hash tables that relate virtual registers, value
numbers, instructions, and expressions.  Since virtual registers are just
integers, we actually use them as value numbers, too.  \factor|vregs>vns| maps
virtual registers to their value numbers.  If a virtual register  is mapped to
itself in this table, its definition is the canonical instruction that we use
to compute the value.  This instruction is stored in the \factor|vns>insns|
table.  Finally, the most important mapping is \factor|exprs>vns|.  True to its
name, it uses expressions as keys, which of course are implicitly hashed.
Thus, we can use this table to determine equivalence of expressions.

Other definitions in \vref{lst:value-numbering-graph} manipulate expressions
and the graph.  The global variable \factor|input-expr-counter| is used in the
generation of unique expressions discussed earlier.  \factor|init-value-graph|
initializes this and all the tables.  \factor|set-vn| establishes a mapping
from a virtual register to a value number in \factor|vregs>vns|.
\factor|vn>insn| gives terse access to the \factor|vns>insns| table.
\factor|vreg>insn| uses \factor|vregs>vns| and \factor|vns>insns| to get the
canonical instruction that defines a given virtual register.  Finally,
\factor|vreg>vn| looks up the value of a key in the \factor|vregs>vns| table.
Importantly, if the key is not yet present in the table, it is automatically
mapped to itself---it's assumed that the virtual register does not correspond
to a redundant instruction.

This is the second shortcoming of the algorithm.  It must make a
\term{pessimistic} assumption about congruences.  That is, it starts by
assuming that every expression has a unique value number, then tries to prove
that there are some values which are actually congruent.

%This is the final version depicted in Algorithm~\vref{alg:hash-vn}.  Clearly,
%it will fail to discover congruences for values that flow along back-edges,
%since we simply ignore back-edges and started with the assumption that values
%are distinct.  For example, refer to Figure~\vref{fig:hash-vn-ex}, which uses
%the SSA form of the CFG in Figure~\vref{fig:cfg-construction}.  We can see that
%the corresponding \lstinline|i|s and \lstinline|j|s are equivalent, but
%Algorithm~\ref{alg:hash-vn} must ignore the back-edge and consider the
%statements in the order shown.  Thus, it only discovers $\code{i$_0$} \cong
%\code{j$_0$}$, but considers $\code{i$_1$} \not\cong \code{j$_1$} \not\cong
%\code{i$_2$} \not\cong \code{j$_2$}$.  This is because
%$\attrib{\code{i$_2$}}{valnum} \ne \attrib{\code{j$_2$}}{valnum}$ (they were
%initialized uniquely) at the time when $\attrib{\code{i$_1$}}{valnum}$ and
%$\attrib{\code{j$_1$}}{valnum}$ were being computed, which gave
%\lstinline|i$_1$| and \lstinline|j$_1$| differing value numbers, which then got
%propagated to the computation of $\attrib{\code{i$_2$}}{valnum}$ and
%$\attrib{\code{j$_2$}}{valnum}$.  Notice that congruence only guarantees
%properties when two values have the same value number.  Nothing can be said of
%incongruent values---not even that they're nonequal (in this case, we have
%incongruent values that actually \emph{are} equal).  This illustrates the
%conservativeness of the solution.

%Factor currently
%  Cocke & Schwartz (effectively what Factor uses)
%
%compiler.cfg.value-numbering
%  process-instruction
%  value-numbering
%  value-numbering-step
%
%compiler.cfg.value-numbering.graph
%  vregs>vns
%  exprs>vns
%  vns>insns
%
%compiler.cfg.value-numbering.expressions
%  value-numbering-step
%    exprs>vns
%    >expr
%    <integer-expr>
%    <reference-expr>
%
%-------------------------------------------------------------------------------
%
%compiler.cfg.value-numbering.rewrite
%  rewrite
%    ! Utilities
%    insn>integer
%    vreg>integer
%    insn>literal
%    vreg>literal
%
%compiler.cfg.value-numbering.alien
%  rewrite
%
%compiler.cfg.value-numbering.comparisons
%  rewrite
%    rewrite-boolean-comparison
%    fold-branch
%    >test
%    >compare-imm
%    simplify-test
%
%compiler.cfg.value-numbering.folding
%  rewrite
%    binary-constant-fold
%    unary-constant-fold
%
%compiler.cfg.value-numbering.math
%  rewrite
%    self-inverse
%    identity
%    compiler.cfg.value-numbering.folding
%    reassociate
%    distribute
%    insn>imm-insn
%
%compiler.cfg.value-numbering.misc
%  rewrite
%
%compiler.cfg.value-numbering.simd
%  rewrite
%
%compiler.cfg.value-numbering.slots
%  rewrite

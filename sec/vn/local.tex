\subsection{Local Value Numbering}\label{sec:vn:local}

Tracing the exact origins of value numbering is difficult.  It's thought to
have originally been invented in the 1960s by Balke\todo{cite Simpson}.  The
earliest tangible reference to a value numbering (at least, the earliest point
where discussions in the literature seem to start) appears in an oft-cited but
unpublished work of Cocke\todo{cite-like}.  The technique is relatively simple,
but not as powerful as other methods for reasons described hereafter.

The algorithm considers a single basic block.  For each instruction (from top
to bottom) in the block, we essentially let the value number of the assignment
target be a hash of the operator and the value numbers of the operands.  That
is, we hash the \term{expression} being computed by an instruction.  Thus,
assuming a proper hash function, two expressions are \term{congruent} if
%
\begin{itemize}
%
  \item they have the same operators and
%
  \item their operands are congruent.
%
\end{itemize}
%
\noindent This is our approximation of runtime equivalence.  The first property
is fulfilled by basing the hash, in part, on the operator.  The second property
holds because the hash is based on the value numbers of the statement's
operands---not just the operands as they appear in code (i.e., \term{lexical}
equivalence).  Any information about congruence is propagated through the value
numbers.  We'll have discovered any such equivalences among the operands before
computing the value number of the assignment target because every value in a
basic block is either defined before it's used, or else defined at some point
in a predecessor of the block, which we don't care about when only considering
one basic block.

This is the first shortcoming of the algorithm.  It is \term{local}, focusing
on only one basic block at a time.  Any definitions outside the boundaries of
the basic block won't be reused, even if they reach the block.  This severely
limits the scope of the redundancies we can discover.  We could improve upon
this by considering the algorithm across an entire loop-free \gls{CFG} in any
\term{topological order}.  In such an ordering, a basic block $B$ comes before
any other block $B'$ to which it has an edge.  Thus, any ``outside'' variables
that instructions in $B'$ rely on must have come from $B$ or earlier, which
will have already been computed in a traversal of such an ordering.  However,
\glsplural{CFG} usually contain cycles or loops (at least interesting ones do),
which make such an ordering impossible.  We could still pick a topological
order that ignores back-edges, but we may encounter operands whose values flow
along those back-edges, so haven't been processed yet.  We'll address this
issue later.

In Factor, expressions are basically instructions (the \Verb|insn| objects
discussed in \cref{sec:compiler:cfg}) that have had their destination registers
stripped.  Instructions can be converted to expressions with the \Verb|>expr|
word defined in the \Verb|compiler.cfg.value-numbering.expressions|
vocabulary.  For instance, an \Verb|##add| instruction with the destination
register \Verb|1| and source registers \Verb|2| and \Verb|3| will be
converted into an array of three elements:
%
\begin{itemize}
%
  \item The \Verb|##add| class word, indicating the expression is derived
        from an \Verb|##add| instruction.
%
  \item The value number of the virtual register \Verb|2|.
%
  \item The value number of the virtual register \Verb|3|.
%
\end{itemize}
%
\noindent Some instructions are not \term{referentially transparent}, meaning
they can't be replaced with the value they compute without changing the
program's behavior.  For example, \Verb|##call| and \Verb|##branch| cannot
reasonably be converted into expressions.  In these cases, \Verb|>expr|
merely returns a unique value.

\inputlst{value-numbering-graph}

The hashing of expressions takes place in the so-called \term{expression graph}
implemented in the vocabulary shown in \vref{lst:value-numbering-graph}.  This
consists of three global hash tables that relate virtual registers, value
numbers, instructions, and expressions.  Since virtual registers are just
integers, we actually use them as value numbers, too.  \Verb|vregs>vns| maps
virtual registers to their value numbers.  If a virtual register  is mapped to
itself in this table, its definition is the canonical instruction that we use
to compute the value.  This instruction is stored in the \Verb|vns>insns| table.  Finally, the most important mapping is \Verb|exprs>vns|.  True to its
name, it uses expressions as keys, which of course are implicitly hashed.
Thus, we can use this table to determine equivalence of expressions.

Other definitions in \vref{lst:value-numbering-graph} manipulate expressions
and the graph.  The global variable \Verb|input-expr-counter| is used in the
generation of unique expressions discussed earlier.  \Verb|init-value-graph|
initializes this and all the tables.  \Verb|set-vn| establishes a mapping
from a virtual register to a value number in \Verb|vregs>vns|.
\Verb|vn>insn| gives terse access to the \Verb|vns>insns| table.
\Verb|vreg>insn| uses \Verb|vregs>vns| and \Verb|vns>insns| to get the
canonical instruction that defines a given virtual register.  Finally,
\Verb|vreg>vn| looks up the value of a key in the \Verb|vregs>vns| table.
Importantly, if the key is not yet present in the table, it is automatically
mapped to itself---it's assumed that the virtual register does not correspond
to a redundant instruction.

This is the second shortcoming of the algorithm.  It must make a
\term{pessimistic} assumption about congruences.  That is, it starts by
assuming that every expression has a unique value number, then tries to prove
that there are some values which are actually congruent.  This fails to
discover congruences for values that flow along back-edges, whether we consider
a single basic block or an entire topological ordering.

One the other hand, an advantage of this local value numbering algorithm is its
simplicity.  It makes a single pass over all the instructions, identifying and
replacing redundancies \term{online} (i.e., rewriting as it goes).  It's
straightforward to write, and even to extend.  In particular, there's nothing
stopping the online replacements from being more complex than \Verb|##copy|
instructions.  At every step, the currently known value numbers will be sound,
and we can use this information for copy/constant propagation, constant
folding, and common subexpression elimination.

\inputlst{value-numbering-step}

To see how Factor accomplishes these extensions, we'll take a look at the
\Verb|compiler.cfg.value-numbering| vocabulary.
\Vref{lst:value-numbering-step} shows the main words that start the
optimization pass.  The \Verb|value-numbering-step| word is called on the
sequence of instructions that comprise each basic block.  It starts the
expression graph from a blank slate with \Verb|init-value-graph|, then
\Verb|map|s the word \Verb|process-instruction| on each of them.  This is a
generic word that we'll study momentarily; it returns either a single
\Verb|insn| object or a sequence of them (in the case that an instruction is
replaced by several others).  Then, the work of \Verb|value-numbering| is to
just call \Verb|value-numbering-step| on each basic block, which is done with
a combinator called \Verb|simple-optimization|.  The words
\Verb|cfg-changed| and \Verb|predecessors-changed| alter some internal
state of the \gls{CFG} that has been potentially invalidated by some
transformations performed by \Verb|process-instruction|.

\inputlst{process-instruction}

The methods of \Verb|process-instruction| are shown in
\vref{lst:process-instruction}.  The default behavior for dispatching on an
\Verb|insn|\todo{line numbers?} is to invoke yet another generic word,
\Verb|rewrite|.  This word will return either a replacement \Verb|insn| (or
sequence thereof) or \factor|f|, indicating that no change has taken place.
Thus, by recursively calling \Verb|process-instruction|, we can do more
specialized processing on this rewritten replacement (e.g., dispatching on
\Verb|insn| again, which applies \Verb|rewrite| once more).  If the
instruction can't be simplified further, we simply return it.  (Note that
%
\factor|[ X ] [ Y ] ?if|
%
is the same as
%
\factor|dup [ nip X ] [ drop Y ] if|.)

For instances of \Verb|foldable-insn| (i.e., \Verb|insn|s that can be
converted to useful expressions with \Verb|>expr|), we similarly invoke
\Verb|rewrite| recursively until no more rewriting occurs.  When that
happens, rather than just return the instruction, we invoke
\Verb|check-redundancy|---though only if the instruction defines exactly $1$
virtual register, which will be stored in a slot named \Verb|dst|.
\Verb|check-redundancy| checks if the expression being computed by the
instruction is already a key of the \Verb|exprs>vns| table.  If it is, the
instruction is redundant, and we call \Verb|redundant-instruction|;
otherwise, we call \Verb|useful-instruction|.  The former uses
\Verb|set-vn| to map the instruction's \Verb|dst| virtual register to the
same value number as the expression that was a key of \Verb|exprs>vns|.
Since value numbers are actually virtual registers, we may also use these two
integers the source and destination registers in a new \Verb|##copy|
instruction, which is then returned.  On the other hand,
\Verb|useful-instruction| saves the instruction's information in the
expression graph by setting the appropriate values in \Verb|vregs>vns|,
\Verb|exprs>vns|, and \Verb|vns>insns|.  Note that in definitions using the
syntax
%
\factor|:: word-name ( stack -- effect ) ... ;|,
%
the input values in the stack effect are actually named lexical variables, like
in most programming languages\todo{move this to primer?}.  Furthermore, the
first line
%
\factor|insn dst>> :> vn|
%
assigns the input instruction's destination register to the variable
\Verb|vn|, which is used later in the definition of
\Verb|useful-instruction|.

The \Verb|##copy| method of \Verb|process-instruction| cannot do anything
to simplify the instruction, but will set the value number of the destination
register to that of the source.  By calling \Verb|vreg>vn| on the source
register, we make sure to call \Verb|set-vn| between the destination and the
canonical value number of the source.

Finally, the \Verb|array| method is used for the purposes of recursion, in
the case that \Verb|rewrite| returns a sequence of replacement instructions.
The correct action is, of course, to descend into this new sequence of
instructions with \Verb|process-instruction|.

Underlying all of the redundancy elimination is the \Verb|rewrite| generic
word.  It has too many methods to look at the source code in-depth here, but
it's instructive to give an overview of the transformations.  These methods
actually make up the bulk of the \Verb|compiler.cfg.value-numbering| code.
They're spread across various sub-vocabularies.
\Verb|compiler.cfg.value-numbering.rewrite| defines the generic itself, along
with a handful of utilities.  The method for the most general instruction
class, \Verb|insn|, is defined to unconditionally return \factor|f|, meaning
no rewriting is performed by default.  That way, we need only define
\Verb|rewrite| methods for more specific instruction classes to get
specialized behavior.

\Verb|compiler.cfg.value-numbering.alien| contains methods that simplify
nodes related to Factor's \gls{FFI}.  Most involve fusing together the results
of intermediate arithmetic.  The instructions that access raw memory (namely
\Verb|##load-memory|, \Verb|##load-memory-imm|, \Verb|##store-memory|,
and \Verb|##store-memory-imm|) tend to have inputs to perform address
arithmetic.  Each has slots for a \Verb|base| register containing an address
and a literal \Verb|offset| from it.  But if \Verb|base| is defined by an
\Verb|##add-imm| instruction, we can just update the \Verb|offset|,
incrementing it by the literal operand of the \Verb|##add-imm|.  Then,
\Verb|base| will just be changed to the register operand of the
\Verb|##add-imm|.  This removes the memory instruction's need for the
\Verb|##add-imm|, increasing the chances that the latter will become dead
code to be removed later.  Unlike the \Verb|-imm| variants,
\Verb|##load-memory| and \Verb|##store-memory| also take a
\Verb|displacement| register, which works like a non-immediate
\Verb|offset|.  Therefore, \Verb|##add|s can be similarly fused into
\Verb|##load-memory-imm| and \Verb|##store-memory-imm| by transforming them
into \Verb|##load-memory| and \Verb|##store-memory| instructions with the
\Verb|##add|'s operand as the \Verb|displacement|.  A few other similar
transformations are also done, including rewrites for
\Verb|##box-displaced-alien|s and \Verb|##unbox-any-c-ptr|s.

\Verb|compiler.cfg.value-numbering.comparisons| defines methods for the
various branching and comparison instructions (which simply store booleans in
registers, rather than branching upon them).  The major optimizations performed
are as follows:
%
\begin{itemize}
%
\item If possible, instructions are converted to more specific forms.  For
example, non-immediate instructions (e.g., \Verb|##compare|) may be turned
into their \Verb|-imm| counterparts (e.g., \Verb|##compare-imm|)  if one of
their source registers corresponds to a literal value.
\Verb|##compare-integer-imm| is also converted to \Verb|##test| if the
architecture supports it.  This corresponds to a special instruction in x86
that performs a bitwise AND for its side effects on particular flags,
discarding the actual result.  This can be more efficient when using the AND
result as a boolean.
%
\item If both inputs to a comparison or branch are literals, we may
constant-fold the instruction.  In the case of comparisons, this means
converting it into a \Verb|##load-reference| of the proper boolean.  In
branches, this modifies the \gls{CFG} so that the path which isn't taken is
removed completely.
%
\item Like a novice programmer writing
%
\mint{java}|if (some_boolean != false) { ... }|
%
in Java, the compiler may generate redundant boolean comparisons that need
cleaning up.  That is, the intermediate boolean values are eliminated when the
result of a comparison is used by another comparison, collapsing the whole
thing into a single instruction.
%
\end{itemize}

\Verb|compiler.cfg.value-numbering.folding| defines some auxiliary words for
constant-folding arithmetic words.  Mainly, \Verb|unary-constant-fold| and
\Verb|binary-constant-fold| perform the actual operation on the one or two
constant inputs provided.  These words are used in
\Verb|compiler.cfg.value-numbering.math|, which predictably simplifies math
via standard rules.  Arithmetic identities are rewritten---conceptually, $x+0$
becomes just $x$, for instance.  If self-inverting instructions (namely
\Verb|##neg| for numerical negation and \Verb|##not| for boolean negation)
are called on registers that themselves correspond to the same instruction, we
can safely rewrite them into \Verb|##copy| instructions.  Non-immediate
instructions are converted to their \Verb|-imm| forms, if possible, and if
both operands are constant, the expression is folded.  The most interesting
math optimizations use the associative and distributive laws.
\term{Reassociation} conceptually converts $(x \otimes y) \otimes z$ into $x
\otimes (y \otimes z)$ when both $y$ and $z$ are constants and $\otimes$ is
associative.  So, for example,
%
\begin{center}
\Verb|##add-imm 1 X Y|\\
\Verb|##add-imm 2 1 Z|
\end{center}
%
\noindent is converted into just
%
\begin{center}
\Verb|##add-imm 2 X (Y+Z)|
\end{center}
%
\noindent where \Verb|X| is a virtual register, and \Verb|Y| and \Verb|Z|
are constants.  \term{Distribution} converts $(x \oplus y) \otimes z$ into $(x
\otimes z) \oplus (y \otimes z)$, where $y$ and $z$ are constants, $\oplus$
corresponds to addition or subtraction, and $\otimes$ to multiplication or left
bitwise shifts.  Therefore,
%
\begin{center}
\Verb|##add-imm 1 X Y|\\
\Verb|##mul-imm 2 1 Z|
\end{center}
%
\noindent is converted into
%
\begin{center}
  \begin{minipage}{0.2\linewidth}
    \begin{factorcode*}{gobble=6,frame=none}
      ##mul-imm 3 X Y
      ##add-imm 2 3 (Y*Z)
    \end{factorcode*}
  \end{minipage}
\end{center}
\noindent Notice that a new intermediate virtual register, \Verb|3|, had to
be created.  However, if the product of \Verb|Y| and \Verb|Z| can be
computed at compile-time and fits in an immediate operand, then we save cycles
by using \Verb|##mul-imm| on a smaller number.

The last few methods of \Verb|rewrite| provide some obvious simplifications.
\Verb|compiler.cfg.value-numbering.simd| performs some limited
constant-folding for vector operations.
\Verb|compiler.cfg.value-numbering.slots| propagates \Verb|##add-imm|
address calculation to \Verb|##slot|, \Verb|##set-slot|, and
\Verb|##write-barrier| instructions in a manner similar to
\Verb|compiler.cfg.value-numbering.alien|.  Finally,
\Verb|compiler.cfg.value-numbering.misc| provides a single method to rewrite
\Verb|##replace| into \Verb|##replace-imm| if possible.

\todo[inline]{make sure margins don't get overflown by long vocab names}

\inputfig{lvn}

\todo[inline]{make sure side-by-side figures aren't too crammed in final draft}

To finish the discussion of local value numbering and Factor's particular
implementation, we'll examine the example from \vref{fig:value-numbering} in
depth.  For convenience, the before/after snapshot of the \gls{CFG} is
reproduced in \vref{fig:lvn}.

\Verb|value-numbering-step| begins at block $1$, where
\Verb|process-instruction| is \factor|map|ped across the instructions.
%
\Verb|##inc-d 3|
%
does not have a \Verb|rewrite| method, so remains untouched; it is also not a
\Verb|foldable-insn|, so it is simply returned.  While
%
\Verb|##load-integer 21 0|
%
doesn't have a \Verb|rewrite| method, it is a \Verb|foldable-insn|, so
\Verb|process-instruction| calls \Verb|check-redundancy|.  At this point,
the expression graph is empty.  Calling \Verb|>expr| converts this
instruction into an \Verb|integer-expr| object representing \Verb|0|.
\Verb|useful-instruction| leaves the tables as follows:
%
  \begin{factorcode}
    ! vregs>vns
    H{ { 21 21 } }

    ! exprs>vns

    H{ { T{ integer-expr { value 0 } } 21 } }

    ! vns>insns
    H{
        { 21 T{ ##load-integer { dst 21 } { val 0 } { insn# 1 } } }
    }
  \end{factorcode}
%
\noindent The next instruction in block $1$,
%
\Verb|##load-integer 22 100|,
%
behaves similarly, leaving:
%
  \begin{factorcode}
    ! vregs>vns
    H{ { 21 21 } { 22 22 } }

    ! exprs>vns
    H{
        { T{ integer-expr { value 0 } } 21 }
        { T{ integer-expr { value 100 } } 22 }
    }

    ! vns>insns
    H{
        { 21 T{ ##load-integer { dst 21 } { val 0 } { insn# 1 } } }
        {
            22
            T{ ##load-integer { dst 22 } { val 100 } { insn# 2 } }
        }
    }
  \end{factorcode}
%
\noindent The following instruction is
%
\Verb|##load-integer 23 0|.
%
In calling \Verb|check-redundancy|, we discover that the integer expression
for \Verb|0| is already in \Verb|exprs>vns|, so this is turned into a
\Verb|##copy|, and the value number is noted.  The remaining instructions in
block $1$ (aside from \Verb|##branch|) are all instances of \Verb|##copy|.
\Verb|process-instruction| thus only sets their value numbers in the
\Verb|vregs>vns| table, leaving them with the following at the end of block
$1$:
%
  \begin{factorcode}
    ! vregs>vns
    H{
        { 21 21 }
        { 22 22 }
        { 23 21 }
        { 24 22 }
        { 25 21 }
        { 26 22 }
        { 27 21 }
    }

    ! exprs>vns
    H{
        { T{ integer-expr { value 0 } } 21 }
        { T{ integer-expr { value 100 } } 22 }
    }

    ! vns>insns
    H{
        { 21 T{ ##load-integer { dst 21 } { val 0 } { insn# 1 } } }
        {
            22
            T{ ##load-integer { dst 22 } { val 100 } { insn# 2 } }
        }
    }
  \end{factorcode}

Next, block $2$ in \vref{fig:lvn} is processed.  The tables are all reset, so
even though block $1$ happens to dominate\todo{yay, another term} block $2$,
none of its definitions are known to \Verb|value-numbering|.  The
\Verb|##phi|s are ignored, as no important methods dispatch upon them.  In
trying to rewrite the \Verb|##compare-integer|, we call \Verb|vreg>vn| on
the operands.  Since they aren't in the \Verb|vregs>vns| table yet, they are
assumed to be unique values.  This assumption is pessimistic---we'd rather the
values be the same, so we can remove redundancy.  It happens to be correct
here, though, as \Verb|26| corresponds to the integer \Verb|100|, while
\Verb|30| is an induction variable of the loop.  However,
\Verb|##compare-integer| cannot be rewritten into an immediate form, since
our focus is local to the basic block, so we don't know that \Verb|26| has
the value \Verb|100|.  The \Verb|##copy| instructions are processed as
usual, and
%
\Verb|##compare-imm-branch 32 f cc/=|
%
is rewritten into a \Verb|##compare-integer-branch|, as the virtual register
\Verb|32| has the same value (through the copies) as the
\Verb|##compare-integer| result.  This is a case of simplifying the 
%
\mint{java}|if (some_boolean != false) { ... }|
%
pattern, and the definition of the register \Verb|31| becomes dead code after
\Verb|rewrite| finishes with this last instruction.  The expression graph is
populated thus by the end:
%
  \begin{factorcode}
    ! vregs>vns
    H{
        { 32 31 }
        { 33 26 }
        { 34 31 }
        { 26 26 }
        { 30 30 }
        { 31 31 }
    }

    ! exprs>vns
    H{ { { ##compare-integer 30 26 cc< } 31 } }

    ! vns>insns
    H{
        {
            31
            T{ ##compare-integer
                { dst 31 }
                { src1 30 }
                { src2 22 }
                { cc cc< }
                { temp 9 }
                { insn# 2 }
            }
        }
    }
  \end{factorcode}

Once again, the tables are reset and we proceed to block $3$.  The first
instruction,
%
\Verb|##load-integer 35 1|,
%
is entered into the expression graph.  Since \Verb|35| is an operand of
%
\Verb|##add 36 29 35|,
%
\Verb|rewrite| changes this instruction into an \Verb|##add-imm|, as we
know the constant value of the operand.  The next \Verb|##load-integer| gets
turned into a \Verb|##copy|, like in block $1$, and the next \Verb|##add|
is similarly changed to \Verb|##add-imm|.  The copies do little but set more
value numbers.  As \Verb|process-instruction| calls \Verb|vreg>vn| on their
sources, we'll insert entries into \Verb|vregs>vns| for those defined outside
of the block, like \Verb|26|.  This leaves us with the following tables:
%
  \begin{factorcode}
    ! vregs>vns
    H{
        { 35 35 }
        { 36 36 }
        { 37 35 }
        { 38 38 }
        { 39 30 }
        { 40 26 }
        { 41 36 }
        { 26 26 }
        { 42 38 }
        { 29 29 }
        { 30 30 }
    }

    ! exprs>vns
    H{
        { { ##add-imm 30 1 } 38 }
        { { ##add-imm 29 1 } 36 }
        { T{ integer-expr { value 1 } } 35 }
    }

    ! vns>insns
    H{
        { 36 T{ ##add-imm { dst 36 } { src1 29 } { src2 1 } } }
        { 38 T{ ##add-imm { dst 38 } { src1 30 } { src2 1 } } }
        { 35 T{ ##load-integer { dst 35 } { val 1 } { insn# 0 } } }
    }
  \end{factorcode}
%
\noindent The fourth invocation of \Verb|value-numbering-step| does not do
anything interesting, as the \Verb|##replace| cannot be changed into a
\Verb|##replace-imm|.

\inputfig{finalize-lvn}

In summary, we managed to replace redundancies within basic blocks online by
maintaining some simple hash tables.  After copy propagation and dead code
elimination, the \gls{CFG} gets finalized to the one shown in
\vref{fig:finalize-lvn}.  Because the value numbering algorithm was local, the
\Verb|##compare-integer-branch| in block $2$ could not be simplified to a
\Verb|##compare-integer-imm-branch|, and we instead have to waste a register
on the integer \Verb|100|.  But it's important to note that even considering
a topological ordering of the \gls{CFG} wouldn't have worked, as we'd have to
ignore back-edges.  The \Verb|##phi|s that used to be in block $2$ had inputs
that flowed along the back-edge, and our pessimistic assumption would have to
classify these values as distinct.  One is for the counter introduced by
\Verb|times|, and the other is from the top value of the stack being
incremented by \Verb|fixnum+fast|.  In this case, however, these induction
variables are actually equal: both start at \Verb|0| and are incremented by
\Verb|1| on each loop.  In terms of the \gls{CFG} in \vref{fig:finalize-lvn},
the \Verb|EAX| and \Verb|EDX| registers are equivalent.  Yet the
combination of the pessimism and locality of the algorithm keep us from
discovering this.

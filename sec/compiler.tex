\section{The Factor Compiler}\label{sec:compiler}

If we could sort programming languages by the fuzzy notions we tend to have
about how ``high-level'' they are, toward the high end we'd find
dynamically-typed languages like Python, Ruby, and PHP---all of which are
generally more interpreted than compiled\todo{Though there are projects for
this}.  Despite being as high-level as these popular languages, Factor's
implementation is driven by performance.  Factor source is always compiled to
native machine code using either its simple, non-optimizing compiler or (more
typically) the optimizing compiler that performs several sorts of data and
control flow analyses.  In this \lcnamecref{sec:compiler}, we look at the
general architecture of Factor's implementation, after which we place a
particular emphasis on the transformations performed by the optimizing
compiler.

%\section{Organization}\label{sec:compiler:vm}

At the lowest level, Factor is written atop a C++ \gls{VM} that is responsible
for basic runtime services.  This is where the non-optimizing base compiler is
implemented.  It's the base compiler's job to compile the simplest primitives:
operations that push literals onto the data stack, \factor|call|, \factor|if|,
\factor|dip|, words that access tuple slots as laid out in memory, stack
shufflers, math operators, functions to allocate/deallocate call stack frames,
etc.  The aim of the base compiler is to generate native machine code as fast
as possible.  To this end, these primitives correspond to their own stubs of
assembly code.  Different stubs are generated by Factor depending on the
instruction set supported by the particular machine in use.  Thus, the base
compiler need only make a single pass over the source code, emitting these
assembly instructions as it goes.

This compiled code is saved in an \term{image file}, which contains a complete
snapshot of the current state of the Factor instance, similar to many Smalltalk
and Lisp systems\todo{cite?}.  As code is parsed and compiled, the image is
updated, serving as a cache for compiled code.  This modified image can be
saved so that future Factor instances needn't recompile vocabularies that are
already contained in the image.

The \gls{VM} also handles method dispatch and memory management.  Method
dispatch incorporates a \term{polymorphic inline cache} to speed up generic
words.  Each generic word's call site is associated with a state:
\begin{itemize}
  \item In the \term{cold} state, the call site's instruction computes the
        right method for the class being dispatched upon, which is the
        operation we're trying to avoid.  As it does this, a polymorphic inline
        cache stub is generated, thus transitioning it to the next state.
  \item In the \term{inline cache} state, a stub has been generated that caches
        the locations of methods for classes that have already been seen.  This
        way, if a generic word at a particular call site is invoked often upon
        only a small number of classes (as is often in the case in loops, for
        example), we don't need to waste as much time doing method lookup.  By
        default, if more than three different classes are dispatched upon, we
        transition to the next state.
  \item In the \term{megamorphic} state, the call instruction points to a
        larger cache that is allocated for the specific generic word (i.e., it
        is shared by all call sites).  While not as efficient as an inline
        cache, this can still improve the performance of method dispatch.
\end{itemize}

To manage memory, the Factor \gls{VM} uses a generational \gls{GC}, which
carves out sections of space on the heap for objects of different ages.
Garbage in the oldest generation is collected with a mark-sweep-compact
algorithm, while younger generations rely on a copying collector\todo{cite?}.
This way, the \gls{GC} is specialized for large numbers of short-lived objects
that will stay in the younger generations without being promoted to the older
generation.  In the oldest space, even compiled code can be compacted.  This is
to avoid heap fragmentation in applications that must call the compiler at
runtime, such as Factor's interactive development environment.

Values are referenced by tagged pointers, which use the three least significant
bits of the pointer's address to store type information.  This is possible
because Factor aligns objects on an eight-byte boundary, so the three least
significant bits of an address are always $0$.  These bits give us eight unique
tags, but since Factor has more than eight data types, two tags are reserved to
indicate that the type information is stored elsewhere.  One is for \gls{VM}
types without their own tag, and the other is for user-defined tuples, each of
which has its own type.  Sufficiently small integers (e.g., $29$-bit integers
on a $32$-bit machine, since the other $3$ bits are used for the type tag) are
stored directly in the pointer, so they needn't be heap-allocated.  Larger
integers and floating point numbers are boxed, but the optimizing compiler may
unbox them to store floats in registers.

The \gls{VM} is meant to be minimal, as Factor is mostly \term{self-hosting}.
That is, the real workhorses of the language are written in Factor itself,
including the standard libraries, parser, object system, and the optimizing
compiler.  It's possible for the compiler to be written in Factor because of
the \term{bootstrapping} process that creates a new image from scratch.  First,
a minimal \term{boot image} is created from an existing \term{host} Factor
instance.  When the \gls{VM} runs the boot image, it initiates the
bootstrapping process.  Using the host's parser, the base compiler will compile
the core vocabularies necessary to load the optimizing compiler.  Once the
optimizing compiler can itself be compiled, it is used to recompile (and thus
optimize) all of the words defined so far.  With the basic vocabularies
recompiled, any additional vocabularies can be loaded using the optimized
compiler and saved into a new, working image.

Thus, while the Factor \gls{VM} is important, it is a small part of the code
base.  Since the bootstrapping process allows the optimizing compiler
(hereafter just ``the compiler'') to be written in the same high-level language
it's compiling, we can avoid the fiddly low-level details of the C++ backend.
This is more conducive to writing advanced compiler optimizations, which are
often complicated enough without having a concise, dynamically-typed,
garbage-collected language like Factor to help us.

\subsection{High-level Optimizations}\label{sec:compiler:tree}

To manipulate source code abstractly, we must have at least one \gls{IR}---a
data structure representing the instructions.  It's common to convert between
several \glsplural{IR} during compilation, as each form offers different
properties that facilitate particular analyses.  The Factor compiler optimizes
code in passes across two different \glsplural{IR}: first at a high-level using
the \factor|compiler.tree| vocabulary, then at a low-level with the
\factor|compiler.cfg| vocabulary.

\inputlst{tree}

The high-level \gls{IR} arranges code into a vector of \factor|node| objects,
which may themselves have children consisting of vectors of node---a tree
structure that lends to the name \factor|compiler.tree|.  This ordered sequence
of nodes represents control flow in a way that's effectively simple, annotated
stack code.  \Vref{lst:tree} shows the definitions of the tuples that represent
the ``instruction set'' of this stack code.  Each object inherits (directly or
indirectly) from the \factor|node| class, which itself inherits from
\factor|identity-tuple|.  This is a tuple whose \factor|equal?| method is
defined to always return \factor|f| so that no two instances are equivalent
unless they are the same instance.

Notice that most nodes define some sort of \factor|in-d| and \factor|out-d|
slots, which mark each of them with the input and output data stacks.  This
represents the flow of data through the program.  Here, stack values are
denoted simply by integers, giving each value a unique identifier.  An
\factor|#introduce| instance is inserted wherever the next node requires stack
values that have not yet been named.  Thus, while \factor|#introduce| has no
\factor|in-d|, its \factor|out-d| introduces the necessary stack values.
Similarly, \factor|#return| is inserted at the end of the sequence to indicate
the final state of the data stack with its \factor|in-d| slot.

The most basic operations of a stack language are, of course, pushing literals
and calling functions that pop inputs and push outputs.  The \factor|#push|
node thus has a \factor|literal| slot and an \factor|out-d| slot, giving a name
to the single element it pushes to the data stack.  \factor|#call|, of course,
is used for normal word invocations.  The \factor|in-d| and \factor|out-d|
slots effectively serve as the stack effect declaration.  In later analyses,
data about the word's definition may be stored across the \factor|body|,
\factor|method|, \factor|class|, and \factor|info| slots.

\inputlst{build-tree-1}

The word \factor|build-tree| takes a Factor quotation and constructs the
equivalent high-level \gls{IR} form.  In \vref{lst:build-tree-1}, we see the
output of the simple example
%
\factor|[ 1 + ] build-tree|.
%
Note that
%
\factor|T{ class { slot1 value1 } { slot2 value2 } ... }|
%
is the syntax for tuple literals.  The first node is a \factor|#push| for the
\factor|1| literal.  Since \factor|+| needs two input values, an
\factor|#introduce| pushes a new ``phantom'' value.  \factor|+| gets turned
into a \factor|#call| instance.  Notice the \factor|in-d| slot refers to the
values in the order that they're passed to the word, not necessarily the order
they've been introduced in the \gls{IR}.  The sum is pushed to the data stack,
so the \factor|out-d| slot is a singleton that names this value.  Finally,
\factor|#return| indicates the end of the routine, its \factor|in-d| containing
the value left on the stack (the sum pushed by \factor|#call|).

\inputlst{build-tree-2}

The next tuples in \vref{lst:tree} reassign existing values on the stack to
fresh identifiers.  The \factor|#renaming| superclass has the two subclasses
\factor|#copy| and \factor|#shuffle|.  The former represents the bijection from
elements of \factor|in-d| to elements of \factor|out-d| in the same position;
corresponding values are copies of each other.  The latter represents a more
general mapping.  Stack shufflers are translated to \factor|#shuffle| nodes
with \factor|mapping| slots that dictate how the fresh values in \factor|out-d|
correspond to the input values in \factor|in-d|.  For instance,
\vref{lst:build-tree-2} shows how \factor|swap| takes in the values
\factor|6256132| and \factor|6256133| and outputs \factor|6256134| and
\factor|6256135|, where the former is mapped to the second element
(\factor|6256133|) and the latter to the first (\factor|6256132|).  Thus,
\factor|out-d| swaps the two elements of \factor|in-d|, mapping them to fresh
identifiers.  The \factor|in-r| and \factor|out-r| slots of \factor|#shuffle|
correspond to the \term{retain} stack, which is an implementation detail beyond
the scope of this discussion.

\inputlst{build-tree-3}
\inputlst{build-tree-4}

\factor|#declare| is a miscellaneous node used for the \factor|declare|
primitive.  It simply annotates type information to stack values, as in
\vref{lst:build-tree-3}.  \factor|#terminate| is another one-off node, but a
much more interesting one.  While Factor normally requires a balanced stack,
sometimes we purposefully want to throw an error.  \factor|#terminate| is
introduced where the program halts prematurely.  When checking the stack
height, it gets to be treated specially so that \term{terminated} stack effects
unify with any other effect.  That way, branches will still be balanced even if
one of them unconditionally throws an error.  \vref{lst:build-tree-4} shows
\factor|#terminate| being introduced by the \factor|throw| word.

Next, \vref{lst:tree} defines nodes for branching based off the superclass
\factor|#branch|.  The \factor|children| slot contains vectors of nodes
representing different branches.  \factor|live-branches| is filled in during
later analyses to indicate which branches are alive so that dead ones may be
removed.  For instance, \factor|#if| will have two elements in its
\factor|children| slot representing the true and false branches.  On the other
hand, \factor|#dispatch| has an arbitrary number of children.  It corresponds
to the \factor|dispatch| primitive, which is an implementation detail of the
generic word system used to speed up method dispatch.

\todo[inline]{Should extract the SSA junk into the general intro of the thesis}

You may have noted the emphasis on introducing new values in \factor|out-d|
slots.  Even \factor|#shuffle|s output fresh identifiers, letting their values
be determined by its \factor|mapping|.  The reason for this is that
\factor|compiler.tree| uses \gls{SSA} form, wherein every variable is defined
by exactly one statement.  This simplifies the properties of variables, which
helps optimizations perform faster and with better results.  By giving unique
names to the targets of each assignment, the \gls{SSA} property is guaranteed.
However, \factor|#branch|es introduce ambiguity: after, say, an \factor|#if|,
what will the identifiers in \factor|out-d| be?  It depends on which branch is
taken.  To remedy this problem, after any \factor|#branch| node, Factor will
place a \factor|#phi| node---the classical \gls{SSA} ``phony function''.  The
\factor|phi-in-d| slot seen in \vref{lst:tree} is a sequence of sequences; each
one corresponds to the \factor|out-d| of the child at the same position in the
\factor|children| of the preceding node.  The \factor|#phi|'s \factor|out-d|
gives unique names to the output values, thus ensuring the \gls{SSA} property.
Though it doesn't perform any literal computation, conceptually it select the
``correct'' \factor|out-d| depending on the control flow.

\inputlst{build-tree-5}

For example, the \factor|#phi| in \vref{lst:build-tree-5} will select between
the
%
\factor|6256248|
%
return value of the first child or the 
%
\factor|6256249|
%
output of the second.  Either way, we can refer to the result as
\factor|6256250| afterwards.  The \factor|terminated| slot of the \factor|#phi|
tells us if there was a \factor|#terminate| in any of the branches.

The \factor|#recursive| node encapsulates \term{inline recursive} words.  In
Factor, words may be annotated with simple compiler declarations, which guide
optimizations.  If we follow a standard colon definition with the
\factor|inline| word, we're saying that its definition can be spliced into the
call-site, rather than generating code to jump to a subroutine.  Inline words
that call themselves must additionally be declared \factor|recursive|.  For
example, we could write
%
\factor|: foo ( -- ) foo ; inline recursive|.
%
The nodes \factor|#enter-recursive|, \factor|#call-recursive|, and
\factor|#return-recursive| denote different stages of the recursion---the
beginning, recursive call, and end, respectively.  They carry around a lot of
metadata about the nature of the recursion, but it doesn't serve our purposes
to get into the details.  Similarly, we gloss over the final nodes of
\vref{lst:tree} correspond to Factor's \gls{FFI} vocabulary, called
\factor|alien|.  At a high level, \factor|#alien-node|, \factor|#alien-invoke|,
\factor|#alien-indirect|, \factor|#alien-assembly|, and
\factor|#alien-callback| are used to make calls to C libraries from within
Factor.

\inputlst{optimize-tree}
\todo[inline]{Shouldn't bold ``cleanup'' in \cref{lst:optimize-tree}}

Now that we're familiar with the structure of the high-level \gls{IR}, we can
turn our attention to optimization.  \Vref{lst:optimize-tree} shows the passes
performed on a sequence of nodes by the word \factor|optimize-tree|.  Before
optimization can begin, we must gather some information and clean up some
oddities in the output of \factor|build-tree|.  \factor|analyze-recursive| is
called first to identify and mark loops in the tree.  Effectively, this means
we detect tail-recursion introduced by \factor|#recursive| nodes.  Future
passes can then use this information for data flow analysis.  Then,
\factor|normalize| makes the tree more consistent by doing two things:
%
\begin{itemize}
%
  \item All \factor|#introduce| nodes are removed and replaced by a single
        \factor|#introduce| at the beginning of the whole program.  This way,
        further passes needn't handle \factor|#introduce| nodes.
%
  \item As constructed, the \factor|in-d| of a \factor|#call-recursive| will be
        the entire stack at the time of the call.  This assumption happens
        because we don't know how many inputs it needs until the
        \factor|#return-recursive| is processed, because of row polymorphism.
        So, here we figure out exactly what stack entries are needed, and trim
        the \factor|in-d| and \factor|out-d| of each \factor|#call-recursive|
        accordingly.
%
\end{itemize}

Once these passes have cleaned up the tree, \factor|propagate| performs
probably the most extensive analysis of all the phases.  In short, it performs
an extended version of \gls{SCCP}\todo{cite}.  The traditional data flow
analysis combines global copy propagation, constant propagation, and some
limited constant folding in a \term{flow-sensitive} way.  That is, it will
propagate information from branches that it knows are definitely taken (e.g.,
because \factor|#if| is always given a true input).  Instead of using the
typical single-level (numeric) constant value lattice, Factor uses a lattice
augmented by information about classes, numeric value ranges, array lengths,
and tuple slots' classes.  Classes can be used in the lattice with the
partial-order protocol described briefly in \cref{sec:primer:oo}.
Additionally, the transfer functions are allowed to inline certain calls if
enough information is present.  This occurs in the transfer function since
generic words' inline expansions into particular methods provide more
information, thus giving us more opportunities for propagation.  This is
particularly useful for arithmetic words.  In Factor, words like \factor|+| and
\factor|*| are generics that work across all sorts of numeric representations,
be they \factor|fixnum|s, \factor|float|s, \factor|bignum|s, etc.  If the
operation overflows, the values are automatically cast up to larger
representations.  But iterated refinement of the inputs' classes can let the
compiler select more specific, efficient methods (e.g., if both arguments are
\factor|fixnum|s).

Interval propagation also helps propagate class information.  By refining the
range of possible values a particular item can have, we might discover that,
say, it's small enough to fit in a \factor|fixnum| rather than a
\factor|bignum|.  There are plenty more things that interval propagation can
tell us, too.  For example, it may give us enough information to remove
overflow checks performed by numeric words.  And if the interval has zero
length, we may replace the value with a constant.  This then continues getting
propagated, contributing to constant folding and so forth.

%cleanup
%dup run-escape-analysis? [
%    escape-analysis
%    unbox-tuples
%] when
%apply-identities
%compute-def-use
%remove-dead-code
%?check
%compute-def-use
%optimize-modular-arithmetic
%finalize

% dls.pdf verbatim:
%
%When static type information is available, Factor’s compiler can eliminate
%runtime method dispatch and allocation of in- termediate objects, generating
%code specialized to the under- lying data structures. This resembles previous
%work in soft typing [10]. Factor provides several mechanisms to facilitate
%static type propagation:
%
%\begin{itemize}
%
%\item Functions can be annotated as inline, causing the compiler to replace
%calls to the function with the function body.
%
%\item Functions can be hinted, causing the compiler to gener- ate multiple
%specialized versions of the function, each assuming different input types, with
%dispatch at the en- try point to choose the best-fitting specialization for the
%given inputs.  
%
%\item Methods on generic functions propagate the type infor- mation for their
%dispatched-on inputs.  
%
%\item Functions can be declared with static input and output types using the
%typed library.
%
%\end{itemize}
%
%The three major optimizations performed on high-level IR are sparse conditional
%constant propagation (SCCP [45]), escape analysis with scalar replacement, and
%overflow check elimination using modular arithmetic properties.  The major
%features of our SCCP implementation are an extended value lattice, rewrite
%rules, and flow sensitivity.  Our SCCP implementation augments the standard
%single- level constant lattice with information about object types, numeric
%intervals, array lengths and tuple slot types. Type transfer functions are
%permitted to replace nodes in the IR with inline expansions. Type functions are
%defined on many of the core language words.  SCCP is used to statically dispatch
%generic word calls by inlining a specific method body at the call site. This
%inlining generates new type information and new opportunities for constant
%folding, simplification and further inlining. In par- ticular, generic
%arithmetic operations which require dynamic dispatch in the general case can be
%lowered to simpler opera- tions as type information is discovered. Overflow
%checks can be removed from integer operations using numeric interval
%information. The analysis can represent flow-sensitive type information.
%Additionally, calls to closures which combina- tor inlining cannot eliminate
%are eliminated when enough in- formation is available [16].

%An escape analysis
%pass is used to discover object alloca- tions which are not stored on the heap
%or returned from the current function. Scalar replacement is performed on such
%allocations, converting tuple slots into SSA values.  The modular arithmetic
%optimization pass identifies in- teger expressions in which the final result is
%taken to be modulo a power of two and removes unnecessary overflow checks from
%any intermediate addition and multiplication operations. This novel
%optimization is global and can operate over loops.

%\subsection{Low-level Optimizations}\label{sec:compiler:cfg}

The low-level \gls{IR} in \factor|compiler.cfg| takes the more conventional
form of a \gls{CFG}.  A \gls{CFG} (not to be confused with ``context-free
grammar'') is an arrangement of instructions into \term{basic blocks}: maximal
sequences of ``straight-line'' code, where control does not transfer out of or
into the middle of the block.  Directed edges are added between blocks to
represent control flow---either from a branching instruction to its target, or
from the end of a basic block to the start of the next one\todo{cite}.
Construction of the low-level \gls{IR} proceeds by analyzing the control flow
of the high-level \gls{IR} and converting the nodes of \cref{sec:compiler:tree}
into lower-level, more conventional instructions modeled after typical assembly
code.  There are over a hundred of these instructions, but many are simply
different versions of the same operation.  For instance, while instructions are
generally called on \term{virtual registers} (represented in Factor simply by
integers), there are \term{immediate} versions of instructions.  The
\factor|##add| instruction, as an example, represents the sum of the contents
of two registers, but \factor|##add-imm| sums the contents of one register and
an integer literal.  Other instructions are inserted to make stack reads and
writes explicit, as well as to balance the height.  Below is a categorized list
of all the instruction objects (each one is a subclass of the \factor|insn|
tuple).

\todo[inline]{Is the complete list really necessary?}
\begin{itemize}
\item
\begin{flushleft}
Loading constants:
\Verb|##load-integer|,
\Verb|##load-reference|
\end{flushleft}

\item
\begin{flushleft}
Optimized loading of constants, inserted by representation selection:
\Verb|##load-tagged|,
\Verb|##load-float|,
\Verb|##load-double|,
\Verb|##load-vector|
\end{flushleft}

\item
\begin{flushleft}
Stack operations:
\Verb|##peek|,
\Verb|##replace|,
\Verb|##replace-imm|,
\Verb|##inc-d|,
\Verb|##inc-r|
\end{flushleft}

\item
\begin{flushleft}
Subroutine calls:
\Verb|##call|,
\Verb|##jump|,
\Verb|##prologue|,
\Verb|##epilogue|,
\Verb|##return|
\end{flushleft}

\item
\begin{flushleft}
Inhibiting \gls{TCO}:
\Verb|##no-tco|
\end{flushleft}

\item
\begin{flushleft}
Jump tables:
\Verb|##dispatch|
\end{flushleft}

\item
\begin{flushleft}
Slot access:
\Verb|##slot|,
\Verb|##slot-imm|,
\Verb|##set-slot|,
\Verb|##set-slot-imm|
\end{flushleft}

\item
\begin{flushleft}
Register transfers:
\Verb|##copy|,
\Verb|##tagged>integer|
\end{flushleft}

\item
\begin{flushleft}
Integer arithmetic:
\Verb|##add|,
\Verb|##add-imm|,
\Verb|##sub|,
\Verb|##sub-imm|,
\Verb|##mul|,
\Verb|##mul-imm|,
\Verb|##and|,
\Verb|##and-imm|,
\Verb|##or|,
\Verb|##or-imm|,
\Verb|##xor|,
\Verb|##xor-imm|,
\Verb|##shl|,
\Verb|##shl-imm|,
\Verb|##shr|,
\Verb|##shr-imm|,
\Verb|##sar|,
\Verb|##sar-imm|,
\Verb|##min|,
\Verb|##max|,
\Verb|##not|,
\Verb|##neg|,
\Verb|##log2|,
\Verb|##bit-count|
\end{flushleft}

\item
\begin{flushleft}
Float arithmetic:
\Verb|##add-float|,
\Verb|##sub-float|,
\Verb|##mul-float|,
\Verb|##div-float|,
\Verb|##min-float|,
\Verb|##max-float|,
\Verb|##sqrt|
\end{flushleft}

\item
\begin{flushleft}
Single/double float conversion:
\Verb|##single>double-float|,
\Verb|##double>single-float|
\end{flushleft}

\item
\begin{flushleft}
Float/integer conversion:
\Verb|##float>integer|,
\Verb|##integer>float|
\end{flushleft}

\item
\begin{flushleft}
\Gls{SIMD} operations:
\Verb|##zero-vector|,
\Verb|##fill-vector|,
\Verb|##gather-vector-2|,
\Verb|##gather-int-vector-2|,
\Verb|##gather-vector-4|,
\Verb|##gather-int-vector-4|,
\Verb|##select-vector|,
\Verb|##shuffle-vector|,
\Verb|##shuffle-vector-halves-imm|,
\Verb|##shuffle-vector-imm|,
\Verb|##tail>head-vector|,
\Verb|##merge-vector-head|,
\Verb|##merge-vector-tail|,
\Verb|##float-pack-vector|,
\Verb|##signed-pack-vector|,
\Verb|##unsigned-pack-vector|,
\Verb|##unpack-vector-head|,
\Verb|##unpack-vector-tail|,
\Verb|##integer>float-vector|,
\Verb|##float>integer-vector|,
\Verb|##compare-vector|,
\Verb|##test-vector|,
\Verb|##test-vector-branch|,
\Verb|##add-vector|,
\Verb|##saturated-add-vector|,
\Verb|##add-sub-vector|,
\Verb|##sub-vector|,
\Verb|##saturated-sub-vector|,
\Verb|##mul-vector|,
\Verb|##mul-high-vector|,
\Verb|##mul-horizontal-add-vector|,
\Verb|##saturated-mul-vector|,
\Verb|##div-vector|,
\Verb|##min-vector|,
\Verb|##max-vector|,
\Verb|##avg-vector|,
\Verb|##dot-vector|,
\Verb|##sad-vector|,
\Verb|##horizontal-add-vector|,
\Verb|##horizontal-sub-vector|,
\Verb|##horizontal-shl-vector-imm|,
\Verb|##horizontal-shr-vector-imm|,
\Verb|##abs-vector|,
\Verb|##sqrt-vector|,
\Verb|##and-vector|,
\Verb|##andn-vector|,
\Verb|##or-vector|,
\Verb|##xor-vector|,
\Verb|##not-vector|,
\Verb|##shl-vector-imm|,
\Verb|##shr-vector-imm|,
\Verb|##shl-vector|,
\Verb|##shr-vector|
\end{flushleft}

\item
\begin{flushleft}
Scalar/vector conversion:
\Verb|##scalar>integer|,
\Verb|##integer>scalar|,
\Verb|##vector>scalar|,
\Verb|##scalar>vector|
\end{flushleft}

\item
\begin{flushleft}
Boxing and unboxing aliens:
\Verb|##box-alien|,
\Verb|##box-displaced-alien|,
\Verb|##unbox-any-c-ptr|,
\Verb|##unbox-alien|
\end{flushleft}

\item
\begin{flushleft}
Zero-extending and sign-extending integers:
\Verb|##convert-integer|
\end{flushleft}

\item
\begin{flushleft}
Raw memory access:
\Verb|##load-memory|,
\Verb|##load-memory-imm|,
\Verb|##store-memory|,
\Verb|##store-memory-imm|
\end{flushleft}

\item
\begin{flushleft}
Memory allocation:
\Verb|##allot|,
\Verb|##write-barrier|,
\Verb|##write-barrier-imm|,
\Verb|##alien-global|,
\Verb|##vm-field|,
\Verb|##set-vm-field|
\end{flushleft}

\item
\begin{flushleft}
The \gls{FFI}:
\Verb|##unbox|,
\Verb|##unbox-long-long|,
\Verb|##local-allot|,
\Verb|##box|,
\Verb|##box-long-long|,
\Verb|##alien-invoke|,
\Verb|##alien-indirect|,
\Verb|##alien-assembly|,
\Verb|##callback-inputs|,
\Verb|##callback-outputs|
\end{flushleft}

\item
\begin{flushleft}
Control flow:
\Verb|##phi|,
\Verb|##branch|
\end{flushleft}

\item
\begin{flushleft}
Tagged conditionals:
\Verb|##compare-branch|,
\Verb|##compare-imm-branch|,
\Verb|##compare|,
\Verb|##compare-imm|
\end{flushleft}

\item
\begin{flushleft}
Integer conditionals:
\Verb|##compare-integer-branch|,
\Verb|##compare-integer-imm-branch|,
\Verb|##test-branch|,
\Verb|##test-imm-branch|,
\Verb|##compare-integer|,
\Verb|##compare-integer-imm|,
\Verb|##test|,
\Verb|##test-imm|
\end{flushleft}

\item
\begin{flushleft}
Float conditionals:
\Verb|##compare-float-ordered-branch|,
\Verb|##compare-float-unordered-branch|,
\Verb|##compare-float-ordered|,
\Verb|##compare-float-unordered|
\end{flushleft}

\item
\begin{flushleft}
Overflowing arithmetic:
\Verb|##fixnum-add|,
\Verb|##fixnum-sub|,
\Verb|##fixnum-mul|
\end{flushleft}

\item
\begin{flushleft}
\Gls{GC} checks:
\Verb|##save-context|,
\Verb|##check-nursery-branch|,
\Verb|##call-gc|
\end{flushleft}

\item
\begin{flushleft}
Spills and reloads, inserted by the register allocator:
\Verb|##spill|,
\Verb|##reload|
\end{flushleft}
\end{itemize}


\inputlst{optimize-cfg}

\inputfig{optimize-tail-calls}

By translating the high-level \gls{IR} into instructions that manipulate
registers directly, we reveal further redundancies that can be optimized away.
The \factor|optimize-cfg| word in \vref{lst:optimize-cfg} shows the passes
performed in doing this.  The first word, \factor|optimize-tail-calls|,
performs tail call elimination on the \gls{CFG}.
%
\term{Tail calls}~\todo{used in \cref{sec:compiler:tree}, not defined} are those
that occur within a procedure and whose results are immediately returned by
that procedure.  Instead of allocating a new call stack frame, we may convert
tail calls into simple jumps, since afterwards the current procedure's call
frame isn't really needed.  In the case of recursive tail calls, we can convert
special cases of recursion into loops in the \gls{CFG}, so that we won't
trigger call stack overflows.  For instance, consider
\vref{fig:optimize-tail-calls}, which shows the effect of
\factor|optimize-tail-calls| on the following definition:
%
\begin{center}
%
  \factor|: tail-call ( -- ) tail-call ;|
%
\end{center}
%
\noindent Note the recursive call (trivially) occurs at the end of the
definition, just before the return point.  When translated to a \gls{CFG}, this
is a \factor|##call| instruction, as seen in block $4$ to the left of
\vref{fig:optimize-tail-calls}.  This is also just before the final
\factor|##epilogue| and \factor|##return| instructions in block $8$, as blocks
$5$--$7$ are effectively empty (these excessive \factor|##branch|es will be
eliminated in a later pass).  Because of this, rather than make a whole new
subroutine call, we can convert it into a \factor|##branch| back to the
beginning of the word, as in the \gls{CFG} to the right.

\inputfig{delete-useless-conditionals}

The next pass in \vref{lst:optimize-cfg} is
\factor|delete-useless-conditionals|, which removes branches that go to the
same basic block.  This situation might occur as a result of optimizations
performed in the high-level \gls{IR}.  To see it in action,
\vref{fig:delete-useless-conditionals} shows the transformation on a
purposefully useless conditional,
%
\factor|[ ] [ ] if|.
%
Before removing the useless conditional, the \gls{CFG} \factor|##peek|s at the
top of the data stack
%
(\factor|D 0|),
%
storing the result in the virtual register \factor|1|.  This value is popped,
so we decrement the stack height
%
(\factor|##inc-d -1|).
%
Then, \factor|##compare-imm-branch| in block $2$ compares the value in the
virtual register \factor|2| (which is a copy of \factor|1|, the top of the
stack) to the immediate value \factor|f| to see if it's not equal (signified by
\factor|cc/=|).  However, both branches jump through several empty blocks and
merge at the same destination.  Thus, we can remove both branches and replace
\factor|##compare-imm-branch| with an unconditional \factor|##branch| to the
eventual destination.  We see this on the right of
\vref{fig:delete-useless-conditionals}.

\inputfig{split-branches}

In order to expose more opportunities for optimization, \factor|split-branches|
will actually duplicate code.  We use the fact that code immediately following
a conditional will be executed along either branch.  If it's sufficiently
short, we copy it up into the branches individually.  That is, we change
%
\factor|[ A ] [ B ] if C|
%
into
%
\factor|[ A C ] [ B C ] if|,
%
as long as \factor|C| is small enough.  Later analyses may then, for example,
more readily eliminate one of the branches if it's never taken.
\vref{fig:split-branches} shows what such a transformation looks like on a
\gls{CFG}.  The example
%
\factor|[ 1 ] [ 2 ] if dup|
%
is essentially changed into
%
\factor|[ 1 dup ] [ 2 dup ] if|,
%
thus splitting the block with two predecessors (block $9$) on the left.

\inputfig{join-blocks}

The next pass, \factor|join-blocks|, compacts the \gls{CFG} by joining together
blocks involved in only a single control flow edge.  Mostly, this is to clean
up the myriad of empty or short blocks introduced during construction, like
sequences of a bunch of \factor|##branch|es.  \Vref{fig:join-blocks} shows this
pass on the \gls{CFG} of
%
\factor|0 100 [ 1 fixnum+fast ] times|.
%
\factor|fixnum+fast| is a specialized version of \factor|+| that suppresses
overflow and type checks.  We use it here to keep the \gls{CFG} simple.  We'll
be using this particular code to illustrate all but one of the remaining
optimization passes in \vref{lst:optimize-cfg}, as it's a motivating example
for the work in this thesis.  The passes before \factor|join-blocks| don't
change the \gls{CFG} seen on the left in \vref{fig:join-blocks}, but we get rid
of the useless \factor|##branch| blocks in the \gls{CFG} on the right.

\inputfig{normalize-height}

\Vref{fig:normalize-height} shows the result of applying
\factor|normalize-height| to the result of \factor|join-blocks|.  This phase
combines and canonicalizes the instructions that track the stack height, like
\factor|##inc-d|.  While the shuffling in this example isn't complex enough to
be interesting, neither is this phase.  It amounts to more cleanup: multiple
height changes are combined into single ones at the beginnings of the basic
blocks.  In \vref{fig:normalize-height}, this means that \factor|##inc-d| is
moved to the top of block $1$, as compared to the right of
\vref{fig:join-blocks}.

\inputfig{construct-ssa}

In converting the high-level \gls{IR} to the low-level, we actually lose the
\gls{SSA} form of \factor|compiler.tree|.  Not only does the construction do
this, but \factor|split-branches| also copies basic blocks verbatim, so any
value defined will have a duplicate definition site, violating the \gls{SSA}
property.  \factor|construct-ssa| recomputes a so-called \term{pruned}
\gls{SSA} form, wherein $\phi$ functions are inserted only if the variables are
live after the insertion point.  This cuts down on useless $\phi$ 
%
functions\todo{cite TDMSC and construction algorithm}.
%
\Vref{fig:construct-ssa} shows the reconstructed \gls{SSA} form of the
\gls{CFG} from \vref{fig:normalize-height}.

The next pass, \factor|alias-analysis|, doesn't change the \gls{CFG} of
%
\factor|0 100 [ 1 fixnum+fast ] times|,
%
so we won't have an accompanying \lcnamecref{fig:construct-ssa}.  At a high
level, \factor|alias-analysis| is easy to understand: it eliminates redundant
memory loads and stores by rewriting certain patterns of memory access.  If the
same location is loaded after being stored, we convert the latter load into a
\factor|##copy| of the value we stored.  Two reads of the same location with no
intermittent write gets the second read turned into a \factor|##copy|.
Similarly, if we see two writes without a read in the middle, the first write
can be removed.

\inputfig{value-numbering}

\factor|value-numbering| is the key focus of this thesis.  It will be detailed
in \vref{sec:vn}.  For now, it does to think of it as a combination of common
subexpression elimination and constant folding.  In \vref{fig:value-numbering}, 
we see several changes:
%
\begin{itemize}
%
  \item \factor|##load-integer 23 0| in block $1$ of \vref{fig:construct-ssa}
  (which assigns the value \factor|0| to the virtual register \factor|23|) is
  redundant, so is replaced by \factor|##copy 23 21|.
%
  \item \begin{flushleft}
  In block $2$, the last instruction
  %
  \factor|##compare-imm-branch 32 f cc/=|
  %
  is the same as
  %
  \factor|##compare-integer-branch 30 26 cc<|.
  %
  The source register (\factor|32|) of the original is a \factor|##copy| of
  \factor|31|, which itself is computed by
  %
  \factor|##compare-integer 31 30 26 cc< 9|.
  %
  So, the \factor|##compare-imm-branch| is equivalent to a simple
  \factor|##compare-integer-branch|, which doesn't use the temporary virtual
  register \factor|9| and doesn't waste time comparing against the \factor|f|
  object.
  \end{flushleft}
%
  \item The second operands in both \factor|##add|s of block $3$ are just
  constants stored by \factor|##load-integer|s.  So, these are turned into
  \factor|##add-imm|s.
\end{itemize}
%
\noindent In \cref{sec:vn}, we'll see how and why this pass fails to identify
other equivalences.

\inputfig{copy-propagation}

\inputfig{eliminate-dead-code}

\inputfig{finalize-cfg}

% dls.pdf verbatim:

%The main optimizations performed on low-level IR are local dead store and
%redundant load elimination, local value numbering, global copy propagation,
%representation selection, and instruction scheduling.  The local value
%numbering pass eliminates common subexpressions and folds expressions with
%constant operands [9].

%Following value numbering and copy propagation, a representation selection
%pass decides when to unbox floating point and SIMD values. A form of
%instruction scheduling intended to reduce register pressure is performed on
%low-level IR as the last step before leaving SSA form [39].  We use the
%second-chance binpacking variation of the linear scan register allocation
%algorithm [43, 47]. Our variant does not take $\phi$ nodes into account, so
%SSA form is destructed first by eliminating $\phi$ nodes while simultaneously
%performing copy coalescing, using the method described in [6].


% dls.pdf verbatim:

%Low-level IR is built from high-level IR by analyzing control flow and making
%stack reads and writes explicit. During this construction phase and a
%subsequent branch splitting phase, the SSA structure of high-level IR is lost.
%SSA form is recon- structed using the SSA construction algorithm described in
%[8], with the minor variation that we construct pruned SSA form rather than
%semi-pruned SSA, by first computing live- ness. To avoid computing iterated
%dominance frontiers, we use the TDMSC algorithm from [13].  The main
%optimizations performed on low-level IR are local dead store and redundant load
%elimination, local value numbering, global copy propagation, representation
%selec- tion, and instruction scheduling.  The local value numbering pass
%eliminates common sub- expressions and folds expressions with constant operands
%[9]. Following value numbering and copy propagation, a representation selection
%pass decides when to unbox floating point and SIMD values. A form of instruction
%scheduling intended to reduce register pressure is performed on low- level IR
%as the last step before leaving SSA form [39].  We use the second-chance
%binpacking variation of the lin- ear scan register allocation algorithm [43,
%47]. Our variant does not take φ nodes into account, so SSA form is destruc-
%ted first by eliminating φ nodes while simultaneously per- forming copy
%coalescing, using the method described in [6].

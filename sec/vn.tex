\chapter{Value Numbering}\label{sec:vn}

At a very basic level, most optimization techniques revolve around avoiding
redundant or unnecessary computation.  Thus, it's vital that we discover which
values in a program are equal.  That way, we can simplify the code that wastes
machine cycles repeatedly calculating the same values.  Classic optimization
phases like constant/copy propagation, common subexpression elimination,
loop-invariant code motion, induction variable elimination, and others
discussed in the de facto treatise, ``The Dragon Book'' \autocite{DragonBook},
perform this sort of redundancy elimination based on information about the
equality of expressions.

In general, the problem of determining whether two expressions in a program are
equivalent is undecidable.  Therefore, we seek a \term{conservative} solution
that doesn't necessarily identify all equivalences, but is nevertheless correct
about any equivalences it does identify.  Solving this equivalence problem is
the work of \term{value numbering} algorithms.  These assign every value in the
program a number such that two values have the same value number if and only if
the compiler can prove they will be equal at runtime.

Value numbering has a long history in literature and practice, spanning many
techniques.  In \cref{sec:compiler:cfg} we saw the \Verb|value-numbering| word,
which is actually based on some of the earliest---and least effective---methods
of value numbering.  \Cref{sec:vn:local} describes the way Factor's current
algorithm works, highlighting its shortcomings to motivate the main work of
this thesis, which is covered in \cref{sec:vn:global,sec:vn:avail}.  We finish
the \lcnamecref{sec:vn} by analyzing the results of these changes and reviewing
the literature for further enhancements that can be made to this optimization
pass.

\section{Local Value Numbering}\label{sec:vn:local}

Tracing the exact origins of value numbering is difficult.  It's thought to
have originally been invented in the 1960s by Balke \autocite{Simpson}.  The
earliest tangible reference to a value numbering (at least, the earliest point
where discussions in the literature seem to start) appears in an oft-cited but
unpublished work of \citeauthor{Cocke} \autocite*{Cocke}.  The technique is
relatively simple, but not as powerful as other methods for reasons described
hereafter.

The algorithm considers a single basic block.  For each instruction (from top
to bottom) in the block, we essentially let the value number of the assignment
target be a hash of the operator and the value numbers of the operands.  That
is, we hash the \term{expression} being computed by an instruction.  Thus,
assuming a proper hash function, two expressions are \term{congruent} if
%
\begin{itemize}
%
  \item they have the same operators and
%
  \item their operands are congruent.
%
\end{itemize}
%
\noindent This is our approximation of runtime equivalence.  The first property
is fulfilled by basing the hash, in part, on the operator.  The second property
holds because the hash is based on the value numbers of the statement's
operands---not just the operands as they appear in code (i.e., \term{lexical}
equivalence).  Any information about congruence is propagated through the value
numbers.  We'll have discovered any such equivalences among the operands before
computing the value number of the assignment target because every value in a
basic block is either defined before it's used, or else defined at some point
in a predecessor of the block, which we don't care about when only considering
one basic block.

This is the first shortcoming of the algorithm.  It is \term{local}, focusing
on only one basic block at a time.  Any definitions outside the boundaries of
the basic block won't be reused, even if they reach the block.  This severely
limits the scope of the redundancies we can discover.  We could improve upon
this by considering the algorithm across an entire loop-free \gls{CFG} in any
\term{topological order}.  In such an ordering, a basic block $B$ comes before
any other block $B'$ to which it has an edge.  Thus, any ``outside'' variables
that instructions in $B'$ rely on must have come from $B$ or earlier, which
will have already been computed in a traversal of such an ordering.  However,
\glsplural{CFG} usually contain cycles or loops (at least interesting ones do),
which make such an ordering impossible.  We could still pick a topological
order that ignores back-edges, but we may encounter operands whose values flow
along those back-edges, so haven't been processed yet.  We'll address this
issue later.

\begin{sloppypar}
In Factor, expressions are basically instructions (the \Verb|insn| objects
discussed in \cref{sec:compiler:cfg}) that have had their destination registers
stripped.  Instructions can be converted to expressions with the \Verb|>expr|
word defined in the \Verb|compiler.cfg.value-numbering.expressions|
vocabulary.  For instance, an \Verb|##add| instruction with the destination
register \Verb|1| and source registers \Verb|2| and \Verb|3| will be
converted into an array of three elements:
%
\begin{itemize}
%
  \item The \Verb|##add| class word, indicating the expression is derived
        from an \Verb|##add| instruction.
%
  \item The value number of the virtual register \Verb|2|.
%
  \item The value number of the virtual register \Verb|3|.
%
\end{itemize}
%
\noindent Some instructions are not \term{referentially transparent}, meaning
they can't be replaced with the value they compute without changing the
program's behavior.  For example, \Verb|##call| and \Verb|##branch| cannot
reasonably be converted into expressions.  In these cases, \Verb|>expr|
merely returns a unique value.
\end{sloppypar}

\inputlst{value-numbering-graph}

The hashing of expressions takes place in the so-called \term{expression graph}
implemented in the vocabulary shown in \vref{lst:value-numbering-graph}.  This
consists of three global hash tables that relate virtual registers, value
numbers, instructions, and expressions.  Since virtual registers are just
integers, we actually use them as value numbers, too.  \Verb|vregs>vns| maps
virtual registers to their value numbers.  If a virtual register  is mapped to
itself in this table, its definition is the canonical instruction that we use
to compute the value.  This instruction is stored in the \Verb|vns>insns| table.  Finally, the most important mapping is \Verb|exprs>vns|.  True to its
name, it uses expressions as keys, which of course are implicitly hashed.
Thus, we can use this table to determine equivalence of expressions.

Other definitions in \vref{lst:value-numbering-graph} manipulate expressions
and the graph.  The global variable \Verb|input-expr-counter| is used in the
generation of unique expressions discussed earlier.  \Verb|init-value-graph|
initializes this and all the tables.  \Verb|set-vn| establishes a mapping
from a virtual register to a value number in \Verb|vregs>vns|.
\Verb|vn>insn| gives terse access to the \Verb|vns>insns| table.
\Verb|vreg>insn| uses \Verb|vregs>vns| and \Verb|vns>insns| to get the
canonical instruction that defines a given virtual register.  Finally,
\Verb|vreg>vn| looks up the value of a key in the \Verb|vregs>vns| table.
Importantly, if the key is not yet present in the table, it is automatically
mapped to itself---it's assumed that the virtual register does not correspond
to a redundant instruction.

This is the second shortcoming of the algorithm.  It must make a
\term{pessimistic} assumption about congruences.  That is, it starts by
assuming that every expression has a unique value number, then tries to prove
that there are some values which are actually congruent.  This fails to
discover congruences for values that flow along back-edges, whether we consider
a single basic block or an entire topological ordering.

One the other hand, an advantage of this local value numbering algorithm is its
simplicity.  It makes a single pass over all the instructions, identifying and
replacing redundancies \term{online} (i.e., rewriting as it goes).  It's
straightforward to write, and even to extend.  In particular, there's nothing
stopping the online replacements from being more complex than \Verb|##copy|
instructions.  At every step, the currently known value numbers will be sound,
and we can use this information for copy/constant propagation, constant
folding, and common subexpression elimination.

\inputlst{value-numbering-step}

\begin{sloppypar}
To see how Factor accomplishes these extensions, we'll take a look at the
\Verb|compiler.cfg.value-numbering| vocabulary.
\Vref{lst:value-numbering-step} shows the main words that start the
optimization pass.  The \Verb|value-numbering-step| word is called on the
sequence of instructions that comprise each basic block.  It starts the
expression graph from a blank slate with \Verb|init-value-graph|, then
\Verb|map|s the word \Verb|process-instruction| on each of them.  This is a
generic word that we'll study momentarily; it returns either a single
\Verb|insn| object or a sequence of them (in the case that an instruction is
replaced by several others).  Then, the work of \Verb|value-numbering| is to
just call \Verb|value-numbering-step| on each basic block, which is done with a
combinator called \Verb|simple-optimization|.  The words \Verb|cfg-changed| and
\Verb|predecessors-changed| alter some internal state of the \gls{CFG} that has
been potentially invalidated by some transformations performed by
\Verb|process-instruction|.
\end{sloppypar}

\inputlst{process-instruction}

The methods of \Verb|process-instruction| are shown in
\vref{lst:process-instruction}.  The default behavior for dispatching on an
\Verb|insn| is to invoke yet another generic word, \Verb|rewrite|.  This word
will return either a replacement \Verb|insn| (or sequence thereof) or
\factor|f|, indicating that no change has taken place.  Thus, by recursively
calling \Verb|process-instruction|, we can do more specialized processing on
this rewritten replacement (e.g., dispatching on \Verb|insn| again, which
applies \Verb|rewrite| once more).  If the instruction can't be simplified
further, we simply return it.  (Note that
%
\factor|[ X ] [ Y ] ?if|
%
is the same as
%
\factor|dup [ nip X ] [ drop Y ] if|.)

For instances of \Verb|foldable-insn| (i.e., \Verb|insn|s that can be
converted to useful expressions with \Verb|>expr|), we similarly invoke
\Verb|rewrite| recursively until no more rewriting occurs.  When that
happens, rather than just return the instruction, we invoke
\Verb|check-redundancy|---though only if the instruction defines exactly $1$
virtual register, which will be stored in a slot named \Verb|dst|.
\Verb|check-redundancy| checks if the expression being computed by the
instruction is already a key of the \Verb|exprs>vns| table.  If it is, the
instruction is redundant, and we call \Verb|redundant-instruction|;
otherwise, we call \Verb|useful-instruction|.  The former uses
\Verb|set-vn| to map the instruction's \Verb|dst| virtual register to the
same value number as the expression that was a key of \Verb|exprs>vns|.
Since value numbers are actually virtual registers, we may also use these two
integers the source and destination registers in a new \Verb|##copy|
instruction, which is then returned.  On the other hand,
\Verb|useful-instruction| saves the instruction's information in the
expression graph by setting the appropriate values in \Verb|vregs>vns|,
\Verb|exprs>vns|, and \Verb|vns>insns|.  Note that in definitions using the
syntax
%
\factor|:: word-name ( stack -- effect ) ... ;|,
%
the input values in the stack effect are actually named lexical variables, like
in most programming languages\todo{move this to primer?}.  Furthermore, the
first line
%
\factor|insn dst>> :> vn|
%
assigns the input instruction's destination register to the variable
\Verb|vn|, which is used later in the definition of
\Verb|useful-instruction|.

The \Verb|##copy| method of \Verb|process-instruction| cannot do anything
to simplify the instruction, but will set the value number of the destination
register to that of the source.  By calling \Verb|vreg>vn| on the source
register, we make sure to call \Verb|set-vn| between the destination and the
canonical value number of the source.

Finally, the \Verb|array| method is used for the purposes of recursion, in
the case that \Verb|rewrite| returns a sequence of replacement instructions.
The correct action is, of course, to descend into this new sequence of
instructions with \Verb|process-instruction|.

Underlying all of the redundancy elimination is the \Verb|rewrite| generic
word.  It has too many methods to look at the source code in-depth here, but
it's instructive to give an overview of the transformations.  These methods
actually make up the bulk of the \Verb|compiler.cfg.value-numbering| code.
They're spread across various sub-vocabularies.
\Verb|compiler.cfg.value-numbering.rewrite| defines the generic itself, along
with a handful of utilities.  The method for the most general instruction
class, \Verb|insn|, is defined to unconditionally return \factor|f|, meaning no
rewriting is performed by default.  That way, we need only define
\Verb|rewrite| methods for more specific instruction classes to get specialized
behavior.

\begin{sloppypar}
\Verb|compiler.cfg.value-numbering.alien| contains methods that simplify nodes
related to Factor's \gls{FFI}.  Most involve fusing together the results of
intermediate arithmetic.  The instructions that access raw memory (namely
\Verb|##load-memory|, \Verb|##load-memory-imm|, \Verb|##store-memory|, and
\Verb|##store-memory-imm|) tend to have inputs to perform address arithmetic.
Each has slots for a \Verb|base| register containing an address and a literal
\Verb|offset| from it.  But if \Verb|base| is defined by an \Verb|##add-imm|
instruction, we can just update the \Verb|offset|, incrementing it by the
literal operand of the \Verb|##add-imm|.  Then, \Verb|base| will just be
changed to the register operand of the \Verb|##add-imm|.  This removes the
memory instruction's need for the \Verb|##add-imm|, increasing the chances that
the latter will become dead code to be removed later.  Unlike the \Verb|-imm|
variants, \Verb|##load-memory| and \Verb|##store-memory| also take a
\Verb|displacement| register, which works like a non-immediate \Verb|offset|.
Therefore, \Verb|##add|s can be similarly fused into \Verb|##load-memory-imm|
and \Verb|##store-memory-imm| by transforming them into \Verb|##load-memory|
and \Verb|##store-memory| instructions with the \Verb|##add|'s operand as the
\Verb|displacement|.  A few other similar transformations are also done,
including rewrites for \Verb|##box-displaced-alien|s and
\Verb|##unbox-any-c-ptr|s.
\end{sloppypar}

\Verb|compiler.cfg.value-numbering.comparisons| defines methods for the
various branching and comparison instructions (which simply store booleans in
registers, rather than branching upon them).  The major optimizations performed
are as follows:
%
\begin{itemize}
%
\item If possible, instructions are converted to more specific forms.  For
example, non-immediate instructions (e.g., \Verb|##compare|) may be turned
into their \Verb|-imm| counterparts (e.g., \Verb|##compare-imm|)  if one of
their source registers corresponds to a literal value.
\Verb|##compare-integer-imm| is also converted to \Verb|##test| if the
architecture supports it.  This corresponds to a special instruction in x86
that performs a bitwise AND for its side effects on particular flags,
discarding the actual result.  This can be more efficient when using the AND
result as a boolean.
%
\item If both inputs to a comparison or branch are literals, we may
constant-fold the instruction.  In the case of comparisons, this means
converting it into a \Verb|##load-reference| of the proper boolean.  In
branches, this modifies the \gls{CFG} so that the path which isn't taken is
removed completely.
%
\item Like a novice programmer writing
%
\mint{java}|if (some_boolean != false) { ... }|
%
in Java, the compiler may generate redundant boolean comparisons that need
cleaning up.  That is, the intermediate boolean values are eliminated when the
result of a comparison is used by another comparison, collapsing the whole
thing into a single instruction.
%
\end{itemize}

\Verb|compiler.cfg.value-numbering.folding| defines some auxiliary words for
constant-folding arithmetic words.  Mainly, \Verb|unary-constant-fold| and
\Verb|binary-constant-fold| perform the actual operation on the one or two
constant inputs provided.  These words are used in
\Verb|compiler.cfg.value-numbering.math|, which predictably simplifies math
via standard rules.  Arithmetic identities are rewritten---conceptually, $x+0$
becomes just $x$, for instance.  If self-inverting instructions (namely
\Verb|##neg| for numerical negation and \Verb|##not| for boolean negation)
are called on registers that themselves correspond to the same instruction, we
can safely rewrite them into \Verb|##copy| instructions.  Non-immediate
instructions are converted to their \Verb|-imm| forms, if possible, and if
both operands are constant, the expression is folded.  The most interesting
math optimizations use the associative and distributive laws.
\term{Reassociation} conceptually converts $(x \otimes y) \otimes z$ into $x
\otimes (y \otimes z)$ when both $y$ and $z$ are constants and $\otimes$ is
associative.  So, for example,
%
\begin{center}
\Verb|##add-imm 1 X Y|\\
\Verb|##add-imm 2 1 Z|
\end{center}
%
\noindent is converted into just
%
\begin{center}
\Verb|##add-imm 2 X (Y+Z)|
\end{center}
%
\noindent where \Verb|X| is a virtual register, and \Verb|Y| and \Verb|Z|
are constants.  \term{Distribution} converts $(x \oplus y) \otimes z$ into $(x
\otimes z) \oplus (y \otimes z)$, where $y$ and $z$ are constants, $\oplus$
corresponds to addition or subtraction, and $\otimes$ to multiplication or left
bitwise shifts.  Therefore,
%
\begin{center}
\Verb|##add-imm 1 X Y|\\
\Verb|##mul-imm 2 1 Z|
\end{center}
%
\noindent is converted into
%
\begin{center}
  \begin{minipage}{0.2\linewidth}
    \begin{factorcode*}{gobble=6,frame=none}
      ##mul-imm 3 X Y
      ##add-imm 2 3 (Y*Z)
    \end{factorcode*}
  \end{minipage}
\end{center}
\noindent Notice that a new intermediate virtual register, \Verb|3|, had to
be created.  However, if the product of \Verb|Y| and \Verb|Z| can be
computed at compile-time and fits in an immediate operand, then we save cycles
by using \Verb|##mul-imm| on a smaller number.

\begin{sloppypar}
The last few methods of \Verb|rewrite| provide some obvious simplifications.
\Verb|compiler.cfg.value-numbering.simd| performs some limited constant-folding
for vector operations.  \Verb|compiler.cfg.value-numbering.slots| propagates
\Verb|##add-imm| address calculation to \Verb|##slot|, \Verb|##set-slot|, and
\Verb|##write-barrier| instructions in a manner similar to
\Verb|compiler.cfg.value-numbering.alien|.  Finally,
\Verb|compiler.cfg.value-numbering.misc| provides a single method to rewrite
\Verb|##replace| into \Verb|##replace-imm| if possible.
\end{sloppypar}

\inputfig{lvn}

To finish the discussion of local value numbering and Factor's particular
implementation, we'll examine the example from \vref{fig:value-numbering} in
depth.  For convenience, the before/after snapshot of the \gls{CFG} is
reproduced in \vref{fig:lvn}.

\Verb|value-numbering-step| begins at block $1$, where
\Verb|process-instruction| is \factor|map|ped across the instructions.
%
\Verb|##inc-d 3|
%
does not have a \Verb|rewrite| method, so remains untouched; it is also not a
\Verb|foldable-insn|, so it is simply returned.  While
%
\Verb|##load-integer 21 0|
%
doesn't have a \Verb|rewrite| method, it is a \Verb|foldable-insn|, so
\Verb|process-instruction| calls \Verb|check-redundancy|.  At this point,
the expression graph is empty.  Calling \Verb|>expr| converts this
instruction into an \Verb|integer-expr| object representing \Verb|0|.
\Verb|useful-instruction| leaves the tables as follows:
%
  \begin{factorcode}
    ! vregs>vns
    H{ { 21 21 } }

    ! exprs>vns
    H{ { T{ integer-expr { value 0 } } 21 } }

    ! vns>insns
    H{
        { 21 T{ ##load-integer { dst 21 } { val 0 } { insn# 1 } } }
    }
  \end{factorcode}
%
\noindent The next instruction in block $1$,
%
\Verb|##load-integer 22 100|,
%
behaves similarly, leaving:
%
  \begin{factorcode}
    ! vregs>vns
    H{ { 21 21 } { 22 22 } }

    ! exprs>vns
    H{
        { T{ integer-expr { value 0 } } 21 }
        { T{ integer-expr { value 100 } } 22 }
    }

    ! vns>insns
    H{
        { 21 T{ ##load-integer { dst 21 } { val 0 } { insn# 1 } } }
        {
            22
            T{ ##load-integer { dst 22 } { val 100 } { insn# 2 } }
        }
    }
  \end{factorcode}
%
\noindent The following instruction is
%
\Verb|##load-integer 23 0|.
%
In calling \Verb|check-redundancy|, we discover that the integer expression
for \Verb|0| is already in \Verb|exprs>vns|, so this is turned into a
\Verb|##copy|, and the value number is noted.  The remaining instructions in
block $1$ (aside from \Verb|##branch|) are all instances of \Verb|##copy|.
\Verb|process-instruction| thus only sets their value numbers in the
\Verb|vregs>vns| table, leaving them with the following at the end of block
$1$:
%
  \begin{factorcode}
    ! vregs>vns
    H{
        { 21 21 }
        { 22 22 }
        { 23 21 }
        { 24 22 }
        { 25 21 }
        { 26 22 }
        { 27 21 }
    }

    ! exprs>vns
    H{
        { T{ integer-expr { value 0 } } 21 }
        { T{ integer-expr { value 100 } } 22 }
    }

    ! vns>insns
    H{
        { 21 T{ ##load-integer { dst 21 } { val 0 } { insn# 1 } } }
        {
            22
            T{ ##load-integer { dst 22 } { val 100 } { insn# 2 } }
        }
    }
  \end{factorcode}

Next, block $2$ in \vref{fig:lvn} is processed.  The tables are all reset, so
even though block $1$ happens to dominate block $2$, none of its definitions
are known to \Verb|value-numbering|.  The \Verb|##phi|s are ignored, as no
important methods dispatch upon them.  In trying to rewrite the
\Verb|##compare-integer|, we call \Verb|vreg>vn| on the operands.  Since they
aren't in the \Verb|vregs>vns| table yet, they are assumed to be unique values.
This assumption is pessimistic---we'd rather the values be the same, so we can
remove redundancy.  It happens to be correct here, though, as \Verb|26|
corresponds to the integer \Verb|100|, while \Verb|30| is an induction variable
of the loop.  However, \Verb|##compare-integer| cannot be rewritten into an
immediate form, since our focus is local to the basic block, so we don't know
that \Verb|26| has the value \Verb|100|.  The \Verb|##copy| instructions are
processed as usual, and
%
\Verb|##compare-imm-branch 32 f cc/=|
%
is rewritten into a \Verb|##compare-integer-branch|, as the virtual register
\Verb|32| has the same value (through the copies) as the
\Verb|##compare-integer| result.  This is a case of simplifying the 
%
\mint{java}|if (some_boolean != false) { ... }|
%
pattern, and the definition of the register \Verb|31| becomes dead code after
\Verb|rewrite| finishes with this last instruction.  The expression graph is
populated thus by the end:
%
  \begin{factorcode}
    ! vregs>vns
    H{
        { 32 31 }
        { 33 26 }
        { 34 31 }
        { 26 26 }
        { 30 30 }
        { 31 31 }
    }

    ! exprs>vns
    H{ { { ##compare-integer 30 26 cc< } 31 } }

    ! vns>insns
    H{
        {
            31
            T{ ##compare-integer
                { dst 31 }
                { src1 30 }
                { src2 22 }
                { cc cc< }
                { temp 9 }
                { insn# 2 }
            }
        }
    }
  \end{factorcode}

Once again, the tables are reset and we proceed to block $3$.  The first
instruction,
%
\Verb|##load-integer 35 1|,
%
is entered into the expression graph.  Since \Verb|35| is an operand of
%
\Verb|##add 36 29 35|,
%
\Verb|rewrite| changes this instruction into an \Verb|##add-imm|, as we
know the constant value of the operand.  The next \Verb|##load-integer| gets
turned into a \Verb|##copy|, like in block $1$, and the next \Verb|##add|
is similarly changed to \Verb|##add-imm|.  The copies do little but set more
value numbers.  As \Verb|process-instruction| calls \Verb|vreg>vn| on their
sources, we'll insert entries into \Verb|vregs>vns| for those defined outside
of the block, like \Verb|26|.  This leaves us with the following tables:
%
  \begin{factorcode}
    ! vregs>vns
    H{
        { 35 35 }
        { 36 36 }
        { 37 35 }
        { 38 38 }
        { 39 30 }
        { 40 26 }
        { 41 36 }
        { 26 26 }
        { 42 38 }
        { 29 29 }
        { 30 30 }
    }

    ! exprs>vns
    H{
        { { ##add-imm 30 1 } 38 }
        { { ##add-imm 29 1 } 36 }
        { T{ integer-expr { value 1 } } 35 }
    }

    ! vns>insns
    H{
        { 36 T{ ##add-imm { dst 36 } { src1 29 } { src2 1 } } }
        { 38 T{ ##add-imm { dst 38 } { src1 30 } { src2 1 } } }
        { 35 T{ ##load-integer { dst 35 } { val 1 } { insn# 0 } } }
    }
  \end{factorcode}
%
\noindent The fourth invocation of \Verb|value-numbering-step| does not do
anything interesting, as the \Verb|##replace| cannot be changed into a
\Verb|##replace-imm|.

\inputfig{finalize-lvn}

In summary, we managed to replace redundancies within basic blocks online by
maintaining some simple hash tables.  After copy propagation and dead code
elimination, the \gls{CFG} gets finalized to the one shown in
\vref{fig:finalize-lvn}.  Because the value numbering algorithm was local, the
\Verb|##compare-integer-branch| in block $2$ could not be simplified to a
\Verb|##compare-integer-imm-branch|, and we instead have to waste a register
on the integer \Verb|100|.  But it's important to note that even considering
a topological ordering of the \gls{CFG} wouldn't have worked, as we'd have to
ignore back-edges.  The \Verb|##phi|s that used to be in block $2$ had inputs
that flowed along the back-edge, and our pessimistic assumption would have to
classify these values as distinct.  One is for the counter introduced by
\Verb|times|, and the other is from the top value of the stack being
incremented by \Verb|fixnum+fast|.  In this case, however, these induction
variables are actually equal: both start at \Verb|0| and are incremented by
\Verb|1| on each loop.  In terms of the \gls{CFG} in \vref{fig:finalize-lvn},
the \Verb|EAX| and \Verb|EDX| registers are equivalent.  Yet the
combination of the pessimism and locality of the algorithm keep us from
discovering this.

\subsection{Global Value Numbering}\label{sec:vn:global}

Answering the challenges of Cocke\todo{cite-like}, AWZ~\todo{cite-like}
described what would be the de facto value numbering algorithm for several
years, and rightly so.  It was a properly \term{global} value numbering
algorithm, working across an entire \gls{CFG} instead of on single basic
blocks.  Their paper was important in another very relevant way: it is the
first published reference to SSA form\todo{cite VanDrunen}, including an
algorithm for its construction.

Though we could try to extend the scope of Factor's local value numbering, it
is still inherently pessimistic.  The algorithm of AWZ\todo{cite-like}, which
is commonly referred to simply as AWZ\todo{gls?}, uses a modification of
Hopcroft's~\todo{cite-like} minimization algorithm for finite state automata.
It works on an \term{optimistic} assumption by first assuming every value has
the same value number, then trying to prove that values are actually different.
It does this by treating value numbers as \term{congruence classes} that
partition the set of virtual registers.  If two values are in the same class,
then they are congruent, where congruence is defined as in \cref{sec:vn:local}.

Such a partition is not unique, in general.  For instance, a trivial one places
each value in its own congruence class.  So, we look for the \term{maximal
fixed point}---the solution that has the most congruent values and therefore
the fewest congruence classes.  We must start with a congruence class for each
operation so that, say, all values computed by \factor|##add|s are grouped
together, those computed by \factor|##mul|s are in the same class, etc.  We
must then iteratively look at our collection of classes, separating them when
we discover incongruent values.  For an \gls{SSA} variable in class $P$, we
look at its defining expression.  If an operand at position $i$ belongs to
class $Q$, then the $i^\text{th}$ operand of every other value in $P$ should
also be in $Q$.  Otherwise, $P$ must be \term{split} by removing those
variables whose $i^\text{th}$ operands are not in class $Q$ and placing them in
a new congruence class.  We keep splitting classes until the partitioning
stabilizes.

The optimistic assumption may seem dangerous.  Is it possible that we're
``overoptimistic''?  That two values assumed to be congruent and not proven
incongruent might actually be inequivalent when the program is run?  The
AWZ~\todo{gls?} paper dedicates a section to proving that two congruent
variables are equivalent at a point $p$ in the program if their definitions
dominate $p$.  The proof is a bit quirky, but reasonable.  They develop a
dynamic notion of dominance in a running program which implies static dominance
in the code, then show that congruence implies runtime equality (though
equivalence does not imply congruence).

AWZ\todo{gls?}~ made the need for \gls{GVN} algorithms apparent.  However,
finite state automata minimization makes for a more complicated algorithm than
hash-based value numbering.  A na\"{i}ve implementation can be quadratic,
although careful data structure and procedure design can make it $O(n\log n)$.
Furthermore, it's resistant to the same improvements we easily added to the
local value numbering.  To even consider the commutativity of operations
requires changes in operand position tracking and splitting---the heart of the
algorithm.  It is generally limited by what the programmer writes down: deeper
congruences due to, say, algebraic identities can't be discovered.

In fact, by performing an optimization that uses the \gls{GVN} information,
more \gls{GVN} congruences may arise.  If we can somehow perform the two
analyses simultaneously, they'll produce better results.  This generalizes to
interdependent compiler optimizations at large, as elucidated in
Click's\todo{cite-like}~ dissertation, which describes a method for formalizing
and combining separate optimizations that make optimistic assumptions (whatever
they happen to be for each particular analysis).  He uses this to merge
\gls{GVN} with \term{conditional constant propagation}, which itself is a
combination of constant propagation and unreachable code elimination (pretty
much like the \factor|propagate| pass from \cref{sec:compiler:tree}).
Furthermore, \gls{GVN} is extended to handle algebraic identities, propagate
constants, and fold redundant $\phi$s.  Unfortunately, the straightforward
algorithm for this is $O(n^2)$, while the $O(n\log n)$ version presented is not
just complicated, but can also miss some congruences between
$\phi$-functions\todo{cite Click, Simpson}.

Hot on the heels of this work, Simpson's\todo{cite-like}~ dissertation provides
probably the most exhaustive treatment of \gls{GVN} algorithms.  He presents
several extensions, such as incorporating hash-based local value numbering into
\gls{SSA} construction, handling commutativity in AWZ\todo{gls?}~ \gls{GVN},
and performing redundant store elimination.  He builds off of the two classical
algorithms independently, which underlines their inherent differences and
limitations.  In general, hash-based value numbering is easy to extend without
greatly impacting the runtime complexity, as is the case in Factor's
implementation.

Drawing from this experience, Simpson's hallmark algorithm combines the best of
both worlds by taking the hash-based algorithm which is easy to understand,
implement, and extend, and making it global, so it identifies more congruences.
Dubbed the ``\gls{RPO} algorithm'', it simply applies hash-based value
numbering iteratively over the \gls{CFG} until we reach the same fixed point
computed by AWZ\todo{gls?}.  (The fact that it computes the \emph{same} fixed
point is proven fairly straightforwardly in the dissertation.)  It could
technically traverse the \gls{CFG} in any topological order, but Simpson
defaults to reverse postorder.

Because it is based off the hashing algorithm, we get the benefits essentially
for free.  The same simplifications can be performed, but with the added
knowledge of global congruences.  Since the majority of Factor's value
numbering code is dedicated to the \factor|rewrite| generic, it makes sense to
reuse as much of that code as possible.  Therefore, to convert Factor's local
algorithm to a global one, I modified the existing code to use the \gls{RPO}
algorithm.

\inputlst{gvn-graph}

The most fundamental change is to the expression graph.  Referring to
\vref{lst:gvn-graph}, we see most of the same code as in
\vref{lst:value-numbering-graph}, with changes indicated by arrows
($\longrightarrow$).  Two more global variables have been added, namely
\factor|changed?| and \factor|final-iteration?|.  The former is what we use to
guide the fixed-point iteration.  As long as value numbers are changing, we
keep iterating.  An important side effect of this is that we can no longer
perform \factor|rewrite| online, since the transformations we make aren't
guaranteed to be sound on any iteration except the final one.  This makes the
\gls{RPO} algorithm work \term{offline}, first discovering redundancies, then
eliminating them in a separate pass.  When this elimination pass starts, we'll
set \factor|final-iteration?| to \factor|t|.

A key change is in the \factor|vreg>vn| word, which now makes an optimistic
assumption about previously unseen values.  Given a new virtual register that
wasn't in the \factor|vregs>vns| table, the old version would map the register
to itself, making the value its own canonical representative.  However, if this
version tries to look up a key that does not exist in the hash table, it will
simply return \factor|f| (which Factor will do by default with the \factor|at|
word).  Therefore, every value in the \gls{CFG} starts off with the same value
``number'', \factor|f|.  By the end of the \gls{GVN} pass, there should be no
value left that hasn't been put in the \factor|vregs>vns| table, as we'll have
processed every definition.

To keep track of whether \factor|vregs>vns| changes, we simply need to alter
\factor|set-vn|.  Here, we use \factor|maybe-set-at|, a utility from the
\factor|assocs| vocabulary.  This works like \factor|set-at|, establishing a
mapping in the hash table.  In addition, it returns a boolean indicating
change: if a new key has been added to the table, we return \factor|t|.
Otherwise, we return \factor|t| only in the case where an old key is mapped to
a new value.  If an old key is mapped to the same value that's already in the
table, \factor|maybe-set-at| returns \factor|f|.  Therefore, when
\factor|vregs>vns| does change, we set \factor|changed?| to \factor|t| (which
is what the \factor|on| word does).

Finally, we define a new utility word, \factor|clear-exprs|, which resets the
\factor|exprs>vns| and \factor|vns>insns| tables.  Unlike the local value
numbering phase, we don't reset the entire expression graph.  Instead, we make
a pass over the whole \gls{CFG} at a time.  The only reason optimism works is
that we keep trying to disprove our foolhardy assumptions.  Really,
\factor|vregs>vns| establishes congruence classes of value numbers.  At first,
every value belongs in one class, \factor|f|.  We make a pass over the
\gls{CFG} to disprove whatever we can about this.  If we've introduced new
congruence classes (new values in the \factor|vregs>vns| hash), we do another
iteration.  But each time, we use the congruence classes discovered from the
previous iteration.  At the start of each new pass, the expressions and
instructions in \factor|exprs>vns| and \factor|vns>insns| are
invalidated---their results are based on old information.  So, these are erased
on each iteration.  Much like AWZ\todo{gls?}, we keep splitting classes until
they can't be split anymore.

\inputlst{gvn-step}

This logic is captured in \vref{lst:gvn-step}.  Rather than reset the tables
when we start processing each basic block in \factor|value-numbering-step| like
before, we call \factor|clear-exprs| on each iteration over the \gls{CFG} in
\factor|value-numbering-iteration|.  Note that \factor|value-numbering-step| no
longer returns the changed instructions, as we aren't replacing them online.
\factor|value-numbering-iteration| uses \factor|simple-analysis| instead of
\factor|simple-optimization|, which only expects global state to change---no
instructions are updated in the block.  Much to our advantage,
\factor|simple-analysis| already traverses the \gls{CFG} in \acrlong{RPO}, so
we needn't worry about traversal order.  The top-level word
\factor|determine-value-numbers| ties this all together by calling
\factor|value-numbering-iteration| until we can get through it with
\factor|changed?| remaining false.

\inputlst{gvn-simplify}
\inputlst{gvn-value-number}

Note that the work of \factor|value-numbering-step| is further divided into two
words, \factor|simplify| and \factor|value-number|.  These combine to do much
the same work as \factor|process-instruction| in
\vref{lst:process-instruction}.  \factor|simplify| makes the repeated calls to
\factor|rewrite| until the instruction cannot be simplified further.  Its
definition is in \vref{lst:gvn-simplify}.  We then pass the simplified
instruction to \factor|value-number|, which is defined in
\vref{lst:gvn-value-number}.  This also has a similar structure to
\factor|process-instruction|.  The main difference is that instructions are no
longer returned (again, they aren't altered in place).  So, the \factor|array|
method uses \factor|each| instead of \factor|map| to recurse into the results
of \factor|rewrite|.

A subtle change is necessary with the \factor|alien-call-insn| and
\factor|##callback-inputs| methods.  Whereas \factor|process-instruction|
merely skipped over certain instructions that could not be rewritten, here we
don't have that luxury.  We need to be careful to \factor|set-vn| every virtual
register that gets defined by any instruction.  While making a pessimistic
assumption, it didn't matter if we did this: any unseen value would be presumed
important by \factor|vreg>vn|.  However, with the optimistic assumption,
\factor|vreg>vn| will give the impression that unseen values are all the same
by returning \factor|f|.  Therefore, we simply record the virtual registers
defined in instructions that may define one or more of them.  Specifically,
\factor|alien-call-insn| and \factor|##callback-inputs| are classes that
correspond to \gls{FFI} instructions.

The \factor|##copy| method uses \factor|set-vn| the same way as before.
\factor|redundant-instruction|, \factor|useful-instruction|, and
\factor|check-redundancy| are also largely the same.  These have just been
tweaked to not return instructions.

\inputlst{phi-expr}

The \factor|##phi| method in \vref{lst:gvn-value-number} represents a major
change. Before, \factor|##phi|s were left uninterpreted.  Congruences between
induction variables that flowed along back-edges would not be identifiable.
But now, by checking for redundant \factor|##phi|s, we may reduce them to
copies.  Each \factor|##phi| object has an \factor|inputs| slot, which is a
hash table from basic block to the virtual register that flows from that block.
Thus, there is one input for each predecessor.  The \factor|values| of the hash
will be the virtual registers that might be selected for the \factor|dst|
value.  We look up the value numbers of these, removing all instances of
\factor|f| with the \factor|sift| word.  If all of the inputs are congruent, we
can call \factor|redundant-instruction|, setting the value number of the
\factor|##phi|'s \factor|dst| to the value number of its first input (without
loss of generality).  The \factor|all-equal?| word will return \factor|t| if
the sequence is empty (as it's vacuously true), so we must make sure not to
call \factor|first| on the sequence, since this will be a runtime error.  If
the sequence is empty, we needn't note the redundancy, as the \factor|##phi|'s
\factor|dst| will already have the optimistic value number \factor|f| anyway.
Otherwise, we call \factor|check-redundancy|.  The purpose of this is to
identify \factor|##phi|s that are equal to each other.  Even if its inputs are
incongruent, a \factor|##phi| might still represent a copy of another induction
variable.  So that \factor|check-redundancy| works, we also define a
\factor|>expr| method in \factor|compiler.cfg.gvn.expressions|, as seen in
\vref{lst:phi-expr}.  Here, the expression is an array consisting of the
\factor|##phi| class word, the current basic block's number, and the inputs'
value numbers.  We include the basic block number because only \factor|##phi|s
within the same block can be considered equivalent to each other.

The final method in \vref{lst:gvn-value-number} defines the default behavior
for \factor|value-number|, which calls \factor|check-redundancy| on the
simplified instruction if it defines a single virtual register.  Note that we
separate the \factor|alien-call-insn| and \factor|##callback-inputs| logic from
this, since they happen to define a variable number of registers.  If
particular instances define only one register, we still don't want to call
\factor|check-redundancy| on them, since they don't have a \factor|dst| slot.
To avoid calling \factor|dst>>| and triggering an error in
\factor|useful-instruction|, we needed separate methods for the \gls{FFI}
classes.

\inputfig{gvn}

With these changes, we can globally identify value numbers, including
equivalences that arise from simplifying instructions (even though no
replacements are actually done yet).  To illustrate this, consider again the
example
%
\factor|0 100 [ 1 fixnum+fast ] times|,
%
reproduced in \vref{fig:gvn}.  As the expression graph changes frequently in
this new algorithm, instead of showing the literal hash tables we'll use a
shorthand notation.  Virtual registers will be integers, and to avoid confusion
value numbers will be written in brackets, like \vn{n}.  Then, we'll show
\factor|vreg>vn| mappings with the notation $n\to\vn{n}$, where $n$ is the
register and \vn{n} is the value number.  If there is a corresponding
expression in \factor|exprs>vns|, it will be denoted after the mapping, like
$n\to\vn{n}~(\textit{expression})$.  With the expressions, the instructions in
\factor|vns>insns| are a bit redundant for understanding the value numbering
process, so they will be elided.  Any mappings to \factor|f| will be elided, as
they're understood to be implicit when a key is absent.

\todo[inline]{Might make separate figures of each block, for easier reference}

\factor|determine-value-numbers| starts the first iteration, which of course
starts at basic block $1$.  \factor|##inc-d| is a no-op, but the first two
\factor|##load-integer|s are established as useful instructions.
%
\factor|##load-integer 23 0|
%
is recognized as redundant, since at this point we know that \factor|21| has
the value \factor|0|.  The \factor|##copy| instructions all pile on value
number mappings, leaving us with the following:
%
\begin{align*}
  21 &\to \vn{21} \quad (0)  \\
  22 &\to \vn{22} \quad (100)\\
  23 &\to \vn{21}            \\
  24 &\to \vn{22}            \\
  25 &\to \vn{21}            \\
  26 &\to \vn{22}            \\
  27 &\to \vn{21}
\end{align*}

At iteration $1$, basic block $2$, the first \factor|##phi| has inputs
\factor|25| (from block $1$) and \factor|41| (from block $3$).  The former has
the value number \vn{21}, while the latter is still at \factor|f|.  We treat
this value number much like a $\top$ element, unifying it with the other input
to give us the assumption that \factor|29| will be a copy of \factor|25|.
Thus, it gets the same value number.  A similar choice happens for the second
\factor|##phi|.  The instruction 
%
\factor|##compare-integer 31 30 26 cc< 9|
%
is an interesting case.  Due to our optimistic assumptions thus far, we believe
\factor|30| is carrying the value \factor|0|, and that \factor|26| is set to
\factor|100|.  Thus, this instruction gets constant-folded by \factor|simplify|
into
%
\factor|##load-reference 31 t|.
%
The \gls{CFG} isn't changed, but the expression graph reflects this belief.
Later, this assumption will be invalidated.  The following copies are processed
as usual, with the distinct difference here that 
%
\factor|##copy 33 26 any-rep|
%
has the global knowledge of the value number of \factor|26|.  Because the
\factor|##compare-integer| was constant-folded, so is the
\factor|##compare-imm-branch|---and to the same value, no less.  This leaves us
with:
%
\begin{align*}
  21 &\to \vn{21} \quad (0)                 \\
  22 &\to \vn{22} \quad (100)               \\
  23 &\to \vn{21}                           \\
  24 &\to \vn{22}                           \\
  25 &\to \vn{21}                           \\
  26 &\to \vn{22}                           \\
  27 &\to \vn{21}                           \\
  29 &\to \vn{21}                           \\
  30 &\to \vn{21}                           \\
  31 &\to \vn{31} \quad (\text{\factor|t|}) \\
  32 &\to \vn{31}                           \\
  33 &\to \vn{22}                           \\
  34 &\to \vn{31}
\end{align*}

Block $3$ of iteration $1$ gives the \factor|##load-integer|s' destinations the
same value number, corresponding to the integer $1$.  Because optimism makes
the algorithm think that \factor|29| and \factor|30| correspond to the integer
$0$, the \factor|##add|s are constant-folded.  This leaves us with:
%
\begin{align*}
  21 &\to \vn{21} \quad (0)                 \\
  22 &\to \vn{22} \quad (100)               \\
  23 &\to \vn{21}                           \\
  24 &\to \vn{22}                           \\
  25 &\to \vn{21}                           \\
  26 &\to \vn{22}                           \\
  27 &\to \vn{21}                           \\
  29 &\to \vn{21}                           \\
  30 &\to \vn{21}                           \\
  31 &\to \vn{31} \quad (\text{\factor|t|}) \\
  32 &\to \vn{31}                           \\
  33 &\to \vn{22}                           \\
  34 &\to \vn{31}                           \\
  35 &\to \vn{35} \quad (1)                 \\
  36 &\to \vn{35}                           \\
  37 &\to \vn{35}                           \\
  38 &\to \vn{35}                           \\
  39 &\to \vn{21}                           \\
  40 &\to \vn{22}                           \\
  41 &\to \vn{35}                           \\
  42 &\to \vn{35}
\end{align*}

While block $4$ is visited in each iteration, it doesn't define any registers,
so doesn't affect the state of value numbering.  Therefore, the above is the
state left at the end of iteration $1$.

Since \factor|vregs>vns| clearly changed, iteration $2$ commences by clearing
the expressions, though the value numbers remain.  Block $1$ doesn't change
from iteration $1$, giving us:
%
\begin{align*}
  21 &\to \vn{21} \quad (0)                 \\
  22 &\to \vn{22} \quad (100)               \\
  23 &\to \vn{21}                           \\
  24 &\to \vn{22}                           \\
  25 &\to \vn{21}                           \\
  26 &\to \vn{22}                           \\
  27 &\to \vn{21}                           \\
  29 &\to \vn{21}                           \\
  30 &\to \vn{21}                           \\
  31 &\to \vn{31}                           \\
  32 &\to \vn{31}                           \\
  33 &\to \vn{22}                           \\
  34 &\to \vn{31}                           \\
  35 &\to \vn{35}                           \\
  36 &\to \vn{35}                           \\
  37 &\to \vn{35}                           \\
  38 &\to \vn{35}                           \\
  39 &\to \vn{21}                           \\
  40 &\to \vn{22}                           \\
  41 &\to \vn{35}                           \\
  42 &\to \vn{35}
\end{align*}

Now that we're in iteration $2$, the inputs to the \factor|##phi|s of block $2$
have been processed once before.  For instance, we still believe that
\factor|25| corresponds to the integer $0$ (which is incidentally correct), but
now that \factor|41| has the value number \vn{35}, we think it corresponds to
the integer $1$.  While this is incorrect, it does break the congruence between
the inputs, making the first \factor|##phi| a useful instruction.  The second
\factor|##phi|, however, still looks like a copy of the first.  Even so, this
is sufficiently different that the following \factor|##compare-integer| cannot
be constant-folded like before.  However, it can still be converted to a
\factor|##compare-integer-imm|, as one of its operands corresponds to an
integer.  The redundant \factor|##compare-imm-branch| gets rewritten to the
same expression as the \factor|##compare-integer|, so winds up getting the same
value number.  This gives us:
%
\begin{align*}
  21 &\to \vn{21} \quad (0)                                                \\
  22 &\to \vn{22} \quad (100)                                              \\
  23 &\to \vn{21}                                                          \\
  24 &\to \vn{22}                                                          \\
  25 &\to \vn{21}                                                          \\
  26 &\to \vn{22}                                                          \\
  27 &\to \vn{21}                                                          \\
  29 &\to \vn{29} \quad (\text{\factor|##phi 2 21 35|})                    \\
  30 &\to \vn{29}                                                          \\
  31 &\to \vn{31} \quad (\text{\factor|##compare-integer-imm 29 100 cc<|}) \\
  32 &\to \vn{31}                                                          \\
  33 &\to \vn{22}                                                          \\
  34 &\to \vn{31}                                                          \\
  35 &\to \vn{35}                                                          \\
  36 &\to \vn{35}                                                          \\
  37 &\to \vn{35}                                                          \\
  38 &\to \vn{35}                                                          \\
  39 &\to \vn{21}                                                          \\
  40 &\to \vn{22}                                                          \\
  41 &\to \vn{35}                                                          \\
  42 &\to \vn{35}
\end{align*}

Block $3$ of iteration $2$ also changes, since the \factor|##add|s can't be
constant-folded like before due to our new discovery about the \factor|##phi|s.
However, the first one can still be converted to an \factor|##add-imm|, and the
second is marked the same as the first.  This leaves the following value
numbers:
%
\begin{align*}
  21 &\to \vn{21} \quad (0)                                                \\
  22 &\to \vn{22} \quad (100)                                              \\
  23 &\to \vn{21}                                                          \\
  24 &\to \vn{22}                                                          \\
  25 &\to \vn{21}                                                          \\
  26 &\to \vn{22}                                                          \\
  27 &\to \vn{21}                                                          \\
  29 &\to \vn{29} \quad (\text{\factor|##phi 2 21 35|})                    \\
  30 &\to \vn{29}                                                          \\
  31 &\to \vn{31} \quad (\text{\factor|##compare-integer-imm 29 100 cc<|}) \\
  32 &\to \vn{31}                                                          \\
  33 &\to \vn{22}                                                          \\
  34 &\to \vn{31}                                                          \\
  35 &\to \vn{35} \quad (1)                                                \\
  36 &\to \vn{36} \quad (\text{\factor|##add-imm 29 1|})                   \\
  37 &\to \vn{35}                                                          \\
  38 &\to \vn{36}                                                          \\
  39 &\to \vn{29}                                                          \\
  40 &\to \vn{22}                                                          \\
  41 &\to \vn{36}                                                          \\
  42 &\to \vn{36}
\end{align*}

Since the value numbers changed, we start iteration $3$.  The expressions are
cleared, and block $1$ once again does not change anything.  The first
\factor|##phi| in block $2$ still gets classified as useful, so no value
numbers change.  The major difference, though, is that the previous iteration's
value numbers for registers in block $3$ update the expression we have for the
\factor|##phi|.  Whereas before we thought it was choosing between \vn{21} (the
integer $0$) and \vn{35} (the integer $1$), the \factor|##add| wasn't
constant-folded in the previous iteration.  Therefore, the virtual register
\factor|41| now corresponds to the result of the \factor|##add| with the value
number \vn{36}.  We still can't disprove that the second \factor|##phi| is
different (because it, in fact, isn't).  So, we're left with the following
after iteration $3$ finishes with block $2$:
%
\begin{align*}
  21 &\to \vn{21} \quad (0)                                                \\
  22 &\to \vn{22} \quad (100)                                              \\
  23 &\to \vn{21}                                                          \\
  24 &\to \vn{22}                                                          \\
  25 &\to \vn{21}                                                          \\
  26 &\to \vn{22}                                                          \\
  27 &\to \vn{21}                                                          \\
  29 &\to \vn{29} \quad (\text{\factor|##phi 2 21 36|})                    \\
  30 &\to \vn{29}                                                          \\
  31 &\to \vn{31} \quad (\text{\factor|##compare-integer-imm 29 100 cc<|}) \\
  32 &\to \vn{31}                                                          \\
  33 &\to \vn{22}                                                          \\
  34 &\to \vn{31}                                                          \\
  35 &\to \vn{35}                                                          \\
  36 &\to \vn{36}                                                          \\
  37 &\to \vn{35}                                                          \\
  38 &\to \vn{36}                                                          \\
  39 &\to \vn{29}                                                          \\
  40 &\to \vn{22}                                                          \\
  41 &\to \vn{36}                                                          \\
  42 &\to \vn{36}
\end{align*}

Blocks $3$ and $4$ do not produce any more changes, so \gls{GVN} has stabilized
after $3$ iterations, with our final congruence classes being:
%
\begin{align*}
  \vn{21} &= \{21, 23, 25, 27\}     \\
  \vn{22} &= \{22, 24, 26, 33, 40\} \\
  \vn{29} &= \{29, 30, 39\}         \\
  \vn{31} &= \{31, 32, 34\}         \\
  \vn{35} &= \{35, 37\}             \\
  \vn{36} &= \{36, 38, 41, 42\}
\end{align*}

\todo[inline]{teletype the numbers in the align*s, I guess}

\subsection{Redundancy Elimination}\label{sec:vn:avail}

Now that we've identified congruences across the entire \gls{CFG}, we must
eliminate any redundancies found.  Since value numbering is now offline, this
entails another pass.  However, replacing instructions is more subtle with
global value numbers than it is with local ones.  Because values come from all
over the \gls{CFG}, we must consider if a definition is \term{available} at the
point where we want to use it.  

\inputfig{not-avail}
\inputfig{is-avail}

\Vref{fig:not-avail,fig:is-avail} show the difference.  In the former, we can
see the \gls{CFG} before value numbering for the code
%
\factor|[ 10 ] [ 20 ] if 10 20 30|.
%
The two extra integers being pushed at the end are there to avoid branch
splitting (see \vref{sec:compiler:cfg}).  In block $4$, there's a
%
\Verb|##load-integer 27 10|,
%
which loads the value \factor|10|.  In globally numbering values, we associate
the
%
\Verb|##load-integer 22 10|
%
in block $2$ with the value \factor|10| first, making it the canonical
representative.  However, we can't replace the instruction in block $4$ with
%
\Verb|##copy 27 22|,
%
because control flow doesn't necessarily go through block $2$, so the virtual
register \factor|22| might not even be defined.  However, in
\vref{fig:is-avail}, we see the \gls{CFG} for the code
%
\factor|10 swap [ 10 ] [ 20 ] if 10 20 30|.
%
In this case, the first definition of the value \factor|10| comes from block
$1$, which dominates block $4$.  So, the definition is available, and we can
replace the \Verb|##load-integer| in block $4$ with a \Verb|##copy|.

There are several ways to decide if we can use a definition at a certain point.
For instance, we could use dominator information, so that if a definition in a
basic block $B$ can be used by any basic block dominated by $B$\todo{cite
Simpson}.  However, here we'll use a data flow analysis called \term{available
expression analysis}, since it was readily implemented.  Mercifully, Factor has
a vocabulary that automatically defines data flow analyses with little more
than a single line of code.

\inputlst{avail}

\Vref{lst:avail} shows the vocabulary that defines the available expression
analysis.  It is a forward analysis\todo{cite?}~ based on the flow equations
below:
\begin{align*}
  \text{\factor|avail-in|}_i &=
    \begin{cases}
      \varnothing
        & \text{if $i=0$} \\
      \bigcap_{j\in\mathrm{pred}(i)}\text{\factor|avail-out|}_j
        & \text{if $i>0$}
    \end{cases} \\
  \text{\factor|avail-out|}_i &= \text{\factor|avail-in|}_i
                                 \cup 
                                 \text{\factor|defined|}_i
\end{align*}
%
\noindent Here, the subscripts indicate the basic block number.
$\text{\factor|defined|}_i$ denotes the result of the \factor|defined| word
from \vref{lst:avail}.  This returns the set of virtual registers defined in a
basic block.

\section{Results}\label{sec:vn:results}

The goal of improving the optimization in Factor is, of course, to reduce the
average running time of programs, and to do so without changing their
semantics.  Short of formal verification, the latter requirement makes it
necessary to thoroughly test any code that gets compiled with the new pass
enabled.  To this end, we'll employ Factor's extensive unit test coverage.
While some vocabularies will have more tests than others, the total number of
unit tests is quite large.  By compiling each vocabulary and running their
tests, we're indirectly testing the compiler: if tests that used to pass no
longer do, then the new pass is changing the semantics of the code somehow.
Though passing all tests does not guarantee the algorithm is correct, it does
let us know that no known regressions have been introduced.  Happily, with the
new \Verb|value-numbering| phase enabled, all the same tests pass as before in
a call to \factor|test-all| from a freshly bootstrapped image.

The efficacy of the changes, on the other hand, must be measured relative to
old benchmarks.  Again, Factor has its bases covered, with a suite of $80$
benchmarks run by the \Verb|benchmark| vocabulary.  Each benchmark is run $5$
times, and the garbage collector is run before each iteration.  The minimum
time from these runs is then used as the benchmark result.  The informal data
below comes from two separate runs on my own personal computer of the
\factor|benchmarks| word, which invokes all the \Verb|benchmark|
sub-vocabularies.  The ``before'' time used the local value numbering, while
``after'' times had \Verb|value-numbering| replaced with the \gls{GVN} pass.
The ``change'' is measured by the formula
%
$$\frac{\text{before} - \text{after}}{\text{before}} \times 100$$
%
to indicate the relative running times.  Negative values in this column are
good, as that means the running time has decreased.

\begin{longtable}{llll}
\toprule
Benchmark & Before (seconds) & After (seconds) & Change (\%) \\
\midrule
\endhead
\texttt{3d-matrix-scalar}         & 3.705816738       & 3.046126696         & $-17.80$    \\
\texttt{3d-matrix-vector}         & 0.161298778       & 0.089539887         & $-44.49$    \\
\texttt{backtrack}                & 4.280001561       & 2.358672591         & $-44.89$    \\
\texttt{base64}                   & 5.127831493       & 2.853612485         & $-44.35$    \\
\texttt{beust1}                   & 7.531546384       & 4.604929188         & $-38.86$    \\
\texttt{beust2}                   & 20.308680548      & 12.843534349        & $-36.76$    \\
\texttt{binary-search}            & 3.729776895       & 2.349520427         & $-37.01$    \\
\texttt{binary-trees}             & 9.403166818       & 6.518867479         & $-30.67$    \\
\texttt{bootstrap1}               & 32.472196349      & 30.887877896        & $-4.88$     \\
\texttt{chameneos-redux}          & 2.923900422       & 2.041007328         & $-30.20$    \\
\texttt{continuations}            & 0.273525202       & 0.200695972         & $-26.63$    \\
\texttt{crc32}                    & 0.010623653       & 0.005282642         & $-50.27$    \\
\texttt{dawes}                    & 1.588111926       & 1.027176578         & $-35.32$    \\
\texttt{dispatch1}                & 7.640720326       & 5.106558985         & $-33.17$    \\
\texttt{dispatch2}                & 5.221652668       & 3.984754032         & $-23.69$    \\
\texttt{dispatch3}                & 9.710520454       & 6.203527737         & $-36.12$    \\
\texttt{dispatch4}                & 8.224931156       & 4.098265543         & $-50.17$    \\
\texttt{dispatch5}                & 4.74357434        & 3.478219608         & $-26.68$    \\
\texttt{e-decimals}               & 3.903754723       & 2.646958072         & $-32.19$    \\
\texttt{e-ratios}                 & 4.774454589       & 3.658075473         & $-23.38$    \\
\texttt{empty-loop-0}             & 0.251816164       & 0.199189271         & $-20.90$    \\
\texttt{empty-loop-1}             & 1.039242509       & 0.857588545         & $-17.48$    \\
\texttt{empty-loop-2}             & 0.472215346       & 0.387974286         & $-17.84$    \\
\texttt{euler150}                 & 37.785852299      & 27.05450689         & $-28.40$    \\
\texttt{fannkuch}                 & 9.627490235       & 6.8970571           & $-28.36$    \\
\texttt{fasta}                    & 7.25292282        & 5.640517069         & $-22.23$    \\
\texttt{fib1}                     & 0.179389215       & 0.164933805         & $-8.06$     \\
\texttt{fib2}                     & 0.205853157       & 0.138174211         & $-32.88$    \\
\texttt{fib3}                     & 0.785036151       & 0.539739186         & $-31.25$    \\
\texttt{fib4}                     & 0.391805799       & 0.260370111         & $-33.55$    \\
\texttt{fib5}                     & 1.508625224       & 1.002724851         & $-33.53$    \\
\texttt{fib6}                     & 19.202504502      & 13.146010511        & $-31.54$    \\
\texttt{gc0}                      & 7.360087104       & 5.508594031         & $-25.16$    \\
\texttt{gc1}                      & 0.418173431       & 0.281497214         & $-32.68$    \\
\texttt{gc2}                      & 25.611210221      & 19.716168704        & $-23.02$    \\
\texttt{gc3}                      & 2.757943071       & 2.210785891         & $-19.84$    \\
\texttt{hashtables}               & 8.068216942       & 7.997106348         & $-0.88$     \\
\texttt{heaps}                    & 4.360368411       & 4.32169158          & $-0.89$     \\
\texttt{iteration}                & 7.875561986       & 6.277891729         & $-20.29$    \\
\texttt{javascript}               & 17.881224721      & 12.74204052         & $-28.74$    \\
\texttt{knucleotide}              & 5.490420772       & 3.5704101           & $-34.97$    \\
\texttt{mandel}                   & 0.251711276       & 0.198695557         & $-21.06$    \\
\texttt{matrix-exponential-scalar}& 16.451432774      & 12.017000042        & $-26.95$    \\
\texttt{matrix-exponential-simd}  & 0.681684747       & 0.536850343         & $-21.25$    \\
\texttt{md5}                      & 10.40516678       & 9.198666403         & $-11.60$    \\
\texttt{mt}                       & 33.91981743       & 29.961085146        & $-11.67$    \\
\texttt{nbody}                    & 9.203478441       & 6.795154145         & $-26.17$    \\
\texttt{nbody-simd}               & 0.845814208       & 0.854773096         & $+1.06$     \\
\texttt{nested-empty-loop-1}      & 0.097090973       & 0.068475608         & $-29.47$    \\
\texttt{nested-empty-loop-2}      & 0.893126911       & 0.861327078         & $-3.56$     \\
\texttt{nsieve}                   & 1.086110659       & 1.137648699         & $+4.75$     \\
\texttt{nsieve-bits}              & 2.707271763       & 2.815509077         & $+4.00$     \\
\texttt{nsieve-bytes}             & 0.785041878       & 1.211421146         & $+54.31$    \\
\texttt{partial-sums}             & 3.762171661       & 4.130144177         & $+9.78$     \\
\texttt{pidigits}                 & 2.182877913       & 2.195385034         & $+0.57$     \\
\texttt{random}                   & 5.66540782        & 5.71913683          & $+0.95$     \\
\texttt{raytracer}                & 5.047070171       & 4.39514879          & $-12.92$    \\
\texttt{raytracer-simd}           & 1.072588515       & 0.980927338         & $-8.55$     \\
\texttt{recursive}                & 2.703509403       & 2.529087637         & $-6.45$     \\
\texttt{regex-dna}                & 2.208584014       & 1.808859571         & $-18.10$    \\
\texttt{reverse-complement}       & 2.801163847       & 2.353254665         & $-15.99$    \\
\texttt{ring}                     & 1.822206473       & 1.62482491          & $-10.83$    \\
\texttt{sfmt}                     & 2.675838657       & 2.463367198         & $-7.94$     \\
\texttt{sha1}                     & 11.964973943      & 11.142380303        & $-6.88$     \\
\texttt{simd-1}                   & 1.857778672       & 1.703206011         & $-8.32$     \\
\texttt{sockets}                  & 10.636346636      & 10.516448454        & $-1.13$     \\
\texttt{sort}                     & 0.695635429       & 0.581855635         & $-16.36$    \\
\texttt{spectral-norm}            & 3.433630383       & 2.960833789         & $-13.77$    \\
\texttt{spectral-norm-simd}       & 2.743240011       & 3.237017655         & $+18.00$    \\
\texttt{stack}                    & 1.580016742       & 2.004478602         & $+26.86$    \\
\texttt{struct-arrays}            & 2.180774222       & 2.421915609         & $+11.06$    \\
\texttt{sum-file}                 & 0.883097981       & 0.957151577         & $+8.39$     \\
\texttt{terrain-generation}       & 1.611800222       & 1.887047663         & $+17.08$    \\
\texttt{tuple-arrays}             & 0.262747557       & 0.329399609         & $+25.37$    \\
\texttt{typecheck1}               & 1.750223408       & 1.674592158         & $-4.32$     \\
\texttt{typecheck2}               & 1.674738245       & 1.553203741         & $-7.26$     \\
\texttt{typecheck3}               & 1.891206648       & 1.735390184         & $-8.24$     \\
\texttt{ui-panes}                 & 0.305595039       & 0.29985214          & $-1.88$     \\
\texttt{xml}                      & 3.013709363       & 2.722223892         & $-9.67$     \\
\texttt{yuv-to-rgb}               & 0.398174487       & 0.318891664         & $-19.91$    \\
\end{longtable}

\begin{sloppypar}
These informal results are promising: the mean speedup was $-16.35\%$ (median
$-18.97\%$), and of $80$ benchmarks, only $13$ showed any increase in running
time.  The mean speedup among those that ran faster was $-22.24\%$ (median
$-22.23\%$).  Of the $13$ that ran slower, even fewer showed significant
increases in running time.  Duplicated below for convenience are the slower
benchmarks, sorted in decreasing order of the percent change.  We can see the
last five or six benchmarks exhibited negligible differences---not only is the
relative change tiny, but the absolute difference in running times is less than
$0.1$ seconds.  (The \Verb|tuple-arrays| results also show a similar absolute
change, but it is relatively much larger.)
\end{sloppypar}

\begin{longtable}{llll}
\toprule
Benchmark & Before (seconds) & After (seconds) & Change (\%) \\
\midrule
\endhead
\texttt{nsieve-bytes}             & 0.785041878       & 1.211421146         & $+54.31$    \\
\texttt{stack}                    & 1.580016742       & 2.004478602         & $+26.86$    \\
\texttt{tuple-arrays}             & 0.262747557       & 0.329399609         & $+25.37$    \\
\texttt{spectral-norm-simd}       & 2.743240011       & 3.237017655         & $+18.00$    \\
\texttt{terrain-generation}       & 1.611800222       & 1.887047663         & $+17.08$    \\
\texttt{struct-arrays}            & 2.180774222       & 2.421915609         & $+11.06$    \\
\texttt{partial-sums}             & 3.762171661       & 4.130144177         & $+9.78$     \\
\texttt{sum-file}                 & 0.883097981       & 0.957151577         & $+8.39$     \\
\texttt{nsieve}                   & 1.086110659       & 1.137648699         & $+4.75$     \\
\texttt{nsieve-bits}              & 2.707271763       & 2.815509077         & $+4.00$     \\
\texttt{nbody-simd}               & 0.845814208       & 0.854773096         & $+1.06$     \\
\texttt{random}                   & 5.66540782        & 5.71913683          & $+0.95$     \\
\texttt{pidigits}                 & 2.182877913       & 2.195385034         & $+0.57$     \\
\end{longtable}

Overall, even transitioning to a relatively simple \gls{GVN} algorithm amounts
to a positive change in Factor's compiler.  More redundancies are eliminated,
resulting in speedier programs.  Judging by unit tests, the implementation is
at least as sound as the previous local value numbering,  as all the same tests
have passed.

% Future
%   SCC (discussion of potential improvement)
%   Click, Gargi, et al. (future directions)


\section{Value Numbering}\label{sec:vn}

At a very basic level, most optimization techniques revolve around avoiding
redundant or unnecessary computation.  Thus, it's vital that we discover which
values in a program are equal.  That way, we can simplify the code that wastes
machine cycles repeatedly calculating the same values.  Classic optimization
phases like constant/copy propagation, common subexpression elimination,
loop-invariant code motion, induction variable elimination, and others
discussed in the de facto treatise, ``The Dragon Book''\todo{cite}, perform
this sort of redundancy elimination based on information about the equality of
expressions.

In general, the problem of determining whether two expressions in a program are
equivalent is undecidable.  Therefore, we seek a \term{conservative} solution
that doesn't necessarily identify all equivalences, but is nevertheless correct
about any equivalences it does identify.  Solving this equivalence problem is
the work of \term{value numbering} algorithms.  These assign every value in the
program a number such that two values have the same value number if and only if
the compiler can prove they will be equal at runtime.

Value numbering has a long history in literature and practice, spanning many
techniques.  In \cref{sec:compiler:cfg} we saw the \factor|value-numbering|
word, which is actually based on some of the earliest---and least
effective---methods of value numbering.  \Cref{sec:vn:local} describes the way
Factor's current algorithm works, and highlights its shortcomings to motivate
the main work of this thesis, which is covered in
\cref{sec:vn:global,sec:vn:avail}.  We finish the \lcnamecref{sec:vn} by
analyzing the results of these changes and reviewing the literature for further
enhancements that can be made to this optimization pass.

%\subsection{Local Value Numbering}\label{sec:vn:local}

Tracing the exact origins of value numbering is difficult.  It's thought to
have originally been invented in the 1960s by Balke\todo{cite Simpson}.  The
earliest tangible reference to a value numbering (at least, the earliest point
where discussions in the literature seem to start) appears in an oft-cited but
unpublished work of Cocke\todo{cite-like}.  The technique is relatively simple,
but not as powerful as other methods for reasons described hereafter.

The algorithm considers a single basic block.  For each instruction (from top
to bottom) in the block, we essentially let the value number of the assignment
target be a hash of the operator and the value numbers of the operands.  That
is, we hash the \term{expression} being computed by an instruction.  Thus,
assuming a proper hash function, two expressions are \term{congruent} (denoted
$\texttt{x} \cong \texttt{y}$) if
%
\begin{itemize}
%
  \item they have the same operators and
%
  \item their operands are congruent.
%
\end{itemize}
%
\noindent This is our approximation of runtime equivalence.  The first property
is fulfilled by basing the hash, in part, on the operator.  The second property
holds because the hash is based on the value numbers of the statement's
operands---not just the operands as they appear in code (i.e., \term{lexical}
equivalence).  Any information about congruence is propagated through the value
numbers.  We'll have discovered any such equivalences among the operands before
computing the value number of the assignment target because every value in a
basic block is either defined before it's used, or else defined at some point
in a predecessor of the block, which we don't care about when only considering
one basic block.

This is the first shortcoming of the algorithm.  It is \term{local}, focusing
on only one basic block at a time.  Any definitions outside the boundaries of
the basic block won't be reused, even if they reach the block.  This severely
limits the scope of the redundancies we can discover.  We could improve upon
this by considering the algorithm across an entire loop-free \gls{CFG} in any
\term{topological order}.  In such an ordering, a basic block $B$ comes before
any other block $B'$ to which it has an edge.  Thus, any ``outside'' variables
that instructions in $B'$ rely on must have come from $B$ or earlier, which
will have already been computed in a traversal of such an ordering.  However,
\glsplural{CFG} usually contain cycles or loops (at least interesting ones do),
which make such an ordering impossible.  We may still pick a topological order
that ignores back-edges, but we may encounter operands whose values flow along
those back-edges.  We'll later address the issue of encountering instructions
whose operands haven't been processed yet.

In Factor, expressions are basically instructions (the \factor|insn| objects
discussed in \cref{sec:compiler:cfg}) that have had their destination registers
stripped.  Instructions can be converted to expressions with the \factor|>expr|
word defined in the \factor|compiler.cfg.value-numbering.expressions|
vocabulary.  For instance, an \factor|##add| instruction with the destination
register \factor|1| and source registers \factor|2| and \factor|3| will be
converted into an array of three elements:
%
\begin{itemize}
%
  \item The \factor|##add| class word, indicating the expression is derived
        from an \factor|##add| instruction.
%
  \item The value number of the virtual register \factor|2|.
%
  \item The value number of the virtual register \factor|3|.
%
\end{itemize}
%
\noindent Some instructions are not \term{referentially transparent}, meaning
they can't be replaced with the value they compute without changing the
program's behavior.  For example, \factor|##call| and \factor|##branch| cannot
reasonably be converted into expressions.  In these cases, \factor|>expr|
merely returns a unique value.

\inputlst{value-numbering-graph}

The hashing of expressions takes place in the so-called \term{expression graph}
implemented in the vocabulary shown in \vref{lst:value-numbering-graph}.  This
consists of three global hash tables that relate virtual registers, value
numbers, instructions, and expressions.  Since virtual registers are just
integers, we actually use them as value numbers, too.  \factor|vregs>vns| maps
virtual registers to their value numbers.  If a virtual register  is mapped to
itself in this table, its definition is the canonical instruction that we use
to compute the value.  This instruction is stored in the \factor|vns>insns|
table.  Finally, the most important mapping is \factor|exprs>vns|.  True to its
name, it uses expressions as keys, which of course are implicitly hashed.
Thus, we can use this table to determine equivalence of expressions.

Other definitions in \vref{lst:value-numbering-graph} manipulate expressions
and the graph.  The global variable \factor|input-expr-counter| is used in the
generation of unique expressions discussed earlier.  \factor|init-value-graph|
initializes this and all the tables.  \factor|set-vn| establishes a mapping
from a virtual register to a value number in \factor|vregs>vns|.
\factor|vn>insn| gives terse access to the \factor|vns>insns| table.
\factor|vreg>insn| uses \factor|vregs>vns| and \factor|vns>insns| to get the
canonical instruction that defines a given virtual register.  Finally,
\factor|vreg>vn| looks up the value of a key in the \factor|vregs>vns| table.
Importantly, if the key is not yet present in the table, it is automatically
mapped to itself---it's assumed that the virtual register does not correspond
to a redundant instruction.

This is the second shortcoming of the algorithm.  It must make a
\term{pessimistic} assumption about congruences.  That is, it starts by
assuming that every expression has a unique value number, then tries to prove
that there are some values which are actually congruent.

%This is the final version depicted in Algorithm~\vref{alg:hash-vn}.  Clearly,
%it will fail to discover congruences for values that flow along back-edges,
%since we simply ignore back-edges and started with the assumption that values
%are distinct.  For example, refer to Figure~\vref{fig:hash-vn-ex}, which uses
%the SSA form of the CFG in Figure~\vref{fig:cfg-construction}.  We can see that
%the corresponding \lstinline|i|s and \lstinline|j|s are equivalent, but
%Algorithm~\ref{alg:hash-vn} must ignore the back-edge and consider the
%statements in the order shown.  Thus, it only discovers $\code{i$_0$} \cong
%\code{j$_0$}$, but considers $\code{i$_1$} \not\cong \code{j$_1$} \not\cong
%\code{i$_2$} \not\cong \code{j$_2$}$.  This is because
%$\attrib{\code{i$_2$}}{valnum} \ne \attrib{\code{j$_2$}}{valnum}$ (they were
%initialized uniquely) at the time when $\attrib{\code{i$_1$}}{valnum}$ and
%$\attrib{\code{j$_1$}}{valnum}$ were being computed, which gave
%\lstinline|i$_1$| and \lstinline|j$_1$| differing value numbers, which then got
%propagated to the computation of $\attrib{\code{i$_2$}}{valnum}$ and
%$\attrib{\code{j$_2$}}{valnum}$.  Notice that congruence only guarantees
%properties when two values have the same value number.  Nothing can be said of
%incongruent values---not even that they're nonequal (in this case, we have
%incongruent values that actually \emph{are} equal).  This illustrates the
%conservativeness of the solution.

%Factor currently
%  Cocke & Schwartz (effectively what Factor uses)
%
%compiler.cfg.value-numbering
%  process-instruction
%  value-numbering
%  value-numbering-step
%
%compiler.cfg.value-numbering.graph
%  vregs>vns
%  exprs>vns
%  vns>insns
%
%compiler.cfg.value-numbering.expressions
%  value-numbering-step
%    exprs>vns
%    >expr
%    <integer-expr>
%    <reference-expr>
%
%-------------------------------------------------------------------------------
%
%compiler.cfg.value-numbering.rewrite
%  rewrite
%    ! Utilities
%    insn>integer
%    vreg>integer
%    insn>literal
%    vreg>literal
%
%compiler.cfg.value-numbering.alien
%  rewrite
%
%compiler.cfg.value-numbering.comparisons
%  rewrite
%    rewrite-boolean-comparison
%    fold-branch
%    >test
%    >compare-imm
%    simplify-test
%
%compiler.cfg.value-numbering.folding
%  rewrite
%    binary-constant-fold
%    unary-constant-fold
%
%compiler.cfg.value-numbering.math
%  rewrite
%    self-inverse
%    identity
%    compiler.cfg.value-numbering.folding
%    reassociate
%    distribute
%    insn>imm-insn
%
%compiler.cfg.value-numbering.misc
%  rewrite
%
%compiler.cfg.value-numbering.simd
%  rewrite
%
%compiler.cfg.value-numbering.slots
%  rewrite

%\section{Global Value Numbering}\label{sec:vn:global}

Answering the challenges of \citeauthor{Cocke}, \citeauthor{AWZ}
\autocite*{AWZ} described what would be the de facto value numbering algorithm
for several years, and rightly so.  It was a properly \term{global} value
numbering algorithm, working across an entire \gls{CFG} instead of on single
basic blocks.  Their paper was important in another very relevant way: it is
the first published reference to SSA form \autocite{VanDrunen}, including an
algorithm for its construction.

Though we could try to extend the scope of Factor's local value numbering, it
is still inherently pessimistic.  The algorithm of \citeauthor{AWZ}, which is
commonly referred to simply as AWZ, uses a modification of a minimization
algorithm for finite state automata \autocite{Hopcroft}.  It works on an
\term{optimistic} assumption by first assuming every value has the same value
number, then trying to prove that values are actually different.  It does this
by treating value numbers as \term{congruence classes} that partition the set
of virtual registers.  If two values are in the same class, then they are
congruent, where congruence is defined as in \cref{sec:vn:local}.

Such a partition is not unique, in general.  For instance, a trivial one places
each value in its own congruence class.  So, we look for the \term{maximal
fixed point}---the solution that has the most congruent values and therefore
the fewest congruence classes.  We must start with a congruence class for each
operation so that, say, all values computed by \Verb|##add|s are grouped
together, those computed by \Verb|##mul|s are in the same class, etc.  We
must then iteratively look at our collection of classes, separating them when
we discover incongruent values.  For an \gls{SSA} variable in class $P$, we
look at its defining expression.  If an operand at position $i$ belongs to
class $Q$, then the $i^\text{th}$ operand of every other value in $P$ should
also be in $Q$.  Otherwise, $P$ must be \term{split} by removing those
variables whose $i^\text{th}$ operands are not in class $Q$ and placing them in
a new congruence class.  We keep splitting classes until the partitioning
stabilizes.

The optimistic assumption may seem dangerous.  Is it possible that we're
``overoptimistic''?  That two values assumed to be congruent and not proven
incongruent might actually be inequivalent when the program is run?  The AWZ
paper dedicates a section to proving that two congruent variables are
equivalent at a point $p$ in the program if their definitions dominate $p$.
The proof is a bit quirky, but reasonable.  They develop a dynamic notion of
dominance in a running program which implies static dominance in the code, then
show that congruence implies runtime equality (though equivalence does not
imply congruence).

AWZ made the need for \gls{GVN} algorithms apparent.  However, finite state
automata minimization makes for a more complicated algorithm than hash-based
value numbering.  A na\"{i}ve implementation can be quadratic, although careful
data structure and procedure design can make it $O(n\log n)$.  Furthermore,
it's resistant to the same improvements we easily added to the local value
numbering.  To even consider the commutativity of operations requires changes
in operand position tracking and splitting---the heart of the algorithm.  It is
generally limited by what the programmer writes down: deeper congruences due
to, say, algebraic identities can't be discovered.

In fact, by performing an optimization that uses the \gls{GVN} information,
more \gls{GVN} congruences may arise.  If we can somehow perform the two
analyses simultaneously, they'll produce better results.  This generalizes to
interdependent compiler optimizations at large, as elucidated in
\citeauthor{Click}'s dissertation \autocite*{Click}, which describes a method
for formalizing and combining separate optimizations that make optimistic
assumptions (whatever they happen to be for each particular analysis).  He uses
this to merge \gls{GVN} with \term{conditional constant propagation}, which
itself is a combination of constant propagation and unreachable code
elimination (pretty much like the \Verb|propagate| pass from
\cref{sec:compiler:tree}).  Furthermore, \gls{GVN} is extended to handle
algebraic identities, propagate constants, and fold redundant $\phi$s.
Unfortunately, the straightforward algorithm for this is $O(n^2)$, while the
$O(n\log n)$ version presented is not just complicated, but can also miss some
congruences between $\phi$-functions \autocites{Click,Simpson}.

Hot on the heels of this work, \citeauthor{Simpson}'s \autocite*{Simpson}
dissertation provides probably the most exhaustive treatment of \gls{GVN}
algorithms.  He presents several extensions, such as incorporating hash-based
local value numbering into \gls{SSA} construction, handling commutativity in
AWZ, and performing redundant store elimination.  He builds off of the two
classical algorithms independently, which underlines their inherent differences
and limitations.  In general, hash-based value numbering is easy to extend
without greatly impacting the runtime complexity, as is the case in Factor's
implementation.

Drawing from this experience, Simpson's hallmark algorithm combines the best of
both worlds by taking the hash-based algorithm which is easy to understand,
implement, and extend, and making it global, so it identifies more congruences.
Dubbed the ``\gls{RPO} algorithm'', it simply applies hash-based value
numbering iteratively over the \gls{CFG} until we reach the same fixed point
computed by AWZ.  (The fact that it computes the exact same fixed point is
proven fairly straightforwardly in the dissertation.)  It could technically
traverse the \gls{CFG} in any topological order, but Simpson defaults to
reverse postorder.

Because it is based off the hashing algorithm, we get the benefits essentially
for free.  The same simplifications can be performed, but with the added
knowledge of global congruences.  Since the majority of Factor's value
numbering code is dedicated to the \Verb|rewrite| generic, it makes sense to
reuse as much of that code as possible.  Therefore, to convert Factor's local
algorithm to a global one, I modified the existing code to use the \gls{RPO}
algorithm.

\inputlst{gvn-graph}

The most fundamental change is to the expression graph.  Referring to
\vref{lst:gvn-graph}, we see most of the same code as in
\vref{lst:value-numbering-graph}, with changes indicated by arrows
($\longrightarrow$).  Two more global variables have been added, namely
\Verb|changed?| and \Verb|final-iteration?|.  The former is what we use to
guide the fixed-point iteration.  As long as value numbers are changing, we
keep iterating.  An important side effect of this is that we can no longer
perform \Verb|rewrite| online, since the transformations we make aren't
guaranteed to be sound on any iteration except the final one.  This makes the
\gls{RPO} algorithm work \term{offline}, first discovering redundancies, then
eliminating them in a separate pass.  When this elimination pass starts, we'll
set \Verb|final-iteration?| to \factor|t|.

A key change is in the \Verb|vreg>vn| word, which now makes an optimistic
assumption about previously unseen values.  Given a new virtual register that
wasn't in the \Verb|vregs>vns| table, the old version would map the register
to itself, making the value its own canonical representative.  However, if this
version tries to look up a key that does not exist in the hash table, it will
simply return \factor|f| (which Factor will do by default with the \factor|at|
word).  Therefore, every value in the \gls{CFG} starts off with the same value
``number'', \factor|f|.  By the end of the \gls{GVN} pass, there should be no
value left that hasn't been put in the \Verb|vregs>vns| table, as we'll have
processed every definition.

To keep track of whether \Verb|vregs>vns| changes, we simply need to alter
\Verb|set-vn|.  Here, we use \factor|maybe-set-at|, a utility from the
\Verb|assocs| vocabulary.  This works like \factor|set-at|, establishing a
mapping in the hash table.  In addition, it returns a boolean indicating
change: if a new key has been added to the table, we return \factor|t|.
Otherwise, we return \factor|t| only in the case where an old key is mapped to
a new value.  If an old key is mapped to the same value that's already in the
table, \factor|maybe-set-at| returns \factor|f|.  Therefore, when
\Verb|vregs>vns| does change, we set \Verb|changed?| to \factor|t| (which
is what the \factor|on| word does).

Finally, we define a new utility word, \Verb|clear-exprs|, which resets the
\Verb|exprs>vns| and \Verb|vns>insns| tables.  Unlike the local value numbering
phase, we don't reset the entire expression graph.  Instead, we make a pass
over the whole \gls{CFG} at a time.  The only reason optimism works is that we
keep trying to disprove our foolhardy assumptions.  Really, \Verb|vregs>vns|
establishes congruence classes of value numbers.  At first, every value belongs
in one class, \factor|f|.  We make a pass over the \gls{CFG} to disprove
whatever we can about this.  If we've introduced new congruence classes (new
values in the \Verb|vregs>vns| hash), we do another iteration.  But each time,
we use the congruence classes discovered from the previous iteration.  At the
start of each new pass, the expressions and instructions in \Verb|exprs>vns|
and \Verb|vns>insns| are invalidated---their results are based on old
information.  So, these are erased on each iteration.  Much like AWZ, we keep
splitting classes until they can't be split anymore.

\inputlst{gvn-step}

This logic is captured in \vref{lst:gvn-step}.  Rather than reset the tables
when we start processing each basic block in \Verb|value-numbering-step| like
before, we call \Verb|clear-exprs| on each iteration over the \gls{CFG} in
\Verb|value-numbering-iteration|.  Note that \Verb|value-numbering-step| no
longer returns the changed instructions, as we aren't replacing them online.
\Verb|value-numbering-iteration| uses \Verb|simple-analysis| instead of
\Verb|simple-optimization|, which only expects global state to change---no
instructions are updated in the block.  Much to our advantage,
\Verb|simple-analysis| already traverses the \gls{CFG} in \acrlong{RPO}, so
we needn't worry about traversal order.  The top-level word
\Verb|determine-value-numbers| ties this all together by calling
\Verb|value-numbering-iteration| until we can get through it with
\Verb|changed?| remaining false.

\inputlst{gvn-simplify}
\inputlst{gvn-value-number}

\begin{sloppypar}
Note that the work of \Verb|value-numbering-step| is further divided into two
words, \Verb|simplify| and \Verb|value-number|.  These combine to do much
the same work as \Verb|process-instruction| in
\cref{lst:process-instruction}.  \Verb|simplify| makes the repeated calls to
\Verb|rewrite| until the instruction cannot be simplified further.  Its
definition is in \vref{lst:gvn-simplify}.  We then pass the simplified
instruction to \Verb|value-number|, which is defined in
\vref{lst:gvn-value-number}.  This also has a similar structure to
\Verb|process-instruction|.  The main difference is that instructions are no
longer returned (again, they aren't altered in place).  So, the \factor|array|
method uses \factor|each| instead of \factor|map| to recurse into the results
of \Verb|rewrite|.
\end{sloppypar}

A subtle change is necessary with the \Verb|alien-call-insn| and
\Verb|##callback-inputs| methods.  Whereas \Verb|process-instruction|
merely skipped over certain instructions that could not be rewritten, here we
don't have that luxury.  We need to be careful to \Verb|set-vn| every virtual
register that gets defined by any instruction.  While making a pessimistic
assumption, it didn't matter if we did this: any unseen value would be presumed
important by \Verb|vreg>vn|.  However, with the optimistic assumption,
\Verb|vreg>vn| will give the impression that unseen values are all the same
by returning \factor|f|.  Therefore, we simply record the virtual registers
defined in instructions that may define one or more of them.  Specifically,
\Verb|alien-call-insn| and \Verb|##callback-inputs| are classes that
correspond to \gls{FFI} instructions.

The \Verb|##copy| method uses \Verb|set-vn| the same way as before.
\Verb|redundant-instruction|, \Verb|useful-instruction|, and
\Verb|check-redundancy| are also largely the same.  These have just been
tweaked to not return instructions.

\inputlst{phi-expr}

The \Verb|##phi| method in \vref{lst:gvn-value-number} represents a major
change. Before, \Verb|##phi|s were left uninterpreted.  Congruences between
induction variables that flowed along back-edges would not be identifiable.
But now, by checking for redundant \Verb|##phi|s, we may reduce them to
copies.  Each \Verb|##phi| object has an \Verb|inputs| slot, which is a
hash table from basic block to the virtual register that flows from that block.
Thus, there is one input for each predecessor.  The \factor|values| of the hash
will be the virtual registers that might be selected for the \Verb|dst|
value.  We look up the value numbers of these, removing all instances of
\factor|f| with the \factor|sift| word.  If all of the inputs are congruent, we
can call \Verb|redundant-instruction|, setting the value number of the
\Verb|##phi|'s \Verb|dst| to the value number of its first input (without
loss of generality).  The \factor|all-equal?| word will return \factor|t| if
the sequence is empty (as it's vacuously true), so we must make sure not to
call \factor|first| on the sequence, since this will be a runtime error.  If
the sequence is empty, we needn't note the redundancy, as the \Verb|##phi|'s
\Verb|dst| will already have the optimistic value number \factor|f| anyway.
Otherwise, we call \Verb|check-redundancy|.  The purpose of this is to
identify \Verb|##phi|s that are equal to each other.  Even if its inputs are
incongruent, a \Verb|##phi| might still represent a copy of another induction
variable.  So that \Verb|check-redundancy| works, we also define a
\Verb|>expr| method in \Verb|compiler.cfg.gvn.expressions|, as seen in
\vref{lst:phi-expr}.  Here, the expression is an array consisting of the
\Verb|##phi| class word, the current basic block's number, and the inputs'
value numbers.  We include the basic block number because only \Verb|##phi|s
within the same block can be considered equivalent to each other.

The final method in \vref{lst:gvn-value-number} defines the default behavior
for \Verb|value-number|, which calls \Verb|check-redundancy| on the
simplified instruction if it defines a single virtual register.  Note that we
separate the \Verb|alien-call-insn| and \Verb|##callback-inputs| logic from
this, since they happen to define a variable number of registers.  If
particular instances define only one register, we still don't want to call
\Verb|check-redundancy| on them, since they don't have a \Verb|dst| slot.
To avoid calling \Verb|dst>>| and triggering an error in
\Verb|useful-instruction|, we needed separate methods for the \gls{FFI}
classes.

\inputfig{gvn}

With these changes, we can globally identify value numbers, including
equivalences that arise from simplifying instructions (even though no
replacements are actually done yet).  To illustrate this, consider again the
example
%
\factor|0 100 [ 1 fixnum+fast ] times|,
%
reproduced in \vref{fig:gvn}.  As the expression graph changes frequently in
this new algorithm, instead of showing the literal hash tables we'll use a
shorthand notation.  Virtual registers will be integers, and to avoid confusion
value numbers will be written in brackets, like \vn{n}.  Then, we'll show
\Verb|vreg>vn| mappings with the notation $n\to\vn{n}$, where $n$ is the
register and \vn{n} is the value number.  If there is a corresponding
expression in \Verb|exprs>vns|, it will be denoted after the mapping, like
$n\to\vn{n}~(\textit{expression})$.  With the expressions, the instructions in
\Verb|vns>insns| are a bit redundant for understanding the value numbering
process, so they will be elided.  Any mappings to \factor|f| will be elided, as
they're understood to be implicit when a key is absent.

\todo[inline]{Might make separate figures of each block, for easier reference}

\Verb|determine-value-numbers| starts the first iteration, which of course
starts at basic block $1$.  \Verb|##inc-d| is a no-op, but the first two
\Verb|##load-integer|s are established as useful instructions.
%
\Verb|##load-integer 23 0|
%
is recognized as redundant, since at this point we know that \Verb|21| has
the value \Verb|0|.  The \Verb|##copy| instructions all pile on value
number mappings, leaving us with the following:
%
\begin{align*}
  21 &\to \vn{21} \quad (0)  \\
  22 &\to \vn{22} \quad (100)\\
  23 &\to \vn{21}            \\
  24 &\to \vn{22}            \\
  25 &\to \vn{21}            \\
  26 &\to \vn{22}            \\
  27 &\to \vn{21}
\end{align*}

At iteration $1$, basic block $2$, the first \Verb|##phi| has inputs
\Verb|25| (from block $1$) and \Verb|41| (from block $3$).  The former has
the value number \vn{21}, while the latter is still at \factor|f|.  We treat
this value number much like a $\top$ element, unifying it with the other input
to give us the assumption that \Verb|29| will be a copy of \Verb|25|.
Thus, it gets the same value number.  A similar choice happens for the second
\Verb|##phi|.  The instruction 
%
\Verb|##compare-integer 31 30 26 cc< 9|
%
is an interesting case.  Due to our optimistic assumptions thus far, we believe
\Verb|30| is carrying the value \Verb|0|, and that \Verb|26| is set to
\Verb|100|.  Thus, this instruction gets constant-folded by \Verb|simplify|
into
%
\Verb|##load-reference 31 t|.
%
The \gls{CFG} isn't changed, but the expression graph reflects this belief.
Later, this assumption will be invalidated.  The following copies are processed
as usual, with the distinct difference here that 
%
\Verb|##copy 33 26 any-rep|
%
has the global knowledge of the value number of \Verb|26|.  Because the
\Verb|##compare-integer| was constant-folded, so is the
\Verb|##compare-imm-branch|---and to the same value, no less.  This leaves us
with:
%
\begin{align*}
  21 &\to \vn{21} \quad (0)                 \\
  22 &\to \vn{22} \quad (100)               \\
  23 &\to \vn{21}                           \\
  24 &\to \vn{22}                           \\
  25 &\to \vn{21}                           \\
  26 &\to \vn{22}                           \\
  27 &\to \vn{21}                           \\
  29 &\to \vn{21}                           \\
  30 &\to \vn{21}                           \\
  31 &\to \vn{31} \quad (\text{\factor|t|}) \\
  32 &\to \vn{31}                           \\
  33 &\to \vn{22}                           \\
  34 &\to \vn{31}
\end{align*}

Block $3$ of iteration $1$ gives the \Verb|##load-integer|s' destinations the
same value number, corresponding to the integer $1$.  Because optimism makes
the algorithm think that \Verb|29| and \Verb|30| correspond to the integer
$0$, the \Verb|##add|s are constant-folded.  This leaves us with:
%
\begin{align*}
  21 &\to \vn{21} \quad (0)                 \\
  22 &\to \vn{22} \quad (100)               \\
  23 &\to \vn{21}                           \\
  24 &\to \vn{22}                           \\
  25 &\to \vn{21}                           \\
  26 &\to \vn{22}                           \\
  27 &\to \vn{21}                           \\
  29 &\to \vn{21}                           \\
  30 &\to \vn{21}                           \\
  31 &\to \vn{31} \quad (\text{\factor|t|}) \\
  32 &\to \vn{31}                           \\
  33 &\to \vn{22}                           \\
  34 &\to \vn{31}                           \\
  35 &\to \vn{35} \quad (1)                 \\
  36 &\to \vn{35}                           \\
  37 &\to \vn{35}                           \\
  38 &\to \vn{35}                           \\
  39 &\to \vn{21}                           \\
  40 &\to \vn{22}                           \\
  41 &\to \vn{35}                           \\
  42 &\to \vn{35}
\end{align*}

While block $4$ is visited in each iteration, it doesn't define any registers,
so doesn't affect the state of value numbering.  Therefore, the above is the
state left at the end of iteration $1$.

Since \Verb|vregs>vns| clearly changed, iteration $2$ commences by clearing
the expressions, though the value numbers remain.  Block $1$ doesn't change
from iteration $1$, giving us:
%
\begin{align*}
  21 &\to \vn{21} \quad (0)                 \\
  22 &\to \vn{22} \quad (100)               \\
  23 &\to \vn{21}                           \\
  24 &\to \vn{22}                           \\
  25 &\to \vn{21}                           \\
  26 &\to \vn{22}                           \\
  27 &\to \vn{21}                           \\
  29 &\to \vn{21}                           \\
  30 &\to \vn{21}                           \\
  31 &\to \vn{31}                           \\
  32 &\to \vn{31}                           \\
  33 &\to \vn{22}                           \\
  34 &\to \vn{31}                           \\
  35 &\to \vn{35}                           \\
  36 &\to \vn{35}                           \\
  37 &\to \vn{35}                           \\
  38 &\to \vn{35}                           \\
  39 &\to \vn{21}                           \\
  40 &\to \vn{22}                           \\
  41 &\to \vn{35}                           \\
  42 &\to \vn{35}
\end{align*}

Now that we're in iteration $2$, the inputs to the \Verb|##phi|s of block $2$
have been processed once before.  For instance, we still believe that
\Verb|25| corresponds to the integer $0$ (which is incidentally correct), but
now that \Verb|41| has the value number \vn{35}, we think it corresponds to
the integer $1$.  While this is incorrect, it does break the congruence between
the inputs, making the first \Verb|##phi| a useful instruction.  The second
\Verb|##phi|, however, still looks like a copy of the first.  Even so, this
is sufficiently different that the following \Verb|##compare-integer| cannot
be constant-folded like before.  However, it can still be converted to a
\Verb|##compare-integer-imm|, as one of its operands corresponds to an
integer.  The redundant \Verb|##compare-imm-branch| gets rewritten to the
same expression as the \Verb|##compare-integer|, so winds up getting the same
value number.  This gives us:
%
\begin{align*}
  21 &\to \vn{21} \quad (0)                                                \\
  22 &\to \vn{22} \quad (100)                                              \\
  23 &\to \vn{21}                                                          \\
  24 &\to \vn{22}                                                          \\
  25 &\to \vn{21}                                                          \\
  26 &\to \vn{22}                                                          \\
  27 &\to \vn{21}                                                          \\
  29 &\to \vn{29} \quad (\text{\Verb|\#\#phi 2 21 35|})                    \\
  30 &\to \vn{29}                                                          \\
  31 &\to \vn{31} \quad (\text{\Verb|\#\#compare-integer-imm 29 100 cc<|}) \\
  32 &\to \vn{31}                                                          \\
  33 &\to \vn{22}                                                          \\
  34 &\to \vn{31}                                                          \\
  35 &\to \vn{35}                                                          \\
  36 &\to \vn{35}                                                          \\
  37 &\to \vn{35}                                                          \\
  38 &\to \vn{35}                                                          \\
  39 &\to \vn{21}                                                          \\
  40 &\to \vn{22}                                                          \\
  41 &\to \vn{35}                                                          \\
  42 &\to \vn{35}
\end{align*}

Block $3$ of iteration $2$ also changes, since the \Verb|##add|s can't be
constant-folded like before due to our new discovery about the \Verb|##phi|s.
However, the first one can still be converted to an \Verb|##add-imm|, and the
second is marked the same as the first.  This leaves the following value
numbers:
%
\begin{align*}
  21 &\to \vn{21} \quad (0)                                                \\
  22 &\to \vn{22} \quad (100)                                              \\
  23 &\to \vn{21}                                                          \\
  24 &\to \vn{22}                                                          \\
  25 &\to \vn{21}                                                          \\
  26 &\to \vn{22}                                                          \\
  27 &\to \vn{21}                                                          \\
  29 &\to \vn{29} \quad (\text{\Verb|\#\#phi 2 21 35|})                    \\
  30 &\to \vn{29}                                                          \\
  31 &\to \vn{31} \quad (\text{\Verb|\#\#compare-integer-imm 29 100 cc<|}) \\
  32 &\to \vn{31}                                                          \\
  33 &\to \vn{22}                                                          \\
  34 &\to \vn{31}                                                          \\
  35 &\to \vn{35} \quad (1)                                                \\
  36 &\to \vn{36} \quad (\text{\Verb|\#\#add-imm 29 1|})                   \\
  37 &\to \vn{35}                                                          \\
  38 &\to \vn{36}                                                          \\
  39 &\to \vn{29}                                                          \\
  40 &\to \vn{22}                                                          \\
  41 &\to \vn{36}                                                          \\
  42 &\to \vn{36}
\end{align*}

Since the value numbers changed, we start iteration $3$.  The expressions are
cleared, and block $1$ once again does not change anything.  The first
\Verb|##phi| in block $2$ still gets classified as useful, so no value
numbers change.  The major difference, though, is that the previous iteration's
value numbers for registers in block $3$ update the expression we have for the
\Verb|##phi|.  Whereas before we thought it was choosing between \vn{21} (the
integer $0$) and \vn{35} (the integer $1$), the \Verb|##add| wasn't
constant-folded in the previous iteration.  Therefore, the virtual register
\Verb|41| now corresponds to the result of the \Verb|##add| with the value
number \vn{36}.  We still can't disprove that the second \Verb|##phi| is
different (because it, in fact, isn't).  So, we're left with the following
after iteration $3$ finishes with block $2$:
%
\begin{align*}
  21 &\to \vn{21} \quad (0)                                                \\
  22 &\to \vn{22} \quad (100)                                              \\
  23 &\to \vn{21}                                                          \\
  24 &\to \vn{22}                                                          \\
  25 &\to \vn{21}                                                          \\
  26 &\to \vn{22}                                                          \\
  27 &\to \vn{21}                                                          \\
  29 &\to \vn{29} \quad (\text{\Verb|\#\#phi 2 21 36|})                    \\
  30 &\to \vn{29}                                                          \\
  31 &\to \vn{31} \quad (\text{\Verb|\#\#compare-integer-imm 29 100 cc<|}) \\
  32 &\to \vn{31}                                                          \\
  33 &\to \vn{22}                                                          \\
  34 &\to \vn{31}                                                          \\
  35 &\to \vn{35}                                                          \\
  36 &\to \vn{36}                                                          \\
  37 &\to \vn{35}                                                          \\
  38 &\to \vn{36}                                                          \\
  39 &\to \vn{29}                                                          \\
  40 &\to \vn{22}                                                          \\
  41 &\to \vn{36}                                                          \\
  42 &\to \vn{36}
\end{align*}

Blocks $3$ and $4$ do not produce any more changes, so \gls{GVN} has stabilized
after $3$ iterations, with our final congruence classes being:
%
\begin{align*}
  \vn{21} &= \{21, 23, 25, 27\}     \\
  \vn{22} &= \{22, 24, 26, 33, 40\} \\
  \vn{29} &= \{29, 30, 39\}         \\
  \vn{31} &= \{31, 32, 34\}         \\
  \vn{35} &= \{35, 37\}             \\
  \vn{36} &= \{36, 38, 41, 42\}
\end{align*}

\todo[inline]{teletype the numbers in the align*s, I guess}

%\section{Redundancy Elimination}\label{sec:vn:avail}

Now that we've identified congruences across the entire \gls{CFG}, we must
eliminate any redundancies found.  Since value numbering is now offline, this
entails another pass.  However, replacing instructions is more subtle with
global value numbers than it is with local ones.  Because values come from all
over the \gls{CFG}, we must consider if a definition is \term{available} at the
point where we want to use it.  

\Vref{fig:not-avail,fig:is-avail} show the difference.  In the former, we can
see the \gls{CFG} before value numbering for the code
%
\factor|[ 10 ] [ 20 ] if 10 20 30|.
%
The two extra integers being pushed at the end don't really illustrate the
point; they're just there to avoid branch splitting (see
\cref{sec:compiler:cfg}).  In block $4$, there's a
%
\Verb|##load-integer 27 10|,
%
which loads the value \Verb|10|.  In globally numbering values, we associate
the
%
\Verb|##load-integer 22 10|
%
in block $2$ with the value \Verb|10| first, making it the canonical
representative.  However, we can't replace the instruction in block $4$ with
%
\Verb|##copy 27 22|,
%
because control flow doesn't necessarily go through block $2$, so the virtual
register \Verb|22| might not even be defined.  However, in \vref{fig:is-avail},
we see the \gls{CFG} for the code
%
\factor|10 swap [ 10 ] [ 20 ] if 10 20 30|.
%
In this case, the first definition of the value \Verb|10| comes from block $1$,
which dominates block $4$.  So, the definition is available, and we can replace
the \Verb|##load-integer| in block $4$ with a \Verb|##copy|.

\inputfig{not-avail}
\inputfig{is-avail}

There are several ways to decide if we can use a definition at a certain point.
For instance, we could use dominator information, so that a definition in a
basic block $B$ can be used by any other block dominated by $B$
\autocite{Simpson}.  However, here we'll use a data flow analysis called
\term{available expression analysis}, since it is readily implemented.
Mercifully, Factor has a vocabulary that automatically defines data flow
analyses with little more than a single line of code.

\Vref{lst:avail} shows the vocabulary that defines the available expression
analysis.  It is a forward analysis \autocite[see][]{DragonBook} based on the
flow equations below:
\begin{align*}
  \text{\Verb|avail-in|}_i &=
    \begin{cases}
      \varnothing
        & \text{if $i=0$} \\
      \bigcap_{j\in\mathrm{pred}(i)}\text{\Verb|avail-out|}_j
        & \text{if $i>0$}
    \end{cases} \\
  \text{\Verb|avail-out|}_i &= \text{\Verb|avail-in|}_i
                                 \cup 
                                 \text{\Verb|defined|}_i
\end{align*}
%
\noindent Here, the subscripts indicate the basic block number.
$\text{\Verb|defined|}_i$ denotes the result of the \Verb|defined| word from
\vref{lst:avail}.  This returns the set of virtual registers defined in a basic
block.  Since we use virtual registers as value numbers, this is the same as
giving us all the value numbers produced by a basic block.  ``Killed''
definitions are impossible by the \gls{SSA} property, so we needn't track
redefinitions of virtual registers, as in other data flow analyses.  Using set
intersection as the confluence operator means that the
$\text{\Verb|avail-in|}_i$ set will contain those values which are available on
all paths from the start of the \gls{CFG} to block $i$.

\inputlst{avail}

\begin{sloppypar}
Using Factor's \Verb|compiler.cfg.dataflow-analysis| vocabulary, the
implementation takes all of two lines of code.  The
%
\factor|FORWARD-ANALYSIS: avail|
%
line automatically defines several objects, variables, words, and methods that
don't warrant full detail here.  One we're immediately concerned with is the
\Verb|transfer-set| generic, which dispatches upon the particular type of
analysis being performed and is invoked on the proper in-set and basic block.
There is no default implementation, as it is the chief difference between
analyses.  So, the next line uses \Verb|defined| and \factor|assoc-union| to
calculate the result of the data flow equation.  Other pieces we'll see used
are the top-level \Verb|compute-avail-sets| word that actually performs the
analysis, the \Verb|avail-ins| hash table that maps basic blocks to their
in-sets, and the \Verb|avail-in| word that is shorthand for looking up a
basic block's in-set.
\end{sloppypar}

We want to use the results of this analysis in the \Verb|rewrite| methods so
that they will only perform correct and meaningful rewrites.  However, we also
want to use \Verb|rewrite| in the \Verb|determine-value-numbers| pass, where we
don't care about availability.  In fact, we want to ignore availability
altogether in that pass, so that we can discover as many congruences as
possible.  In order to separate these concerns, we need to have two modes for
checking availability.  \Vref{lst:avail} defines the \Verb|available?| word to
do just this.  It will only check the actual availability if
\Verb|final-iteration?| is true, otherwise defaulting to \factor|t|.
Therefore, during the value numbering phase, everything is considered
available.  We further define the utilities \Verb|available-uses?| and
\Verb|with-available-uses?|.  The former checks if all an instruction's uses
are available, and the latter does this only if another quotation first returns
a true value.  That way, we can guard instruction predicates with a test for
available uses, like
%
\factor|[ ##add-imm? ] with-available-uses?|.

Finding all the instances where \Verb|rewrite| needed to be altered was subtle.
Since the old \Verb|value-numbering| was an online optimization, it didn't need
to worry about modifying an instruction in memory.  But by doing the
fixed-point iteration, we cannot permit \Verb|rewrite| to destructively modify
any object until the final iteration.  An obvious instance was in
\Verb|compiler.cfg.value-numbering.comparisons| with the word
\Verb|fold-branch|, responsible for modifying the \gls{CFG} to remove an
untaken branch.  We definitely would not want the branch removed while doing
the fixed-point iteration, because the transformation is not necessarily sound.
So, we can protect it with a check for \Verb|final-iteration?|, as in
\vref{lst:fold-branch}.

\inputlst{fold-branch}

\begin{sloppypar}
More typical instances of the problems that occurred were in words like
\Verb|self-inverse| from \Verb|compiler.cfg.value-numbering.math| (refer to
\vref{lst:self-inverse}).  The idea is essentially to change
%
\begin{center}
  \begin{minipage}{0.2\linewidth}
    \begin{factorcode*}{gobble=6,frame=none}
      ##neg 1 2
      ##neg 3 1
    \end{factorcode*}
  \end{minipage}
\end{center}
%
\noindent into
%
\begin{center}
  \begin{minipage}{0.2\linewidth}
    \begin{factorcode*}{gobble=6,frame=none}
      ##neg 1 2
      ##copy 3 2 any-rep
    \end{factorcode*}
  \end{minipage}
\end{center}
%
\noindent since \Verb|##neg| undoes itself.  But \Verb|rewrite| only has
knowledge of one instruction at a time, so it looks up the redundant
\Verb|##neg|'s source register in the \Verb|vns>insns| table to see if it's
computed by another \Verb|##neg| instruction.  For straight-line code this is
alright, but the input to the originating \Verb|##neg| (in the example, the
virtual register \Verb|2|) isn't necessarily available.  So, we have to use
\Verb|with-available-uses?| to make sure the virtual registers used by the
result of a \Verb|vreg>insn| can themselves be utilized before we rewrite
anything.
\end{sloppypar}

\inputlst{self-inverse}

An even subtler issue that led to infinite loops occured in simplifcations like
the arithmetic distribution in \Verb|compiler.cfg.value-numbering.math|.  The
problem is that the \Verb|rewrite| method would generate instructions that
assigned to entirely brand new registers.  These, of course, would invariably
get value numbered, triggering a change in the \Verb|vregs>vns| table.  A new
iteration would begin, and (since it gets called on the same instructions as
the previous iteration) \Verb|rewrite| would generate new virtual registers
all over again.  Therefore, the \Verb|vregs>vns| table would never stop
changing.  As a stop-gap, distribution had to be disabled altogether until the
final iteration.

Armed with the correct rewrite rules, availability information, and global
value numbers, we can perform \gls{GCSE}.  The logic in the \Verb|gcse| generic
in \vref{lst:gcse} is similar to \Verb|process-instruction| from
\vref{lst:process-instruction} and \Verb|value-number| from
\vref{lst:gvn-value-number}.  Unlike \Verb|value-number|, we do return an
instruction (or sequence thereof) representing the replacement.  Thus, the
\Verb|array| method of \Verb|gcse| uses \factor|map| instead of \factor|each|,
to hold onto the resulting sequence when recursing into several instructions.

\inputlst{gcse}

\Verb|defs-available| is similar to \Verb|record-defs| from
\vref{lst:gvn-value-number}, except that value numbers have already stabilized,
so we don't call \Verb|set-vn|.  Instead, we use the \Verb|make-available|
word, which was the last one defined in \vref{lst:avail}.  We have to ensure
that after processing an instruction, any register it defines is available to
future instructions in the same block, thus enabling rewrites.  So, we add
the virtual register to that block's \Verb|avail-in| (which acts like a set,
even though it's implemented by a hash table by Factor's data flow analysis
framework).  \Verb|alien-call-insn|s, \Verb|##callback-inputs| instructions,
and instances of \Verb|##copy| don't get rewritten any further, so we simply
note that their definitions are available and move on.

The \Verb|?eliminate| word transforms an instruction into a \Verb|##copy|
of the canonical value number that computes it.  If the value number isn't
available, we don't do anything but post-process with \Verb|defs-available|.
If it is, a \Verb|##copy| is produced and its destination is made available.
Thus, \Verb|eliminate-redundancy| works like \Verb|check-redundancy| from
\vref{lst:gvn-value-number}.  We look up the expression computed by the
instruction in the \Verb|exprs>vns| table.  If it's there, we call
\Verb|?eliminate|, but otherwise we leave the instruction alone and make its
definitions available.

The rest of the logic mirrors that of \Verb|value-number|.  If the inputs to a
\Verb|##phi| are all congruent, we'll call \Verb|?eliminate| to transform it
into a \Verb|##copy| of its first input (without loss of generality).
Otherwise, we check for equivalent \Verb|##phi|s with
\Verb|eliminate-redundancy|.  Finally, the \Verb|insn| method will default to
calling \Verb|eliminate-redundancy| on instructions that define only one value,
much like how \Verb|value-number| worked.

\begin{sloppypar}
The main word that performs the pass is \Verb|eliminate-common-subexpressions|.
\Verb|final-iteration?| is turned on (set to \factor|t|), and we make sure to
compute the \Verb|avail-in| sets needed to make \Verb|available?| work.  Then,
using \Verb|simple-optimization|, we iterate over each basic block.  For each
instruction, we first use \Verb|simplify| (refer to \vref{lst:gvn-simplify}),
then call \Verb|gcse| on the rewritten instruction.  Thus, \Verb|rewrite| does
the work of simplifying instructions, then \Verb|gcse| cleans up redundant ones
by converting them into \Verb|##copy| instructions if possible.  The new
\Verb|value-numbering| word can be seen in \vref{lst:new-value-numbering}.
\end{sloppypar}

\inputlst{new-value-numbering}

\begin{sloppypar}
Consider for the last time the example
%
\factor|0 100 [ 1 fixnum+fast ] times|.
%
Again, we have the \gls{CFG} in \vref{fig:gcse-before}.  Making a final pass
with \Verb|eliminate-common-subexpressions| gives us the \gls{CFG} in
\vref{fig:gcse-after}.  Compared to the \gls{CFG} after the old
\Verb|value-numbering| word was called (see \vref{fig:value-numbering}), we
have identified several more redundancies:
\begin{itemize}
  \item The second \Verb|##phi| in block $2$ has been turned into a
  \Verb|##copy| of the first.
%
  \item The \Verb|##compare-integer| of block $2$ has been simplified to
  a \Verb|##compare-integer-imm|, since its operand \Verb|26| is both
  available and known to correspond to the integer value \Verb|100|.
%
  \item Similarly, we've managed to convert the \Verb|##compare-integer-branch|
  at the end of block $2$ into a \Verb|##compare-integer-imm-branch|.
%
  \item Because the \Verb|##phi|s have been recognized as copies (i.e., the
  induction variables are congruent), the second \Verb|##add| in block $3$ is
  turned into a \Verb|##copy| of the first (which itself is still turned into
  an \Verb|##add-imm|).
\end{itemize}
\end{sloppypar}

\inputfig{gcse-before}
\inputfig{gcse-after}

Afterwards, the \Verb|copy-propagation| pass cleans up all of these newly
identified copies, as seen in \vref{fig:gcse-copy-prop}.
\Verb|eliminate-dead-code| now gets rid of more instructions than before, such
as the second \Verb|##load-integer| in block $1$, since it has been propagated
to the \Verb|-imm| instructions in block $2$.  See \vref{fig:gcse-dce}.  At
last, after \Verb|finalize-cfg| in \vref{fig:gcse-finalize}, we see a loop that
uses a single register---down from the three in \vref{fig:finalize-cfg}.

\inputfig{gcse-copy-prop}
\inputfig{gcse-dce}
\inputfig{gcse-finalize}
\clearpage

%\section{Results}\label{sec:vn:results}

The goal of improving the optimization in Factor is, of course, to reduce the
average running time of programs, and to do so without changing their
semantics.  Short of formal verification, the latter requirement makes it
necessary to thoroughly test any code that gets compiled with the new pass
enabled.  To this end, we'll employ Factor's extensive unit test coverage.
Because Factor is (largely) self-hosting, its standard vocabularies are written
in Factor code, typically coupled with tests.  While some vocabularies will
have more test coverage than others, the total amount of tests is quite large.
By compiling each vocabulary and running their tests, we're indirectly testing
the compiler: if tests that used to pass no longer do, then the new pass is
changing the semantics of the code somehow.  Though passing all tests does not
guarantee the algorithm is correct, it does let us know that no known
regressions have been introduced.  Happily, with the new
\Verb|value-numbering| phase enabled, all the same tests pass as before in a
call to \factor|test-all| from a freshly bootstrapped image.

The efficacy of the changes, on the other hand, must be measured relative to
old benchmarks.  Again, Factor has its bases covered, with a suite of $80$
benchmarks run by the \Verb|benchmark| vocabulary.  Each benchmark is run $5$
times, where the garbage collector is run before each iteration.  The minimum
time from these runs is then used as the benchmark result.  The data below
comes from two separate runs of the \factor|benchmarks| word, which invokes all
the benchmark sub-vocabularies.  The ``before'' time used the local value
numbering, while ``after'' times had \Verb|value-numbering| replaced with the
\gls{GVN} pass.  The ``change'' is measured by the formula
%
$$\frac{\text{before} - \text{after}}{\text{before}} \times 100$$
%
to indicate the relative running times.  Negative values in this column are
good, as that means the running time has decreased.

\todo[inline]{Guess I should provide my PC's specs}

\begin{longtable}{llll}
\toprule
Benchmark & Before (seconds) & After (seconds) & Change (\%) \\
\midrule
\endhead
\texttt{benchmark.3d-matrix-scalar}         & 3.705816738       & 3.046126696         & $-17.80$    \\
\texttt{benchmark.3d-matrix-vector}         & 0.161298778       & 0.089539887         & $-44.49$    \\
\texttt{benchmark.backtrack}                & 4.280001561       & 2.358672591         & $-44.89$    \\
\texttt{benchmark.base64}                   & 5.127831493       & 2.853612485         & $-44.35$    \\
\texttt{benchmark.beust1}                   & 7.531546384       & 4.604929188         & $-38.86$    \\
\texttt{benchmark.beust2}                   & 20.308680548      & 12.843534349        & $-36.76$    \\
\texttt{benchmark.binary-search}            & 3.729776895       & 2.349520427         & $-37.01$    \\
\texttt{benchmark.binary-trees}             & 9.403166818       & 6.518867479         & $-30.67$    \\
\texttt{benchmark.bootstrap1}               & 32.472196349      & 30.887877896        & $-4.88$     \\
\texttt{benchmark.chameneos-redux}          & 2.923900422       & 2.041007328         & $-30.20$    \\
\texttt{benchmark.continuations}            & 0.273525202       & 0.200695972         & $-26.63$    \\
\texttt{benchmark.crc32}                    & 0.010623653       & 0.005282642         & $-50.27$    \\
\texttt{benchmark.dawes}                    & 1.588111926       & 1.027176578         & $-35.32$    \\
\texttt{benchmark.dispatch1}                & 7.640720326       & 5.106558985         & $-33.17$    \\
\texttt{benchmark.dispatch2}                & 5.221652668       & 3.984754032         & $-23.69$    \\
\texttt{benchmark.dispatch3}                & 9.710520454       & 6.203527737         & $-36.12$    \\
\texttt{benchmark.dispatch4}                & 8.224931156       & 4.098265543         & $-50.17$    \\
\texttt{benchmark.dispatch5}                & 4.74357434        & 3.478219608         & $-26.68$    \\
\texttt{benchmark.e-decimals}               & 3.903754723       & 2.646958072         & $-32.19$    \\
\texttt{benchmark.e-ratios}                 & 4.774454589       & 3.658075473         & $-23.38$    \\
\texttt{benchmark.empty-loop-0}             & 0.251816164       & 0.199189271         & $-20.90$    \\
\texttt{benchmark.empty-loop-1}             & 1.039242509       & 0.857588545         & $-17.48$    \\
\texttt{benchmark.empty-loop-2}             & 0.472215346       & 0.387974286         & $-17.84$    \\
\texttt{benchmark.euler150}                 & 37.785852299      & 27.05450689         & $-28.40$    \\
\texttt{benchmark.fannkuch}                 & 9.627490235       & 6.8970571           & $-28.36$    \\
\texttt{benchmark.fasta}                    & 7.25292282        & 5.640517069         & $-22.23$    \\
\texttt{benchmark.fib1}                     & 0.179389215       & 0.164933805         & $-8.06$     \\
\texttt{benchmark.fib2}                     & 0.205853157       & 0.138174211         & $-32.88$    \\
\texttt{benchmark.fib3}                     & 0.785036151       & 0.539739186         & $-31.25$    \\
\texttt{benchmark.fib4}                     & 0.391805799       & 0.260370111         & $-33.55$    \\
\texttt{benchmark.fib5}                     & 1.508625224       & 1.002724851         & $-33.53$    \\
\texttt{benchmark.fib6}                     & 19.202504502      & 13.146010511        & $-31.54$    \\
\texttt{benchmark.gc0}                      & 7.360087104       & 5.508594031         & $-25.16$    \\
\texttt{benchmark.gc1}                      & 0.418173431       & 0.281497214         & $-32.68$    \\
\texttt{benchmark.gc2}                      & 25.611210221      & 19.716168704        & $-23.02$    \\
\texttt{benchmark.gc3}                      & 2.757943071       & 2.210785891         & $-19.84$    \\
\texttt{benchmark.hashtables}               & 8.068216942       & 7.997106348         & $-0.88$     \\
\texttt{benchmark.heaps}                    & 4.360368411       & 4.32169158          & $-0.89$     \\
\texttt{benchmark.iteration}                & 7.875561986       & 6.277891729         & $-20.29$    \\
\texttt{benchmark.javascript}               & 17.881224721      & 12.74204052         & $-28.74$    \\
\texttt{benchmark.knucleotide}              & 5.490420772       & 3.5704101           & $-34.97$    \\
\texttt{benchmark.mandel}                   & 0.251711276       & 0.198695557         & $-21.06$    \\
\texttt{benchmark.matrix-exponential-scalar}& 16.451432774      & 12.017000042        & $-26.95$    \\
\texttt{benchmark.matrix-exponential-simd}  & 0.681684747       & 0.536850343         & $-21.25$    \\
\texttt{benchmark.md5}                      & 10.40516678       & 9.198666403         & $-11.60$    \\
\texttt{benchmark.mt}                       & 33.91981743       & 29.961085146        & $-11.67$    \\
\texttt{benchmark.nbody}                    & 9.203478441       & 6.795154145         & $-26.17$    \\
\texttt{benchmark.nbody-simd}               & 0.845814208       & 0.854773096         & $+1.06$     \\
\texttt{benchmark.nested-empty-loop-1}      & 0.097090973       & 0.068475608         & $-29.47$    \\
\texttt{benchmark.nested-empty-loop-2}      & 0.893126911       & 0.861327078         & $-3.56$     \\
\texttt{benchmark.nsieve}                   & 1.086110659       & 1.137648699         & $+4.75$     \\
\texttt{benchmark.nsieve-bits}              & 2.707271763       & 2.815509077         & $+4.00$     \\
\texttt{benchmark.nsieve-bytes}             & 0.785041878       & 1.211421146         & $+54.31$    \\
\texttt{benchmark.partial-sums}             & 3.762171661       & 4.130144177         & $+9.78$     \\
\texttt{benchmark.pidigits}                 & 2.182877913       & 2.195385034         & $+0.57$     \\
\texttt{benchmark.random}                   & 5.66540782        & 5.71913683          & $+0.95$     \\
\texttt{benchmark.raytracer}                & 5.047070171       & 4.39514879          & $-12.92$    \\
\texttt{benchmark.raytracer-simd}           & 1.072588515       & 0.980927338         & $-8.55$     \\
\texttt{benchmark.recursive}                & 2.703509403       & 2.529087637         & $-6.45$     \\
\texttt{benchmark.regex-dna}                & 2.208584014       & 1.808859571         & $-18.10$    \\
\texttt{benchmark.reverse-complement}       & 2.801163847       & 2.353254665         & $-15.99$    \\
\texttt{benchmark.ring}                     & 1.822206473       & 1.62482491          & $-10.83$    \\
\texttt{benchmark.sfmt}                     & 2.675838657       & 2.463367198         & $-7.94$     \\
\texttt{benchmark.sha1}                     & 11.964973943      & 11.142380303        & $-6.88$     \\
\texttt{benchmark.simd-1}                   & 1.857778672       & 1.703206011         & $-8.32$     \\
\texttt{benchmark.sockets}                  & 10.636346636      & 10.516448454        & $-1.13$     \\
\texttt{benchmark.sort}                     & 0.695635429       & 0.581855635         & $-16.36$    \\
\texttt{benchmark.spectral-norm}            & 3.433630383       & 2.960833789         & $-13.77$    \\
\texttt{benchmark.spectral-norm-simd}       & 2.743240011       & 3.237017655         & $+18.00$    \\
\texttt{benchmark.stack}                    & 1.580016742       & 2.004478602         & $+26.86$    \\
\texttt{benchmark.struct-arrays}            & 2.180774222       & 2.421915609         & $+11.06$    \\
\texttt{benchmark.sum-file}                 & 0.883097981       & 0.957151577         & $+8.39$     \\
\texttt{benchmark.terrain-generation}       & 1.611800222       & 1.887047663         & $+17.08$    \\
\texttt{benchmark.tuple-arrays}             & 0.262747557       & 0.329399609         & $+25.37$    \\
\texttt{benchmark.typecheck1}               & 1.750223408       & 1.674592158         & $-4.32$     \\
\texttt{benchmark.typecheck2}               & 1.674738245       & 1.553203741         & $-7.26$     \\
\texttt{benchmark.typecheck3}               & 1.891206648       & 1.735390184         & $-8.24$     \\
\texttt{benchmark.ui-panes}                 & 0.305595039       & 0.29985214          & $-1.88$     \\
\texttt{benchmark.xml}                      & 3.013709363       & 2.722223892         & $-9.67$     \\
\texttt{benchmark.yuv-to-rgb}               & 0.398174487       & 0.318891664         & $-19.91$    \\
\end{longtable}

The results are promising: of $80$ benchmarks, only $13$ showed any increase in
running time.  And of those, even fewer showed significant increases.
Duplicated below for convenience are the benchmarks that ran slower, sorted in
decreasing order of the percent difference between running times.  We can see
the last five or six benchmarks exhibited negligible differences---not only is
the relative change tiny, but the absolute difference in running times is less
than $0.1$ seconds.  (The \Verb|benchmark.tuple-arrays| results also show a
similar absolute change, but it is relatively much larger.)

\begin{longtable}{llll}
\toprule
Benchmark & Before (seconds) & After (seconds) & Change (\%) \\
\midrule
\endhead
\texttt{benchmark.nsieve-bytes}             & 0.785041878       & 1.211421146         & $+54.31$    \\
\texttt{benchmark.stack}                    & 1.580016742       & 2.004478602         & $+26.86$    \\
\texttt{benchmark.tuple-arrays}             & 0.262747557       & 0.329399609         & $+25.37$    \\
\texttt{benchmark.spectral-norm-simd}       & 2.743240011       & 3.237017655         & $+18.00$    \\
\texttt{benchmark.terrain-generation}       & 1.611800222       & 1.887047663         & $+17.08$    \\
\texttt{benchmark.struct-arrays}            & 2.180774222       & 2.421915609         & $+11.06$    \\
\texttt{benchmark.partial-sums}             & 3.762171661       & 4.130144177         & $+9.78$     \\
\texttt{benchmark.sum-file}                 & 0.883097981       & 0.957151577         & $+8.39$     \\
\texttt{benchmark.nsieve}                   & 1.086110659       & 1.137648699         & $+4.75$     \\
\texttt{benchmark.nsieve-bits}              & 2.707271763       & 2.815509077         & $+4.00$     \\
\texttt{benchmark.nbody-simd}               & 0.845814208       & 0.854773096         & $+1.06$     \\
\texttt{benchmark.random}                   & 5.66540782        & 5.71913683          & $+0.95$     \\
\texttt{benchmark.pidigits}                 & 2.182877913       & 2.195385034         & $+0.57$     \\
\end{longtable}

Overall, even transitioning to a relatively simple \gls{GVN} algorithm amounts
to a positive change in Factor's compiler.  More redundancies are eliminated,
resulting in speedier programs.  Judging by unit tests, the implementation is
at least as sound as the previous local value numbering,  as all the same tests
have passed.

% Future
%   SCC (discussion of potential improvement)
%   Click, Gargi, et al. (future directions)


\documentclass[11pt,article,oldfontcommands]{cpp-thesis}

\setlength{\parskip}{\medskipamount}

\title{Global Value Numbering in Factor}
\author{Alex Vondrak}
\degree{Master of Science}
\field{Computer Science}
\chair{Dr. Craig Rich}{Computer Science}
\memberA{Dr. Daisy Sang}{Computer Science}
\memberB{Dr. Amar Raheja}{Computer Science}
\quarter{Summer 2011}

\usepackage{todonotes}

\usepackage{bold-extra}

\usepackage{glossaries}

\usepackage{fullpage}
\usepackage{indentfirst}

\usepackage{tikz}
%\usetikzlibrary{external}
%\tikzexternalize

\usepackage{lib/minted}
\usemintedstyle{bw}
\newminted{factor}{gobble=4,frame=single}
\newmint{factor}{}

\usepackage{float}
\floatstyle{ruled}
\newfloat{algorithm}{tbp}{alg}
\floatname{algorithm}{Algorithm}

%TODO figure out the "right" way to do terminology typesetting
\newcommand{\term}[1]{\emph{#1}}

\newcommand{\inputfig}[1]{
  \begin{figure}
    \input{fig/#1}
    \label{fig:#1}
  \end{figure}
}

\newcommand{\inputlst}[2][tbph]{
  \begin{listing}[#1]
    \input{lst/#2}
    \label{lst:#2}
  \end{listing}
}

\newcommand{\inputalg}[1]{
  \begin{algorithm}
    \input{alg/#1}
    \label{alg:#1}
  \end{algorithm}
}

\usepackage{array}
\newcolumntype{k}[1]{|>{\centering\arraybackslash\hspace{0pt}}p{#1}|}

\usepackage{varioref}
\usepackage{hyperref}
\usepackage[capitalise]{cleveref}
\def\reftextcurrent{\unskip}
\crefname{figure}{Figure}{Figures}
\crefname{subfigure}{Figure}{Figures}

\usepackage{graphicx}
\newcommand{\whenNewer}[3]{%
  \ifnum\pdfstrcmp{\pdffilemoddate{#1}}%
  {\pdffilemoddate{#2}}>0%
  {\immediate\write18{#3}}\fi%
}
\newcommand{\includesvg}[2][]{%
  \whenNewer{#2.svg}{#2.pdf}%
    {inkscape -z -D --file=#2.svg --export-pdf=#2.pdf}%
  \includegraphics[#1]{#2.pdf}%
}

\newacronym{JVM}{JVM}{Java Virtual Machine}
\newacronym{RPN}{RPN}{Reverse Polish Notation}
\newacronym{TCO}{TCO}{tail-call optimization}
\newacronym{GC}{GC}{garbage collector}
\newacronym{FFI}{FFI}{foreign function interface}
\newacronym{VM}{VM}{virtual machine}
\newacronym{IR}{IR}{intermediate representation}
\newacronym{CFG}{CFG}{control flow graph}
\newacronym{SSA}{SSA}{static single assignment}


\begin{document}

\frontmatter

\begin{ack}
  \null
  % Bah; ack header effectively kills joke---I want a dedication, dammit
  \begin{vplace}[0.1]
    \begin{center}
      \textit{\Large To Lindsay---he is my rock}
    \end{center}
  \end{vplace}
\end{ack}

\begin{abstract}
  \normalsize
  Compilers translate code in one programming language into semantically
  equivalent code in another language---canonically from a high-level language
  to low-level machine primitives.  Generally, the further removed a
  language's abstractions get from those of a computer, the harder it gets to
  compile code into an efficient representation.  What isn't redundant in the
  source language may map to repetitive target instructions that waste time
  recomputing results.  To combat this, compilers try to optimize away
  redundancies by looking for values that are provably equivalent when the
  program is run.

  This thesis explores the theory and implementation of a particularly
  aggressive analysis called global value numbering in a particularly
  high-level language called Factor.  Factor is a stack-based,
  dynamically-typed, object-oriented language born in late $2003$.  A baby
  among languages (now at version $0.94$), its compiler craves all the
  optimizations it can get.  By altering the existing local value numbering
  pass, redundancies can be identified and eliminated across entire programs,
  rather than isolated regions of code.  This induces speedups as high as
  $45\%$ across the majority of benchmarks.  The results from these
  comparatively simple changes hold much promise for future improvements in
  making Factor programs more efficient.
  \newpage
\end{abstract}

\tableofcontents*
\listoffigures

\mainmatter

        \chapter{Introduction}\label{sec:intro}

Compilers translate programs written in a source language (e.g., Java) into
semantically equivalent programs in some target language (e.g., assembly code).
They let us make our source language arbitrarily abstract so we can write
programs in ways that humans understand while letting the computer execute
programs in ways that machines understand.  In a perfect world, such
translation would be straightforward.  Reality, however, is unforgiving.
Straightforward compilation results in clunky target code that performs a lot
of redundant computations.  To produce efficient code, we must rely on
less-than-straightforward methods.  Typical compilers go through a stage of
\term{optimization}, whereby a number of semantics-preserving transformations
are applied to an \term{\acrlong{IR}} of the source code.  These then
(hopefully) produce a more efficient version of said representation.
Optimizers tend to work in \term{phases}, applying specific transformations
during any given phase.

\Gls{GVN} is such an analysis performed by many highly-optimizing compilers.
Its roots run deep through both the theoretical and the practical.  Using the
results of this analysis, the compiler can identify expressions in the source
code that produce the same value---not just by lexical comparison (i.e.,
variables having the same name), but by proving equivalences between what's
actually computed at runtime.  These expressions can then be simplified by
further algorithms for redundancy elimination.  This is the very essence of
most compiler optimizations: avoid redundant computation, giving us code that
runs as quickly as possible while still following what the programmer
originally wrote.

High-level, dynamic languages tend to suffer from efficiency issues: they're
often interpreted rather than compiled, and perform no heavy optimization of
the source code.  However, the Factor language (\url{http://factorcode.org})
fills an intriguing design niche, as it's very high-level yet still fully
compiled.  It's still young, though, so its compiler craves all the
improvements it can get.  In particular, while Factor currently has a
\term{local} value numbering analysis, it is inferior to \gls{GVN} in several
significant ways.

In this thesis, we explore the implementation and use of \gls{GVN} in improving
the strength of optimizations in Factor.  Because Factor is a young and
relatively unknown language, \cref{sec:primer} provides a short tutorial,
laying a foundation for understanding the changes.  \Cref{sec:compiler}
describes the overall architecture of the Factor compiler, highlighting where
the exact contributions of this thesis fit in.  Finally, \cref{sec:vn} goes
into detail about the existing and new value numbering passes, closing with a
look at the results achieved and directions for future work. 

All the code for the \gls{GVN} phase was written atop Factor version $0.94$,
and a copy of it can be found in the appendix\todo{ref}.  In the unlikely event
that you want to cite this thesis, you may use the following
\textsc{Bib}\TeX~entry\todo{make sure it's single-page}:
\begin{Verbatim}[gobble=2,frame=single]
  @mastersthesis{vondrak:11,
    author = {Alex Vondrak},
    title  = {Global Value Numbering in Factor},
    school = {California Polytechnic State University, Pomona},
    month  = sep,
    year   = {2011},
  }
\end{Verbatim}

%\newpage\chapter{Language Primer}\label{sec:primer}

Factor is a rather young language created by Slava Pestov in September 2003
\autocite{young-factor}.  Its first incarnation was an embedded scripting
language for a game that targeted the \gls{JVM}.  As such, its feature set was
minimal.  Factor has since evolved into a general-purpose programming language,
gaining new features and redesigning old ones as necessary for larger programs.
Today's implementation sports an extensive standard library and has moved away
from the \gls{JVM} in favor of native code generation.  In this
\lcnamecref{sec:primer}, we cover the basic syntax and semantics of Factor for
those unfamiliar with the language.  This should be just enough to understand
the later material in this thesis.  More thorough documentation can be found
via Factor's website, \url{http://factorcode.org}.

\section{Stack-Based Languages}\label{sec:primer:stack-based}

\inputfig{rpn}

Like \gls{RPN} calculators, Factor's evaluation model uses a global stack upon
which operands are pushed before operators are called.  This naturally
facilitates postfix notation, in which operators are written after their
operands.  For example, instead of \texttt{1~+~2}, we write \texttt{1~2~+}.
\vref{fig:rpn} shows how \texttt{1~2~+} works conceptually:
\begin{itemize}
  \item \texttt{1} is pushed onto the stack
  \item \texttt{2} is pushed onto the stack
  \item \texttt{+} is called, so two values are popped from the stack, added,
        and the result (\texttt{3}) is pushed back onto the stack
\end{itemize}
Other stack-based programming languages include Forth \autocite{Forth},
Cat \autocite{Cat}, and PostScript \autocite{PostScript}.

The strength of this model is its simplicity.  Evaluation essentially goes
left-to-right: literals (like \texttt{1} and \texttt{2}) are pushed onto the
stack, and operators (like \texttt{+}) perform some computation using values
currently on the stack.  This ``flatness'' makes parsing easier, since we don't
need complex grammars with subtle ambiguities and precedence issues.  Rather,
we basically just scan left-to-right for tokens separated by whitespace.  In
the Forth tradition, functions are called \term{words} since they're made up of
any contiguous non-whitespace characters.  This also lends to the term
\term{vocabulary} instead of ``module'' or ``library''.   In Factor, the parser
works as follows.
\begin{itemize}
  \item If the current character is a double-quote (\texttt{"}), try to
        parse ahead for a string literal.
  \item Otherwise, scan ahead for a single token.
        \begin{itemize}
          \item If the token is the name of a \term{parsing word}, that word is
                invoked with the parser's current state.
          \item If the token is the name of an ordinary (i.e., non-parsing)
                word, that word is added to the parse tree.
          \item Otherwise, try to parse the token as a numeric literal.
        \end{itemize}
\end{itemize}

Parsing words serve as hooks into the parser, letting Factor users extend the
syntax dynamically.  For instance, instead of having special knowledge of
comments built into the parser, the parsing word \texttt{!}~scans forward for a
newline and discards any characters read (adding nothing to the parse tree).

Similarly, there are parsing words for what might otherwise be hard-coded
syntax for data structure literals.  Many act as sided delimeters: the parsing
word for the left-delimiter will parse ahead until it reaches the
right-delimiter, using whatever was read in between to add objects to the data
structure.  For example, \factor|{ 1 2 3 }| denotes an array of three numbers.
Note the deliberate spaces in between the tokens, so that the delimeters are
themselves distinct words.  In
%
\Verb[showspaces]|{ 1 2 3 }| (with spaces as marked), the parsing word \Verb|{|
parses objects until it reaches \Verb|}|, collecting the results into an array.
The \verb|{| word would not be called if not for that space, whereas
%
\Verb[showspaces]|{1 2 3}| parses as the word \Verb|{1|, the number \Verb|2|,
and the word \Verb|3}|---not an array.  Further, since the left-delimeter words
parse recursively, such literals can be nested, contain comments, etc.  Other
literals include \vpageref[the following][those in
\vref{lst:literals}~]{lst:literals}.

\inputlst[h]{literals}

\inputfig{quot}

A particularly important set of parsing words in Factor are the square
brackets, \Verb|[| and \Verb|]|.  Any code in between such brackets is
collected up into a special sequence called a \term{quotation}.  Essentially,
it's a snippet of code whose execution is suppressed.  The code inside a
quotation can then be run with the \factor|call| word.  Quotations are like
anonymous functions in other languages, but the stack model makes them
conceptually simpler, since we don't have to worry about variable binding and
the like.  Consider a small example like \factor|1 2 [ + ] call|.  You can
think of \factor|call| working by ``erasing'' the brackets around a quotation,
so this example behaves just like \factor|1 2 +|.  \vref{fig:quot} shows its
evaluation: instead of adding the numbers immediately, \factor|+| is placed in
a quotation, which is pushed to the stack.  The quotation is then invoked by
\factor|call|, so \factor|+| pops and adds the two numbers and pushes the
result onto the stack.  We'll show how quotations are used in
\cref{sec:primer:combinators}.

\subsection{Stack Effects}\label{sec:primer:effects}

Everything else about Factor follows from the stack-based structure outlined in
\cref{sec:primer:stack-based}.  Consecutive words transform the stack in
discrete steps, thereby shaping a result.  In a way, words are functions from
stacks to stacks---from ``before'' to ``after''---and whitespace is effectively
function composition.  Even literals (numbers, strings, arrays, quotations,
etc.) can be thought of as functions that take in a stack and return that stack
with an extra element pushed onto it.

With this in mind, Factor requires that the number of elements on the stack
(the \term{stack height}) is known at each point of the program in order to
ensure consistency.  To this end, every word is associated with a \term{stack
effect} declaration using a notation implemented by parsing words.  In general,
a stack effect declaration has the form
%
\begin{center} \factor|( input1 input2 ... -- output1 output2 ... )|
\end{center}
%
\noindent where the parsing word \Verb|(| scans forward for the special token
\Verb|--| to separate the two sides of the declaration, and then for the
\Verb|)| token to end the declaration.  The names of the intermediate tokens
don't technically matter---only how many of them there are.  However, names
should be meaningful for clarity's sake.  The number of tokens on the left side
of the declaration (before the \Verb|--|) indicates the minimum stack height
expected before executing the word.  Given exactly this number of inputs, the
number of tokens on the right side is the stack height after executing the
word.

For instance, the stack effect of the \factor|+| word is
%
\factor|( x y -- z )|,
%
as it pops two numbers off the stack and pushes one number (their sum) onto the
stack.  This could be written any number of ways, though.
%
\factor|( x x -- x )|,
%
\factor|( number1 number2 -- sum )|,
%
and
%
\factor|( m n -- m+n )|
%
are all equally valid.  Further, while the stack effect
%
\factor|( junk x y -- junk z )|
%
has the same relative height change, this declaration would be wrong, since
\factor|+| might legitimately be called on only two inputs.

\inputfig{shufflers}

For the purposes of documentation, of course, the names in stack effects do
matter.  They correspond to elements of the stack from bottom-to-top.  So, the
rightmost value on either side of the declaration names the top element of the
stack.  We can see this in \vref{fig:shufflers}, which shows the effects of
standard \term{stack shuffler} words.  These words are used for basic data flow
in Factor programs.  For example, to discard the top element of the stack, we
use the \factor|drop| word, whose effect is simply
%
\factor|( x -- )|.
%
To discard the element just below the top of the stack, we use \factor|nip|,
whose effect is
%
\factor|( x y -- y )|.
%
This stack effect indicates that there are at least two elements on the stack
before \factor|nip| is called: the top element is \factor|y|, and the next
element is \factor|x|.  After calling the word, \factor|x| is removed, leaving
the original \factor|y| still on top of the stack.  Other shuffler words that
remove data from the stack are
%
\factor|2drop| with the effect \factor|( x y -- )|,
%
\factor|3drop| with the effect \factor|( x y z -- )|, and
%
\factor|2nip| with the effect \factor|( x y z -- z )|.

The next stack shufflers duplicate data.  \factor|dup| copies the top element
of the stack, as indicated by its effect \factor|( x -- x x )|.  \factor|over|
has the effect \factor|( x y -- x y x )|, which tells us that it expects at
least two inputs: the top of the stack is \factor|y|, and the next object is
\factor|x|.  \factor|x| is copied and pushed on top of the two original
elements, sandwiching \factor|y| between two \factor|x|s.  Other shuffler words
that duplicate data on the stack are
%
\factor|2dup| with the effect \factor|( x y -- x y x y )|,
%
\factor|3dup| with the effect \factor|( x y z -- x y z x y z )|,
%
\factor|2over| with the effect \factor|( x y z -- x y z x y )|, and
%
\factor|pick| with the effect \factor|( x y z -- x y z x )|.

True to the name \factor|swap|, the final shuffler in \vref{fig:shufflers}
permutes the top two elements of the stack, reversing their order.  The stack
effect
%
\factor|( x y -- y x )|
%
indicates as much.  The left side denotes that two inputs are on the stack (the
top is \factor|y|, the next is \factor|x|), and the right side shows the
outputs are swapped (the top element is \factor|x| and the next is \factor|y|).
Factor has other words that permute elements deeper into the stack.  However,
their use is discouraged because it's harder for the programmer to mentally
keep track of more than a couple items on the stack.  We'll see how more
complex data flow patterns are handled in \cref{sec:primer:data-flow}.

\subsection{Definitions}\label{sec:primer:colon-defs}

\inputlst{hello-world}

\inputlst{norm}

Using the basic syntax of stack effect declarations described in
\cref{sec:primer:effects}, we can now understand how to define words.  Most
words are defined with the parsing word \factor|:|, which scans for a name, a
stack effect, and then any words up until the \factor|;| token, which together
become the body of the definition.  Thus, the classic example in
\vref{lst:hello-world} defines a word named \factor|hello-world| which expects
no inputs and pushes no outputs onto the stack.  When called, this word will
display the canonical greeting on standard output using the \factor|print|
word.

\inputfig{norm-steps}

A slightly more interesting example is the \factor|norm| word in
\vref{lst:norm}.  This squares each of the top two numbers on the stack, adds
them, then takes the square root of the sum.  \vref{fig:norm-steps} shows this
in action.  By defining a word to perform these steps, we can replace virtually
any instance of
%
\factor|dup * swap dup * + sqrt|
%
in a program simply with \factor|norm|.  This is a deceptively important point.
Data flow is made explicit via stack manipulation rather than being hidden in
variable assignments, so repetitive patterns become painfully evident.  This
makes identifying, extracting, and replacing redundant code easy.  Often, you
can just copy a repetitive sequence of words into its own definition verbatim.
This emphasis on ``factoring'' your code is what gives Factor its name.

\inputlst{norm-factored}

As a simple case in point, we see the subexpression \factor|dup *| appears
twice in the definition of \factor|norm| in \vref{lst:norm}.  We can easily
factor that out into a new word and substitute it for the old expressions, as
in \vref{lst:norm-factored}.  By contrast, programs in more traditional
languages are laden with variables and syntactic noise that require more work
to refactor: identifying free variables, pulling out the right functions
without causing finicky syntax errors, calling a new function with the right
variables, etc.  Though Factor's stack-based paradigm is atypical, it is part
of a design philosophy that aims to facilitate readable code focusing on short,
reusable definitions.

\subsection{Object Orientation}\label{sec:primer:oo}

You may have noticed that the examples in \cref{sec:primer:colon-defs} did not
use type declarations.  While Factor is dynamically typed for the sake of
simplicity, it does not do away with types altogether.  In fact, Factor is
object-oriented.  However, its object system doesn't rely on classes possessing
particular methods, as is common.  Instead, it uses \term{generic words} with
methods implemented for particular classes.  To start, though, we must see how
classes are defined.

\subsubsection{Tuples}

\inputlst{tuples}

The central data type of Factor's object system is called a \term{tuple}, which
is a class composed of named \term{slots}---like instance variables in other
languages.  Tuples are defined with the \factor|TUPLE:| parsing word as shown
in \vref{lst:tuples}.  A class name is specified first; if it is followed by
the \factor|<| token and a superclass name, the tuple inherits the slots of the
superclass.  If no superclass is specified, the default is the \factor|tuple|
class.

Slots can be specified in several ways.  The simplest is to just provide a
single token, which is the name of the slot.  This slot can then hold any type
of object.  Using the syntax
%
\factor|{ name class }|,
%
a slot can be limited to hold only instances of a particular class, like
\factor|integer| or \factor|string|.  There are other forms of slot specifiers,
which we will cover after some examples.

\inputlst{colors}

Consider the two tuples defined in \vref{lst:colors}.  The first,
\factor|color|, has no slots.  With every tuple, a class predicate is defined
with the stack effect
%
\verb|( object -- ? )|
%
whose name is the class suffixed by a question mark.  Here, the word
\factor|color?| is defined, which pushes a boolean (in Factor, either
\factor|t| or \factor|f|) indicating whether the top element of the stack is an
instance of the \factor|color| class.  The second tuple, \factor|rgb|, inherits
from the \factor|color| class.  While this doesn't give \factor|rgb| any
different slots, it does mean that an instance of \factor|rgb| will return
\factor|t| for \factor|color?| due to the ``is-a'' relationship between
subclass and superclass.  The word \factor|rgb?| is similarly defined.

Notice that the \factor|rgb| tuple declares three slots named \factor|red|,
\factor|green|, and \factor|blue|.  Since the slots' classes aren't declared,
any sort of object can be stored in them.  A set of methods are defined to
manipulate an \factor|rgb| instance's slots.  Three \term{reader} words are
defined (one for each slot), analogous to ``getter'' methods in other
languages.  Following the template for naming reader words, this example
defines \factor|red>>|, \factor|green>>|, and \factor|blue>>|.  Each word has
the stack effect
%
\factor|( object -- value )|,
%
and extracts the value corresponding to the eponymous slot.  Similarly, the
\term{writer} words \factor|red<<|, \factor|green<<|, and \factor|blue<<| each
have the stack effect
%
\factor|( value object -- )|,
%
and store values in the corresponding \factor|rgb| slots destructively.  To
leave the modified \factor|rgb| instance on the stack while setting slots, the
\term{setter} words \factor|>>red|, \factor|>>green|, and \factor|>>blue| are
also defined, each with the stack effect
%
\factor|( object value -- object' )|.
%
These words are defined in terms of writers.  For instance, \factor|>>red| is
the same as \factor|over red<<|, since \factor|over| copies a reference to the
tuple (i.e., it doesn't make a ``deep'' copy).

To construct an instance of a tuple, we can use either \factor|new| or
\factor|boa|.  \factor|new| will not initialize any of the slots to a
particular input value---all slots will default to Factor's canonical false
value, \factor|f|.  \factor|new| is used in \vref{lst:colors} to define
\factor|<color>| (by convention, the constructor for \factor|foo| is named
\factor|<foo>|).  First, we push the class \factor|color| onto the stack (this
word is also automatically defined by \factor|TUPLE:|), then just call
\factor|new|, leaving a new instance on the stack.  Since this particular tuple
has no slots, using \factor|new| makes sense.  We might also use it to
initialize a class, then use setter words to only assign a particular subset of
slots' values.

However, we often want to initialize a tuple with values for each of its slots.
For this, we have \factor|boa|, which works similarly to \factor|new|.  This is
used in the definition of \factor|<rgb>| in \vref{lst:colors}.  The difference
here is the additional inputs on the stack---one for each slot, in the order
they're declared.  That is, we're constructing the tuple \textbf{b}y
\textbf{o}rder of \textbf{a}rguments, giving us the fun pun ``\factor|boa|
constructor''.  So, \factor|1 2 3 <rgb>| will construct an \factor|rgb|
instance with the \factor|red| slot set to \factor|1|, the \factor|green| slot
set to \factor|2|, and the \factor|blue| slot set to \factor|3|.

\inputlst{email}

Now that we've seen the various words defined for tuples, we can explore more
complex slot specifiers.  Using the array-like syntax from before, slot
specifiers may be marked with certain \term{attributes}---both with the class
declared (like
%
\factor|{ name class attributes... }|)
%
and without the class declared (as in 
%
\factor|{ name attributes... }|).
%
In particular, Factor recognizes two different attributes.  If a slot marked
\factor|read-only|, the writer (and thus setter) for the slot will not be
defined, so the slot cannot be altered.  A slot may also provide an initial
value using the syntax \factor|initial: some-literal|.  This will be the slot's
value when instantiated with \factor|new|. 

For example, \vref{lst:email} shows a tuple definition from Factor's
\factor|smtp| vocabulary that defines an \factor|email| object.  The
\factor|from| address, \factor|subject|, and \factor|body| must be instances of
\factor|string|, while \factor|to|, \factor|cc|, and \factor|bcc| are
\factor|array|s of destination addresses.  The \factor|content-type| slot must
also be a \factor|string|, but if unspecified, it defaults to
\factor|"text/plain"|.  The \factor|encoding| must be a \factor|word| (in
Factor, even words are first-class objects), which by default is \factor|utf8|,
a word from the \factor|io.encodings.utf8| vocabulary for a Unicode format.

\subsubsection{Generics and Methods}

Unlike more common object systems, we do not define individual methods that
``belong'' to particular tuples.  In Factor, you define a method that
specializes on a class for a particular generic word.  That way, when the
generic word is called, it dispatches on the class of the object, invoking the
most specific method for the object.

Generic words are declared with the syntax
%
\factor|GENERIC: word-name ( stack -- effect )|.
%
Words defined this way will then dispatch on the class of the top element of
the stack (necessarily the rightmost input in the stack effect).  To define a
new method with which to control this dispatch, we use the syntax
%
\factor|M: class word-name definition... ;|.

\inputlst{sets}

An accessible example of a generic word is in Factor's \factor|sets|
vocabulary.  \factor|set| is a \term{mixin} class---a union of other classes
whose members may be extended by the user.  We can see the standard definition
in \vref{lst:sets}.  Note that the \factor|USING:| form specifies vocabularies
being used (like Java's \mint{java}|import|), and \factor|IN:| specifies the
vocabulary in which the definitions appear (like Java's \mint{java}|package|).
We can see here that instances of the \factor|sequence|, \factor|hash-set|, and
\factor|bit-set| classes are all instances of \factor|set|, so will respond
\factor|t| to the predicate \factor|set?|.  Similarly, \factor|sequence| is a
mixin class with many more members, including \factor|array|, \factor|vector|,
and \factor|string|.

\inputlst{cardinality}

\vref{lst:cardinality} shows the \factor|cardinality| generic from Factor's
\factor|sets| vocabulary, along with its methods.  This generic word takes a
\factor|set| instance from the top of the stack and pushes the number of
elements it contains.  For instance, if the top element is a \factor|bit-set|,
we extract its \factor|table| slot and invoke another word, \factor|bit-count|,
on that.  But if the top element is \factor|f| (the canonical false/empty
value), we know the cardinality is $0$.  For any \factor|sequence|, we may
offshore the work to a different generic, \factor|length|, defined in the
\factor|sequences| vocabulary.  The final method gives a default behavior for
any other \factor|set| instance, which simply uses \factor|members| to obtain
an equivalent \factor|sequence| of set members, then calls \factor|length|.

By viewing a class as a set of all objects that respond positively to the class
predicate, we may partially order classes with the subset relationship.  Method
dispatch will use this ordering when \factor|cardinality| is called to select
the most specific method for the object being dispatched upon.  For instance,
while no explicit method for \factor|array| is defined, any instance of
\factor|array| is also an instance of \factor|sequence|.  In turn, every
instance of \factor|sequence| is also an instance of \factor|set|.  We have
methods that dispatch on both \factor|set| and \factor|sequence|, but the
latter is more specific, so that is the method invoked.  If we define our own
class, \factor|foo|, and declare it as an instance of \factor|set| but not as
an instance of \factor|sequence|, then the \factor|set| method of
\factor|cardinality| will be invoked.  Sometimes resolving the precedence gets
more complicated, but these edge-cases are beyond the scope of our discussion.

%\section{Combinators}\label{sec:primer:combinators}

Quotations, introduced in \cref{sec:primer:stack-based}, form the basis of both
control flow and data flow in Factor.  Not only are they the equivalent of
anonymous functions, but the stack model also makes them syntactically
lightweight enough to serve as blocks akin to the code between curly braces in
C or Java.  Higher-order words that make use of quotations on the stack are
called \term{combinators}.  It's simple to express familiar conditional logic
and loops using combinators, as we'll show in \cref{sec:primer:control-flow}.
In the presence of explicit data flow via stack operations, even more patterns
arise that can be abstracted away.  \cref{sec:primer:data-flow} explores how we
can use combinators to express otherwise convoluted stack-shuffling logic more
succinctly.

\subsection{Control Flow}\label{sec:primer:control-flow}

\inputlst{if}

The most primitive form of control flow in typical programming languages is, of
course, the \mint{java}|if| statement, and the same holds true for Factor.  The
only difference is that Factor's \factor|if| isn't syntactically
significant---it's just another word, albeit implemented as a primitive.  For
the moment, it will do to think of \factor|if| as having the stack effect
%
\Verb|( ? true false -- )|.
%
The third element from the top of the stack is a condition, and it's followed
by two quotations.  The first quotation (second element from the top of the
stack) is called if the condition is true, and the second quotation (the top of
the stack) is called if the condition is false.  Specifically, \factor|f| is a
special object in Factor for falsity.  It is a singleton object---the sole
instance of the \factor|f| class---and is the only false value in the entire
language.  Any other object is necessarily boolean true.  For a canonical
boolean, there is the \factor|t| object, but its truth value exists only
because it is not \factor|f|.  Basic \factor|if| use is shown in
\cref{lst:if}\todo{vref}.  The first example will print ``odd'', the second
``empty'', and the third ``isn't f''.  All of them leave nothing on the stack.

\inputlst{if-effects}

However, the simplified stack effect for \factor|if| is quite restrictive.
%
\Verb|( ? true false -- )|
%
intuitively means that both the \Verb|true| and \Verb|false| quotations
can't take any inputs or produce any outputs---that their effects are
%
\Verb|( -- )|.
%
We'd like to loosen this restriction, but per \cref{sec:primer:effects}, Factor
must know the stack height after the \factor|if| call.  We could give
\factor|if| the effect
%
\Verb|( x ? true false -- y )|,
%
so that the two quotations could each have the stack effect
%
\Verb|( x -- y )|.
%
This would work for the \Verb|example1| word in \vref{lst:if-effects}, yet
it's just as restrictive.  For instance, the \Verb|example2| word would need
\factor|if| to have the effect
%
\Verb|( x y ? true false -- z )|,
%
since each branch has the effect
%
\Verb|( x y -- z )|.
%
Furthermore, the quotations might even have different effects, but still leave
the overall stack height balanced.  Only one item is left on the stack after a
call to \Verb|example3| regardless, even though the two quotations have
different stack effects: \factor|+| has the effect
%
\Verb|( x y -- z )|,
%
while \factor|drop| has the effect
%
\Verb|( x -- )|.

In reality, there are infinitely many correct stack effects for \factor|if|.
Factor has a special notation for such \term{row-polymorphic} stack effects.
If a token in a stack effect begins with two dots, like \Verb|..a| or
\Verb|..b|, it is a \term{row variable}.  If either side of a stack effect
begins with a row variable, it represents any number inputs/outputs.  Thus, we
could give \factor|if| the stack effect
%
\begin{center} \Verb|( ..a ? true false -- ..b )| \end{center}
%
\noindent to indicate that there may be any number of inputs below the
condition on the stack, and any number of outputs will be present after the
call to \factor|if|.  Note that these numbers aren't necessarily equal, which
is why we use distinct row variables in this case.  However, this still isn't
quite enough to capture the stack height requirements.  It doesn't communicate
that \Verb|true| and \Verb|false| must affect the stack in the same ways.
For this, we can use the notation
%
\Verb|quot: ( stack -- effect )|,
%
giving quotations a nested stack effect.  Using the same names for row
variables in both the ``inner'' and ``outer'' stack effects will refer to the
same number of inputs or outputs.  Thus, our final (correct) stack effect for
\factor|if| is 
%
\begin{center}
%
  \Verb|( ..a ? true: ( ..a -- ..b ) false: ( ..a -- ..b ) -- ..b )|
%
\end{center}
%
\noindent This tells us that the \Verb|true| quotation and the \Verb|false|
quotation will each create the same relative change in stack height as
\factor|if| does overall.

\inputlst{loops}

Though \factor|if| is necessarily a language primitive, other control flow
constructs are defined in Factor itself.  It's simple to write combinators for
iteration and looping as tail-recursive words that invoke quotations.
\vref{lst:loops} showcases some common looping patterns.  The most basic yet
versatile word is \factor|each|.  Its stack effect is
%
\begin{center}
%
  \Verb|( ... seq quot: ( ... x -- ... ) -- ... )|
%
\end{center}
%
\noindent Each element \Verb|x| of the sequence \Verb|seq| will be passed
to \Verb|quot|, which may use any of the underlying stack elements.  Here,
unlike \factor|if|, we enforce that the input stack height is exactly the same
as the output (since we use the same row variable).  Otherwise, depending on
the number of elements in \Verb|seq|, we might dig arbitrarily deep into the
stack or flood it with a varying number of values.  The first use of
\factor|each| in \vref{lst:loops} is balanced, as the quotation has the effect
%
\Verb|( str -- )|
%
and no additional items were on the stack to begin with.  Essentially, it's
equivalent to
%
\factor|"Lorem" print "ipsum" print "dolor" print|.
%
On the other hand, the quotation in the second example has the stack effect
%
\Verb|( total n -- total+n )|.
%
This is still balanced, since there is one additional item below the sequence
on the stack (namely \Verb|0|), and one element is left by the end (the sum
of the sequence elements).  So, this example is the same as
%
\factor|0 1 + 2 + 3 +|.

Any instance of the extensive \factor|sequence| mixin will work with
\factor|each|, making it very flexible.  The third example in \vref{lst:loops}
shows \factor|iota|, which is used here to create a \term{virtual} sequence of
integers from $0$ to $9$ (inclusive).  No actual sequence is allocated, merely
an object that behaves like a sequence.  In Factor, it's common practice to use
\factor|iota| and \factor|each| in favor of repetitive C-like \mint{c}|for|
loops.

Of course, we sometimes don't need the induction variable in loops.  That is,
we just want to execute a body of code a certain number of times.  For these
cases, there's the \factor|times| combinator, with the stack effect
%
\begin{center}
%
  \Verb|( ... n quot: ( ... -- ... ) -- ... )|
%
\end{center}
%
\noindent This is similar to \factor|each|, except that \Verb|n| is a number
(so we needn't use \factor|iota|) and the quotation doesn't expect an extra
argument (i.e., a sequence element).  Therefore, the example in
\vref{lst:loops} is equivalent to
%
\factor|"Ho!" print "Ho!" print "Ho!" print|.
%

Naturally, Factor also has the \factor|while| combinator, whose stack effect is
%
\begin{center}
%
  \Verb|( ..a pred: ( ..a -- ..b ? ) body: ( ..b -- ..a ) -- ..b )|
%
\end{center}
%
\noindent The row variables are a bit messy, but it works as you'd expected:
the \Verb|pred| quotation is invoked on each iteration to determine whether
\Verb|body| should be called.  The \factor|do| word is a handy modifier for
\factor|while| that simply executes the body of the loop once before leaving
\factor|while| to test the precondition as per usual.  Thus, the last example
in \vref{lst:loops} executes the body once, despite the condition being
immediately false.

\inputlst{high-order}

In the preceding combinators, quotations were used like blocks of code.  But
really, they're the same as anonymous functions from other languages.  As such,
Factor borrows classic tools from functional languages, like \factor|map| and
\factor|filter|, as shown in \vref{lst:high-order}.  \factor|map| is like
\factor|each|, except that the quotation should produce a single output.  Each
such output is collected up into a new sequence of the same class as the input
sequence.  Here, the example produces
%
\factor|{ 2 3 4 }|.
%
\factor|filter| selects only those elements from the sequence for which the
quotation returns a true value.  Thus, the \factor|filter| in
\vref{lst:high-order} outputs
%
\factor|{ 2 4 }|.
%
Even \factor|reduce| is in Factor, also known as a \term{left fold}.  An
initial element is iteratively updated by pushing a value from the sequence and
invoking the quotation.  In fact, \factor|reduce| is defined as
%
\factor|swapd each|,
%
where \factor|swapd| is a shuffler word with the stack effect
%
\Verb|( x y z -- y x z )|.
%
Thus, the example in \vref{lst:high-order} is the same as
%
\factor|0 { 1 2 3 } [ + ] each|,
%
as in \vref{lst:loops}.

These are just some of the control flow combinators defined in Factor.  Several
variants exist that meld stack shuffling with control flow, or can be used to
shorten common patterns like empty false branches.  An entire list is beyond
the scope of our discussion, but the ones we've studied should give a solid
view of what standard conditional execution, iteration, and looping looks like
in a stack-based language.

\subsection{Data Flow}\label{sec:primer:data-flow}

While avoiding variables and additional syntax makes it easier to refactor
code, keeping mental track of the stack can be taxing.  If we need to
manipulate more than the top few elements of the stack, code gets harder to
read and write.  Since the flow of data is made explicit via stack shufflers,
we actually wind up with redundant patterns of data flow that we otherwise
couldn't identify.  In Factor, there are several combinators that clean up
common stack-shuffling logic, making code easier to understand.

\inputlst{preserve}

The first combinators we'll look at are \factor|dip| and \factor|keep|.  These
are used to preserve elements of the stack.  When working with several values,
sometimes we don't want to use all of them at quite the same time.  Using
\factor|drop| and the like wouldn't help, as we'd lose the data altogether.
Rather, we want to retain certain stack elements, do a computation, then
restore them.  For an uncompelling but illustrative example, suppose we have
two values on the stack, but we want to increment the second element from the
top.  \Verb|without-dip1| in \vref{lst:preserve} shows one strategy, where we
shuffle the top element away with \factor|swap|, perform the computation, then
\factor|swap| the top back to its original place.  A cleaner way is to call
\factor|dip| on a quotation, which will execute that quotation just under the
top of the stack, as in \Verb|with-dip1|.  While the stack shuffling in
\Verb|without-dip1| isn't terribly complicated, it doesn't convey our meaning
very well.  Shuffling the top element out of the way becomes increasingly
difficult with more complex computations.  In \Verb|without-dip2|, we want to
call \factor|-| on the two elements below the top.  For lack of a more robust
stack shuffler, we use \factor|2over| to isolate the two values so we can call
\factor|-|.  The rest of the word consists of shuffling to get rid of excess
values on the stack.  It's also worth noting that \factor|swapd| is a
deprecated word in Factor, since its use starts making code harder to reason
about.  Alternatively, we could dream up a more complex stack shuffler with
exactly the stack effect we wanted in this situation.  But this solution
doesn't scale: what if we had to calculate something that required more inputs
or produced more outputs?  Clearly, \factor|dip| provides a cleaner alternative
in \Verb|with-dip2|.

\factor|keep| provides a way to hold onto the top element of the stack, but
still use it to perform a computation.  In general,
%
\factor|[ ... ] keep|
%
is equivalent to
%
\factor|dup [ ... ] dip|.
%
Thus, the current top of the stack remains on top after the use of
\factor|keep|, but the quotation is still invoked with that value.  In
\Verb|with-keep1| in \vref{lst:preserve}, we want to increment the top, but
stash the result below.  Again, this logic isn't terribly complicated, though
\Verb|with-keep1| does away with the shuffling.  \Verb|without-keep2| shows
a messier example where a simple \factor|dup| will not save us, as we're using
more than just the top element in the call to \factor|-|.  Rather, three of the
four words in the definition are dedicated to rearranging the stack in just the
right way, obscuring the call to \factor|-| that we really want to focus on.
On the other hand, \Verb|with-keep2| places the subtraction word
front-and-center in its own quotation, while \factor|keep| does the work of
retaining the top of the stack.

\inputlst{cleave}

The next set of combinators apply multiple quotations to a single value.  The
most general form of these so-called \term{cleave} combinators is the word
\factor|cleave|, which takes an array of quotations as input, and calls each
one in turn on the top element of the stack.  Of course, for only a couple of
quotations, wrapping them in an array literal becomes cumbersome.  The word
\factor|bi| exists for the two-quotation case, and \factor|tri| for the three
quotations.  Cleave combinators are often used to extract multiple slots from a
tuple.  \vref{lst:cleave} shows such a case in the \Verb|with-bi| word, which
improves upon using just \factor|keep| in the \Verb|without-bi| word.  In
general, a series of \factor|keep|s like
%
\factor|[ a ] keep [ b ] keep c|
%
is the same as
%
\factor|{ [ a ] [ b ] [ c ] } cleave|,
%
%XXX fuck this dumb-shit fucking cunt what the fuck does it fucking want
which is more readable.  We can see this in action in the difference between
\Verb|without-tri| and \Verb|with-tri| in \cref{lst:cleave}.  In cases
where we need to apply multiple quotations to a set of values instead of just a
single one, there are also the variants \factor|2cleave| and \factor|3cleave|
(and the corresponding \factor|2bi|, \factor|2tri|, \factor|3bi|, and
\factor|3tri|), which apply the quotations to the top two and three elements of
the stack, respectively.

\inputlst{spread}

To apply multiple quotations to multiple values, Factor has \term{spread}
combinators.  Whereas cleave combinators abstract away repeated instances of
\factor|keep|, spread combinators replace nested calls to \factor|dip|.  The
archetypical combinator, \factor|spread|, takes an array of quotations, like
\factor|cleave|.  However, instead of applying each one to the top element of
the stack, each one corresponds to a separate element of the stack.  Thus,
%
\factor|{ [ a ] [ b ] } spread|
%
invokes \Verb|b| on the top element, and \Verb|a| on the element beneath
the top.  Much like \factor|cleave|, there are shorthand words for the two- and
three-quotation cases.  These are suffixed with asterisks to indicate the
spread variants, so we have \factor|bi*| and \factor|tri*|.  In
\vref{lst:spread}, the \Verb|without-bi*| word shows the simple \factor|dip|
pattern that \factor|bi*| encapsulates.  Here, we're converting the string
\Verb|str1| (the second element from the top) into uppercase and
\Verb|str2| (the top element) to lowercase.  In \Verb|with-bi*|, the
\factor|>upper| and \factor|>lower| words are seen first, uninterrupted by an
extra word, making the code easier to read.  More compelling is the way that
\factor|tri*| replaces the \factor|dip|s that can be seen in
\Verb|without-tri*|.  In comparison, \Verb|with-tri*| is less nested and
easier to comprehend at first glance.  While there are \factor|2bi*| and
\factor|2tri*| variants that spread quotations to two values apiece on the
stack, they are uncommon in practice.

\inputlst{apply}

Finally, \term{apply} combinators invoke a single quotation on multiple stack
entries in turn.  While there is a generalized word, it's more common to use
the corresponding shorthands.  Here, they are suffixed with at-signs, so
\factor|bi@| applies a quotation to each of the top two stack values, and
\factor|tri@| to each of the top three.  This way, rather than duplicate code
for each time we want to call a word, we need only specify it once.  This is
demonstrated clearly in \vref{lst:apply}.  In \Verb|without-bi@|, we see that
the quotation \factor|[ sq ]| (for squaring numbers) appears twice for the call
to \factor|bi*|.  In general, we can replace spread combinators whose
quotations are all the same with a single quotation and an apply combinator.
Thus, \Verb|with-bi@| cuts down on the duplicated \factor|[ sq ]| in
\Verb|without-bi@|.  Similarly, we can replace the three repeated quotations
passed to \factor|tri*| in \Verb|without-tri@| with a single instance passed
to \factor|tri@| in \Verb|with-tri@|.  Like other data flow combinators, we
have the numbered variants.  \factor|2bi@| has the stack effect
%
\Verb|( w x y z quot -- )|,
%
where \Verb|quot| expects two inputs, and is thus applied to \Verb|w| and
\Verb|x| first, then to \Verb|y| and \Verb|z|.  Similarly, \factor|2tri@|
applies the quotation to the top six objects of the stack in groups of two.
Like their spread counterparts, they are not used very much.

\todo[inline]{Some wrap-up that isn't completely lame.}


%\newpage\chapter{The Factor Compiler}\label{sec:compiler}

If we could sort programming languages by the fuzzy notions we tend to have
about how ``high-level'' they are, toward the high end we'd find
dynamically-typed languages like Python, Ruby, and PHP---all of which are
generally more interpreted than compiled (though there has been compelling work
on this front \autocite[e.g.,][]{Biggar}).  Despite being as high-level as these
popular languages, Factor's implementation is driven by performance.  Factor
source is always compiled to native machine code using either its simple,
non-optimizing compiler or (more typically) the optimizing compiler that
performs several sorts of data and control flow analyses.  In this
\lcnamecref{sec:compiler}, we look at the general architecture of Factor's
implementation, after which we place a particular emphasis on the
transformations performed by the optimizing compiler.

\subsection{Organization}\label{sec:compiler:vm}

At the lowest level, Factor is written atop a C++ \gls{VM} that is responsible
for basic runtime services.  This is where the non-optimizing base compiler is
implemented.  It's the base compiler's job to compile the simplest primitives:
operations that push literals onto the data stack, \factor|call|, \factor|if|,
\factor|dip|, words that access tuple slots as laid out in memory, stack
shufflers, math operators, functions to allocate/deallocate call stack frames,
etc.  The aim of the base compiler is to generate native machine code as fast
as possible.  To this end, these primitives correspond to their own stubs of
assembly code.  Different stubs are generated by Factor depending on the
instruction set supported by the particular machine in use.  Thus, the base
compiler need only make a single pass over the source code, emitting these
assembly instructions as it goes.

This compiled code is saved in an \term{image file}, which contains a complete
snapshot of the current state of the Factor instance, similar to many Smalltalk
and Lisp systems\todo{cite?}.  As code is parsed and compiled, the image is
updated, serving as a cache for compiled code.  This modified image can be
saved so that future Factor instances needn't recompile vocabularies that are
already contained in the image.

The \gls{VM} also handles method dispatch and memory management.  Method
dispatch incorporates a \term{polymorphic inline cache} to speed up generic
words.  Each generic word's call site is associated with a state:
\begin{itemize}
  \item In the \term{cold} state, the call site's instruction computes the
        right method for the class being dispatched upon, which is the
        operation we're trying to avoid.  As it does this, a polymorphic inline
        cache stub is generated, thus transitioning it to the next state.
  \item In the \term{inline cache} state, a stub has been generated that caches
        the locations of methods for classes that have already been seen.  This
        way, if a generic word at a particular call site is invoked often upon
        only a small number of classes (as is often in the case in loops, for
        example), we don't need to waste as much time doing method lookup.  By
        default, if more than three different classes are dispatched upon, we
        transition to the next state.
  \item In the \term{megamorphic} state, the call instruction points to a
        larger cache that is allocated for the specific generic word (i.e., it
        is shared by all call sites).  While not as efficient as an inline
        cache, this can still improve the performance of method dispatch.
\end{itemize}

To manage memory, the Factor \gls{VM} uses a generational \gls{GC}, which
carves out sections of space on the heap for objects of different ages.
Garbage in the oldest generation is collected with a mark-sweep-compact
algorithm, while younger generations rely on a copying collector\todo{cite?}.
This way, the \gls{GC} is specialized for large numbers of short-lived objects
that will stay in the younger generations without being promoted to the older
generation.  In the oldest space, even compiled code can be compacted.  This is
to avoid heap fragmentation in applications that must call the compiler at
runtime, such as Factor's interactive development environment.

Values are referenced by tagged pointers, which use the three least significant
bits of the pointer's address to store type information.  This is possible
because Factor aligns objects on an eight-byte boundary, so the three least
significant bits of an address are always $0$.  These bits give us eight unique
tags, but since Factor has more than eight data types, two tags are reserved to
indicate that the type information is stored elsewhere.  One is for \gls{VM}
types without their own tag, and the other is for user-defined tuples, each of
which has its own type.  Sufficiently small integers (e.g., $29$-bit integers
on a $32$-bit machine, since the other $3$ bits are used for the type tag) are
stored directly in the pointer, so they needn't be heap-allocated.  Larger
integers and floating point numbers are boxed, but the optimizing compiler may
unbox them to store floats in registers.

The \gls{VM} is meant to be minimal, as Factor is mostly \term{self-hosting}.
That is, the real workhorses of the language are written in Factor itself,
including the standard libraries, parser, object system, and the optimizing
compiler.  It's possible for the compiler to be written in Factor because of
the \term{bootstrapping} process that creates a new image from scratch.  First,
a minimal \term{boot image} is created from an existing \term{host} Factor
instance.  When the \gls{VM} runs the boot image, it initiates the
bootstrapping process.  Using the host's parser, the base compiler will compile
the core vocabularies necessary to load the optimizing compiler.  Once the
optimizing compiler can itself be compiled, it is used to recompile (and thus
optimize) all of the words defined so far.  With the basic vocabularies
recompiled, any additional vocabularies can be loaded using the optimized
compiler and saved into a new, working image.

Thus, while the Factor \gls{VM} is important, it is a small part of the code
base.  Since the bootstrapping process allows the optimizing compiler
(hereafter just ``the compiler'') to be written in the same high-level language
it's compiling, we can avoid the fiddly low-level details of the C++ backend.
This is more conducive to writing advanced compiler optimizations, which are
often complicated enough without having a concise, dynamically-typed,
garbage-collected language like Factor to help us.

\section{High-level Optimizations}\label{sec:compiler:tree}

To manipulate source code abstractly, we must have at least one \gls{IR}---a
data structure representing the instructions.  It's common to convert between
several \glsplural{IR} during compilation, as each form offers different
properties that facilitate particular analyses.  The Factor compiler optimizes
code in passes across two different \glsplural{IR}: first at a high-level using
the \Verb|compiler.tree| vocabulary, then at a low-level with the
\Verb|compiler.cfg| vocabulary.

\inputlst{tree}

The high-level \gls{IR} arranges code into a vector of \Verb|node| objects,
which may themselves have children consisting of vectors of node---a tree
structure that lends to the name \Verb|compiler.tree|.  This ordered sequence
of nodes represents control flow in a way that's effectively simple, annotated
stack code.  \Vref{lst:tree} shows the definitions of the tuples that represent
the ``instruction set'' of this stack code.  Each object inherits (directly or
indirectly) from the \Verb|node| class, which itself inherits from
\factor|identity-tuple|.  This is a tuple whose \factor|equal?| method is
defined to always return \factor|f| so that no two instances are equivalent
unless they are the same instance.

Notice that most nodes define some sort of \Verb|in-d| and \Verb|out-d|
slots, which mark each of them with the input and output data stacks.  This
represents the flow of data through the program.  Here, stack values are
denoted simply by integers, giving each value a unique identifier.  An
\Verb|#introduce| instance is inserted wherever the next node requires stack
values that have not yet been named.  Thus, while \Verb|#introduce| has no
\Verb|in-d|, its \Verb|out-d| introduces the necessary stack values.
Similarly, \Verb|#return| is inserted at the end of the sequence to indicate
the final state of the data stack with its \Verb|in-d| slot.

The most basic operations of a stack language are, of course, pushing literals
and calling functions.  The \Verb|#push| node thus has a \Verb|literal|
slot and an \Verb|out-d| slot, giving a name to the single element it pushes
to the data stack.  \Verb|#call|, of course, is used for normal word
invocations.  The \Verb|in-d| and \Verb|out-d| slots effectively serve as
the stack effect declaration.  In later analyses, data about the word's
definition may be stored across the \Verb|body|, \Verb|method|,
\Verb|class|, and \Verb|info| slots.

\inputlst{build-tree-1}

The word \Verb|build-tree| takes a Factor quotation and constructs the
equivalent high-level \gls{IR} form.  In \vref{lst:build-tree-1}, we see the
output of the simple example
%
\factor|[ 1 + ] build-tree|.
%
Note that
%
\factor|T{ class { slot1 value1 } { slot2 value2 } ... }|
%
is the syntax for tuple literals.  The first node is a \Verb|#push| for the
\Verb|1| literal.  Since \factor|+| needs two input values, an
\Verb|#introduce| pushes a new ``phantom'' value.  \factor|+| gets turned
into a \Verb|#call| instance.  Notice the \Verb|in-d| slot refers to the
values in the order that they're passed to the word, not necessarily the order
they've been introduced in the \gls{IR}.  The sum is pushed to the data stack,
so the \Verb|out-d| slot is a singleton that names this value.  Finally,
\Verb|#return| indicates the end of the routine, its \Verb|in-d| containing
the value left on the stack (the sum pushed by \Verb|#call|).

\inputlst{build-tree-2}

The next tuples in \vref{lst:tree} reassign existing values on the stack to
fresh identifiers.  The \Verb|#renaming| superclass has the two subclasses
\Verb|#copy| and \Verb|#shuffle|.  The former represents the bijection from
elements of \Verb|in-d| to elements of \Verb|out-d| in the same position;
corresponding values are copies of each other.  The latter represents a more
general mapping.  Stack shufflers are translated to \Verb|#shuffle| nodes
with \Verb|mapping| slots that dictate how the fresh values in \Verb|out-d|
correspond to the input values in \Verb|in-d|.  For instance,
\vref{lst:build-tree-2} shows how \factor|swap| takes in the values
\Verb|6256132| and \Verb|6256133| and outputs \Verb|6256134| and
\Verb|6256135|, where the former is mapped to the second element
(\Verb|6256133|) and the latter to the first (\Verb|6256132|).  Thus,
\Verb|out-d| swaps the two elements of \Verb|in-d|, mapping them to fresh
identifiers.  The \Verb|in-r| and \Verb|out-r| slots of \Verb|#shuffle|
correspond to the \term{retain} stack, which is an implementation detail beyond
the scope of this discussion.

\inputlst{build-tree-3}
\inputlst{build-tree-4}

\Verb|#declare| is a miscellaneous node used for the \factor|declare|
primitive.  It simply annotates type information to stack values, as in
\vref{lst:build-tree-3}.  \Verb|#terminate| is another one-off node, but a
much more interesting one.  While Factor normally requires a balanced stack,
sometimes we purposefully want to throw an error.  \Verb|#terminate| is
introduced where the program halts prematurely.  When checking the stack
height, it gets to be treated specially so that \term{terminated} stack effects
unify with any other effect.  That way, branches will still be balanced even if
one of them unconditionally throws an error.  \vref{lst:build-tree-4} shows
\Verb|#terminate| being introduced by the \factor|throw| word.

Next, \vref{lst:tree} defines nodes for branching based off the superclass
\Verb|#branch|.  The \Verb|children| slot contains vectors of nodes
representing different branches.  \Verb|live-branches| is filled in during
later analyses to indicate which branches are alive so that dead ones may be
removed.  For instance, \Verb|#if| will have two elements in its
\Verb|children| slot representing the true and false branches.  On the other
hand, \Verb|#dispatch| has an arbitrary number of children.  It corresponds
to the \factor|dispatch| primitive, which is an implementation detail of the
generic word system used to speed up method dispatch.

You may have noted the emphasis on introducing new values, instead of
reassigning old ones.  Even \Verb|#shuffle|s output fresh identifiers,
letting their values be determined by the \Verb|mapping|.  The reason for
this is that \Verb|compiler.tree| uses \gls{SSA} form, wherein every variable
is defined by exactly one statement.  This simplifies the properties of
variables, which helps optimizations perform faster and with better
results\todo{cite?}.  By giving unique names to the targets of each assignment,
the \gls{SSA} property is guaranteed.  However, \Verb|#branch|es introduce
ambiguity: after, say, an \Verb|#if|, what will the \Verb|out-d| be?  It
depends on which branch is taken.  To remedy this problem, after any
\Verb|#branch| node, Factor will place a \Verb|#phi| node---the classical
\gls{SSA} ``phony function'', $\phi$.  While it doesn't perform any literal
computation, conceptually $\phi$ selects between its inputs, choosing the
``correct'' argument depending on control flow.  This can then be assigned to a
unique value, preserving the \gls{SSA} property.  In Factor, this is
represented by a \Verb|phi-in-d| slot, which is a sequence of sequences.
Each element corresponds to the \Verb|out-d| of the child at the same
position in the \Verb|children| of the preceding \Verb|#branch| node.  The
\Verb|#phi|'s \Verb|out-d| gives unique names to the output values.

\inputlst{build-tree-5}

For example, the \Verb|#phi| in \vref{lst:build-tree-5} will select between
the \Verb|6256248| return value of the first child or the \Verb|6256249|
output of the second.  Either way, we can refer to the result as
\Verb|6256250| afterwards.  The \Verb|terminated| slot of the \Verb|#phi|
tells us if there was a \Verb|#terminate| in any of the branches.

The \Verb|#recursive| node encapsulates \term{inline recursive} words.  In
Factor, words may be annotated with simple compiler declarations, which guide
optimizations.  If we follow a standard colon definition with the
\factor|inline| word, we're saying that its definition can be spliced into the
call-site, rather than generating code to jump to a subroutine.  Inline words
that call themselves must additionally be declared \factor|recursive|.  For
example, we could write
%
\factor|: foo ( -- ) foo ; inline recursive|.
%
The nodes \Verb|#enter-recursive|, \Verb|#call-recursive|, and
\Verb|#return-recursive| denote different stages of the recursion---the
beginning, recursive call, and end, respectively.  They carry around a lot of
metadata about the nature of the recursion, but it doesn't serve our purposes
to get into the details.  Similarly, we gloss over the final nodes of
\vref{lst:tree} correspond to Factor's \gls{FFI} vocabulary, called
\Verb|alien|.  At a high level, \Verb|#alien-node|, \Verb|#alien-invoke|,
\Verb|#alien-indirect|, \Verb|#alien-assembly|, and
\Verb|#alien-callback| are used to make calls to C libraries from within
Factor.

\inputlst{optimize-tree}
\todo[inline]{Shouldn't bold ``cleanup'' in \cref{lst:optimize-tree}}

Now that we're familiar with the structure of the high-level \gls{IR}, we can
turn our attention to optimization.  \Vref{lst:optimize-tree} shows the passes
performed on a sequence of nodes by the word \Verb|optimize-tree|.  Before
optimization can begin, we must gather some information and clean up some
oddities in the output of \Verb|build-tree|.  \Verb|analyze-recursive| is
called first to identify and mark loops in the tree.  Effectively, this means
we detect tail-recursion introduced by \Verb|#recursive| nodes.  Future
passes can then use this information for data flow analysis.  Then,
\Verb|normalize| makes the tree more consistent by doing two things:
%
\begin{itemize}
%
  \item All \Verb|#introduce| nodes are removed and replaced by a single
        \Verb|#introduce| at the beginning of the whole program.  This way,
        further passes needn't handle \Verb|#introduce| nodes.
%
  \item As constructed, the \Verb|in-d| of a \Verb|#call-recursive| will be
        the entire stack at the time of the call.  This assumption happens
        because we don't know how many inputs it needs until the
        \Verb|#return-recursive| is processed, because of row polymorphism.
        So, here we figure out exactly what stack entries are needed, and trim
        the \Verb|in-d| and \Verb|out-d| of each \Verb|#call-recursive|
        accordingly.
%
\end{itemize}

Once these passes have cleaned up the tree, \Verb|propagate| performs
probably the most extensive analysis of all the phases.  In short, it performs
an extended version of \gls{SCCP}\todo{cite}.  The traditional data flow
analysis combines global copy propagation, constant propagation, and constant
folding in a \term{flow-sensitive} way.  That is, it will propagate information
from branches that it knows are definitely taken (e.g., because \Verb|#if| is
always given a true input).  Instead of using the typical single-level
(numeric) constant value lattice, Factor uses a lattice augmented by
information about classes, numeric value ranges, array lengths, and tuple
slots' classes.  Classes can be used in the lattice with the partial-order
protocol described briefly in \cref{sec:primer:oo}.  Additionally, the transfer
functions are allowed to inline certain calls if enough information is present.
This occurs in the transfer function since generic words' inline expansions
into particular methods provide more information, thus giving us more
opportunities for propagation.  This is particularly useful for arithmetic
words.  In Factor, words like \factor|+| and \factor|*| are generics that work
across all sorts of numeric representations, be they \factor|fixnum|s,
\factor|float|s, \factor|bignum|s, etc.  If the operation overflows, the values
are automatically cast up to larger representations.  But iterated refinement
of the inputs' classes can let the compiler select more specific, efficient
methods (e.g., if both arguments are \factor|fixnum|s).

Interval propagation also helps propagate class information.  By refining the
range of possible values a particular item can have, we might discover that,
say, it's small enough to fit in a \factor|fixnum| rather than a
\factor|bignum|.  There are plenty more things that interval propagation can
tell us, too.  For example, it may give us enough information to remove
overflow checks performed by numeric words.  And if the interval has zero
length, we may replace the value with a constant.  This then continues getting
propagated, contributing to constant folding and so forth.

\Verb|propagate| iterates through the nodes collecting all of this data until
reaching a stable point where inferences can no longer be drawn.  Technically,
this information doesn't alter the tree at all; we merely store it so that
speculative decisions may be realized later.  The next word in
\vref{lst:optimize-tree}, \Verb|cleanup|, does just this by inlining words,
folding constants, removing overflow checks, deleting unreachable branches, and
flattening inline-recursive words that don't actually wind up calling
themselves (e.g., because the calls got constant-folded).

\inputlst{escaping}

The next major pass is \Verb|escape-analysis|, whose information is used for
the actual transformation \Verb|unbox-tuples|.  This discovers tuples that
\term{escape} by getting passed outside of a word.  For instance, the inputs to
\Verb|#return| obviously escape, as they are passed to the world outside of
the word in question.  Similarly, inputs to the \Verb|#call| of another word
escape.  So, though the tuples in \Verb|escaping-via-#return| and
\Verb|escaping-via-#call| in \vref{lst:escaping} both escape, we can see the
one in \Verb|non-escaping| does not.  In fact, the last allocation is
unnecessary.  By identifying this, \Verb|unbox-tuples| can then rewrite the
code to avoid allocating a \Verb|data-struct| altogether, instead
manipulating the slots' values directly.  Note that this only happens for
immutable tuples, all of whose slots are \factor|read-only|.  Otherwise, we
would need to perform more advanced pointer analyses to discover aliases.

\Verb|apply-identities| follows to simplify words with known identity
elements.  If, say, an argument to \factor|+| is \Verb|0|, we can simply
return the other argument.  This converts the \Verb|#call| to \factor|+| into
a simple \Verb|#shuffle|.  These identities are defined for most arithmetic
words.

Another simple few passes come next in \vref{lst:optimize-tree}.  True to its
name, \Verb|compute-def-use| computes where \gls{SSA} values are defined and
used.  Values that are never used are eliminated by \Verb|remove-dead-code|.
\Verb|?check| conditionally performs some consistency checks on the tree,
mostly to make sure that no errors were introduced in the stack flow.  If a
global variable isn't toggled on, this part is skipped.  We run
\Verb|compute-def-use| again to update the information after altering the
tree with dead code elimination.

Finally, \Verb|optimize-modular-arithmetic| performs a form of
strength-reduction on artihmetic words that only use the low-order bits of
their inputs/results, which may also remove more unnecessary overflow checks.
\Verb|finalize| cleans up a few random miscellaneous bits of the tree
(removing empty shufflers, deleting \Verb|#copy| nodes, etc.) in preparation
for lower-level optimizations.

\todo[inline]{Double-check zealous syntax-highlighting}

\section{Low-level Optimizations}\label{sec:compiler:cfg}

The low-level \gls{IR} in \Verb|compiler.cfg| takes the more conventional form
of a \gls{CFG}.  A \gls{CFG} (not to be confused with ``context-free grammar'')
is an arrangement of instructions into \term{basic blocks}: maximal sequences
of ``straight-line'' code, where control does not transfer out of or into the
middle of the block.  Directed edges are added between blocks to represent
control flow---either from a branching instruction to its target, or from the
end of a basic block to the start of the next one \autocite{DragonBook}.
Construction of the low-level \gls{IR} proceeds by analyzing the control flow
of the high-level \gls{IR} and converting the nodes of \cref{sec:compiler:tree}
into lower-level, more conventional instructions modeled after typical assembly
code.  There are over a hundred of these instructions, but many are simply
different versions of the same operation.  For instance, while instructions are
generally called on \term{virtual registers} (represented in Factor simply by
integers), there are \term{immediate} versions of instructions.  The
\Verb|##add| instruction, as an example, represents the sum of the contents of
two registers, but \Verb|##add-imm| sums the contents of one register and an
integer literal.  Other instructions are inserted to make stack reads and
writes explicit, as well as to balance the height.  Below is a categorized list
of all the instruction objects (each one is a subclass of the \Verb|insn|
tuple).

\todo[inline]{Is the complete list really necessary?}
\begin{itemize}
\item
\begin{flushleft}
Loading constants:
\Verb|##load-integer|,
\Verb|##load-reference|
\end{flushleft}

\item
\begin{flushleft}
Optimized loading of constants, inserted by representation selection:
\Verb|##load-tagged|,
\Verb|##load-float|,
\Verb|##load-double|,
\Verb|##load-vector|
\end{flushleft}

\item
\begin{flushleft}
Stack operations:
\Verb|##peek|,
\Verb|##replace|,
\Verb|##replace-imm|,
\Verb|##inc-d|,
\Verb|##inc-r|
\end{flushleft}

\item
\begin{flushleft}
Subroutine calls:
\Verb|##call|,
\Verb|##jump|,
\Verb|##prologue|,
\Verb|##epilogue|,
\Verb|##return|
\end{flushleft}

\item
\begin{flushleft}
Inhibiting \gls{TCO}:
\Verb|##no-tco|
\end{flushleft}

\item
\begin{flushleft}
Jump tables:
\Verb|##dispatch|
\end{flushleft}

\item
\begin{flushleft}
Slot access:
\Verb|##slot|,
\Verb|##slot-imm|,
\Verb|##set-slot|,
\Verb|##set-slot-imm|
\end{flushleft}

\item
\begin{flushleft}
Register transfers:
\Verb|##copy|,
\Verb|##tagged>integer|
\end{flushleft}

\item
\begin{flushleft}
Integer arithmetic:
\Verb|##add|,
\Verb|##add-imm|,
\Verb|##sub|,
\Verb|##sub-imm|,
\Verb|##mul|,
\Verb|##mul-imm|,
\Verb|##and|,
\Verb|##and-imm|,
\Verb|##or|,
\Verb|##or-imm|,
\Verb|##xor|,
\Verb|##xor-imm|,
\Verb|##shl|,
\Verb|##shl-imm|,
\Verb|##shr|,
\Verb|##shr-imm|,
\Verb|##sar|,
\Verb|##sar-imm|,
\Verb|##min|,
\Verb|##max|,
\Verb|##not|,
\Verb|##neg|,
\Verb|##log2|,
\Verb|##bit-count|
\end{flushleft}

\item
\begin{flushleft}
Float arithmetic:
\Verb|##add-float|,
\Verb|##sub-float|,
\Verb|##mul-float|,
\Verb|##div-float|,
\Verb|##min-float|,
\Verb|##max-float|,
\Verb|##sqrt|
\end{flushleft}

\item
\begin{flushleft}
Single/double float conversion:
\Verb|##single>double-float|,
\Verb|##double>single-float|
\end{flushleft}

\item
\begin{flushleft}
Float/integer conversion:
\Verb|##float>integer|,
\Verb|##integer>float|
\end{flushleft}

\item
\begin{flushleft}
\Gls{SIMD} operations:
\Verb|##zero-vector|,
\Verb|##fill-vector|,
\Verb|##gather-vector-2|,
\Verb|##gather-int-vector-2|,
\Verb|##gather-vector-4|,
\Verb|##gather-int-vector-4|,
\Verb|##select-vector|,
\Verb|##shuffle-vector|,
\Verb|##shuffle-vector-halves-imm|,
\Verb|##shuffle-vector-imm|,
\Verb|##tail>head-vector|,
\Verb|##merge-vector-head|,
\Verb|##merge-vector-tail|,
\Verb|##float-pack-vector|,
\Verb|##signed-pack-vector|,
\Verb|##unsigned-pack-vector|,
\Verb|##unpack-vector-head|,
\Verb|##unpack-vector-tail|,
\Verb|##integer>float-vector|,
\Verb|##float>integer-vector|,
\Verb|##compare-vector|,
\Verb|##test-vector|,
\Verb|##test-vector-branch|,
\Verb|##add-vector|,
\Verb|##saturated-add-vector|,
\Verb|##add-sub-vector|,
\Verb|##sub-vector|,
\Verb|##saturated-sub-vector|,
\Verb|##mul-vector|,
\Verb|##mul-high-vector|,
\Verb|##mul-horizontal-add-vector|,
\Verb|##saturated-mul-vector|,
\Verb|##div-vector|,
\Verb|##min-vector|,
\Verb|##max-vector|,
\Verb|##avg-vector|,
\Verb|##dot-vector|,
\Verb|##sad-vector|,
\Verb|##horizontal-add-vector|,
\Verb|##horizontal-sub-vector|,
\Verb|##horizontal-shl-vector-imm|,
\Verb|##horizontal-shr-vector-imm|,
\Verb|##abs-vector|,
\Verb|##sqrt-vector|,
\Verb|##and-vector|,
\Verb|##andn-vector|,
\Verb|##or-vector|,
\Verb|##xor-vector|,
\Verb|##not-vector|,
\Verb|##shl-vector-imm|,
\Verb|##shr-vector-imm|,
\Verb|##shl-vector|,
\Verb|##shr-vector|
\end{flushleft}

\item
\begin{flushleft}
Scalar/vector conversion:
\Verb|##scalar>integer|,
\Verb|##integer>scalar|,
\Verb|##vector>scalar|,
\Verb|##scalar>vector|
\end{flushleft}

\item
\begin{flushleft}
Boxing and unboxing aliens:
\Verb|##box-alien|,
\Verb|##box-displaced-alien|,
\Verb|##unbox-any-c-ptr|,
\Verb|##unbox-alien|
\end{flushleft}

\item
\begin{flushleft}
Zero-extending and sign-extending integers:
\Verb|##convert-integer|
\end{flushleft}

\item
\begin{flushleft}
Raw memory access:
\Verb|##load-memory|,
\Verb|##load-memory-imm|,
\Verb|##store-memory|,
\Verb|##store-memory-imm|
\end{flushleft}

\item
\begin{flushleft}
Memory allocation:
\Verb|##allot|,
\Verb|##write-barrier|,
\Verb|##write-barrier-imm|,
\Verb|##alien-global|,
\Verb|##vm-field|,
\Verb|##set-vm-field|
\end{flushleft}

\item
\begin{flushleft}
The \gls{FFI}:
\Verb|##unbox|,
\Verb|##unbox-long-long|,
\Verb|##local-allot|,
\Verb|##box|,
\Verb|##box-long-long|,
\Verb|##alien-invoke|,
\Verb|##alien-indirect|,
\Verb|##alien-assembly|,
\Verb|##callback-inputs|,
\Verb|##callback-outputs|
\end{flushleft}

\item
\begin{flushleft}
Control flow:
\Verb|##phi|,
\Verb|##branch|
\end{flushleft}

\item
\begin{flushleft}
Tagged conditionals:
\Verb|##compare-branch|,
\Verb|##compare-imm-branch|,
\Verb|##compare|,
\Verb|##compare-imm|
\end{flushleft}

\item
\begin{flushleft}
Integer conditionals:
\Verb|##compare-integer-branch|,
\Verb|##compare-integer-imm-branch|,
\Verb|##test-branch|,
\Verb|##test-imm-branch|,
\Verb|##compare-integer|,
\Verb|##compare-integer-imm|,
\Verb|##test|,
\Verb|##test-imm|
\end{flushleft}

\item
\begin{flushleft}
Float conditionals:
\Verb|##compare-float-ordered-branch|,
\Verb|##compare-float-unordered-branch|,
\Verb|##compare-float-ordered|,
\Verb|##compare-float-unordered|
\end{flushleft}

\item
\begin{flushleft}
Overflowing arithmetic:
\Verb|##fixnum-add|,
\Verb|##fixnum-sub|,
\Verb|##fixnum-mul|
\end{flushleft}

\item
\begin{flushleft}
\Gls{GC} checks:
\Verb|##save-context|,
\Verb|##check-nursery-branch|,
\Verb|##call-gc|
\end{flushleft}

\item
\begin{flushleft}
Spills and reloads, inserted by the register allocator:
\Verb|##spill|,
\Verb|##reload|
\end{flushleft}
\end{itemize}


\inputlst{optimize-cfg}

\inputfig{optimize-tail-calls}

By translating the high-level \gls{IR} into instructions that manipulate
registers directly, we reveal further redundancies that can be optimized away.
The \Verb|optimize-cfg| word in \cref{lst:optimize-cfg} shows the passes
performed in doing this.  The first word, \Verb|optimize-tail-calls|,
performs tail call elimination on the \gls{CFG}.
%
\term{Tail calls}~\todo{used in \cref{sec:compiler:tree}, not defined} are those
that occur within a procedure and whose results are immediately returned by
that procedure.  Instead of allocating a new call stack frame, we may convert
tail calls into simple jumps, since afterwards the current procedure's call
frame isn't really needed.  In the case of recursive tail calls, we can convert
special cases of recursion into loops in the \gls{CFG}, so that we won't
trigger call stack overflows.  For instance, consider
\cref{fig:optimize-tail-calls}, which shows the effect of
\Verb|optimize-tail-calls| on the following definition:
%
\begin{center}
%
  \factor|: tail-call ( -- ) tail-call ;|
%
\end{center}
%
\noindent Note the recursive call (trivially) occurs at the end of the
definition, just before the return point.  When translated to a \gls{CFG}, this
is a \Verb|##call| instruction, as seen in block $4$ to the left of
\cref{fig:optimize-tail-calls}.  This is also just before the final
\Verb|##epilogue| and \Verb|##return| instructions in block $8$, as blocks
$5$--$7$ are effectively empty (these excessive \Verb|##branch|es will be
eliminated in a later pass).  Because of this, rather than make a whole new
subroutine call, we can convert it into a \Verb|##branch| back to the
beginning of the word, as in the \gls{CFG} to the right.

\inputfig{delete-useless-conditionals}

The next pass in \cref{lst:optimize-cfg} is
\Verb|delete-useless-conditionals|, which removes branches that go to the
same basic block.  This situation might occur as a result of optimizations
performed in the high-level \gls{IR}.  To see it in action,
\cref{fig:delete-useless-conditionals} shows the transformation on a
purposefully useless conditional,
%
\factor|[ ] [ ] if|.
%
Before removing the useless conditional, the \gls{CFG} \Verb|##peek|s at the
top of the data stack
%
(\Verb|D 0|),
%
storing the result in the virtual register \Verb|1|.  This value is popped,
so we decrement the stack height
%
(\Verb|##inc-d -1|).
%
Then, \Verb|##compare-imm-branch| in block $2$ compares the value in the
virtual register \Verb|2| (which is a copy of \Verb|1|, the top of the
stack) to the immediate value \factor|f| to see if it's not equal (signified by
\Verb|cc/=|).  However, both branches jump through several empty blocks and
merge at the same destination.  Thus, we can remove both branches and replace
\Verb|##compare-imm-branch| with an unconditional \Verb|##branch| to the
eventual destination.  We see this on the right of
\cref{fig:delete-useless-conditionals}.

\inputfig{split-branches}

In order to expose more opportunities for optimization, \Verb|split-branches|
will actually duplicate code.  We use the fact that code immediately following
a conditional will be executed along either branch.  If it's sufficiently
short, we copy it up into the branches individually.  That is, we change
%
\factor|[ A ] [ B ] if C|
%
into
%
\factor|[ A C ] [ B C ] if|,
%
as long as \Verb|C| is small enough.  Later analyses may then, for example,
more readily eliminate one of the branches if it's never taken.
\cref{fig:split-branches} shows what such a transformation looks like on a
\gls{CFG}.  The example
%
\factor|[ 1 ] [ 2 ] if dup|
%
is essentially changed into
%
\factor|[ 1 dup ] [ 2 dup ] if|,
%
thus splitting the block with two predecessors (block $9$) on the left.

\inputfig{join-blocks}

The next pass, \Verb|join-blocks|, compacts the \gls{CFG} by joining together
blocks involved in only a single control flow edge.  Mostly, this is to clean
up the myriad of empty or short blocks introduced during construction, like
sequences of a bunch of \Verb|##branch|es.  \Vref{fig:join-blocks} shows this
pass on the \gls{CFG} of
%
\factor|0 100 [ 1 fixnum+fast ] times|.
%
\factor|fixnum+fast| is a specialized version of \factor|+| that suppresses
overflow and type checks.  We use it here to keep the \gls{CFG} simple.  We'll
be using this particular code to illustrate all but one of the remaining
optimization passes in \cref{lst:optimize-cfg}, as it's a motivating example
for the work in this thesis.  The passes before \Verb|join-blocks| don't
change the \gls{CFG} seen on the left in \cref{fig:join-blocks}, but we get rid
of the useless \Verb|##branch| blocks in the \gls{CFG} on the right.

\inputfig{normalize-height}

\Vref{fig:normalize-height} shows the result of applying
\Verb|normalize-height| to the result of \Verb|join-blocks|.  This phase
combines and canonicalizes the instructions that track the stack height, like
\Verb|##inc-d|.  While the shuffling in this example isn't complex enough to
be interesting, neither is this phase.  It amounts to more cleanup: multiple
height changes are combined into single ones at the beginnings of the basic
blocks.  In \cref{fig:normalize-height}, this means that \Verb|##inc-d| is
moved to the top of block $1$, as compared to the right of
\cref{fig:join-blocks}.

\inputfig{construct-ssa}

In converting the high-level \gls{IR} to the low-level, we actually lose the
\gls{SSA} form of \Verb|compiler.tree|.  Not only does the construction do
this, but \Verb|split-branches| also copies basic blocks verbatim, so any
value defined will have a duplicate definition site, violating the \gls{SSA}
property.  \Verb|construct-ssa| recomputes a so-called \term{pruned}
\gls{SSA} form, wherein $\phi$ functions are inserted only if the variables are
live after the insertion point.  This cuts down on useless $\phi$ 
%
functions \autocites{TDMSC,construct-ssa}.
%
\Vref{fig:construct-ssa} shows the reconstructed \gls{SSA} form of the
\gls{CFG} from \cref{fig:normalize-height}.

The next pass, \Verb|alias-analysis|, doesn't change the \gls{CFG} of
%
\factor|0 100 [ 1 fixnum+fast ] times|,
%
so we won't have an accompanying \lcnamecref{fig:construct-ssa}.  At a high
level, \Verb|alias-analysis| is easy to understand: it eliminates redundant
memory loads and stores by rewriting certain patterns of memory access.  If the
same location is loaded after being stored, we convert the latter load into a
\Verb|##copy| of the value we stored.  Two reads of the same location with no
intermittent write gets the second read turned into a \Verb|##copy|.
Similarly, if we see two writes without a read in the middle, the first write
can be removed.

\inputfig{value-numbering}

\Verb|value-numbering| is the key focus of this thesis.  It will be detailed
in \cref{sec:vn}.  For now, it does to think of it as a combination of common
subexpression elimination and constant folding.  In \cref{fig:value-numbering}, 
we see several changes:
%
\begin{itemize}
%
  \item \Verb|##load-integer 23 0| in block $1$ of \cref{fig:construct-ssa}
  (which assigns the value \Verb|0| to the virtual register \Verb|23|) is
  redundant, so is replaced by \Verb|##copy 23 21|.
%
  \item \begin{sloppypar}
  In block $2$, the last instruction
  %
  \Verb|##compare-imm-branch 32 f cc/=|
  %
  is the same as
  %
  \Verb|##compare-integer-branch 30 26 cc<|.
  %
  The source register (\Verb|32|) of the original is a \Verb|##copy| of
  \Verb|31|, which itself is computed by
  %
  \Verb|##compare-integer 31 30 26 cc< 9|.
  %
  So, the \Verb|##compare-imm-branch| is equivalent to a simple
  \Verb|##compare-integer-branch|, which doesn't use the temporary virtual
  register \Verb|9| and doesn't waste time comparing against the \factor|f|
  object.
  \end{sloppypar}
%
  \item \begin{sloppypar} The second operands in both \Verb|##add|s of block
  $3$ are just constants stored by \Verb|##load-integer|s.  So, these are
  turned into \Verb|##add-imm|s.
  \end{sloppypar}
%
  \item Also, the second \Verb|##load-integer| in block $3$ just loads
  \Verb|1| like the first instruction.  Therefore, it's replaced by a
  \Verb|##copy|.
\end{itemize}
%
\noindent In \cref{sec:vn}, we'll see how and why this pass fails to identify
other equivalences.

\inputfig{copy-propagation}

Following \Verb|value-numbering|, \Verb|copy-propagation| performs a global
pass that eliminates \Verb|##copy| instructions.  Uses of the copies are
replaced by the originals.  So, in \cref{fig:copy-propagation}, we can see that
all of the \Verb|##copy| instructions have been removed and, for instance,
the use of the virtual register \Verb|25| in block $2$ has been replaced by
\Verb|21|, since \Verb|25| was a copy of it.

\inputfig{eliminate-dead-code}

Next, dead code is removed by \Verb|eliminate-dead-code|.
\Vref{fig:eliminate-dead-code} shows that the \Verb|##compare-integer| in
block $2$ and the \Verb|##load-integer| in block $3$ were removed, since they
defined values that were never used.

\inputfig{finalize-cfg}

The final pass in \cref{lst:optimize-cfg}, \Verb|finalize-cfg|, itself
consists of several more passes.  We will not get into many details here, but
at a high level, the most important passes figure out how virtual registers
should map to machine registers.  We first figure out when certain values can
be unboxed.  Then, instructions are reordered in order to reduce \term{register
pressure}.  That is, we try to schedule instructions around each other so that
we don't need to store more values than we have machine registers.  That way,
we avoid \term{spilling} registers onto the heap, which wastes time.  After
leaving \gls{SSA} form, we perform a \term{linear scan} register allocation,
which replaces virtual registers with machine registers and inserts
\Verb|##spill| and \Verb|##reload| instructions for the cases we can't
avoid.  \Vref{fig:finalize-cfg} shows an example on an Intel x86 machine, which
has enough registers that we needn't spill anything.


%\newpage\chapter{Value Numbering}\label{sec:vn}

At a very basic level, most optimization techniques revolve around avoiding
redundant or unnecessary computation.  Thus, it's vital that we discover which
values in a program are equal.  That way, we can simplify the code that wastes
machine cycles repeatedly calculating the same values.  Classic optimization
phases like constant/copy propagation, common subexpression elimination,
loop-invariant code motion, induction variable elimination, and others
discussed in the de facto treatise, ``The Dragon Book'' \autocite{DragonBook},
perform this sort of redundancy elimination based on information about the
equality of expressions.

In general, the problem of determining whether two expressions in a program are
equivalent is undecidable.  Therefore, we seek a \term{conservative} solution
that doesn't necessarily identify all equivalences, but is nevertheless correct
about any equivalences it does identify.  Solving this equivalence problem is
the work of \term{value numbering} algorithms.  These assign every value in the
program a number such that two values have the same value number if and only if
the compiler can prove they will be equal at runtime.

Value numbering has a long history in literature and practice, spanning many
techniques.  In \cref{sec:compiler:cfg} we saw the \Verb|value-numbering| word,
which is actually based on some of the earliest---and least effective---methods
of value numbering.  \Cref{sec:vn:local} describes the way Factor's current
algorithm works, highlighting its shortcomings to motivate the main work of
this thesis, which is covered in \cref{sec:vn:global,sec:vn:avail}.  We finish
the \lcnamecref{sec:vn} by analyzing the results of these changes and reviewing
the literature for further enhancements that can be made to this optimization
pass.

\subsection{Local Value Numbering}\label{sec:vn:local}

Tracing the exact origins of value numbering is difficult.  It's thought to
have originally been invented in the 1960s by Balke\todo{cite Simpson}.  The
earliest tangible reference to a value numbering (at least, the earliest point
where discussions in the literature seem to start) appears in an oft-cited but
unpublished work of Cocke\todo{cite-like}.  The technique is relatively simple,
but not as powerful as other methods for reasons described hereafter.

The algorithm considers a single basic block.  For each instruction (from top
to bottom) in the block, we essentially let the value number of the assignment
target be a hash of the operator and the value numbers of the operands.  That
is, we hash the \term{expression} being computed by an instruction.  Thus,
assuming a proper hash function, two expressions are \term{congruent} (denoted
$\texttt{x} \cong \texttt{y}$) if
%
\begin{itemize}
%
  \item they have the same operators and
%
  \item their operands are congruent.
%
\end{itemize}
%
\noindent This is our approximation of runtime equivalence.  The first property
is fulfilled by basing the hash, in part, on the operator.  The second property
holds because the hash is based on the value numbers of the statement's
operands---not just the operands as they appear in code (i.e., \term{lexical}
equivalence).  Any information about congruence is propagated through the value
numbers.  We'll have discovered any such equivalences among the operands before
computing the value number of the assignment target because every value in a
basic block is either defined before it's used, or else defined at some point
in a predecessor of the block, which we don't care about when only considering
one basic block.

This is the first shortcoming of the algorithm.  It is \term{local}, focusing
on only one basic block at a time.  Any definitions outside the boundaries of
the basic block won't be reused, even if they reach the block.  This severely
limits the scope of the redundancies we can discover.  We could improve upon
this by considering the algorithm across an entire loop-free \gls{CFG} in any
\term{topological order}.  In such an ordering, a basic block $B$ comes before
any other block $B'$ to which it has an edge.  Thus, any ``outside'' variables
that instructions in $B'$ rely on must have come from $B$ or earlier, which
will have already been computed in a traversal of such an ordering.  However,
\glsplural{CFG} usually contain cycles or loops (at least interesting ones do),
which make such an ordering impossible.  We may still pick a topological order
that ignores back-edges, but we may encounter operands whose values flow along
those back-edges.  We'll later address the issue of encountering instructions
whose operands haven't been processed yet.

In Factor, expressions are basically instructions (the \factor|insn| objects
discussed in \cref{sec:compiler:cfg}) that have had their destination registers
stripped.  Instructions can be converted to expressions with the \factor|>expr|
word defined in the \factor|compiler.cfg.value-numbering.expressions|
vocabulary.  For instance, an \factor|##add| instruction with the destination
register \factor|1| and source registers \factor|2| and \factor|3| will be
converted into an array of three elements:
%
\begin{itemize}
%
  \item The \factor|##add| class word, indicating the expression is derived
        from an \factor|##add| instruction.
%
  \item The value number of the virtual register \factor|2|.
%
  \item The value number of the virtual register \factor|3|.
%
\end{itemize}
%
\noindent Some instructions are not \term{referentially transparent}, meaning
they can't be replaced with the value they compute without changing the
program's behavior.  For example, \factor|##call| and \factor|##branch| cannot
reasonably be converted into expressions.  In these cases, \factor|>expr|
merely returns a unique value.

\inputlst{value-numbering-graph}

The hashing of expressions takes place in the so-called \term{expression graph}
implemented in the vocabulary shown in \vref{lst:value-numbering-graph}.  This
consists of three global hash tables that relate virtual registers, value
numbers, instructions, and expressions.  Since virtual registers are just
integers, we actually use them as value numbers, too.  \factor|vregs>vns| maps
virtual registers to their value numbers.  If a virtual register  is mapped to
itself in this table, its definition is the canonical instruction that we use
to compute the value.  This instruction is stored in the \factor|vns>insns|
table.  Finally, the most important mapping is \factor|exprs>vns|.  True to its
name, it uses expressions as keys, which of course are implicitly hashed.
Thus, we can use this table to determine equivalence of expressions.

Other definitions in \vref{lst:value-numbering-graph} manipulate expressions
and the graph.  The global variable \factor|input-expr-counter| is used in the
generation of unique expressions discussed earlier.  \factor|init-value-graph|
initializes this and all the tables.  \factor|set-vn| establishes a mapping
from a virtual register to a value number in \factor|vregs>vns|.
\factor|vn>insn| gives terse access to the \factor|vns>insns| table.
\factor|vreg>insn| uses \factor|vregs>vns| and \factor|vns>insns| to get the
canonical instruction that defines a given virtual register.  Finally,
\factor|vreg>vn| looks up the value of a key in the \factor|vregs>vns| table.
Importantly, if the key is not yet present in the table, it is automatically
mapped to itself---it's assumed that the virtual register does not correspond
to a redundant instruction.

This is the second shortcoming of the algorithm.  It must make a
\term{pessimistic} assumption about congruences.  That is, it starts by
assuming that every expression has a unique value number, then tries to prove
that there are some values which are actually congruent.

%This is the final version depicted in Algorithm~\vref{alg:hash-vn}.  Clearly,
%it will fail to discover congruences for values that flow along back-edges,
%since we simply ignore back-edges and started with the assumption that values
%are distinct.  For example, refer to Figure~\vref{fig:hash-vn-ex}, which uses
%the SSA form of the CFG in Figure~\vref{fig:cfg-construction}.  We can see that
%the corresponding \lstinline|i|s and \lstinline|j|s are equivalent, but
%Algorithm~\ref{alg:hash-vn} must ignore the back-edge and consider the
%statements in the order shown.  Thus, it only discovers $\code{i$_0$} \cong
%\code{j$_0$}$, but considers $\code{i$_1$} \not\cong \code{j$_1$} \not\cong
%\code{i$_2$} \not\cong \code{j$_2$}$.  This is because
%$\attrib{\code{i$_2$}}{valnum} \ne \attrib{\code{j$_2$}}{valnum}$ (they were
%initialized uniquely) at the time when $\attrib{\code{i$_1$}}{valnum}$ and
%$\attrib{\code{j$_1$}}{valnum}$ were being computed, which gave
%\lstinline|i$_1$| and \lstinline|j$_1$| differing value numbers, which then got
%propagated to the computation of $\attrib{\code{i$_2$}}{valnum}$ and
%$\attrib{\code{j$_2$}}{valnum}$.  Notice that congruence only guarantees
%properties when two values have the same value number.  Nothing can be said of
%incongruent values---not even that they're nonequal (in this case, we have
%incongruent values that actually \emph{are} equal).  This illustrates the
%conservativeness of the solution.

%Factor currently
%  Cocke & Schwartz (effectively what Factor uses)
%
%compiler.cfg.value-numbering
%  process-instruction
%  value-numbering
%  value-numbering-step
%
%compiler.cfg.value-numbering.graph
%  vregs>vns
%  exprs>vns
%  vns>insns
%
%compiler.cfg.value-numbering.expressions
%  value-numbering-step
%    exprs>vns
%    >expr
%    <integer-expr>
%    <reference-expr>
%
%-------------------------------------------------------------------------------
%
%compiler.cfg.value-numbering.rewrite
%  rewrite
%    ! Utilities
%    insn>integer
%    vreg>integer
%    insn>literal
%    vreg>literal
%
%compiler.cfg.value-numbering.alien
%  rewrite
%
%compiler.cfg.value-numbering.comparisons
%  rewrite
%    rewrite-boolean-comparison
%    fold-branch
%    >test
%    >compare-imm
%    simplify-test
%
%compiler.cfg.value-numbering.folding
%  rewrite
%    binary-constant-fold
%    unary-constant-fold
%
%compiler.cfg.value-numbering.math
%  rewrite
%    self-inverse
%    identity
%    compiler.cfg.value-numbering.folding
%    reassociate
%    distribute
%    insn>imm-insn
%
%compiler.cfg.value-numbering.misc
%  rewrite
%
%compiler.cfg.value-numbering.simd
%  rewrite
%
%compiler.cfg.value-numbering.slots
%  rewrite

%\section{Global Value Numbering}\label{sec:vn:global}

Answering the challenges of \citeauthor{Cocke}, \citeauthor{AWZ}
\autocite*{AWZ} described what would be the de facto value numbering algorithm
for several years, and rightly so.  It was a properly \term{global} value
numbering algorithm, working across an entire \gls{CFG} instead of on single
basic blocks.  Their paper was important in another very relevant way: it is
the first published reference to SSA form \autocite{VanDrunen}, including an
algorithm for its construction.

Though we could try to extend the scope of Factor's local value numbering, it
is still inherently pessimistic.  The algorithm of \citeauthor{AWZ}, which is
commonly referred to simply as AWZ, uses a modification of a minimization
algorithm for finite state automata \autocite{Hopcroft}.  It works on an
\term{optimistic} assumption by first assuming every value has the same value
number, then trying to prove that values are actually different.  It does this
by treating value numbers as \term{congruence classes} that partition the set
of virtual registers.  If two values are in the same class, then they are
congruent, where congruence is defined as in \cref{sec:vn:local}.

Such a partition is not unique, in general.  For instance, a trivial one places
each value in its own congruence class.  So, we look for the \term{maximal
fixed point}---the solution that has the most congruent values and therefore
the fewest congruence classes.  We must start with a congruence class for each
operation so that, say, all values computed by \Verb|##add|s are grouped
together, those computed by \Verb|##mul|s are in the same class, etc.  We
must then iteratively look at our collection of classes, separating them when
we discover incongruent values.  For an \gls{SSA} variable in class $P$, we
look at its defining expression.  If an operand at position $i$ belongs to
class $Q$, then the $i^\text{th}$ operand of every other value in $P$ should
also be in $Q$.  Otherwise, $P$ must be \term{split} by removing those
variables whose $i^\text{th}$ operands are not in class $Q$ and placing them in
a new congruence class.  We keep splitting classes until the partitioning
stabilizes.

The optimistic assumption may seem dangerous.  Is it possible that we're
``overoptimistic''?  That two values assumed to be congruent and not proven
incongruent might actually be inequivalent when the program is run?  The AWZ
paper dedicates a section to proving that two congruent variables are
equivalent at a point $p$ in the program if their definitions dominate $p$.
The proof is a bit quirky, but reasonable.  They develop a dynamic notion of
dominance in a running program which implies static dominance in the code, then
show that congruence implies runtime equality (though equivalence does not
imply congruence).

AWZ made the need for \gls{GVN} algorithms apparent.  However, finite state
automata minimization makes for a more complicated algorithm than hash-based
value numbering.  A na\"{i}ve implementation can be quadratic, although careful
data structure and procedure design can make it $O(n\log n)$.  Furthermore,
it's resistant to the same improvements we easily added to the local value
numbering.  To even consider the commutativity of operations requires changes
in operand position tracking and splitting---the heart of the algorithm.  It is
generally limited by what the programmer writes down: deeper congruences due
to, say, algebraic identities can't be discovered.

In fact, by performing an optimization that uses the \gls{GVN} information,
more \gls{GVN} congruences may arise.  If we can somehow perform the two
analyses simultaneously, they'll produce better results.  This generalizes to
interdependent compiler optimizations at large, as elucidated in
\citeauthor{Click}'s dissertation \autocite*{Click}, which describes a method
for formalizing and combining separate optimizations that make optimistic
assumptions (whatever they happen to be for each particular analysis).  He uses
this to merge \gls{GVN} with \term{conditional constant propagation}, which
itself is a combination of constant propagation and unreachable code
elimination (pretty much like the \Verb|propagate| pass from
\cref{sec:compiler:tree}).  Furthermore, \gls{GVN} is extended to handle
algebraic identities, propagate constants, and fold redundant $\phi$s.
Unfortunately, the straightforward algorithm for this is $O(n^2)$, while the
$O(n\log n)$ version presented is not just complicated, but can also miss some
congruences between $\phi$-functions \autocites{Click,Simpson}.

Hot on the heels of this work, \citeauthor{Simpson}'s \autocite*{Simpson}
dissertation provides probably the most exhaustive treatment of \gls{GVN}
algorithms.  He presents several extensions, such as incorporating hash-based
local value numbering into \gls{SSA} construction, handling commutativity in
AWZ, and performing redundant store elimination.  He builds off of the two
classical algorithms independently, which underlines their inherent differences
and limitations.  In general, hash-based value numbering is easy to extend
without greatly impacting the runtime complexity, as is the case in Factor's
implementation.

Drawing from this experience, Simpson's hallmark algorithm combines the best of
both worlds by taking the hash-based algorithm which is easy to understand,
implement, and extend, and making it global, so it identifies more congruences.
Dubbed the ``\gls{RPO} algorithm'', it simply applies hash-based value
numbering iteratively over the \gls{CFG} until we reach the same fixed point
computed by AWZ.  (The fact that it computes the exact same fixed point is
proven fairly straightforwardly in the dissertation.)  It could technically
traverse the \gls{CFG} in any topological order, but Simpson defaults to
reverse postorder.

Because it is based off the hashing algorithm, we get the benefits essentially
for free.  The same simplifications can be performed, but with the added
knowledge of global congruences.  Since the majority of Factor's value
numbering code is dedicated to the \Verb|rewrite| generic, it makes sense to
reuse as much of that code as possible.  Therefore, to convert Factor's local
algorithm to a global one, I modified the existing code to use the \gls{RPO}
algorithm.

\inputlst{gvn-graph}

The most fundamental change is to the expression graph.  Referring to
\vref{lst:gvn-graph}, we see most of the same code as in
\vref{lst:value-numbering-graph}, with changes indicated by arrows
($\longrightarrow$).  Two more global variables have been added, namely
\Verb|changed?| and \Verb|final-iteration?|.  The former is what we use to
guide the fixed-point iteration.  As long as value numbers are changing, we
keep iterating.  An important side effect of this is that we can no longer
perform \Verb|rewrite| online, since the transformations we make aren't
guaranteed to be sound on any iteration except the final one.  This makes the
\gls{RPO} algorithm work \term{offline}, first discovering redundancies, then
eliminating them in a separate pass.  When this elimination pass starts, we'll
set \Verb|final-iteration?| to \factor|t|.

A key change is in the \Verb|vreg>vn| word, which now makes an optimistic
assumption about previously unseen values.  Given a new virtual register that
wasn't in the \Verb|vregs>vns| table, the old version would map the register
to itself, making the value its own canonical representative.  However, if this
version tries to look up a key that does not exist in the hash table, it will
simply return \factor|f| (which Factor will do by default with the \factor|at|
word).  Therefore, every value in the \gls{CFG} starts off with the same value
``number'', \factor|f|.  By the end of the \gls{GVN} pass, there should be no
value left that hasn't been put in the \Verb|vregs>vns| table, as we'll have
processed every definition.

To keep track of whether \Verb|vregs>vns| changes, we simply need to alter
\Verb|set-vn|.  Here, we use \factor|maybe-set-at|, a utility from the
\Verb|assocs| vocabulary.  This works like \factor|set-at|, establishing a
mapping in the hash table.  In addition, it returns a boolean indicating
change: if a new key has been added to the table, we return \factor|t|.
Otherwise, we return \factor|t| only in the case where an old key is mapped to
a new value.  If an old key is mapped to the same value that's already in the
table, \factor|maybe-set-at| returns \factor|f|.  Therefore, when
\Verb|vregs>vns| does change, we set \Verb|changed?| to \factor|t| (which
is what the \factor|on| word does).

Finally, we define a new utility word, \Verb|clear-exprs|, which resets the
\Verb|exprs>vns| and \Verb|vns>insns| tables.  Unlike the local value numbering
phase, we don't reset the entire expression graph.  Instead, we make a pass
over the whole \gls{CFG} at a time.  The only reason optimism works is that we
keep trying to disprove our foolhardy assumptions.  Really, \Verb|vregs>vns|
establishes congruence classes of value numbers.  At first, every value belongs
in one class, \factor|f|.  We make a pass over the \gls{CFG} to disprove
whatever we can about this.  If we've introduced new congruence classes (new
values in the \Verb|vregs>vns| hash), we do another iteration.  But each time,
we use the congruence classes discovered from the previous iteration.  At the
start of each new pass, the expressions and instructions in \Verb|exprs>vns|
and \Verb|vns>insns| are invalidated---their results are based on old
information.  So, these are erased on each iteration.  Much like AWZ, we keep
splitting classes until they can't be split anymore.

\inputlst{gvn-step}

This logic is captured in \vref{lst:gvn-step}.  Rather than reset the tables
when we start processing each basic block in \Verb|value-numbering-step| like
before, we call \Verb|clear-exprs| on each iteration over the \gls{CFG} in
\Verb|value-numbering-iteration|.  Note that \Verb|value-numbering-step| no
longer returns the changed instructions, as we aren't replacing them online.
\Verb|value-numbering-iteration| uses \Verb|simple-analysis| instead of
\Verb|simple-optimization|, which only expects global state to change---no
instructions are updated in the block.  Much to our advantage,
\Verb|simple-analysis| already traverses the \gls{CFG} in \acrlong{RPO}, so
we needn't worry about traversal order.  The top-level word
\Verb|determine-value-numbers| ties this all together by calling
\Verb|value-numbering-iteration| until we can get through it with
\Verb|changed?| remaining false.

\inputlst{gvn-simplify}
\inputlst{gvn-value-number}

\begin{sloppypar}
Note that the work of \Verb|value-numbering-step| is further divided into two
words, \Verb|simplify| and \Verb|value-number|.  These combine to do much
the same work as \Verb|process-instruction| in
\cref{lst:process-instruction}.  \Verb|simplify| makes the repeated calls to
\Verb|rewrite| until the instruction cannot be simplified further.  Its
definition is in \vref{lst:gvn-simplify}.  We then pass the simplified
instruction to \Verb|value-number|, which is defined in
\vref{lst:gvn-value-number}.  This also has a similar structure to
\Verb|process-instruction|.  The main difference is that instructions are no
longer returned (again, they aren't altered in place).  So, the \factor|array|
method uses \factor|each| instead of \factor|map| to recurse into the results
of \Verb|rewrite|.
\end{sloppypar}

A subtle change is necessary with the \Verb|alien-call-insn| and
\Verb|##callback-inputs| methods.  Whereas \Verb|process-instruction|
merely skipped over certain instructions that could not be rewritten, here we
don't have that luxury.  We need to be careful to \Verb|set-vn| every virtual
register that gets defined by any instruction.  While making a pessimistic
assumption, it didn't matter if we did this: any unseen value would be presumed
important by \Verb|vreg>vn|.  However, with the optimistic assumption,
\Verb|vreg>vn| will give the impression that unseen values are all the same
by returning \factor|f|.  Therefore, we simply record the virtual registers
defined in instructions that may define one or more of them.  Specifically,
\Verb|alien-call-insn| and \Verb|##callback-inputs| are classes that
correspond to \gls{FFI} instructions.

The \Verb|##copy| method uses \Verb|set-vn| the same way as before.
\Verb|redundant-instruction|, \Verb|useful-instruction|, and
\Verb|check-redundancy| are also largely the same.  These have just been
tweaked to not return instructions.

\inputlst{phi-expr}

The \Verb|##phi| method in \vref{lst:gvn-value-number} represents a major
change. Before, \Verb|##phi|s were left uninterpreted.  Congruences between
induction variables that flowed along back-edges would not be identifiable.
But now, by checking for redundant \Verb|##phi|s, we may reduce them to
copies.  Each \Verb|##phi| object has an \Verb|inputs| slot, which is a
hash table from basic block to the virtual register that flows from that block.
Thus, there is one input for each predecessor.  The \factor|values| of the hash
will be the virtual registers that might be selected for the \Verb|dst|
value.  We look up the value numbers of these, removing all instances of
\factor|f| with the \factor|sift| word.  If all of the inputs are congruent, we
can call \Verb|redundant-instruction|, setting the value number of the
\Verb|##phi|'s \Verb|dst| to the value number of its first input (without
loss of generality).  The \factor|all-equal?| word will return \factor|t| if
the sequence is empty (as it's vacuously true), so we must make sure not to
call \factor|first| on the sequence, since this will be a runtime error.  If
the sequence is empty, we needn't note the redundancy, as the \Verb|##phi|'s
\Verb|dst| will already have the optimistic value number \factor|f| anyway.
Otherwise, we call \Verb|check-redundancy|.  The purpose of this is to
identify \Verb|##phi|s that are equal to each other.  Even if its inputs are
incongruent, a \Verb|##phi| might still represent a copy of another induction
variable.  So that \Verb|check-redundancy| works, we also define a
\Verb|>expr| method in \Verb|compiler.cfg.gvn.expressions|, as seen in
\vref{lst:phi-expr}.  Here, the expression is an array consisting of the
\Verb|##phi| class word, the current basic block's number, and the inputs'
value numbers.  We include the basic block number because only \Verb|##phi|s
within the same block can be considered equivalent to each other.

The final method in \vref{lst:gvn-value-number} defines the default behavior
for \Verb|value-number|, which calls \Verb|check-redundancy| on the
simplified instruction if it defines a single virtual register.  Note that we
separate the \Verb|alien-call-insn| and \Verb|##callback-inputs| logic from
this, since they happen to define a variable number of registers.  If
particular instances define only one register, we still don't want to call
\Verb|check-redundancy| on them, since they don't have a \Verb|dst| slot.
To avoid calling \Verb|dst>>| and triggering an error in
\Verb|useful-instruction|, we needed separate methods for the \gls{FFI}
classes.

\inputfig{gvn}

With these changes, we can globally identify value numbers, including
equivalences that arise from simplifying instructions (even though no
replacements are actually done yet).  To illustrate this, consider again the
example
%
\factor|0 100 [ 1 fixnum+fast ] times|,
%
reproduced in \vref{fig:gvn}.  As the expression graph changes frequently in
this new algorithm, instead of showing the literal hash tables we'll use a
shorthand notation.  Virtual registers will be integers, and to avoid confusion
value numbers will be written in brackets, like \vn{n}.  Then, we'll show
\Verb|vreg>vn| mappings with the notation $n\to\vn{n}$, where $n$ is the
register and \vn{n} is the value number.  If there is a corresponding
expression in \Verb|exprs>vns|, it will be denoted after the mapping, like
$n\to\vn{n}~(\textit{expression})$.  With the expressions, the instructions in
\Verb|vns>insns| are a bit redundant for understanding the value numbering
process, so they will be elided.  Any mappings to \factor|f| will be elided, as
they're understood to be implicit when a key is absent.

\todo[inline]{Might make separate figures of each block, for easier reference}

\Verb|determine-value-numbers| starts the first iteration, which of course
starts at basic block $1$.  \Verb|##inc-d| is a no-op, but the first two
\Verb|##load-integer|s are established as useful instructions.
%
\Verb|##load-integer 23 0|
%
is recognized as redundant, since at this point we know that \Verb|21| has
the value \Verb|0|.  The \Verb|##copy| instructions all pile on value
number mappings, leaving us with the following:
%
\begin{align*}
  21 &\to \vn{21} \quad (0)  \\
  22 &\to \vn{22} \quad (100)\\
  23 &\to \vn{21}            \\
  24 &\to \vn{22}            \\
  25 &\to \vn{21}            \\
  26 &\to \vn{22}            \\
  27 &\to \vn{21}
\end{align*}

At iteration $1$, basic block $2$, the first \Verb|##phi| has inputs
\Verb|25| (from block $1$) and \Verb|41| (from block $3$).  The former has
the value number \vn{21}, while the latter is still at \factor|f|.  We treat
this value number much like a $\top$ element, unifying it with the other input
to give us the assumption that \Verb|29| will be a copy of \Verb|25|.
Thus, it gets the same value number.  A similar choice happens for the second
\Verb|##phi|.  The instruction 
%
\Verb|##compare-integer 31 30 26 cc< 9|
%
is an interesting case.  Due to our optimistic assumptions thus far, we believe
\Verb|30| is carrying the value \Verb|0|, and that \Verb|26| is set to
\Verb|100|.  Thus, this instruction gets constant-folded by \Verb|simplify|
into
%
\Verb|##load-reference 31 t|.
%
The \gls{CFG} isn't changed, but the expression graph reflects this belief.
Later, this assumption will be invalidated.  The following copies are processed
as usual, with the distinct difference here that 
%
\Verb|##copy 33 26 any-rep|
%
has the global knowledge of the value number of \Verb|26|.  Because the
\Verb|##compare-integer| was constant-folded, so is the
\Verb|##compare-imm-branch|---and to the same value, no less.  This leaves us
with:
%
\begin{align*}
  21 &\to \vn{21} \quad (0)                 \\
  22 &\to \vn{22} \quad (100)               \\
  23 &\to \vn{21}                           \\
  24 &\to \vn{22}                           \\
  25 &\to \vn{21}                           \\
  26 &\to \vn{22}                           \\
  27 &\to \vn{21}                           \\
  29 &\to \vn{21}                           \\
  30 &\to \vn{21}                           \\
  31 &\to \vn{31} \quad (\text{\factor|t|}) \\
  32 &\to \vn{31}                           \\
  33 &\to \vn{22}                           \\
  34 &\to \vn{31}
\end{align*}

Block $3$ of iteration $1$ gives the \Verb|##load-integer|s' destinations the
same value number, corresponding to the integer $1$.  Because optimism makes
the algorithm think that \Verb|29| and \Verb|30| correspond to the integer
$0$, the \Verb|##add|s are constant-folded.  This leaves us with:
%
\begin{align*}
  21 &\to \vn{21} \quad (0)                 \\
  22 &\to \vn{22} \quad (100)               \\
  23 &\to \vn{21}                           \\
  24 &\to \vn{22}                           \\
  25 &\to \vn{21}                           \\
  26 &\to \vn{22}                           \\
  27 &\to \vn{21}                           \\
  29 &\to \vn{21}                           \\
  30 &\to \vn{21}                           \\
  31 &\to \vn{31} \quad (\text{\factor|t|}) \\
  32 &\to \vn{31}                           \\
  33 &\to \vn{22}                           \\
  34 &\to \vn{31}                           \\
  35 &\to \vn{35} \quad (1)                 \\
  36 &\to \vn{35}                           \\
  37 &\to \vn{35}                           \\
  38 &\to \vn{35}                           \\
  39 &\to \vn{21}                           \\
  40 &\to \vn{22}                           \\
  41 &\to \vn{35}                           \\
  42 &\to \vn{35}
\end{align*}

While block $4$ is visited in each iteration, it doesn't define any registers,
so doesn't affect the state of value numbering.  Therefore, the above is the
state left at the end of iteration $1$.

Since \Verb|vregs>vns| clearly changed, iteration $2$ commences by clearing
the expressions, though the value numbers remain.  Block $1$ doesn't change
from iteration $1$, giving us:
%
\begin{align*}
  21 &\to \vn{21} \quad (0)                 \\
  22 &\to \vn{22} \quad (100)               \\
  23 &\to \vn{21}                           \\
  24 &\to \vn{22}                           \\
  25 &\to \vn{21}                           \\
  26 &\to \vn{22}                           \\
  27 &\to \vn{21}                           \\
  29 &\to \vn{21}                           \\
  30 &\to \vn{21}                           \\
  31 &\to \vn{31}                           \\
  32 &\to \vn{31}                           \\
  33 &\to \vn{22}                           \\
  34 &\to \vn{31}                           \\
  35 &\to \vn{35}                           \\
  36 &\to \vn{35}                           \\
  37 &\to \vn{35}                           \\
  38 &\to \vn{35}                           \\
  39 &\to \vn{21}                           \\
  40 &\to \vn{22}                           \\
  41 &\to \vn{35}                           \\
  42 &\to \vn{35}
\end{align*}

Now that we're in iteration $2$, the inputs to the \Verb|##phi|s of block $2$
have been processed once before.  For instance, we still believe that
\Verb|25| corresponds to the integer $0$ (which is incidentally correct), but
now that \Verb|41| has the value number \vn{35}, we think it corresponds to
the integer $1$.  While this is incorrect, it does break the congruence between
the inputs, making the first \Verb|##phi| a useful instruction.  The second
\Verb|##phi|, however, still looks like a copy of the first.  Even so, this
is sufficiently different that the following \Verb|##compare-integer| cannot
be constant-folded like before.  However, it can still be converted to a
\Verb|##compare-integer-imm|, as one of its operands corresponds to an
integer.  The redundant \Verb|##compare-imm-branch| gets rewritten to the
same expression as the \Verb|##compare-integer|, so winds up getting the same
value number.  This gives us:
%
\begin{align*}
  21 &\to \vn{21} \quad (0)                                                \\
  22 &\to \vn{22} \quad (100)                                              \\
  23 &\to \vn{21}                                                          \\
  24 &\to \vn{22}                                                          \\
  25 &\to \vn{21}                                                          \\
  26 &\to \vn{22}                                                          \\
  27 &\to \vn{21}                                                          \\
  29 &\to \vn{29} \quad (\text{\Verb|\#\#phi 2 21 35|})                    \\
  30 &\to \vn{29}                                                          \\
  31 &\to \vn{31} \quad (\text{\Verb|\#\#compare-integer-imm 29 100 cc<|}) \\
  32 &\to \vn{31}                                                          \\
  33 &\to \vn{22}                                                          \\
  34 &\to \vn{31}                                                          \\
  35 &\to \vn{35}                                                          \\
  36 &\to \vn{35}                                                          \\
  37 &\to \vn{35}                                                          \\
  38 &\to \vn{35}                                                          \\
  39 &\to \vn{21}                                                          \\
  40 &\to \vn{22}                                                          \\
  41 &\to \vn{35}                                                          \\
  42 &\to \vn{35}
\end{align*}

Block $3$ of iteration $2$ also changes, since the \Verb|##add|s can't be
constant-folded like before due to our new discovery about the \Verb|##phi|s.
However, the first one can still be converted to an \Verb|##add-imm|, and the
second is marked the same as the first.  This leaves the following value
numbers:
%
\begin{align*}
  21 &\to \vn{21} \quad (0)                                                \\
  22 &\to \vn{22} \quad (100)                                              \\
  23 &\to \vn{21}                                                          \\
  24 &\to \vn{22}                                                          \\
  25 &\to \vn{21}                                                          \\
  26 &\to \vn{22}                                                          \\
  27 &\to \vn{21}                                                          \\
  29 &\to \vn{29} \quad (\text{\Verb|\#\#phi 2 21 35|})                    \\
  30 &\to \vn{29}                                                          \\
  31 &\to \vn{31} \quad (\text{\Verb|\#\#compare-integer-imm 29 100 cc<|}) \\
  32 &\to \vn{31}                                                          \\
  33 &\to \vn{22}                                                          \\
  34 &\to \vn{31}                                                          \\
  35 &\to \vn{35} \quad (1)                                                \\
  36 &\to \vn{36} \quad (\text{\Verb|\#\#add-imm 29 1|})                   \\
  37 &\to \vn{35}                                                          \\
  38 &\to \vn{36}                                                          \\
  39 &\to \vn{29}                                                          \\
  40 &\to \vn{22}                                                          \\
  41 &\to \vn{36}                                                          \\
  42 &\to \vn{36}
\end{align*}

Since the value numbers changed, we start iteration $3$.  The expressions are
cleared, and block $1$ once again does not change anything.  The first
\Verb|##phi| in block $2$ still gets classified as useful, so no value
numbers change.  The major difference, though, is that the previous iteration's
value numbers for registers in block $3$ update the expression we have for the
\Verb|##phi|.  Whereas before we thought it was choosing between \vn{21} (the
integer $0$) and \vn{35} (the integer $1$), the \Verb|##add| wasn't
constant-folded in the previous iteration.  Therefore, the virtual register
\Verb|41| now corresponds to the result of the \Verb|##add| with the value
number \vn{36}.  We still can't disprove that the second \Verb|##phi| is
different (because it, in fact, isn't).  So, we're left with the following
after iteration $3$ finishes with block $2$:
%
\begin{align*}
  21 &\to \vn{21} \quad (0)                                                \\
  22 &\to \vn{22} \quad (100)                                              \\
  23 &\to \vn{21}                                                          \\
  24 &\to \vn{22}                                                          \\
  25 &\to \vn{21}                                                          \\
  26 &\to \vn{22}                                                          \\
  27 &\to \vn{21}                                                          \\
  29 &\to \vn{29} \quad (\text{\Verb|\#\#phi 2 21 36|})                    \\
  30 &\to \vn{29}                                                          \\
  31 &\to \vn{31} \quad (\text{\Verb|\#\#compare-integer-imm 29 100 cc<|}) \\
  32 &\to \vn{31}                                                          \\
  33 &\to \vn{22}                                                          \\
  34 &\to \vn{31}                                                          \\
  35 &\to \vn{35}                                                          \\
  36 &\to \vn{36}                                                          \\
  37 &\to \vn{35}                                                          \\
  38 &\to \vn{36}                                                          \\
  39 &\to \vn{29}                                                          \\
  40 &\to \vn{22}                                                          \\
  41 &\to \vn{36}                                                          \\
  42 &\to \vn{36}
\end{align*}

Blocks $3$ and $4$ do not produce any more changes, so \gls{GVN} has stabilized
after $3$ iterations, with our final congruence classes being:
%
\begin{align*}
  \vn{21} &= \{21, 23, 25, 27\}     \\
  \vn{22} &= \{22, 24, 26, 33, 40\} \\
  \vn{29} &= \{29, 30, 39\}         \\
  \vn{31} &= \{31, 32, 34\}         \\
  \vn{35} &= \{35, 37\}             \\
  \vn{36} &= \{36, 38, 41, 42\}
\end{align*}

\todo[inline]{teletype the numbers in the align*s, I guess}

%\section{Redundancy Elimination}\label{sec:vn:avail}

Now that we've identified congruences across the entire \gls{CFG}, we must
eliminate any redundancies found.  Since value numbering is now offline, this
entails another pass.  However, replacing instructions is more subtle with
global value numbers than it is with local ones.  Because values come from all
over the \gls{CFG}, we must consider if a definition is \term{available} at the
point where we want to use it.  

\Vref{fig:not-avail,fig:is-avail} show the difference.  In the former, we can
see the \gls{CFG} before value numbering for the code
%
\factor|[ 10 ] [ 20 ] if 10 20 30|.
%
The two extra integers being pushed at the end don't really illustrate the
point; they're just there to avoid branch splitting (see
\cref{sec:compiler:cfg}).  In block $4$, there's a
%
\Verb|##load-integer 27 10|,
%
which loads the value \Verb|10|.  In globally numbering values, we associate
the
%
\Verb|##load-integer 22 10|
%
in block $2$ with the value \Verb|10| first, making it the canonical
representative.  However, we can't replace the instruction in block $4$ with
%
\Verb|##copy 27 22|,
%
because control flow doesn't necessarily go through block $2$, so the virtual
register \Verb|22| might not even be defined.  However, in \vref{fig:is-avail},
we see the \gls{CFG} for the code
%
\factor|10 swap [ 10 ] [ 20 ] if 10 20 30|.
%
In this case, the first definition of the value \Verb|10| comes from block $1$,
which dominates block $4$.  So, the definition is available, and we can replace
the \Verb|##load-integer| in block $4$ with a \Verb|##copy|.

\inputfig{not-avail}
\inputfig{is-avail}

There are several ways to decide if we can use a definition at a certain point.
For instance, we could use dominator information, so that a definition in a
basic block $B$ can be used by any other block dominated by $B$
\autocite{Simpson}.  However, here we'll use a data flow analysis called
\term{available expression analysis}, since it is readily implemented.
Mercifully, Factor has a vocabulary that automatically defines data flow
analyses with little more than a single line of code.

\Vref{lst:avail} shows the vocabulary that defines the available expression
analysis.  It is a forward analysis \autocite[see][]{DragonBook} based on the
flow equations below:
\begin{align*}
  \text{\Verb|avail-in|}_i &=
    \begin{cases}
      \varnothing
        & \text{if $i=0$} \\
      \bigcap_{j\in\mathrm{pred}(i)}\text{\Verb|avail-out|}_j
        & \text{if $i>0$}
    \end{cases} \\
  \text{\Verb|avail-out|}_i &= \text{\Verb|avail-in|}_i
                                 \cup 
                                 \text{\Verb|defined|}_i
\end{align*}
%
\noindent Here, the subscripts indicate the basic block number.
$\text{\Verb|defined|}_i$ denotes the result of the \Verb|defined| word from
\vref{lst:avail}.  This returns the set of virtual registers defined in a basic
block.  Since we use virtual registers as value numbers, this is the same as
giving us all the value numbers produced by a basic block.  ``Killed''
definitions are impossible by the \gls{SSA} property, so we needn't track
redefinitions of virtual registers, as in other data flow analyses.  Using set
intersection as the confluence operator means that the
$\text{\Verb|avail-in|}_i$ set will contain those values which are available on
all paths from the start of the \gls{CFG} to block $i$.

\inputlst{avail}

\begin{sloppypar}
Using Factor's \Verb|compiler.cfg.dataflow-analysis| vocabulary, the
implementation takes all of two lines of code.  The
%
\factor|FORWARD-ANALYSIS: avail|
%
line automatically defines several objects, variables, words, and methods that
don't warrant full detail here.  One we're immediately concerned with is the
\Verb|transfer-set| generic, which dispatches upon the particular type of
analysis being performed and is invoked on the proper in-set and basic block.
There is no default implementation, as it is the chief difference between
analyses.  So, the next line uses \Verb|defined| and \factor|assoc-union| to
calculate the result of the data flow equation.  Other pieces we'll see used
are the top-level \Verb|compute-avail-sets| word that actually performs the
analysis, the \Verb|avail-ins| hash table that maps basic blocks to their
in-sets, and the \Verb|avail-in| word that is shorthand for looking up a
basic block's in-set.
\end{sloppypar}

We want to use the results of this analysis in the \Verb|rewrite| methods so
that they will only perform correct and meaningful rewrites.  However, we also
want to use \Verb|rewrite| in the \Verb|determine-value-numbers| pass, where we
don't care about availability.  In fact, we want to ignore availability
altogether in that pass, so that we can discover as many congruences as
possible.  In order to separate these concerns, we need to have two modes for
checking availability.  \Vref{lst:avail} defines the \Verb|available?| word to
do just this.  It will only check the actual availability if
\Verb|final-iteration?| is true, otherwise defaulting to \factor|t|.
Therefore, during the value numbering phase, everything is considered
available.  We further define the utilities \Verb|available-uses?| and
\Verb|with-available-uses?|.  The former checks if all an instruction's uses
are available, and the latter does this only if another quotation first returns
a true value.  That way, we can guard instruction predicates with a test for
available uses, like
%
\factor|[ ##add-imm? ] with-available-uses?|.

Finding all the instances where \Verb|rewrite| needed to be altered was subtle.
Since the old \Verb|value-numbering| was an online optimization, it didn't need
to worry about modifying an instruction in memory.  But by doing the
fixed-point iteration, we cannot permit \Verb|rewrite| to destructively modify
any object until the final iteration.  An obvious instance was in
\Verb|compiler.cfg.value-numbering.comparisons| with the word
\Verb|fold-branch|, responsible for modifying the \gls{CFG} to remove an
untaken branch.  We definitely would not want the branch removed while doing
the fixed-point iteration, because the transformation is not necessarily sound.
So, we can protect it with a check for \Verb|final-iteration?|, as in
\vref{lst:fold-branch}.

\inputlst{fold-branch}

\begin{sloppypar}
More typical instances of the problems that occurred were in words like
\Verb|self-inverse| from \Verb|compiler.cfg.value-numbering.math| (refer to
\vref{lst:self-inverse}).  The idea is essentially to change
%
\begin{center}
  \begin{minipage}{0.2\linewidth}
    \begin{factorcode*}{gobble=6,frame=none}
      ##neg 1 2
      ##neg 3 1
    \end{factorcode*}
  \end{minipage}
\end{center}
%
\noindent into
%
\begin{center}
  \begin{minipage}{0.2\linewidth}
    \begin{factorcode*}{gobble=6,frame=none}
      ##neg 1 2
      ##copy 3 2 any-rep
    \end{factorcode*}
  \end{minipage}
\end{center}
%
\noindent since \Verb|##neg| undoes itself.  But \Verb|rewrite| only has
knowledge of one instruction at a time, so it looks up the redundant
\Verb|##neg|'s source register in the \Verb|vns>insns| table to see if it's
computed by another \Verb|##neg| instruction.  For straight-line code this is
alright, but the input to the originating \Verb|##neg| (in the example, the
virtual register \Verb|2|) isn't necessarily available.  So, we have to use
\Verb|with-available-uses?| to make sure the virtual registers used by the
result of a \Verb|vreg>insn| can themselves be utilized before we rewrite
anything.
\end{sloppypar}

\inputlst{self-inverse}

An even subtler issue that led to infinite loops occured in simplifcations like
the arithmetic distribution in \Verb|compiler.cfg.value-numbering.math|.  The
problem is that the \Verb|rewrite| method would generate instructions that
assigned to entirely brand new registers.  These, of course, would invariably
get value numbered, triggering a change in the \Verb|vregs>vns| table.  A new
iteration would begin, and (since it gets called on the same instructions as
the previous iteration) \Verb|rewrite| would generate new virtual registers
all over again.  Therefore, the \Verb|vregs>vns| table would never stop
changing.  As a stop-gap, distribution had to be disabled altogether until the
final iteration.

Armed with the correct rewrite rules, availability information, and global
value numbers, we can perform \gls{GCSE}.  The logic in the \Verb|gcse| generic
in \vref{lst:gcse} is similar to \Verb|process-instruction| from
\vref{lst:process-instruction} and \Verb|value-number| from
\vref{lst:gvn-value-number}.  Unlike \Verb|value-number|, we do return an
instruction (or sequence thereof) representing the replacement.  Thus, the
\Verb|array| method of \Verb|gcse| uses \factor|map| instead of \factor|each|,
to hold onto the resulting sequence when recursing into several instructions.

\inputlst{gcse}

\Verb|defs-available| is similar to \Verb|record-defs| from
\vref{lst:gvn-value-number}, except that value numbers have already stabilized,
so we don't call \Verb|set-vn|.  Instead, we use the \Verb|make-available|
word, which was the last one defined in \vref{lst:avail}.  We have to ensure
that after processing an instruction, any register it defines is available to
future instructions in the same block, thus enabling rewrites.  So, we add
the virtual register to that block's \Verb|avail-in| (which acts like a set,
even though it's implemented by a hash table by Factor's data flow analysis
framework).  \Verb|alien-call-insn|s, \Verb|##callback-inputs| instructions,
and instances of \Verb|##copy| don't get rewritten any further, so we simply
note that their definitions are available and move on.

The \Verb|?eliminate| word transforms an instruction into a \Verb|##copy|
of the canonical value number that computes it.  If the value number isn't
available, we don't do anything but post-process with \Verb|defs-available|.
If it is, a \Verb|##copy| is produced and its destination is made available.
Thus, \Verb|eliminate-redundancy| works like \Verb|check-redundancy| from
\vref{lst:gvn-value-number}.  We look up the expression computed by the
instruction in the \Verb|exprs>vns| table.  If it's there, we call
\Verb|?eliminate|, but otherwise we leave the instruction alone and make its
definitions available.

The rest of the logic mirrors that of \Verb|value-number|.  If the inputs to a
\Verb|##phi| are all congruent, we'll call \Verb|?eliminate| to transform it
into a \Verb|##copy| of its first input (without loss of generality).
Otherwise, we check for equivalent \Verb|##phi|s with
\Verb|eliminate-redundancy|.  Finally, the \Verb|insn| method will default to
calling \Verb|eliminate-redundancy| on instructions that define only one value,
much like how \Verb|value-number| worked.

\begin{sloppypar}
The main word that performs the pass is \Verb|eliminate-common-subexpressions|.
\Verb|final-iteration?| is turned on (set to \factor|t|), and we make sure to
compute the \Verb|avail-in| sets needed to make \Verb|available?| work.  Then,
using \Verb|simple-optimization|, we iterate over each basic block.  For each
instruction, we first use \Verb|simplify| (refer to \vref{lst:gvn-simplify}),
then call \Verb|gcse| on the rewritten instruction.  Thus, \Verb|rewrite| does
the work of simplifying instructions, then \Verb|gcse| cleans up redundant ones
by converting them into \Verb|##copy| instructions if possible.  The new
\Verb|value-numbering| word can be seen in \vref{lst:new-value-numbering}.
\end{sloppypar}

\inputlst{new-value-numbering}

\begin{sloppypar}
Consider for the last time the example
%
\factor|0 100 [ 1 fixnum+fast ] times|.
%
Again, we have the \gls{CFG} in \vref{fig:gcse-before}.  Making a final pass
with \Verb|eliminate-common-subexpressions| gives us the \gls{CFG} in
\vref{fig:gcse-after}.  Compared to the \gls{CFG} after the old
\Verb|value-numbering| word was called (see \vref{fig:value-numbering}), we
have identified several more redundancies:
\begin{itemize}
  \item The second \Verb|##phi| in block $2$ has been turned into a
  \Verb|##copy| of the first.
%
  \item The \Verb|##compare-integer| of block $2$ has been simplified to
  a \Verb|##compare-integer-imm|, since its operand \Verb|26| is both
  available and known to correspond to the integer value \Verb|100|.
%
  \item Similarly, we've managed to convert the \Verb|##compare-integer-branch|
  at the end of block $2$ into a \Verb|##compare-integer-imm-branch|.
%
  \item Because the \Verb|##phi|s have been recognized as copies (i.e., the
  induction variables are congruent), the second \Verb|##add| in block $3$ is
  turned into a \Verb|##copy| of the first (which itself is still turned into
  an \Verb|##add-imm|).
\end{itemize}
\end{sloppypar}

\inputfig{gcse-before}
\inputfig{gcse-after}

Afterwards, the \Verb|copy-propagation| pass cleans up all of these newly
identified copies, as seen in \vref{fig:gcse-copy-prop}.
\Verb|eliminate-dead-code| now gets rid of more instructions than before, such
as the second \Verb|##load-integer| in block $1$, since it has been propagated
to the \Verb|-imm| instructions in block $2$.  See \vref{fig:gcse-dce}.  At
last, after \Verb|finalize-cfg| in \vref{fig:gcse-finalize}, we see a loop that
uses a single register---down from the three in \vref{fig:finalize-cfg}.

\inputfig{gcse-copy-prop}
\inputfig{gcse-dce}
\inputfig{gcse-finalize}
\clearpage

%\section{Results}\label{sec:vn:results}

The goal of improving the optimization in Factor is, of course, to reduce the
average running time of programs, and to do so without changing their
semantics.  Short of formal verification, the latter requirement makes it
necessary to thoroughly test any code that gets compiled with the new pass
enabled.  To this end, we'll employ Factor's extensive unit test coverage.
Because Factor is (largely) self-hosting, its standard vocabularies are written
in Factor code, typically coupled with tests.  While some vocabularies will
have more test coverage than others, the total amount of tests is quite large.
By compiling each vocabulary and running their tests, we're indirectly testing
the compiler: if tests that used to pass no longer do, then the new pass is
changing the semantics of the code somehow.  Though passing all tests does not
guarantee the algorithm is correct, it does let us know that no known
regressions have been introduced.  Happily, with the new
\Verb|value-numbering| phase enabled, all the same tests pass as before in a
call to \factor|test-all| from a freshly bootstrapped image.

The efficacy of the changes, on the other hand, must be measured relative to
old benchmarks.  Again, Factor has its bases covered, with a suite of $80$
benchmarks run by the \Verb|benchmark| vocabulary.  Each benchmark is run $5$
times, where the garbage collector is run before each iteration.  The minimum
time from these runs is then used as the benchmark result.  The data below
comes from two separate runs of the \factor|benchmarks| word, which invokes all
the benchmark sub-vocabularies.  The ``before'' time used the local value
numbering, while ``after'' times had \Verb|value-numbering| replaced with the
\gls{GVN} pass.  The ``change'' is measured by the formula
%
$$\frac{\text{before} - \text{after}}{\text{before}} \times 100$$
%
to indicate the relative running times.  Negative values in this column are
good, as that means the running time has decreased.

\todo[inline]{Guess I should provide my PC's specs}

\begin{longtable}{llll}
\toprule
Benchmark & Before (seconds) & After (seconds) & Change (\%) \\
\midrule
\endhead
\texttt{benchmark.3d-matrix-scalar}         & 3.705816738       & 3.046126696         & $-17.80$    \\
\texttt{benchmark.3d-matrix-vector}         & 0.161298778       & 0.089539887         & $-44.49$    \\
\texttt{benchmark.backtrack}                & 4.280001561       & 2.358672591         & $-44.89$    \\
\texttt{benchmark.base64}                   & 5.127831493       & 2.853612485         & $-44.35$    \\
\texttt{benchmark.beust1}                   & 7.531546384       & 4.604929188         & $-38.86$    \\
\texttt{benchmark.beust2}                   & 20.308680548      & 12.843534349        & $-36.76$    \\
\texttt{benchmark.binary-search}            & 3.729776895       & 2.349520427         & $-37.01$    \\
\texttt{benchmark.binary-trees}             & 9.403166818       & 6.518867479         & $-30.67$    \\
\texttt{benchmark.bootstrap1}               & 32.472196349      & 30.887877896        & $-4.88$     \\
\texttt{benchmark.chameneos-redux}          & 2.923900422       & 2.041007328         & $-30.20$    \\
\texttt{benchmark.continuations}            & 0.273525202       & 0.200695972         & $-26.63$    \\
\texttt{benchmark.crc32}                    & 0.010623653       & 0.005282642         & $-50.27$    \\
\texttt{benchmark.dawes}                    & 1.588111926       & 1.027176578         & $-35.32$    \\
\texttt{benchmark.dispatch1}                & 7.640720326       & 5.106558985         & $-33.17$    \\
\texttt{benchmark.dispatch2}                & 5.221652668       & 3.984754032         & $-23.69$    \\
\texttt{benchmark.dispatch3}                & 9.710520454       & 6.203527737         & $-36.12$    \\
\texttt{benchmark.dispatch4}                & 8.224931156       & 4.098265543         & $-50.17$    \\
\texttt{benchmark.dispatch5}                & 4.74357434        & 3.478219608         & $-26.68$    \\
\texttt{benchmark.e-decimals}               & 3.903754723       & 2.646958072         & $-32.19$    \\
\texttt{benchmark.e-ratios}                 & 4.774454589       & 3.658075473         & $-23.38$    \\
\texttt{benchmark.empty-loop-0}             & 0.251816164       & 0.199189271         & $-20.90$    \\
\texttt{benchmark.empty-loop-1}             & 1.039242509       & 0.857588545         & $-17.48$    \\
\texttt{benchmark.empty-loop-2}             & 0.472215346       & 0.387974286         & $-17.84$    \\
\texttt{benchmark.euler150}                 & 37.785852299      & 27.05450689         & $-28.40$    \\
\texttt{benchmark.fannkuch}                 & 9.627490235       & 6.8970571           & $-28.36$    \\
\texttt{benchmark.fasta}                    & 7.25292282        & 5.640517069         & $-22.23$    \\
\texttt{benchmark.fib1}                     & 0.179389215       & 0.164933805         & $-8.06$     \\
\texttt{benchmark.fib2}                     & 0.205853157       & 0.138174211         & $-32.88$    \\
\texttt{benchmark.fib3}                     & 0.785036151       & 0.539739186         & $-31.25$    \\
\texttt{benchmark.fib4}                     & 0.391805799       & 0.260370111         & $-33.55$    \\
\texttt{benchmark.fib5}                     & 1.508625224       & 1.002724851         & $-33.53$    \\
\texttt{benchmark.fib6}                     & 19.202504502      & 13.146010511        & $-31.54$    \\
\texttt{benchmark.gc0}                      & 7.360087104       & 5.508594031         & $-25.16$    \\
\texttt{benchmark.gc1}                      & 0.418173431       & 0.281497214         & $-32.68$    \\
\texttt{benchmark.gc2}                      & 25.611210221      & 19.716168704        & $-23.02$    \\
\texttt{benchmark.gc3}                      & 2.757943071       & 2.210785891         & $-19.84$    \\
\texttt{benchmark.hashtables}               & 8.068216942       & 7.997106348         & $-0.88$     \\
\texttt{benchmark.heaps}                    & 4.360368411       & 4.32169158          & $-0.89$     \\
\texttt{benchmark.iteration}                & 7.875561986       & 6.277891729         & $-20.29$    \\
\texttt{benchmark.javascript}               & 17.881224721      & 12.74204052         & $-28.74$    \\
\texttt{benchmark.knucleotide}              & 5.490420772       & 3.5704101           & $-34.97$    \\
\texttt{benchmark.mandel}                   & 0.251711276       & 0.198695557         & $-21.06$    \\
\texttt{benchmark.matrix-exponential-scalar}& 16.451432774      & 12.017000042        & $-26.95$    \\
\texttt{benchmark.matrix-exponential-simd}  & 0.681684747       & 0.536850343         & $-21.25$    \\
\texttt{benchmark.md5}                      & 10.40516678       & 9.198666403         & $-11.60$    \\
\texttt{benchmark.mt}                       & 33.91981743       & 29.961085146        & $-11.67$    \\
\texttt{benchmark.nbody}                    & 9.203478441       & 6.795154145         & $-26.17$    \\
\texttt{benchmark.nbody-simd}               & 0.845814208       & 0.854773096         & $+1.06$     \\
\texttt{benchmark.nested-empty-loop-1}      & 0.097090973       & 0.068475608         & $-29.47$    \\
\texttt{benchmark.nested-empty-loop-2}      & 0.893126911       & 0.861327078         & $-3.56$     \\
\texttt{benchmark.nsieve}                   & 1.086110659       & 1.137648699         & $+4.75$     \\
\texttt{benchmark.nsieve-bits}              & 2.707271763       & 2.815509077         & $+4.00$     \\
\texttt{benchmark.nsieve-bytes}             & 0.785041878       & 1.211421146         & $+54.31$    \\
\texttt{benchmark.partial-sums}             & 3.762171661       & 4.130144177         & $+9.78$     \\
\texttt{benchmark.pidigits}                 & 2.182877913       & 2.195385034         & $+0.57$     \\
\texttt{benchmark.random}                   & 5.66540782        & 5.71913683          & $+0.95$     \\
\texttt{benchmark.raytracer}                & 5.047070171       & 4.39514879          & $-12.92$    \\
\texttt{benchmark.raytracer-simd}           & 1.072588515       & 0.980927338         & $-8.55$     \\
\texttt{benchmark.recursive}                & 2.703509403       & 2.529087637         & $-6.45$     \\
\texttt{benchmark.regex-dna}                & 2.208584014       & 1.808859571         & $-18.10$    \\
\texttt{benchmark.reverse-complement}       & 2.801163847       & 2.353254665         & $-15.99$    \\
\texttt{benchmark.ring}                     & 1.822206473       & 1.62482491          & $-10.83$    \\
\texttt{benchmark.sfmt}                     & 2.675838657       & 2.463367198         & $-7.94$     \\
\texttt{benchmark.sha1}                     & 11.964973943      & 11.142380303        & $-6.88$     \\
\texttt{benchmark.simd-1}                   & 1.857778672       & 1.703206011         & $-8.32$     \\
\texttt{benchmark.sockets}                  & 10.636346636      & 10.516448454        & $-1.13$     \\
\texttt{benchmark.sort}                     & 0.695635429       & 0.581855635         & $-16.36$    \\
\texttt{benchmark.spectral-norm}            & 3.433630383       & 2.960833789         & $-13.77$    \\
\texttt{benchmark.spectral-norm-simd}       & 2.743240011       & 3.237017655         & $+18.00$    \\
\texttt{benchmark.stack}                    & 1.580016742       & 2.004478602         & $+26.86$    \\
\texttt{benchmark.struct-arrays}            & 2.180774222       & 2.421915609         & $+11.06$    \\
\texttt{benchmark.sum-file}                 & 0.883097981       & 0.957151577         & $+8.39$     \\
\texttt{benchmark.terrain-generation}       & 1.611800222       & 1.887047663         & $+17.08$    \\
\texttt{benchmark.tuple-arrays}             & 0.262747557       & 0.329399609         & $+25.37$    \\
\texttt{benchmark.typecheck1}               & 1.750223408       & 1.674592158         & $-4.32$     \\
\texttt{benchmark.typecheck2}               & 1.674738245       & 1.553203741         & $-7.26$     \\
\texttt{benchmark.typecheck3}               & 1.891206648       & 1.735390184         & $-8.24$     \\
\texttt{benchmark.ui-panes}                 & 0.305595039       & 0.29985214          & $-1.88$     \\
\texttt{benchmark.xml}                      & 3.013709363       & 2.722223892         & $-9.67$     \\
\texttt{benchmark.yuv-to-rgb}               & 0.398174487       & 0.318891664         & $-19.91$    \\
\end{longtable}

The results are promising: of $80$ benchmarks, only $13$ showed any increase in
running time.  And of those, even fewer showed significant increases.
Duplicated below for convenience are the benchmarks that ran slower, sorted in
decreasing order of the percent difference between running times.  We can see
the last five or six benchmarks exhibited negligible differences---not only is
the relative change tiny, but the absolute difference in running times is less
than $0.1$ seconds.  (The \Verb|benchmark.tuple-arrays| results also show a
similar absolute change, but it is relatively much larger.)

\begin{longtable}{llll}
\toprule
Benchmark & Before (seconds) & After (seconds) & Change (\%) \\
\midrule
\endhead
\texttt{benchmark.nsieve-bytes}             & 0.785041878       & 1.211421146         & $+54.31$    \\
\texttt{benchmark.stack}                    & 1.580016742       & 2.004478602         & $+26.86$    \\
\texttt{benchmark.tuple-arrays}             & 0.262747557       & 0.329399609         & $+25.37$    \\
\texttt{benchmark.spectral-norm-simd}       & 2.743240011       & 3.237017655         & $+18.00$    \\
\texttt{benchmark.terrain-generation}       & 1.611800222       & 1.887047663         & $+17.08$    \\
\texttt{benchmark.struct-arrays}            & 2.180774222       & 2.421915609         & $+11.06$    \\
\texttt{benchmark.partial-sums}             & 3.762171661       & 4.130144177         & $+9.78$     \\
\texttt{benchmark.sum-file}                 & 0.883097981       & 0.957151577         & $+8.39$     \\
\texttt{benchmark.nsieve}                   & 1.086110659       & 1.137648699         & $+4.75$     \\
\texttt{benchmark.nsieve-bits}              & 2.707271763       & 2.815509077         & $+4.00$     \\
\texttt{benchmark.nbody-simd}               & 0.845814208       & 0.854773096         & $+1.06$     \\
\texttt{benchmark.random}                   & 5.66540782        & 5.71913683          & $+0.95$     \\
\texttt{benchmark.pidigits}                 & 2.182877913       & 2.195385034         & $+0.57$     \\
\end{longtable}

Overall, even transitioning to a relatively simple \gls{GVN} algorithm amounts
to a positive change in Factor's compiler.  More redundancies are eliminated,
resulting in speedier programs.  Judging by unit tests, the implementation is
at least as sound as the previous local value numbering,  as all the same tests
have passed.

%% Future
%   SCC (discussion of potential improvement)
%   Click, Gargi, et al. (future directions)



\renewcommand{\bibname}{References}\newpage\printbibliography

\end{document}

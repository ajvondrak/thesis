\documentclass[11pt,article,oldfontcommands]{cpp-thesis}

\setlength{\parskip}{\medskipamount}

\title{Global Value Numbering in Factor}
\author{Alex Vondrak}
\degree{Master of Science}
\field{Computer Science}
\chair{Dr. Craig Rich}{Computer Science}
\memberA{Dr. Daisy Sang}{Computer Science}
\memberB{Dr. Amar Raheja}{Computer Science}
\quarter{Summer 2011}

\usepackage{todonotes}

\usepackage{bold-extra}

\usepackage{glossaries}
\glsdisablehyper

\usepackage{indentfirst}

\usepackage{tikz}

\usepackage{lib/minted}
\usemintedstyle{bw}
\newminted{factor}{gobble=4,frame=single}
\newmint{factor}{}

%TODO figure out the "right" way to do terminology typesetting
\newcommand{\term}[1]{\emph{#1}}

\newcommand{\inputfig}[1]{
  \begin{figure}
    \input{fig/#1}
    \label{fig:#1}
  \end{figure}
}

\newcommand{\inputlst}[2][tbp]{
  \begin{figure}[#1]
    \input{lst/#2}
    \label{lst:#2}
  \end{figure}
}

\usepackage{array}
\newcolumntype{k}[1]{|>{\centering\arraybackslash\hspace{0pt}}p{#1}|}

\usepackage{varioref}
\usepackage{hyperref}
\usepackage[capitalise]{cleveref}
\def\reftextcurrent{\unskip}
\crefname{figure}{Figure}{Figures}
\crefname{subfigure}{Figure}{Figures}

\usepackage{graphicx}
\newcommand{\whenNewer}[3]{%
  \ifnum\pdfstrcmp{\pdffilemoddate{#1}}%
  {\pdffilemoddate{#2}}>0%
  {\immediate\write18{#3}}\fi%
}
\newcommand{\includesvg}[2][]{%
  \whenNewer{#2.svg}{#2.pdf}%
    {inkscape -z -D --file=#2.svg --export-pdf=#2.pdf}%
  \includegraphics[#1]{#2.pdf}%
}

\usepackage{amsmath,amssymb}
\allowdisplaybreaks[1]
\newcommand{\vn}[1]{\ensuremath{\langle\texttt{#1}\rangle}}
\newcommand{\mem}[1]{\ensuremath{\textsc{reg}[\texttt{#1}]}}

\usepackage{longtable,booktabs}

\usepackage[
  style=authoryear,
  sortcites=true, % not that I care a whole lot
  %backref=true,
  abbreviate=false,
  alldates=long,
  dateabbrev=false,
  maxbibnames=5,
  url=true,
  doi=false,
  firstinits=true,
  dashed=false,
  uniquename=init
]{biblatex}

\DeclareNameAlias{sortname}{last-first}

\renewbibmacro*{date+extrayear}{%
  \iffieldundef{year}
    {}
    {\printtext{%
   \addperiod\space\printfield{labelyear}%
   \printfield{extrayear}}}}

\renewcommand*{\mkbibparens}[1]{%
  {\ifcitation{\bibleftbracket#1\bibrightbracket}%
              {\bibleftparen#1\bibrightparen}}%
}

\renewcommand*{\bibopenparen}[1]{%
  {\ifcitation{\bibleftbracket#1}{\bibleftparen#1}}%
}

\renewcommand*{\bibcloseparen}{%
  {\ifcitation{\bibrightbracket}{\bibrightparen}}%
}

\renewcommand{\contentsname}{Table of Contents}

\defbibheading{bibliography}{\bibsection}

\addbibresource{thesis.bib}

\setpnumwidth{2em}
\setrmarg{3em}

\overfullrule=1mm %XXX

\newacronym{JVM}{JVM}{Java Virtual Machine}
\newacronym{RPN}{RPN}{Reverse Polish Notation}
\newacronym{TCO}{TCO}{tail-call optimization}
\newacronym{GC}{GC}{garbage collector}
\newacronym{FFI}{FFI}{foreign function interface}
\newacronym{VM}{VM}{virtual machine}
\newacronym{IR}{IR}{intermediate representation}
\newacronym{CFG}{CFG}{control flow graph}
\newacronym{SSA}{SSA}{static single assignment}
\newacronym{SCCP}{SCCP}{sparse conditional constant propagation}
\newacronym{GVN}{GVN}{global value numbering}
\newacronym{RPO}{RPO}{reverse postorder}
\newacronym{GCSE}{GCSE}{global common subexpression elimination}
\newacronym{SCC}{SCC}{strongly connected component}


\begin{document}

\frontmatter

\begin{ack}
  \null
  % Bah; ack header effectively kills joke---I want a dedication, dammit
  \begin{vplace}[0.1]
    \begin{center}
      \textit{\Large To Lindsay---he is my rock}
    \end{center}
  \end{vplace}
\end{ack}

\begin{abstract}

  Compilers translate code in one programming language into semantically
  equivalent code in another language---canonically from a high-level language
  to low-level machine primitives.  Generally, the further removed a
  language's abstractions get from those of a computer, the harder it gets to
  compile code into an efficient representation.  What isn't redundant in the
  source language may map to repetitive target instructions that waste time
  recomputing results.  To combat this, compilers try to optimize away
  redundancies by looking for values that are provably equivalent when the
  program is run.

  This thesis explores the theory and implementation of a particularly
  aggressive analysis called global value numbering in a particularly
  high-level language called Factor.  Factor is a stack-based,
  dynamically-typed, object-oriented language born in late $2003$.  A baby
  among languages (now at version $0.94$), its compiler craves all the
  optimizations it can get.  By altering the existing local value numbering
  pass, redundancies can be identified and eliminated across entire programs,
  rather than isolated regions of code.  This induces speedups as high as
  $45\%$ across the majority of benchmarks.  The results from these
  comparatively simple changes hold much promise for future improvements in
  making Factor programs more efficient.
  \newpage
\end{abstract}

\tableofcontents*
\listoffigures

\mainmatter

        \section{Introduction}\label{sec:intro}

% CFGs, SSA form

Compilers translate programs written in a source language (e.g., Java) into
semantically equivalent programs in some target language (e.g., assembly code).
They let us make our source language arbitrarily abstract so we can write
programs in ways that humans understand while letting the computer execute
programs in ways that machines understand.  In a perfect world, such
translation would be straightforward.  Reality, however, is unforgiving.
Straightforward compilation results in clunky target code that performs a lot
of redundant computations.  To produce efficient code, we must rely on
less-than-straightforward methods.  Typical compilers go through a stage of
\term{optimization}, whereby a number of semantics-preserving transformations
are applied to an \term{intermediate representation} of the source code.  These
then (hopefully) produce a more efficient version of said representation.
Optimizers tend to work in \term{phases}, applying specific transformations
during any given phase.

\Gls{GVN} is such an analysis performed by many highly-optimizing compilers.
Its roots run deep through both the theoretical and the practical.  Using the
results of this analysis, the compiler can identify expressions in the source
code that produce the same value---not just by lexical comparison (i.e.,
variables having the same name), but by proving equivalences between what's
actually computed at runtime.  These expressions can then be simplified by
further algorithms for redundancy elimination.  This is the very essence of
most compiler optimizations: avoid redundant computation, giving us code that
runs as quickly as possible while still following what the programmer
originally wrote.

High-level, dynamic languages tend to suffer from efficiency issues: they're
often interpreted rather than compiled, and perform no heavy optimization of
the source code.  However, the Factor language (\url{http://factorcode.org})
fills an intriguing design niche, as it's very high-level yet still fully
compiled.  It's still young, though, so its compiler craves all the
improvements it can get.  In particular, while Factor currently has a
\term{local} value numbering analysis, it is inferior to \gls{GVN} in several
significant ways.

In this thesis, we explore the implementation and use of \gls{GVN} in improving
the strength of optimizations in Factor.  After establishing some preliminary
terminology and concepts in \cref{sec:intro:prelim}, we turn our attention to
the details.  Because Factor is a young and relatively unknown language,
\cref{sec:primer} provides an short tutorial, laying a foundation for
understanding the changes.  \Cref{sec:compiler} describes the overall
architecture of the Factor compiler, highlighting where the exact contributions
of this thesis fit in.  Finally, \cref{sec:vn} goes into detail about the
existing and new value numbering passes, closing with a look at the results
achieved and directions for future work.  In total, the following Factor
libraries were written for this thesis:
\begin{itemize}
  \item A \factor|graphviz| library, which provides bindings to create and
        manipulate graphs in Factor and output them using Graphviz
        (\url{http://graphviz.org}).
  \item The \factor|compiler.cfg.graphviz| library, which sues the Graphviz
        bindings to output images of Factor's low-level intermediate
        representation.  This is responsible for the illustrations of
        optimization passes seen in \cref{sec:vn}.
  \item The \factor|compiler.cfg.gvn| module, which was created by copying the
        existing value numbering code (from
        \factor|compiler.cfg.value-numbering|) to make the pass global.
\end{itemize}

All code was written atop Factor version $0.94$, and copies of it can be found
in the appendices.  In the unlikely event that you want to cite this thesis,
you may use the following \textsc{Bib}\TeX~entry:
\begin{center}
  \begin{Verbatim}[gobble=4,frame=single]
    @mastersthesis{vondrak:11,
      author = {Alex Vondrak},
      title  = {Global Value Numbering in Factor},
      school = {California Polytechnic State University, Pomona},
      month  = sep,
      year   = {2011},
    }
  \end{Verbatim}
\end{center}

%\subsection{Preliminaries}

%The most common intermediate representation (or at least the one we'll use
%here) is the \defn{control flow graph} (or \defn{CFG}, not to be confused with
%the abbreviation for ``context-free grammar'').  

%\begin{figure}
%  \begin{center}
%    \begin{minipage}{0.3\linewidth}
%      \begin{lstlisting}
%    i := 0
%    j := 0
%L1: i := i + 1
%    j := j + 1
%    if $P$ goto L1
%      \end{lstlisting}
%    \end{minipage}
%    \vrule
%    \begin{minipage}{0.3\linewidth}
%      \begin{tikzpicture}[node distance=.85in,>=latex]
%        \node[draw,label=right:$B_1$] (0) at (0,-1.25) {
%          \begin{lstlisting}
%i := 0
%j := 0
%          \end{lstlisting}
%        };
%        \node[draw,label=right:$B_2$] (1) [below of=0] {
%          \begin{lstlisting}
%i := i + 1
%j := j + 1
%if $P$ goto $B_2$
%          \end{lstlisting}
%        };
%
%        \draw[->] (0,0) -- (0);
%        \draw[->] (0) -- (1);
%        \draw (1.south)
%              edge[->,out=180,in=270,controls=+(35:-4) and +(-35:-4)]
%              (1.north);
%      \end{tikzpicture}
%    \end{minipage}
%  \end{center}
%  \caption{CFG Construction}
%  \label{fig:cfg-construction}
%\end{figure}
%
%For the purposes of this survey, we'll assume the compiler is transforming a
%simple three-address language, so-called because it's generally composed of
%instructions of the form \lstinline|x := y $op$ z|.  The semantics aren't
%terribly important, but the code should be clear to anyone familiar with basic
%assembly, as the language is modeled after common RISC machine code.  CFGs are
%arrangements of instructions into \defn{basic blocks}: maximal sequences of
%``straight-line'' code, where control does not transfer out of or into the
%middle of the block (e.g., by a \lstinline|goto|).  Then, directed edges are
%added between blocks to represent control flow---either from a \lstinline|goto|
%to its target, or from the end of a basic block to the start of the next one
%\cite{DragonBook}.  See Figure~\vref{fig:cfg-construction}.
%
%Probably the most popular intermediate representation in modern compilers is a
%variation of the CFG called \defn{static single assignment} (or \defn{SSA})
%form, wherein every variable in the program is defined by exactly one
%statement.  This simplifies the properties of variables, which helps
%optimizations perform faster and with better results.  The optimizations in
%this thesis will operate on SSA form.
%
%\begin{figure}
%\begin{center}
%  \begin{minipage}{0.4\linewidth}
%    \begin{tikzpicture}[node distance=1.0in,>=latex]
%    \node[draw,label=right:$B_1$] (1) at (0,-1.25)
%      {\lstinline|if $P$ goto $B_3$|};
%    \node[draw,label=left:$B_2$] (2) [below left of=1]
%      {\lstinline|x := 5|};
%    \node[draw,label=right:$B_3$] (3) [below right of=1]
%      {\lstinline|x := 10|};
%    \node[draw,label=right:$B_4$] (4) [below right of=2]
%      {\lstinline|y := x + 1|};
%
%    \draw[->] (0,0) -- (1);
%    \draw (1) edge[->] (2)
%              edge[->] (3)
%          (2) edge[->] (4)
%          (3) edge[->] (4);
%    \end{tikzpicture}
%  \end{minipage}
%  \vrule
%  \begin{minipage}{0.4\linewidth}
%    \begin{tikzpicture}[node distance=1.0in,>=latex]
%    \node[draw,label=right:$B_1$] (1) at (0,-1.25)
%      {\lstinline|if $P$ goto $B_3$|};
%    \node[draw,label=left:$B_2$] (2) [below left of=1]
%      {\lstinline|x$_0$ := 5|};
%    \node[draw,label=right:$B_3$] (3) [below right of=1]
%      {\lstinline|x$_1$ := 10|};
%    \node[draw,label=right:$B_4$] (4) [below right of=2]
%      {\lstinline|y$_{~}$ := x$_?$ + 1|};
%
%    \draw[->] (0,0) -- (1);
%    \draw (1) edge[->] (2)
%              edge[->] (3)
%          (2) edge[->] (4)
%          (3) edge[->] (4);
%    \end{tikzpicture}
%  \end{minipage}
%\end{center}
%\caption{SSA Construction Ambiguity}
%\label{fig:ssa-construction}
%\end{figure}
%
%At a high level, SSA form is constructed from a non-SSA CFG by giving unique
%names to the targets of each assignment (thus guaranteeing the SSA property),
%and by replacing uses of the original assignment with this new name.  But
%control flow can produce ambiguity.  In Figure~\vref{fig:ssa-construction}, the
%CFG on the left is being transformed into SSA form.  The original has two
%definitions of \lstinline|x| (in $B_2$ and $B_3$), either of which may reach
%the use of \lstinline|x| in $B_4$.  In SSA form, we give unique names to these
%two \lstinline|x| assignments, so which ``version'' do we use in $B_4$:
%\lstinline|x$_0$| or \lstinline|x$_1$|?
%
%\begin{figure}
%\begin{center}
%\begin{tikzpicture}[node distance=1.0in,>=latex]
%\node[draw,label=right:$B_1$] (1) at (0,-1.25) {\lstinline|if $P$ goto $B_3$|};
%\node[draw,label=left:$B_2$] (2) [below left of=1] {\lstinline|x$_0$ := 5|};
%\node[draw,label=right:$B_3$] (3) [below right of=1] {\lstinline|x$_1$ := 10|};
%\node[draw,label=right:$B_4$] (4) [below right of=2] {
%\begin{lstlisting}
%x$_2$ := $\phi$(x$_0$,x$_1$)
%y$_{~}$ := x$_2$ + 1
%\end{lstlisting}
%};
%
%\draw[->] (0,0) -- (1);
%\draw (1) edge[->] (2)
%          edge[->] (3)
%      (2) edge[->] (4)
%      (3) edge[->] (4);
%\end{tikzpicture}
%\end{center}
%\caption{$\phi$ Insertion}
%\label{fig:ssa-phi}
%\end{figure}
%
%To remedy this problem, SSA introduces a ``phony function'', $\phi$.  It's
%applied to two versions of variables that may reach the point where it's
%inserted.  While $\phi$ doesn't perform any literal computation, conceptually
%it selects the ``correct'' operand, depending on the control flow.  In
%Figure~\vref{fig:ssa-phi}, we correct the CFG of
%Figure~\ref{fig:ssa-construction} by inserting a new variable that's assigned
%to the result of the $\phi$ function and replacing the use of \lstinline|x|
%with this new variable.  Thus, if control flow goes $B_1 \to B_2 \to B_4$,
%$\phi$ selects \lstinline|x$_0$|.  If control flow goes $B_1 \to B_3 \to B_4$,
%$\phi$ selects \lstinline|x$_1$|.
%
%In principle, we could place any number of $\phi$-functions anywhere in the
%program (e.g., trivial ones like \lstinline|y := $\phi($x,x$)$|).  In practice,
%we insert them in as few places as possible---only at the beginnings of blocks
%where we need to.  There are many algorithms for efficient SSA construction
%with proper $\phi$ insertion (e.g., see \citeasnoun{SSAConstruction}) which are
%beyond the scope of this thesis.  For the rest of this proposal, it's assumed
%that CFGs are in SSA form.

\newpage\section{Language Primer}\label{sec:primer}

\todo[inline]{citations for this history are fragmented across the internet}

Factor is a rather young language created by Slava Pestov in September of 2003.
Its first incarnation targeted the \gls{JVM} as an embedded scripting language
for a game.  As such, its feature set was minimal.  Factor has since evolved
into a general-purpose programming language, gaining new features and
redesigning old ones as necessary for larger programs.  Today's implementation
sports an extensive standard library and has moved away from the \gls{JVM} in
favor of native code generation.  In this \lcnamecref{sec:primer}, we cover the
basic syntax and semantics of Factor for those unfamiliar with the language.
This should be just enough to understand the later material in this thesis.
More thorough documentation can be found via Factor's website,
\url{http://factorcode.org}.

%\subsection{Stack-Based Languages}\label{sec:primer:stack-based}

\inputfig{rpn}

Like \gls{RPN} calculators, Factor's evaluation model uses a global stack upon
which operands are pushed before operators are called.  This naturally
facilitates postfix notation, in which operators are written after their
operands.  For example, instead of \texttt{1~+~2}, we write \texttt{1~2~+}.
\vref{fig:rpn} shows how \texttt{1~2~+} works conceptually:
\begin{itemize}
  \item \texttt{1} is pushed onto the stack
  \item \texttt{2} is pushed onto the stack
  \item \texttt{+} is called, so two values are popped from the stack, added,
        and the result (\texttt{3}) is pushed back onto the stack
\end{itemize}
Other stack-based programming languages include Forth\todo{cite},
Joy\todo{cite}, Cat\todo{cite}, and PostScript\todo{cite}.

The strength of this model is its simplicity.  Evaluation essentially goes
left-to-right: literals (like \texttt{1} and \texttt{2}) are pushed onto the
stack, and operators (like \texttt{+}) perform some computation using values
currently on the stack.  This ``flatness'' makes parsing easier, since we don't
need complex grammars with subtle ambiguities and precedence issues.  Rather,
we basically just scan left-to-right for tokens separated by whitespace.  In
the Forth tradition, functions are called \term{words} since they're made up of
any contiguous non-whitespace characters.  This also lends to the term
\term{vocabulary} instead of ``module'' or ``library''.   In Factor, the parser
works as follows.
\begin{itemize}
  \item If the current character is a double-quote (\texttt{"}), try to
        parse ahead for a string literal.
  \item Otherwise, scan ahead for a single token.
        \begin{itemize}
          \item If the token is the name of a \term{parsing word}, that word is
                invoked with the parser's current state.
          \item If the token is the name of an ordinary (i.e., non-parsing)
                word, that word is added to the parse tree.
          \item Otherwise, try to parse the token as a numeric literal.
        \end{itemize}
\end{itemize}

Parsing words serve as hooks into the parser, letting Factor users extend the
syntax dynamically.  For instance, instead of having special knowledge of
comments built into the parser, the parsing word \texttt{!}~scans forward for a
newline and discards any characters read (adding nothing to the parse tree).

Similarly, there are parsing words for what might otherwise be hard-coded
syntax for data structure literals.  Many act as sided delimeters: the parsing
word for the left-delimiter will parse ahead until it reaches the
right-delimiter, using whatever was read in between to add objects to the data
structure.  For example, \factor|{ 1 2 3 }| denotes an array of three numbers.
Note the deliberate spaces in between the tokens, so that the delimeters are
themselves distinct words.  In
%
\Verb[showspaces]|{ 1 2 3 }| (with spaces as marked), the parsing word \Verb|{|
parses objects until it reaches \Verb|}|, collecting the results into an array.
The \verb|{| word would not be called if not for that space, whereas
%
\Verb[showspaces]|{1 2 3}| parses as the word \Verb|{1|, the number \Verb|2|,
and the word \Verb|3}|---not an array.  Further, since the left-delimeter words
parse recursively, such literals can be nested, contain comments, etc.  Other
literals include \vpageref[the following][those in
\vref{lst:literals}~]{lst:literals}.

\inputlst[h]{literals}

\inputfig{quot}

A particularly important set of parsing words in Factor are the square
brackets, \Verb|[| and \Verb|]|.  Any code in between such brackets is
collected up into a special sequence called a \term{quotation}.  Essentially,
it's a snippet of code whose execution is suppressed.  The code inside a
quotation can then be run with the \factor|call| word.  Quotations are like
anonymous functions in other languages, but the stack model makes them
conceptually simpler, since we don't have to worry about variable binding and
the like.  Consider a small example like \factor|1 2 [ + ] call|.  You can
think of \factor|call| working by ``erasing'' the brackets around a quotation,
so this example behaves just like \factor|1 2 +|.  \vref{fig:quot} shows its
evaluation: instead of adding the numbers immediately, \factor|+| is placed in
a quotation, which is pushed to the stack.  The quotation is then invoked by
\factor|call|, so \factor|+| pops and adds the two numbers and pushes the
result onto the stack.  We'll show how quotations are used in
\cref{sec:primer:combinators}.

%\section{Stack Effects}\label{sec:primer:effects}

Everything else about Factor follows from the stack-based structure outlined in
\cref{sec:primer:stack-based}.  Consecutive words transform the stack in
discrete steps, thereby shaping a result.  In a way, words are functions from
stacks to stacks---from ``before'' to ``after''---and whitespace is effectively
function composition.  Even literals (numbers, strings, arrays, quotations,
etc.) can be thought of as functions that take in a stack and return that stack
with an extra element pushed onto it.

With this in mind, Factor requires that the number of elements on the stack
(the \term{stack height}) is known at each point of the program in order to
ensure consistency.  To this end, every word is associated with a \term{stack
effect} declaration using a notation implemented by parsing words.  In general,
a stack effect declaration has the form
%
\begin{center} \Verb|( input1 input2 ... -- output1 output2 ... )| \end{center}
%
\noindent where the parsing word \Verb|(| scans forward for the special token
\Verb|--| to separate the two sides of the declaration, and then for the
\Verb|)| token to end the declaration.  The names of the intermediate tokens
don't technically matter---only how many of them there are.  However, names
should be meaningful for clarity's sake.  The number of tokens on the left side
of the declaration (before the \Verb|--|) indicates the minimum stack height
expected before executing the word.  Given exactly this number of inputs, the
number of tokens on the right side is the stack height after executing the
word.

For instance, the stack effect of the \factor|+| word is
%
\Verb|( x y -- z )|,
%
as it pops two numbers off the stack and pushes one number (their sum) onto the
stack.  This could be written any number of ways, though.
%
\Verb|( x x -- x )|,
%
\Verb|( number1 number2 -- sum )|,
%
and
%
\Verb|( m n -- m+n )|
%
are all equally valid.  Further, while the stack effect
%
\Verb|( junk x y -- junk z )|
%
has the same relative height change, this declaration would be wrong, since it
requires at least three inputs but \factor|+| might legitimately be called on
only two.

\inputfig{shufflers}

For the purposes of documentation, of course, the names in stack effects do
matter.  They correspond to elements of the stack from bottom-to-top.  So, the
rightmost value on either side of the declaration names the top element of the
stack.  We can see this in \vref{fig:shufflers}, which shows the effects of
standard \term{stack shuffler} words.  These words are used for basic data flow
in Factor programs.  For example, to discard the top element of the stack, we
use the \factor|drop| word, whose effect is simply
%
\Verb|( x -- )|.
%
To discard the element just below the top of the stack, we use \factor|nip|,
whose effect is
%
\Verb|( x y -- y )|.
%
This stack effect indicates that there are at least two elements on the stack
before \factor|nip| is called: the top element is \Verb|y|, and the next
element is \Verb|x|.  After calling the word, \Verb|x| is removed, leaving
the original \Verb|y| still on top of the stack.  Other shuffler words that
remove data from the stack are
%
\factor|2drop| with the effect \Verb|( x y -- )|,
%
\factor|3drop| with the effect \Verb|( x y z -- )|, and
%
\factor|2nip| with the effect \Verb|( x y z -- z )|.

The next stack shufflers duplicate data.  \factor|dup| copies the top element
of the stack, as indicated by its effect \Verb|( x -- x x )|.  \factor|over|
has the effect \Verb|( x y -- x y x )|, which tells us that it expects at
least two inputs: the top of the stack is \Verb|y|, and the next object is
\Verb|x|.  \Verb|x| is copied and pushed on top of the two original
elements, sandwiching \Verb|y| between two \Verb|x|s.  Other shuffler words
that duplicate data on the stack are
%
\factor|2dup| with the effect \Verb|( x y -- x y x y )|,
%
\factor|3dup| with the effect \Verb|( x y z -- x y z x y z )|,
%
\factor|2over| with the effect \Verb|( x y z -- x y z x y )|, and
%
\factor|pick| with the effect \Verb|( x y z -- x y z x )|.

True to the name \factor|swap|, the final shuffler in \vref{fig:shufflers}
permutes the top two elements of the stack, reversing their order.  The stack
effect
%
\Verb|( x y -- y x )|
%
indicates as much.  The left side denotes that two inputs are on the stack (the
top is \Verb|y|, the next is \Verb|x|), and the right side shows the
outputs are swapped (the top element is \Verb|x| and the next is \Verb|y|).
Factor has other words that permute elements deeper into the stack.  However,
their use is discouraged because it's harder for the programmer to mentally
keep track of more than a couple items on the stack.  We'll see how more
complex data flow patterns are handled in \vref{sec:primer:combinators}.

%\section{Definitions}\label{sec:primer:colon-defs}

\inputlst{hello-world}

\inputlst{norm}

Using the basic syntax of stack effect declarations described in
\cref{sec:primer:effects}, we can now understand how to define words.  Most
words are defined with the parsing word \factor|:|, which scans for a name, a
stack effect, and then any words up until the \factor|;| token, which together
become the body of the definition.  Thus, the classic example in
\vref{lst:hello-world} defines a word named \Verb|hello-world| which expects
no inputs and pushes no outputs onto the stack.  When called, this word will
display the canonical greeting on standard output using the \factor|print|
word.

\inputfig{norm-steps}

A slightly more interesting example is the \Verb|norm| word in
\vref{lst:norm}.  This squares each of the top two numbers on the stack, adds
them, then takes the square root of the sum.  \vref{fig:norm-steps} shows this
in action.  By defining a word to perform these steps, we can replace virtually
any instance of
%
\factor|dup * swap dup * + sqrt|
%
in a program simply with \Verb|norm|.  This is a deceptively important point.
Data flow is made explicit via stack manipulation rather than being hidden in
variable assignments, so repetitive patterns become painfully evident.  This
makes identifying, extracting, and replacing redundant code easy.  Often, you
can just copy a repetitive sequence of words into its own definition verbatim.
This emphasis on ``factoring'' your code is what gives Factor its name.

\inputlst{norm-factored}

As a simple case in point, we see the subexpression \factor|dup *| appears
twice in the definition of \Verb|norm| in \vref{lst:norm}.  We can easily
factor that out into a new word and substitute it for the old expressions, as
in \vref{lst:norm-factored}.  By contrast, programs in more traditional
languages are laden with variables and syntactic noise that require more work
to refactor: identifying free variables, pulling out the right functions
without causing finicky syntax errors, calling a new function with the right
variables, etc.  Though Factor's stack-based paradigm is atypical, it is part
of a design philosophy that aims to facilitate readable code focusing on short,
reusable definitions.

%\section{Object Orientation}\label{sec:primer:oo}

You may have noticed that the examples in \cref{sec:primer:colon-defs} did not
use type declarations.  While Factor is dynamically typed for the sake of
simplicity, it does not do away with types altogether.  In fact, Factor is
object-oriented.  However, its object system doesn't rely on classes possessing
particular methods, as is common.  Instead, it uses \term{generic words} with
methods implemented for particular classes.  To start, though, we must see how
classes are defined.

\subsection{Tuples}

The central data type of Factor's object system is called a \term{tuple}, which
is a class composed of named \term{slots}---like instance variables in other
languages.  Tuples are defined with the \texttt{\textbf{TUPLE:}} parsing word
as shown in \vref{lst:tuples}.  A class name is specified first; if it is
followed by the \factor|<| token and a superclass name, the tuple inherits the
slots of the superclass.  If no superclass is specified, the default is the
\factor|tuple| class.  Any number of slot specifiers follow, and the definition
is terminated by the \factor|;| token.

\inputlst{tuples}

Tuple definitions automatically generate several different words, most of which
depend on how slots are specified.  There are various ways to specify slots,
but we use only two basic forms in later code examples.  We can see both in the
first tuple of \vref{lst:regexp}, which defines an object to represent regular
expressions.  The first three slots have the form
%
\factor|{ name read-only }|,
%
which specifies a slot named \Verb|name| that can't be modified once
initialized, akin to a \mint{java}|final| variable in Java.  The next two
specifiers are simpler, being just the names of the slots.  Such slots can be
modified freely.  The following words are automatically defined for the first
tuple:
\begin{itemize}
  \item The \Verb|regexp| \term{class word} acts like a literal representing
  the class.  This gets used for instantiation and method definitions, which
  we'll see later.
%
  \item The \Verb|regexp?| \term{class predicate} is a word with the stack
  effect \Verb|( object -- ? )|.  That is, it returns a boolean (either
  \factor|t| or \factor|f|, conventionally written in stack effects as a single
  question mark) indicating whether the top of the stack is an instance of the
  \Verb|regexp| class.  This is like a class-specific variant of Java's
  \mint{java}|instanceof|.
%
  \item Each slot has an associated \term{reader} word with the stack effect %
  \Verb|( object -- value )|.  These are analogous  to ``getter'' methods in
  other languages.  Each one is named after the slot whose value is extracted,
  so this example defines \Verb|raw>>|, \Verb|parse-tree>>|, \Verb|options>>|,
  \Verb|dfa>>|, and \Verb|next-match>>|.
%
  \item Similarly, any slot that is not marked \Verb|read-only| has a
  corresponding \term{writer} word with the stack effect                      %
  \Verb|( value object -- )|.  These destructively write the value into the
  eponymous slot of the object.  Here, only two are defined, named \Verb|dfa<<|
  and \Verb|next-match<<|.
%
  \item Extra \term{setter} words are defined in terms of writers.  These will
  have the stack effect \Verb|( object value -- object' )|, leaving the
  modified instance on top of the stack.  The first tuple in \vref{lst:regexp}
  defines \Verb|>>dfa| and \Verb|>>next-match|, which are equivalent to
  \factor|over dfa<<| and \factor|over next-match<<|, respectively.  The
  shuffler duplicates \Verb|object| and pushes it to the top of the stack.
  More accurately, it duplicates a reference to \Verb|object|, as Factor's data
  stack is actually a stack of pointers.  That way, changes to the new top of
  the stack with \Verb|dfa<<| or \Verb|next-match<<| will be reflected in the
  original \Verb|object|, which is left over at the end.
%
  \item \term{Changer} words are also created with the stack effect           %
  \Verb|( object quot -- object' )|.  Here, \Verb|change-dfa| and
  \Verb|change-next-match| are defined.  The quotation is called on the slot's
  current value in \Verb|object|.  The result of calling the quotation is then
  stored in the slot.  For instance, incrementing an integer slot named
  \Verb|foo| could be done with \factor|[ 1 + ] change-foo|.
\end{itemize}

\inputlst{regexp}

The second tuple in \vref{lst:regexp} also defines a class word and predicate.
Since it inherits from \Verb|regexp|, \Verb|reverse-regexp| gets the same five
slots.  If we had any other slot specifiers in the definition, it would have
those in addition to the slots of its parent class.  The reader, writer,
setter, and changer methods will work on instances of \Verb|reverse-regexp|,
since inheritance establishes an ``is-a'' relationship from subclass to
superclass---any instance of \Verb|reverse-regexp| is also an instance of
\Verb|regexp|, though the reverse is not necessarily true.  That is,
\Verb|regexp?| will return \factor|t| on instances of \Verb|reverse-regexp|,
but \Verb|reverse-regexp?| will only return \factor|t| on instances of
\Verb|regexp| that are also \Verb|reverse-regexp|s.  By viewing a class as the
set of all objects that respond positively to the class predicate, we may
partially order classes with the subset relationship.  This fact will be
important later.

To construct an instance of a tuple, we can use either \factor|new| or
\factor|boa|.  \factor|new| will not initialize any of the slots to a
particular input value---all slots will default to Factor's canonical false
value, \factor|f|.  For example, \factor|new| is used in \vref{lst:colors} to
define \Verb|<color>| (by convention, the constructor for \Verb|foo| is named
\Verb|<foo>|).  First, we push the class \Verb|color|, then just call
\factor|new|, leaving a new instance on the stack.  Since this particular tuple
has no slots, using \factor|new| makes sense.  We might also use it to
initialize a class, then use setter words to only assign a particular subset of
slots' values (as long as the slots aren't \Verb|read-only|).

\inputlst{colors}

However, we often want to initialize a tuple with values for each of its slots.
For this, we have \factor|boa|, which works similarly to \factor|new|.  This is
used in the definition of \Verb|<rgb>| in \vref{lst:colors}.  The difference
here is the additional inputs on the stack---one for each slot, in the order
they're declared.  That is, we're constructing the tuple \textbf{b}y
\textbf{o}rder of \textbf{a}rguments, giving us the fun pun ``\factor|boa|
constructor''.  So, \Verb|1 2 3 <rgb>| will construct an \Verb|rgb|
instance with the \Verb|red| slot set to \Verb|1|, the \Verb|green| slot
set to \Verb|2|, and the \Verb|blue| slot set to \Verb|3|.

\subsection{Generics and Methods}

Unlike more common object systems, we do not define individual methods that
``belong'' to particular tuples.  In Factor, for a given generic word you
define a method that specializes on a class.  When the generic word is called
on an object, it selects the method most specific to the object's class.  This
is determined by the aforementioned partial ordering of classes by their
inheritance relationships.

Generic words are declared with the syntax
%
\begin{center} \factor|GENERIC: word-name ( stack -- effect )| \end{center}
%
Words defined this way will then dispatch on the class of the top element of
the stack (necessarily the rightmost input in the stack effect).  To define a
new method with which to control this dispatch, we use the syntax
%
\begin{center} \factor|M: class word-name definition... ;| \end{center}

Factor's \Verb|sets| vocabulary gives us an accessible example of a generic
word.  \Verb|set| is a \term{mixin} class, defined by the \factor|MIXIN:|
parsing word.  That is, the \Verb|set| class is a union of other classes, and
users may extend the members of this union with the \factor|INSTANCE:| word.
We can this in \vref{lst:sets}, which shows the standard members of the
\Verb|set| mixin.  Note that the \factor|USING:| form specifies vocabularies
being used (like Java's \mint{java}|import|) and \texttt{\textbf{IN:}}
specifies the vocabulary in which the definitions appear (like Java's
\mint{java}|package|).  We can see here that instances of the \Verb|sequence|,
\Verb|hash-set|, and \Verb|bit-set| classes are all instances of \Verb|set|, so
will respond \factor|t| to the predicate \Verb|set?|.  Similarly,
\Verb|sequence| is a mixin class with many more members, including
\Verb|array|, \Verb|vector|, and \Verb|string|.

\inputlst{sets}

\Vref{lst:cardinality} shows the \Verb|cardinality| generic from Factor's
\Verb|sets| vocabulary, along with its methods.  This generic word takes a
\Verb|set| instance from the top of the stack and pushes the number of elements
it contains.  For instance, if the top element is a \Verb|bit-set|, we extract
its \Verb|table| slot and invoke another word, \Verb|bit-count|, on that.  But
if the top element is \factor|f| (the canonical false/empty value), we know the
cardinality is $0$.  For any \Verb|sequence|, we may offshore the work to a
different generic, \factor|length|, defined in the \Verb|sequences| vocabulary.
The final method gives a default behavior for any other \Verb|set| instance,
which simply uses \Verb|members| to obtain an equivalent \Verb|sequence| of set
members, then calls \factor|length|.

\inputlst{cardinality}

We can see how the class ordering is used when \Verb|cardinality| selects the
proper method for the object being dispatched upon.  For instance, while no
explicit method for \Verb|array| is defined, any instance of \Verb|array| is
also an instance of \Verb|sequence|.  In turn, every instance of
\Verb|sequence| is also an instance of \Verb|set|.  We have methods that
dispatch on both \Verb|set| and \Verb|sequence|, but the latter is more
specific, so that is the method invoked on an \Verb|array|.  If we define our
own class, \Verb|foo|, and declare it as an instance of \Verb|set| but not as
an instance of \Verb|sequence|, then the \Verb|set| method of
\Verb|cardinality| will be invoked.  Sometimes resolving the precedence gets
more complicated, but these edge-cases are beyond the scope of our discussion.

\subsection{Combinators}\label{sec:primer:combinators}

Quotations, introduced in \cref{sec:primer:stack-based}, form the basis of both
control flow and data flow in Factor.  Not only are they the equivalent of
anonymous functions, but the stack model also makes them syntactically
lightweight enough to serve as blocks akin to the code between curly braces in
C or Java.  Higher-order words that make use of quotations on the stack are
called \term{combinators}.  It's simple to express familiar conditional logic
and loops using combinators, as we'll show in \cref{sec:primer:control-flow}.
In the presence of explicit data flow via stack operations, even more patterns
arise that can be abstracted away.  \cref{sec:primer:data-flow} explores how we
can use combinators to express otherwise convoluted stack-shuffling logic more
succinctly.

\subsubsection{Control Flow}\label{sec:primer:control-flow}

\inputlst{if}

The most primitive form of control flow in typical programming languages is, of
course, the \mint{java}|if| statement, and the same holds true for Factor.  The
only difference is that Factor's \factor|if| isn't syntactically
significant---it's just another word, albeit implemented as a primitive.  For
the moment, it will do to think of \factor|if| as having the stack effect
%
\Verb|( ? true false -- )|.
%
The third element from the top of the stack is a condition, and it's followed
by two quotations.  The first quotation (second element from the top of the
stack) is called if the condition is true, and the second quotation (the top of
the stack) is called if the condition is false.  Specifically, \factor|f| is a
special object in Factor for falsity.  It is a singleton object---the sole
instance of the \factor|f| class---and is the only false value in the entire
language.  Any other object is necessarily boolean true.  For a canonical
boolean, there is the \factor|t| object, but its truth value exists only
because it is not \factor|f|.  Basic \factor|if| use is shown in
\cref{lst:if}\todo{vref}.  The first example will print ``odd'', the second
``empty'', and the third ``isn't f''.  All of them leave nothing on the stack.

\inputlst{if-effects}

However, the simplified stack effect for \factor|if| is quite restrictive.
%
\Verb|( ? true false -- )| \todo{extra space after {\tt ?} with minted}
%
intuitively means that both the \factor|true| and \factor|false| quotations
can't take any inputs or produce any outputs---that their effects are
%
\factor|( -- )|.
%
We'd like to loosen this restriction, but per \cref{sec:primer:effects}, Factor
must know the stack height after the \factor|if| call.  We could give
\factor|if| the effect
%
\Verb|( x ? true false -- y )|,
%
so that the two quotations could each have the stack effect
%
\factor|( x -- y )|.
%
This would work for the \factor|example1| word in \vref{lst:if-effects}, yet
it's just as restrictive.  For instance, the \factor|example2| word would need
\factor|if| to have the effect
%
\Verb|( x y ? true false -- z )|,
%
since each branch has the effect
%
\factor|( x y -- z )|.
%
Furthermore, the quotations might even have different effects, but still leave
the overall stack height balanced.  Only one item is left on the stack after a
call to \factor|example3| regardless, even though the two quotations have
different stack effects: \factor|+| has the effect
%
\factor|( x y -- z )|,
%
while \factor|drop| has the effect
%
\factor|( x -- )|.

In reality, there are infinitely many correct stack effects for \factor|if|.
Factor has a special notation for such \term{row-polymorphic} stack effects.
If a token in a stack effect begins with two dots, like \factor|..a| or
\factor|..b|, it is a \term{row variable}.  If either side of a stack effect
begins with a row variable, it represents any number inputs/outputs.  Thus, we
could give \factor|if| the stack effect
%
\begin{center} \Verb|( ..a ? true false -- ..b )| \end{center}
%
\noindent to indicate that there may be any number of inputs below the
condition on the stack, and any number of outputs will be present after the
call to \factor|if|.  Note that these numbers aren't necessarily equal, which
is why we use distinct row variables in this case.  However, this still isn't
quite enough to capture the stack height requirements.  It doesn't communicate
that \factor|true| and \factor|false| must affect the stack in the same ways.
For this, we can use the notation
%
\factor|quot: ( stack -- effect )|,
%
giving quotations a nested stack effect.  Using the same names for row
variables in both the ``inner'' and ``outer'' stack effects will refer to the
same number of inputs or outputs.  Thus, our final (correct) stack effect for
\factor|if| is 
%
\begin{center}
%
  \Verb|( ..a ? true: ( ..a -- ..b ) false: ( ..a -- ..b ) -- ..b )|
%
\end{center}
%
\noindent This tells us that the \factor|true| quotation and the \factor|false|
quotation will each create the same relative change in stack height as
\factor|if| does overall.

\inputlst{loops}

Though \factor|if| is necessarily a language primitive, other control flow
constructs are defined in Factor itself.  It's simple to write combinators for
iteration and looping as tail-recursive words that invoke quotations.
\vref{lst:loops} showcases some common looping patterns.  The most basic yet
versatile word is \factor|each|.  Its stack effect is
%
\begin{center}
%
  \factor|( ... seq quot: ( ... x -- ... ) -- ... )|
%
\end{center}
%
\noindent Each element \factor|x| of the sequence \factor|seq| will be passed
to \factor|quot|, which may use any of the underlying stack elements.  Here,
unlike \factor|if|, we enforce that the input stack height is exactly the same
as the output (since we use the same row variable).  Otherwise, depending on
the number of elements in \factor|seq|, we might dig arbitrarily deep into the
stack or flood it with a varying number of values.  The first use of
\factor|each| in \vref{lst:loops} is balanced, as the quotation has the effect
%
\factor|( str -- )|
%
and no additional items were on the stack to begin with.  Essentially, it's
equivalent to
%
\factor|"Lorem" print "ipsum" print "dolor" print|.
%
On the other hand, the quotation in the second example has the stack effect
%
\factor|( total n -- total+n )|.
%
This is still balanced, since there is one additional item below the sequence
on the stack (namely \factor|0|), and one element is left by the end (the sum
of the sequence elements).  So, this example is the same as
%
\factor|0 1 + 2 + 3 +|.

Any instance of the extensive \factor|sequence| mixin will work with
\factor|each|, making it very flexible.  The third example in \vref{lst:loops}
shows \factor|iota|, which is used here to create a \term{virtual} sequence of
integers from $0$ to $9$ (inclusive).  No actual sequence is allocated, merely
an object that behaves like a sequence.  In Factor, it's common practice to use
\factor|iota| and \factor|each| in favor of repetitive C-like \mint{c}|for|
loops.

Of course, we sometimes don't need the induction variable in loops.  That is,
we just want to execute a body of code a certain number of times.  For these
cases, there's the \factor|times| combinator, with the stack effect
%
\begin{center}
%
  \factor|( ... n quot: ( ... -- ... ) -- ... )|
%
\end{center}
%
\noindent This is similar to \factor|each|, except that \factor|n| is a number
(so we needn't use \factor|iota|) and the quotation doesn't expect an extra
argument (i.e., a sequence element).  Therefore, the example in
\vref{lst:loops} is equivalent to
%
\factor|"Ho!" print "Ho!" print "Ho!" print|.
%

Naturally, Factor also has the \factor|while| combinator, whose stack effect is
%
\begin{center}
%
  \Verb|( ..a pred: ( ..a -- ..b ? ) body: ( ..b -- ..a ) -- ..b )|
%
\end{center}
%
\noindent The row variables are a bit messy, but it works as you'd expected:
the \factor|pred| quotation is invoked on each iteration to determine whether
\factor|body| should be called.  The \factor|do| word is a handy modifier for
\factor|while| that simply executes the body of the loop once before leaving
\factor|while| to test the precondition as per usual.  Thus, the last example
in \vref{lst:loops} executes the body once, despite the condition being
immediately false.

\inputlst{high-order}

In the preceding combinators, quotations were used like blocks of code.  But
really, they're the same as anonymous functions from other languages.  As such,
Factor borrows classic tools from functional languages, like \factor|map| and
\factor|filter|, as shown in \vref{lst:high-order}.  \factor|map| is like
\factor|each|, except that the quotation should produce a single output.  Each
such output is collected up into a new sequence of the same class as the input
sequence.  Here, the example produces
%
\factor|{ 2 3 4 }|.
%
\factor|filter| selects only those elements from the sequence for which the
quotation returns a true value.  Thus, the \factor|filter| in
\vref{lst:high-order} outputs
%
\factor|{ 2 4 }|.
%
Even \factor|reduce| is in Factor, also known as a \term{left fold}.  An
initial element is iteratively updated by pushing a value from the sequence and
invoking the quotation.  In fact, \factor|reduce| is defined as
%
\factor|swapd each|,
%
where \factor|swapd| is a shuffler word with the stack effect
%
\factor|( x y z -- y x z )|.
%
Thus, the example in \vref{lst:high-order} is the same as
%
\factor|0 { 1 2 3 } [ + ] each|,
%
as in \vref{lst:loops}.

These are just some of the control flow combinators defined in Factor.  Several
variants exist that meld stack shuffling with control flow, or can be used to
shorten common patterns like empty false branches.  An entire list is beyond
the scope of our discussion, but the ones we've studied should give a solid
view of what standard conditional execution, iteration, and looping looks like
in a stack-based language.

\subsubsection{Data Flow}\label{sec:primer:data-flow}

While avoiding variables and additional syntax makes it easier to refactor
code, keeping mental track of the stack can be taxing.  If we need to
manipulate more than the top few elements of the stack, code gets harder to
read and write.  Since the flow of data is made explicit via stack shufflers,
we actually wind up with redundant patterns of data flow that we otherwise
couldn't identify.  In Factor, there are several combinators that clean up
common stack-shuffling logic, making code easier to understand.

\inputlst{preserve}

The first combinators we'll look at are \factor|dip| and \factor|keep|.  These
are used to preserve elements of the stack.  When working with several values,
sometimes we don't want to use all of them at quite the same time.  Using
\factor|drop| and the like wouldn't help, as we'd lose the data altogether.
Rather, we want to retain certain stack elements, do a computation, then
restore them.  For an uncompelling but illustrative example, suppose we have
two values on the stack, but we want to increment the second element from the
top.  \factor|without-dip1| in \vref{lst:preserve} shows one strategy, where we
shuffle the top element away with \factor|swap|, perform the computation, then
\factor|swap| the top back to its original place.  A cleaner way is to call
\factor|dip| on a quotation, which will execute that quotation just under the
top of the stack, as in \factor|with-dip1|.  While the stack shuffling in
\factor|without-dip1| isn't terribly complicated, it doesn't convey our meaning
very well.  Shuffling the top element out of the way becomes increasingly
difficult with more complex computations.  In \factor|without-dip2|, we want to
call \factor|-| on the two elements below the top.  For lack of a more robust
stack shuffler, we use \factor|2over| to isolate the two values so we can call
\factor|-|.  The rest of the word consists of shuffling to get rid of excess
values on the stack.  It's also worth noting that \factor|swapd| is a
deprecated word in Factor, since its use starts making code harder to reason
about.  Alternatively, we could dream up a more complex stack shuffler with
exactly the stack effect we wanted in this situation.  But this solution
doesn't scale: what if we had to calculate something that required more inputs
or produced more outputs?  Clearly, \factor|dip| provides a cleaner alternative
in \factor|with-dip2|.

\factor|keep| provides a way to hold onto the top element of the stack, but
still use it to perform a computation.  In general,
%
\factor|[ ... ] keep|
%
is equivalent to
%
\factor|dup [ ... ] dip|.
%
Thus, the current top of the stack remains on top after the use of
\factor|keep|, but the quotation is still invoked with that value.  In
\factor|with-keep1| in \vref{lst:preserve}, we want to increment the top, but
stash the result below.  Again, this logic isn't terribly complicated, though
\factor|with-keep1| does away with the shuffling.  \factor|without-keep2| shows
a messier example where a simple \factor|dup| will not save us, as we're using
more than just the top element in the call to \factor|-|.  Rather, three of the
four words in the definition are dedicated to rearranging the stack in just the
right way, obscuring the call to \factor|-| that we really want to focus on.
On the other hand, \factor|with-keep2| places the subtraction word
front-and-center in its own quotation, while \factor|keep| does the work of
retaining the top of the stack.


%\begin{itemize}
%
%\item Stack effects
%      \begin{itemize}
%        \item Complex stack effects: row polymorphism \& types
%        \item Stack checker
%      \end{itemize}
%
%\item Combinators
%      \begin{itemize}
%        \item Control flow
%              \begin{itemize}
%                \item if
%                \item each/map
%                \item while
%                \item times
%              \end{itemize}
%        \item Data flow
%              \begin{itemize}
%                \item Dip/keep
%                \item Cleave
%                \item Spread
%                \item Apply
%              \end{itemize}
%      \end{itemize}
%
%\item Libraries \& metaprogramming
%      \begin{itemize}
%        \item Results of evolution
%        \item locals?
%        \item fry?
%        \item macros?
%        \item functors?
%        \item ffi?
%      \end{itemize}
%
%\end{itemize}

\newpage\section{The Factor Compiler}\label{sec:compiler}

If we could sort programming languages by the fuzzy notions we tend to have
about how ``high-level'' they are, toward the high end we'd find
dynamically-typed languages like Python, Ruby, and PHP---all of which are
generally more interpreted than compiled\todo{Though there are projects for
this}.  Despite being as high-level as these popular languages, Factor's
implementation is driven by performance.  Factor source is always compiled to
native machine code using either its simple, non-optimizing compiler or (more
typically) the optimizing compiler that performs several sorts of data and
control flow analyses.  In this \lcnamecref{sec:compiler}, we look at the
general architecture of Factor's implementation, after which we place a
particular emphasis on the transformations performed by the optimizing
compiler.

%\section{Organization}\label{sec:compiler:vm}

At the lowest level, Factor is written atop a C++ \gls{VM} that is responsible
for basic runtime services.  This is where the non-optimizing base compiler is
implemented.  It's the base compiler's job to compile the simplest primitives:
operations that push literals onto the data stack, \factor|call|, \factor|if|,
\factor|dip|, words that access tuple slots as laid out in memory, stack
shufflers, math operators, functions to allocate/deallocate call stack frames,
etc.  The aim of the base compiler is to generate native machine code as fast
as possible.  To this end, these primitives correspond to their own stubs of
assembly code.  Different stubs are generated by Factor depending on the
instruction set supported by the particular machine in use.  Thus, the base
compiler need only make a single pass over the source code, emitting these
assembly instructions as it goes.

This compiled code is saved in an \term{image file}, which contains a complete
snapshot of the current state of the Factor instance, similar to many Smalltalk
and Lisp systems\todo{cite?}.  As code is parsed and compiled, the image is
updated, serving as a cache for compiled code.  This modified image can be
saved so that future Factor instances needn't recompile vocabularies that are
already contained in the image.

The \gls{VM} also handles method dispatch and memory management.  Method
dispatch incorporates a \term{polymorphic inline cache} to speed up generic
words.  Each generic word's call site is associated with a state:
\begin{itemize}
  \item In the \term{cold} state, the call site's instruction computes the
        right method for the class being dispatched upon, which is the
        operation we're trying to avoid.  As it does this, a polymorphic inline
        cache stub is generated, thus transitioning it to the next state.
  \item In the \term{inline cache} state, a stub has been generated that caches
        the locations of methods for classes that have already been seen.  This
        way, if a generic word at a particular call site is invoked often upon
        only a small number of classes (as is often in the case in loops, for
        example), we don't need to waste as much time doing method lookup.  By
        default, if more than three different classes are dispatched upon, we
        transition to the next state.
  \item In the \term{megamorphic} state, the call instruction points to a
        larger cache that is allocated for the specific generic word (i.e., it
        is shared by all call sites).  While not as efficient as an inline
        cache, this can still improve the performance of method dispatch.
\end{itemize}

To manage memory, the Factor \gls{VM} uses a generational \gls{GC}, which
carves out sections of space on the heap for objects of different ages.
Garbage in the oldest generation is collected with a mark-sweep-compact
algorithm, while younger generations rely on a copying collector\todo{cite?}.
This way, the \gls{GC} is specialized for large numbers of short-lived objects
that will stay in the younger generations without being promoted to the older
generation.  In the oldest space, even compiled code can be compacted.  This is
to avoid heap fragmentation in applications that must call the compiler at
runtime, such as Factor's interactive development environment.

Values are referenced by tagged pointers, which use the three least significant
bits of the pointer's address to store type information.  This is possible
because Factor aligns objects on an eight-byte boundary, so the three least
significant bits of an address are always $0$.  These bits give us eight unique
tags, but since Factor has more than eight data types, two tags are reserved to
indicate that the type information is stored elsewhere.  One is for \gls{VM}
types without their own tag, and the other is for user-defined tuples, each of
which has its own type.  Sufficiently small integers (e.g., $29$-bit integers
on a $32$-bit machine, since the other $3$ bits are used for the type tag) are
stored directly in the pointer, so they needn't be heap-allocated.  Larger
integers and floating point numbers are boxed, but the optimizing compiler may
unbox them to store floats in registers.

The \gls{VM} is meant to be minimal, as Factor is mostly \term{self-hosting}.
That is, the real workhorses of the language are written in Factor itself,
including the standard libraries, parser, object system, and the optimizing
compiler.  It's possible for the compiler to be written in Factor because of
the \term{bootstrapping} process that creates a new image from scratch.  First,
a minimal \term{boot image} is created from an existing \term{host} Factor
instance.  When the \gls{VM} runs the boot image, it initiates the
bootstrapping process.  Using the host's parser, the base compiler will compile
the core vocabularies necessary to load the optimizing compiler.  Once the
optimizing compiler can itself be compiled, it is used to recompile (and thus
optimize) all of the words defined so far.  With the basic vocabularies
recompiled, any additional vocabularies can be loaded using the optimized
compiler and saved into a new, working image.

Thus, while the Factor \gls{VM} is important, it is a small part of the code
base.  Since the bootstrapping process allows the optimizing compiler
(hereafter just ``the compiler'') to be written in the same high-level language
it's compiling, we can avoid the fiddly low-level details of the C++ backend.
This is more conducive to writing advanced compiler optimizations, which are
often complicated enough without having a concise, dynamically-typed,
garbage-collected language like Factor to help us.

\subsection{High-level Optimizations}\label{sec:compiler:tree}

To manipulate source code abstractly, we must have at least one \gls{IR}---a
data structure representing the instructions.  It's common to convert between
several \glsplural{IR} during compilation, as each form offers different
properties that facilitate particular analyses.  The Factor compiler optimizes
code in passes across two different \glsplural{IR}: first at a high-level using
the \factor|compiler.tree| vocabulary, then at a low-level with the
\factor|compiler.cfg| vocabulary.

\inputlst{tree}

The high-level \gls{IR} arranges code into a vector of \factor|node| objects,
which may themselves have children consisting of vectors of node---a tree
structure that lends to the name \factor|compiler.tree|.  This ordered sequence
of nodes represents control flow in a way that's effectively simple, annotated
stack code.  \Vref{lst:tree} shows the definitions of the tuples that represent
the ``instruction set'' of this stack code.  Each object inherits (directly or
indirectly) from the \factor|node| class, which itself inherits from
\factor|identity-tuple|.  This is a tuple whose \factor|equal?| method is
defined to always return \factor|f| so that no two instances are equivalent
unless they are the same instance.

Notice that most nodes define some sort of \factor|in-d| and \factor|out-d|
slots, which mark each of them with the input and output data stacks.  This
represents the flow of data through the program.  Here, stack values are
denoted simply by integers, giving each value a unique identifier.  An
\factor|#introduce| instance is inserted wherever the next node requires stack
values that have not yet been named.  Thus, while \factor|#introduce| has no
\factor|in-d|, its \factor|out-d| introduces the necessary stack values.
Similarly, \factor|#return| is inserted at the end of the sequence to indicate
the final state of the data stack with its \factor|in-d| slot.

The most basic operations of a stack language are, of course, pushing literals
and calling functions that pop inputs and push outputs.  The \factor|#push|
node thus has a \factor|literal| slot and an \factor|out-d| slot, giving a name
to the single element it pushes to the data stack.  \factor|#call|, of course,
is used for normal word invocations.  The \factor|in-d| and \factor|out-d|
slots effectively serve as the stack effect declaration.  In later analyses,
data about the word's definition may be stored across the \factor|body|,
\factor|method|, \factor|class|, and \factor|info| slots.

\inputlst{build-tree-1}

The word \factor|build-tree| takes a Factor quotation and constructs the
equivalent high-level \gls{IR} form.  In \vref{lst:build-tree-1}, we see the
output of the simple example
%
\factor|[ 1 + ] build-tree|.
%
Note that
%
\factor|T{ class { slot1 value1 } { slot2 value2 } ... }|
%
is the syntax for tuple literals.  The first node is a \factor|#push| for the
\factor|1| literal.  Since \factor|+| needs two input values, an
\factor|#introduce| pushes a new ``phantom'' value.  \factor|+| gets turned
into a \factor|#call| instance.  Notice the \factor|in-d| slot refers to the
values in the order that they're passed to the word, not necessarily the order
they've been introduced in the \gls{IR}.  The sum is pushed to the data stack,
so the \factor|out-d| slot is a singleton that names this value.  Finally,
\factor|#return| indicates the end of the routine, its \factor|in-d| containing
the value left on the stack (the sum pushed by \factor|#call|).

\inputlst{build-tree-2}

The next tuples in \vref{lst:tree} reassign existing values on the stack to
fresh identifiers.  The \factor|#renaming| superclass has the two subclasses
\factor|#copy| and \factor|#shuffle|.  The former represents the bijection from
elements of \factor|in-d| to elements of \factor|out-d| in the same position;
corresponding values are copies of each other.  The latter represents a more
general mapping.  Stack shufflers are translated to \factor|#shuffle| nodes
with \factor|mapping| slots that dictate how the fresh values in \factor|out-d|
correspond to the input values in \factor|in-d|.  For instance,
\vref{lst:build-tree-2} shows how \factor|swap| takes in the values
\factor|6256132| and \factor|6256133| and outputs \factor|6256134| and
\factor|6256135|, where the former is mapped to the second element
(\factor|6256133|) and the latter to the first (\factor|6256132|).  Thus,
\factor|out-d| swaps the two elements of \factor|in-d|, mapping them to fresh
identifiers.  The \factor|in-r| and \factor|out-r| slots of \factor|#shuffle|
correspond to the \term{retain} stack, which is an implementation detail beyond
the scope of this discussion.

\inputlst{build-tree-3}
\inputlst{build-tree-4}

\factor|#declare| is a miscellaneous node used for the \factor|declare|
primitive.  It simply annotates type information to stack values, as in
\vref{lst:build-tree-3}.  \factor|#terminate| is another one-off node, but a
much more interesting one.  While Factor normally requires a balanced stack,
sometimes we purposefully want to throw an error.  \factor|#terminate| is
introduced where the program halts prematurely.  When checking the stack
height, it gets to be treated specially so that \term{terminated} stack effects
unify with any other effect.  That way, branches will still be balanced even if
one of them unconditionally throws an error.  \vref{lst:build-tree-4} shows
\factor|#terminate| being introduced by the \factor|throw| word.

Next, \vref{lst:tree} defines nodes for branching based off the superclass
\factor|#branch|.  The \factor|children| slot contains vectors of nodes
representing different branches.  \factor|live-branches| is filled in during
later analyses to indicate which branches are alive so that dead ones may be
removed.  For instance, \factor|#if| will have two elements in its
\factor|children| slot representing the true and false branches.  On the other
hand, \factor|#dispatch| has an arbitrary number of children.  It corresponds
to the \factor|dispatch| primitive, which is an implementation detail of the
generic word system used to speed up method dispatch.

\todo[inline]{Should extract the SSA junk into the general intro of the thesis}

You may have noted the emphasis on introducing new values in \factor|out-d|
slots.  Even \factor|#shuffle|s output fresh identifiers, letting their values
be determined by its \factor|mapping|.  The reason for this is that
\factor|compiler.tree| uses \gls{SSA} form, wherein every variable is defined
by exactly one statement.  This simplifies the properties of variables, which
helps optimizations perform faster and with better results.  By giving unique
names to the targets of each assignment, the \gls{SSA} property is guaranteed.
However, \factor|#branch|es introduce ambiguity: after, say, an \factor|#if|,
what will the identifiers in \factor|out-d| be?  It depends on which branch is
taken.  To remedy this problem, after any \factor|#branch| node, Factor will
place a \factor|#phi| node---the classical \gls{SSA} ``phony function''.  The
\factor|phi-in-d| slot seen in \vref{lst:tree} is a sequence of sequences; each
one corresponds to the \factor|out-d| of the child at the same position in the
\factor|children| of the preceding node.  The \factor|#phi|'s \factor|out-d|
gives unique names to the output values, thus ensuring the \gls{SSA} property.
Though it doesn't perform any literal computation, conceptually it select the
``correct'' \factor|out-d| depending on the control flow.

\inputlst{build-tree-5}

For example, the \factor|#phi| in \vref{lst:build-tree-5} will select between
the
%
\factor|6256248|
%
return value of the first child or the 
%
\factor|6256249|
%
output of the second.  Either way, we can refer to the result as
\factor|6256250| afterwards.  The \factor|terminated| slot of the \factor|#phi|
tells us if there was a \factor|#terminate| in any of the branches.

The \factor|#recursive| node encapsulates \term{inline recursive} words.  In
Factor, words may be annotated with simple compiler declarations, which guide
optimizations.  If we follow a standard colon definition with the
\factor|inline| word, we're saying that its definition can be spliced into the
call-site, rather than generating code to jump to a subroutine.  Inline words
that call themselves must additionally be declared \factor|recursive|.  For
example, we could write
%
\factor|: foo ( -- ) foo ; inline recursive|.
%
The nodes \factor|#enter-recursive|, \factor|#call-recursive|, and
\factor|#return-recursive| denote different stages of the recursion---the
beginning, recursive call, and end, respectively.  They carry around a lot of
metadata about the nature of the recursion, but it doesn't serve our purposes
to get into the details.  Similarly, we gloss over the final nodes of
\vref{lst:tree} correspond to Factor's \gls{FFI} vocabulary, called
\factor|alien|.  At a high level, \factor|#alien-node|, \factor|#alien-invoke|,
\factor|#alien-indirect|, \factor|#alien-assembly|, and
\factor|#alien-callback| are used to make calls to C libraries from within
Factor.

\inputlst{optimize-tree}
\todo[inline]{Shouldn't bold ``cleanup'' in \cref{lst:optimize-tree}}

Now that we're familiar with the structure of the high-level \gls{IR}, we can
turn our attention to optimization.  \Vref{lst:optimize-tree} shows the passes
performed on a sequence of nodes by the word \factor|optimize-tree|.  Before
optimization can begin, we must gather some information and clean up some
oddities in the output of \factor|build-tree|.  \factor|analyze-recursive| is
called first to identify and mark loops in the tree.  Effectively, this means
we detect tail-recursion introduced by \factor|#recursive| nodes.  Future
passes can then use this information for data flow analysis.  Then,
\factor|normalize| makes the tree more consistent by doing two things:
%
\begin{itemize}
%
  \item All \factor|#introduce| nodes are removed and replaced by a single
        \factor|#introduce| at the beginning of the whole program.  This way,
        further passes needn't handle \factor|#introduce| nodes.
%
  \item As constructed, the \factor|in-d| of a \factor|#call-recursive| will be
        the entire stack at the time of the call.  This assumption happens
        because we don't know how many inputs it needs until the
        \factor|#return-recursive| is processed, because of row polymorphism.
        So, here we figure out exactly what stack entries are needed, and trim
        the \factor|in-d| and \factor|out-d| of each \factor|#call-recursive|
        accordingly.
%
\end{itemize}

Once these passes have cleaned up the tree, \factor|propagate| performs
probably the most extensive analysis of all the phases.  In short, it performs
an extended version of \gls{SCCP}\todo{cite}.  The traditional data flow
analysis combines global copy propagation, constant propagation, and some
limited constant folding in a \term{flow-sensitive} way.  That is, it will
propagate information from branches that it knows are definitely taken (e.g.,
because \factor|#if| is always given a true input).  Instead of using the
typical single-level (numeric) constant value lattice, Factor uses a lattice
augmented by information about classes, numeric value ranges, array lengths,
and tuple slots' classes.  Classes can be used in the lattice with the
partial-order protocol described briefly in \cref{sec:primer:oo}.
Additionally, the transfer functions are allowed to inline certain calls if
enough information is present.  This occurs in the transfer function since
generic words' inline expansions into particular methods provide more
information, thus giving us more opportunities for propagation.  This is
particularly useful for arithmetic words.  In Factor, words like \factor|+| and
\factor|*| are generics that work across all sorts of numeric representations,
be they \factor|fixnum|s, \factor|float|s, \factor|bignum|s, etc.  If the
operation overflows, the values are automatically cast up to larger
representations.  But iterated refinement of the inputs' classes can let the
compiler select more specific, efficient methods (e.g., if both arguments are
\factor|fixnum|s).

Interval propagation also helps propagate class information.  By refining the
range of possible values a particular item can have, we might discover that,
say, it's small enough to fit in a \factor|fixnum| rather than a
\factor|bignum|.  There are plenty more things that interval propagation can
tell us, too.  For example, it may give us enough information to remove
overflow checks performed by numeric words.  And if the interval has zero
length, we may replace the value with a constant.  This then continues getting
propagated, contributing to constant folding and so forth.

%cleanup
%dup run-escape-analysis? [
%    escape-analysis
%    unbox-tuples
%] when
%apply-identities
%compute-def-use
%remove-dead-code
%?check
%compute-def-use
%optimize-modular-arithmetic
%finalize

% dls.pdf verbatim:
%
%When static type information is available, Factor’s compiler can eliminate
%runtime method dispatch and allocation of in- termediate objects, generating
%code specialized to the under- lying data structures. This resembles previous
%work in soft typing [10]. Factor provides several mechanisms to facilitate
%static type propagation:
%
%\begin{itemize}
%
%\item Functions can be annotated as inline, causing the compiler to replace
%calls to the function with the function body.
%
%\item Functions can be hinted, causing the compiler to gener- ate multiple
%specialized versions of the function, each assuming different input types, with
%dispatch at the en- try point to choose the best-fitting specialization for the
%given inputs.  
%
%\item Methods on generic functions propagate the type infor- mation for their
%dispatched-on inputs.  
%
%\item Functions can be declared with static input and output types using the
%typed library.
%
%\end{itemize}
%
%The three major optimizations performed on high-level IR are sparse conditional
%constant propagation (SCCP [45]), escape analysis with scalar replacement, and
%overflow check elimination using modular arithmetic properties.  The major
%features of our SCCP implementation are an extended value lattice, rewrite
%rules, and flow sensitivity.  Our SCCP implementation augments the standard
%single- level constant lattice with information about object types, numeric
%intervals, array lengths and tuple slot types. Type transfer functions are
%permitted to replace nodes in the IR with inline expansions. Type functions are
%defined on many of the core language words.  SCCP is used to statically dispatch
%generic word calls by inlining a specific method body at the call site. This
%inlining generates new type information and new opportunities for constant
%folding, simplification and further inlining. In par- ticular, generic
%arithmetic operations which require dynamic dispatch in the general case can be
%lowered to simpler opera- tions as type information is discovered. Overflow
%checks can be removed from integer operations using numeric interval
%information. The analysis can represent flow-sensitive type information.
%Additionally, calls to closures which combina- tor inlining cannot eliminate
%are eliminated when enough in- formation is available [16].

%An escape analysis
%pass is used to discover object alloca- tions which are not stored on the heap
%or returned from the current function. Scalar replacement is performed on such
%allocations, converting tuple slots into SSA values.  The modular arithmetic
%optimization pass identifies in- teger expressions in which the final result is
%taken to be modulo a power of two and removes unnecessary overflow checks from
%any intermediate addition and multiplication operations. This novel
%optimization is global and can operate over loops.

%\subsection{Low-level Optimizations}\label{sec:compiler:cfg}

The low-level \gls{IR} in \factor|compiler.cfg| takes the more conventional
form of a \gls{CFG}.  A \gls{CFG} (not to be confused with ``context-free
grammar'') is an arrangement of instructions into \term{basic blocks}: maximal
sequences of ``straight-line'' code, where control does not transfer out of or
into the middle of the block.  Directed edges are added between blocks to
represent control flow---either from a branching instruction to its target, or
from the end of a basic block to the start of the next one\todo{cite}.
Construction of the low-level \gls{IR} proceeds by analyzing the control flow
of the high-level \gls{IR} and converting the nodes of \cref{sec:compiler:tree}
into lower-level, more conventional instructions modeled after typical assembly
code.  There are over a hundred of these instructions, but many are simply
different versions of the same operation.  For instance, while instructions are
generally called on \term{virtual registers} (represented in Factor simply by
integers), there are \term{immediate} versions of instructions.  The
\factor|##add| instruction, as an example, represents the sum of the contents
of two registers, but \factor|##add-imm| sums the contents of one register and
an integer literal.  Other instructions are inserted to make stack reads and
writes explicit, as well as to balance the height.  Below is a categorized list
of all the instruction objects (each one is a subclass of the \factor|insn|
tuple).

\todo[inline]{Is the complete list really necessary?}
\begin{itemize}
\item
\begin{flushleft}
Loading constants:
\Verb|##load-integer|,
\Verb|##load-reference|
\end{flushleft}

\item
\begin{flushleft}
Optimized loading of constants, inserted by representation selection:
\Verb|##load-tagged|,
\Verb|##load-float|,
\Verb|##load-double|,
\Verb|##load-vector|
\end{flushleft}

\item
\begin{flushleft}
Stack operations:
\Verb|##peek|,
\Verb|##replace|,
\Verb|##replace-imm|,
\Verb|##inc-d|,
\Verb|##inc-r|
\end{flushleft}

\item
\begin{flushleft}
Subroutine calls:
\Verb|##call|,
\Verb|##jump|,
\Verb|##prologue|,
\Verb|##epilogue|,
\Verb|##return|
\end{flushleft}

\item
\begin{flushleft}
Inhibiting \gls{TCO}:
\Verb|##no-tco|
\end{flushleft}

\item
\begin{flushleft}
Jump tables:
\Verb|##dispatch|
\end{flushleft}

\item
\begin{flushleft}
Slot access:
\Verb|##slot|,
\Verb|##slot-imm|,
\Verb|##set-slot|,
\Verb|##set-slot-imm|
\end{flushleft}

\item
\begin{flushleft}
Register transfers:
\Verb|##copy|,
\Verb|##tagged>integer|
\end{flushleft}

\item
\begin{flushleft}
Integer arithmetic:
\Verb|##add|,
\Verb|##add-imm|,
\Verb|##sub|,
\Verb|##sub-imm|,
\Verb|##mul|,
\Verb|##mul-imm|,
\Verb|##and|,
\Verb|##and-imm|,
\Verb|##or|,
\Verb|##or-imm|,
\Verb|##xor|,
\Verb|##xor-imm|,
\Verb|##shl|,
\Verb|##shl-imm|,
\Verb|##shr|,
\Verb|##shr-imm|,
\Verb|##sar|,
\Verb|##sar-imm|,
\Verb|##min|,
\Verb|##max|,
\Verb|##not|,
\Verb|##neg|,
\Verb|##log2|,
\Verb|##bit-count|
\end{flushleft}

\item
\begin{flushleft}
Float arithmetic:
\Verb|##add-float|,
\Verb|##sub-float|,
\Verb|##mul-float|,
\Verb|##div-float|,
\Verb|##min-float|,
\Verb|##max-float|,
\Verb|##sqrt|
\end{flushleft}

\item
\begin{flushleft}
Single/double float conversion:
\Verb|##single>double-float|,
\Verb|##double>single-float|
\end{flushleft}

\item
\begin{flushleft}
Float/integer conversion:
\Verb|##float>integer|,
\Verb|##integer>float|
\end{flushleft}

\item
\begin{flushleft}
\Gls{SIMD} operations:
\Verb|##zero-vector|,
\Verb|##fill-vector|,
\Verb|##gather-vector-2|,
\Verb|##gather-int-vector-2|,
\Verb|##gather-vector-4|,
\Verb|##gather-int-vector-4|,
\Verb|##select-vector|,
\Verb|##shuffle-vector|,
\Verb|##shuffle-vector-halves-imm|,
\Verb|##shuffle-vector-imm|,
\Verb|##tail>head-vector|,
\Verb|##merge-vector-head|,
\Verb|##merge-vector-tail|,
\Verb|##float-pack-vector|,
\Verb|##signed-pack-vector|,
\Verb|##unsigned-pack-vector|,
\Verb|##unpack-vector-head|,
\Verb|##unpack-vector-tail|,
\Verb|##integer>float-vector|,
\Verb|##float>integer-vector|,
\Verb|##compare-vector|,
\Verb|##test-vector|,
\Verb|##test-vector-branch|,
\Verb|##add-vector|,
\Verb|##saturated-add-vector|,
\Verb|##add-sub-vector|,
\Verb|##sub-vector|,
\Verb|##saturated-sub-vector|,
\Verb|##mul-vector|,
\Verb|##mul-high-vector|,
\Verb|##mul-horizontal-add-vector|,
\Verb|##saturated-mul-vector|,
\Verb|##div-vector|,
\Verb|##min-vector|,
\Verb|##max-vector|,
\Verb|##avg-vector|,
\Verb|##dot-vector|,
\Verb|##sad-vector|,
\Verb|##horizontal-add-vector|,
\Verb|##horizontal-sub-vector|,
\Verb|##horizontal-shl-vector-imm|,
\Verb|##horizontal-shr-vector-imm|,
\Verb|##abs-vector|,
\Verb|##sqrt-vector|,
\Verb|##and-vector|,
\Verb|##andn-vector|,
\Verb|##or-vector|,
\Verb|##xor-vector|,
\Verb|##not-vector|,
\Verb|##shl-vector-imm|,
\Verb|##shr-vector-imm|,
\Verb|##shl-vector|,
\Verb|##shr-vector|
\end{flushleft}

\item
\begin{flushleft}
Scalar/vector conversion:
\Verb|##scalar>integer|,
\Verb|##integer>scalar|,
\Verb|##vector>scalar|,
\Verb|##scalar>vector|
\end{flushleft}

\item
\begin{flushleft}
Boxing and unboxing aliens:
\Verb|##box-alien|,
\Verb|##box-displaced-alien|,
\Verb|##unbox-any-c-ptr|,
\Verb|##unbox-alien|
\end{flushleft}

\item
\begin{flushleft}
Zero-extending and sign-extending integers:
\Verb|##convert-integer|
\end{flushleft}

\item
\begin{flushleft}
Raw memory access:
\Verb|##load-memory|,
\Verb|##load-memory-imm|,
\Verb|##store-memory|,
\Verb|##store-memory-imm|
\end{flushleft}

\item
\begin{flushleft}
Memory allocation:
\Verb|##allot|,
\Verb|##write-barrier|,
\Verb|##write-barrier-imm|,
\Verb|##alien-global|,
\Verb|##vm-field|,
\Verb|##set-vm-field|
\end{flushleft}

\item
\begin{flushleft}
The \gls{FFI}:
\Verb|##unbox|,
\Verb|##unbox-long-long|,
\Verb|##local-allot|,
\Verb|##box|,
\Verb|##box-long-long|,
\Verb|##alien-invoke|,
\Verb|##alien-indirect|,
\Verb|##alien-assembly|,
\Verb|##callback-inputs|,
\Verb|##callback-outputs|
\end{flushleft}

\item
\begin{flushleft}
Control flow:
\Verb|##phi|,
\Verb|##branch|
\end{flushleft}

\item
\begin{flushleft}
Tagged conditionals:
\Verb|##compare-branch|,
\Verb|##compare-imm-branch|,
\Verb|##compare|,
\Verb|##compare-imm|
\end{flushleft}

\item
\begin{flushleft}
Integer conditionals:
\Verb|##compare-integer-branch|,
\Verb|##compare-integer-imm-branch|,
\Verb|##test-branch|,
\Verb|##test-imm-branch|,
\Verb|##compare-integer|,
\Verb|##compare-integer-imm|,
\Verb|##test|,
\Verb|##test-imm|
\end{flushleft}

\item
\begin{flushleft}
Float conditionals:
\Verb|##compare-float-ordered-branch|,
\Verb|##compare-float-unordered-branch|,
\Verb|##compare-float-ordered|,
\Verb|##compare-float-unordered|
\end{flushleft}

\item
\begin{flushleft}
Overflowing arithmetic:
\Verb|##fixnum-add|,
\Verb|##fixnum-sub|,
\Verb|##fixnum-mul|
\end{flushleft}

\item
\begin{flushleft}
\Gls{GC} checks:
\Verb|##save-context|,
\Verb|##check-nursery-branch|,
\Verb|##call-gc|
\end{flushleft}

\item
\begin{flushleft}
Spills and reloads, inserted by the register allocator:
\Verb|##spill|,
\Verb|##reload|
\end{flushleft}
\end{itemize}


\inputlst{optimize-cfg}

\inputfig{optimize-tail-calls}

By translating the high-level \gls{IR} into instructions that manipulate
registers directly, we reveal further redundancies that can be optimized away.
The \factor|optimize-cfg| word in \vref{lst:optimize-cfg} shows the passes
performed in doing this.  The first word, \factor|optimize-tail-calls|,
performs tail call elimination on the \gls{CFG}.
%
\term{Tail calls}~\todo{used in \cref{sec:compiler:tree}, not defined} are those
that occur within a procedure and whose results are immediately returned by
that procedure.  Instead of allocating a new call stack frame, we may convert
tail calls into simple jumps, since afterwards the current procedure's call
frame isn't really needed.  In the case of recursive tail calls, we can convert
special cases of recursion into loops in the \gls{CFG}, so that we won't
trigger call stack overflows.  For instance, consider
\vref{fig:optimize-tail-calls}, which shows the effect of
\factor|optimize-tail-calls| on the following definition:
%
\begin{center}
%
  \factor|: tail-call ( -- ) tail-call ;|
%
\end{center}
%
\noindent Note the recursive call (trivially) occurs at the end of the
definition, just before the return point.  When translated to a \gls{CFG}, this
is a \factor|##call| instruction, as seen in block $4$ to the left of
\vref{fig:optimize-tail-calls}.  This is also just before the final
\factor|##epilogue| and \factor|##return| instructions in block $8$, as blocks
$5$--$7$ are effectively empty (these excessive \factor|##branch|es will be
eliminated in a later pass).  Because of this, rather than make a whole new
subroutine call, we can convert it into a \factor|##branch| back to the
beginning of the word, as in the \gls{CFG} to the right.

\inputfig{delete-useless-conditionals}

The next pass in \vref{lst:optimize-cfg} is
\factor|delete-useless-conditionals|, which removes branches that go to the
same basic block.  This situation might occur as a result of optimizations
performed in the high-level \gls{IR}.  To see it in action,
\vref{fig:delete-useless-conditionals} shows the transformation on a
purposefully useless conditional,
%
\factor|[ ] [ ] if|.
%
Before removing the useless conditional, the \gls{CFG} \factor|##peek|s at the
top of the data stack
%
(\factor|D 0|),
%
storing the result in the virtual register \factor|1|.  This value is popped,
so we decrement the stack height
%
(\factor|##inc-d -1|).
%
Then, \factor|##compare-imm-branch| in block $2$ compares the value in the
virtual register \factor|2| (which is a copy of \factor|1|, the top of the
stack) to the immediate value \factor|f| to see if it's not equal (signified by
\factor|cc/=|).  However, both branches jump through several empty blocks and
merge at the same destination.  Thus, we can remove both branches and replace
\factor|##compare-imm-branch| with an unconditional \factor|##branch| to the
eventual destination.  We see this on the right of
\vref{fig:delete-useless-conditionals}.

\inputfig{split-branches}

In order to expose more opportunities for optimization, \factor|split-branches|
will actually duplicate code.  We use the fact that code immediately following
a conditional will be executed along either branch.  If it's sufficiently
short, we copy it up into the branches individually.  That is, we change
%
\factor|[ A ] [ B ] if C|
%
into
%
\factor|[ A C ] [ B C ] if|,
%
as long as \factor|C| is small enough.  Later analyses may then, for example,
more readily eliminate one of the branches if it's never taken.
\vref{fig:split-branches} shows what such a transformation looks like on a
\gls{CFG}.  The example
%
\factor|[ 1 ] [ 2 ] if dup|
%
is essentially changed into
%
\factor|[ 1 dup ] [ 2 dup ] if|,
%
thus splitting the block with two predecessors (block $9$) on the left.

\inputfig{join-blocks}

The next pass, \factor|join-blocks|, compacts the \gls{CFG} by joining together
blocks involved in only a single control flow edge.  Mostly, this is to clean
up the myriad of empty or short blocks introduced during construction, like
sequences of a bunch of \factor|##branch|es.  \Vref{fig:join-blocks} shows this
pass on the \gls{CFG} of
%
\factor|0 100 [ 1 fixnum+fast ] times|.
%
\factor|fixnum+fast| is a specialized version of \factor|+| that suppresses
overflow and type checks.  We use it here to keep the \gls{CFG} simple.  We'll
be using this particular code to illustrate all but one of the remaining
optimization passes in \vref{lst:optimize-cfg}, as it's a motivating example
for the work in this thesis.  The passes before \factor|join-blocks| don't
change the \gls{CFG} seen on the left in \vref{fig:join-blocks}, but we get rid
of the useless \factor|##branch| blocks in the \gls{CFG} on the right.

\inputfig{normalize-height}

\Vref{fig:normalize-height} shows the result of applying
\factor|normalize-height| to the result of \factor|join-blocks|.  This phase
combines and canonicalizes the instructions that track the stack height, like
\factor|##inc-d|.  While the shuffling in this example isn't complex enough to
be interesting, neither is this phase.  It amounts to more cleanup: multiple
height changes are combined into single ones at the beginnings of the basic
blocks.  In \vref{fig:normalize-height}, this means that \factor|##inc-d| is
moved to the top of block $1$, as compared to the right of
\vref{fig:join-blocks}.

\inputfig{construct-ssa}

In converting the high-level \gls{IR} to the low-level, we actually lose the
\gls{SSA} form of \factor|compiler.tree|.  Not only does the construction do
this, but \factor|split-branches| also copies basic blocks verbatim, so any
value defined will have a duplicate definition site, violating the \gls{SSA}
property.  \factor|construct-ssa| recomputes a so-called \term{pruned}
\gls{SSA} form, wherein $\phi$ functions are inserted only if the variables are
live after the insertion point.  This cuts down on useless $\phi$ 
%
functions\todo{cite TDMSC and construction algorithm}.
%
\Vref{fig:construct-ssa} shows the reconstructed \gls{SSA} form of the
\gls{CFG} from \vref{fig:normalize-height}.

The next pass, \factor|alias-analysis|, doesn't change the \gls{CFG} of
%
\factor|0 100 [ 1 fixnum+fast ] times|,
%
so we won't have an accompanying \lcnamecref{fig:construct-ssa}.  At a high
level, \factor|alias-analysis| is easy to understand: it eliminates redundant
memory loads and stores by rewriting certain patterns of memory access.  If the
same location is loaded after being stored, we convert the latter load into a
\factor|##copy| of the value we stored.  Two reads of the same location with no
intermittent write gets the second read turned into a \factor|##copy|.
Similarly, if we see two writes without a read in the middle, the first write
can be removed.

\inputfig{value-numbering}

\factor|value-numbering| is the key focus of this thesis.  It will be detailed
in \vref{sec:vn}.  For now, it does to think of it as a combination of common
subexpression elimination and constant folding.  In \vref{fig:value-numbering}, 
we see several changes:
%
\begin{itemize}
%
  \item \factor|##load-integer 23 0| in block $1$ of \vref{fig:construct-ssa}
  (which assigns the value \factor|0| to the virtual register \factor|23|) is
  redundant, so is replaced by \factor|##copy 23 21|.
%
  \item \begin{flushleft}
  In block $2$, the last instruction
  %
  \factor|##compare-imm-branch 32 f cc/=|
  %
  is the same as
  %
  \factor|##compare-integer-branch 30 26 cc<|.
  %
  The source register (\factor|32|) of the original is a \factor|##copy| of
  \factor|31|, which itself is computed by
  %
  \factor|##compare-integer 31 30 26 cc< 9|.
  %
  So, the \factor|##compare-imm-branch| is equivalent to a simple
  \factor|##compare-integer-branch|, which doesn't use the temporary virtual
  register \factor|9| and doesn't waste time comparing against the \factor|f|
  object.
  \end{flushleft}
%
  \item The second operands in both \factor|##add|s of block $3$ are just
  constants stored by \factor|##load-integer|s.  So, these are turned into
  \factor|##add-imm|s.
\end{itemize}
%
\noindent In \cref{sec:vn}, we'll see how and why this pass fails to identify
other equivalences.

\inputfig{copy-propagation}

\inputfig{eliminate-dead-code}

\inputfig{finalize-cfg}

% dls.pdf verbatim:

%The main optimizations performed on low-level IR are local dead store and
%redundant load elimination, local value numbering, global copy propagation,
%representation selection, and instruction scheduling.  The local value
%numbering pass eliminates common subexpressions and folds expressions with
%constant operands [9].

%Following value numbering and copy propagation, a representation selection
%pass decides when to unbox floating point and SIMD values. A form of
%instruction scheduling intended to reduce register pressure is performed on
%low-level IR as the last step before leaving SSA form [39].  We use the
%second-chance binpacking variation of the linear scan register allocation
%algorithm [43, 47]. Our variant does not take $\phi$ nodes into account, so
%SSA form is destructed first by eliminating $\phi$ nodes while simultaneously
%performing copy coalescing, using the method described in [6].


% dls.pdf verbatim:

%Low-level IR is built from high-level IR by analyzing control flow and making
%stack reads and writes explicit. During this construction phase and a
%subsequent branch splitting phase, the SSA structure of high-level IR is lost.
%SSA form is recon- structed using the SSA construction algorithm described in
%[8], with the minor variation that we construct pruned SSA form rather than
%semi-pruned SSA, by first computing live- ness. To avoid computing iterated
%dominance frontiers, we use the TDMSC algorithm from [13].  The main
%optimizations performed on low-level IR are local dead store and redundant load
%elimination, local value numbering, global copy propagation, representation
%selec- tion, and instruction scheduling.  The local value numbering pass
%eliminates common sub- expressions and folds expressions with constant operands
%[9]. Following value numbering and copy propagation, a representation selection
%pass decides when to unbox floating point and SIMD values. A form of instruction
%scheduling intended to reduce register pressure is performed on low- level IR
%as the last step before leaving SSA form [39].  We use the second-chance
%binpacking variation of the lin- ear scan register allocation algorithm [43,
%47]. Our variant does not take φ nodes into account, so SSA form is destruc-
%ted first by eliminating φ nodes while simultaneously per- forming copy
%coalescing, using the method described in [6].

\newpage\chapter{Value Numbering}\label{sec:vn}

At a very basic level, most optimization techniques revolve around avoiding
redundant or unnecessary computation.  Thus, it's vital that we discover which
values in a program are equal.  That way, we can simplify the code that wastes
machine cycles repeatedly calculating the same values.  Classic optimization
phases like constant/copy propagation, common subexpression elimination,
loop-invariant code motion, induction variable elimination, and others
discussed in the de facto treatise, ``The Dragon Book'' \autocite{DragonBook},
perform this sort of redundancy elimination based on information about the
equality of expressions.

In general, the problem of determining whether two expressions in a program are
equivalent is undecidable.  Therefore, we seek a \term{conservative} solution
that doesn't necessarily identify all equivalences, but is nevertheless correct
about any equivalences it does identify.  Solving this equivalence problem is
the work of \term{value numbering} algorithms.  These assign every value in the
program a number such that two values have the same value number if and only if
the compiler can prove they will be equal at runtime.

Value numbering has a long history in literature and practice, spanning many
techniques.  In \cref{sec:compiler:cfg} we saw the \Verb|value-numbering| word,
which is actually based on some of the earliest---and least effective---methods
of value numbering.  \Cref{sec:vn:local} describes the way Factor's current
algorithm works, highlighting its shortcomings to motivate the main work of
this thesis, which is covered in \cref{sec:vn:global,sec:vn:avail}.  We finish
the \lcnamecref{sec:vn} by analyzing the results of these changes and reviewing
the literature for further enhancements that can be made to this optimization
pass.

\section{Local Value Numbering}\label{sec:vn:local}

Tracing the exact origins of value numbering is difficult.  It's thought to
have originally been invented in the 1960s by Balke \autocite{Simpson}.  The
earliest tangible reference to a value numbering (at least, the earliest point
where discussions in the literature seem to start) appears in an oft-cited but
unpublished work of \citeauthor{Cocke} \autocite*{Cocke}.  The technique is
relatively simple, but not as powerful as other methods for reasons described
hereafter.

The algorithm considers a single basic block.  For each instruction (from top
to bottom) in the block, we essentially let the value number of the assignment
target be a hash of the operator and the value numbers of the operands.  That
is, we hash the \term{expression} being computed by an instruction.  Thus,
assuming a proper hash function, two expressions are \term{congruent} if
%
\begin{itemize}
%
  \item they have the same operators and
%
  \item their operands are congruent.
%
\end{itemize}
%
\noindent This is our approximation of runtime equivalence.  The first property
is fulfilled by basing the hash, in part, on the operator.  The second property
holds because the hash is based on the value numbers of the statement's
operands---not just the operands as they appear in code (i.e., \term{lexical}
equivalence).  Any information about congruence is propagated through the value
numbers.  We'll have discovered any such equivalences among the operands before
computing the value number of the assignment target because every value in a
basic block is either defined before it's used, or else defined at some point
in a predecessor of the block, which we don't care about when only considering
one basic block.

This is the first shortcoming of the algorithm.  It is \term{local}, focusing
on only one basic block at a time.  Any definitions outside the boundaries of
the basic block won't be reused, even if they reach the block.  This severely
limits the scope of the redundancies we can discover.  We could improve upon
this by considering the algorithm across an entire loop-free \gls{CFG} in any
\term{topological order}.  In such an ordering, a basic block $B$ comes before
any other block $B'$ to which it has an edge.  Thus, any ``outside'' variables
that instructions in $B'$ rely on must have come from $B$ or earlier, which
will have already been computed in a traversal of such an ordering.  However,
\glsplural{CFG} usually contain cycles or loops (at least interesting ones do),
which make such an ordering impossible.  We could still pick a topological
order that ignores back-edges, but we may encounter operands whose values flow
along those back-edges, so haven't been processed yet.  We'll address this
issue later.

\begin{sloppypar}
In Factor, expressions are basically instructions (the \Verb|insn| objects
discussed in \cref{sec:compiler:cfg}) that have had their destination registers
stripped.  Instructions can be converted to expressions with the \Verb|>expr|
word defined in the \Verb|compiler.cfg.value-numbering.expressions|
vocabulary.  For instance, an \Verb|##add| instruction with the destination
register \Verb|1| and source registers \Verb|2| and \Verb|3| will be
converted into an array of three elements:
%
\begin{itemize}
%
  \item The \Verb|##add| class word, indicating the expression is derived
        from an \Verb|##add| instruction.
%
  \item The value number of the virtual register \Verb|2|.
%
  \item The value number of the virtual register \Verb|3|.
%
\end{itemize}
%
\noindent Some instructions are not \term{referentially transparent}, meaning
they can't be replaced with the value they compute without changing the
program's behavior.  For example, \Verb|##call| and \Verb|##branch| cannot
reasonably be converted into expressions.  In these cases, \Verb|>expr|
merely returns a unique value.
\end{sloppypar}

\inputlst{value-numbering-graph}

The hashing of expressions takes place in the so-called \term{expression graph}
implemented in the vocabulary shown in \vref{lst:value-numbering-graph}.  This
consists of three global hash tables that relate virtual registers, value
numbers, instructions, and expressions.  Since virtual registers are just
integers, we actually use them as value numbers, too.  \Verb|vregs>vns| maps
virtual registers to their value numbers.  If a virtual register  is mapped to
itself in this table, its definition is the canonical instruction that we use
to compute the value.  This instruction is stored in the \Verb|vns>insns| table.  Finally, the most important mapping is \Verb|exprs>vns|.  True to its
name, it uses expressions as keys, which of course are implicitly hashed.
Thus, we can use this table to determine equivalence of expressions.

Other definitions in \vref{lst:value-numbering-graph} manipulate expressions
and the graph.  The global variable \Verb|input-expr-counter| is used in the
generation of unique expressions discussed earlier.  \Verb|init-value-graph|
initializes this and all the tables.  \Verb|set-vn| establishes a mapping
from a virtual register to a value number in \Verb|vregs>vns|.
\Verb|vn>insn| gives terse access to the \Verb|vns>insns| table.
\Verb|vreg>insn| uses \Verb|vregs>vns| and \Verb|vns>insns| to get the
canonical instruction that defines a given virtual register.  Finally,
\Verb|vreg>vn| looks up the value of a key in the \Verb|vregs>vns| table.
Importantly, if the key is not yet present in the table, it is automatically
mapped to itself---it's assumed that the virtual register does not correspond
to a redundant instruction.

This is the second shortcoming of the algorithm.  It must make a
\term{pessimistic} assumption about congruences.  That is, it starts by
assuming that every expression has a unique value number, then tries to prove
that there are some values which are actually congruent.  This fails to
discover congruences for values that flow along back-edges, whether we consider
a single basic block or an entire topological ordering.

One the other hand, an advantage of this local value numbering algorithm is its
simplicity.  It makes a single pass over all the instructions, identifying and
replacing redundancies \term{online} (i.e., rewriting as it goes).  It's
straightforward to write, and even to extend.  In particular, there's nothing
stopping the online replacements from being more complex than \Verb|##copy|
instructions.  At every step, the currently known value numbers will be sound,
and we can use this information for copy/constant propagation, constant
folding, and common subexpression elimination.

\inputlst{value-numbering-step}

\begin{sloppypar}
To see how Factor accomplishes these extensions, we'll take a look at the
\Verb|compiler.cfg.value-numbering| vocabulary.
\Vref{lst:value-numbering-step} shows the main words that start the
optimization pass.  The \Verb|value-numbering-step| word is called on the
sequence of instructions that comprise each basic block.  It starts the
expression graph from a blank slate with \Verb|init-value-graph|, then
\Verb|map|s the word \Verb|process-instruction| on each of them.  This is a
generic word that we'll study momentarily; it returns either a single
\Verb|insn| object or a sequence of them (in the case that an instruction is
replaced by several others).  Then, the work of \Verb|value-numbering| is to
just call \Verb|value-numbering-step| on each basic block, which is done with a
combinator called \Verb|simple-optimization|.  The words \Verb|cfg-changed| and
\Verb|predecessors-changed| alter some internal state of the \gls{CFG} that has
been potentially invalidated by some transformations performed by
\Verb|process-instruction|.
\end{sloppypar}

\inputlst{process-instruction}

The methods of \Verb|process-instruction| are shown in
\vref{lst:process-instruction}.  The default behavior for dispatching on an
\Verb|insn| is to invoke yet another generic word, \Verb|rewrite|.  This word
will return either a replacement \Verb|insn| (or sequence thereof) or
\factor|f|, indicating that no change has taken place.  Thus, by recursively
calling \Verb|process-instruction|, we can do more specialized processing on
this rewritten replacement (e.g., dispatching on \Verb|insn| again, which
applies \Verb|rewrite| once more).  If the instruction can't be simplified
further, we simply return it.  (Note that
%
\factor|[ X ] [ Y ] ?if|
%
is the same as
%
\factor|dup [ nip X ] [ drop Y ] if|.)

For instances of \Verb|foldable-insn| (i.e., \Verb|insn|s that can be
converted to useful expressions with \Verb|>expr|), we similarly invoke
\Verb|rewrite| recursively until no more rewriting occurs.  When that
happens, rather than just return the instruction, we invoke
\Verb|check-redundancy|---though only if the instruction defines exactly $1$
virtual register, which will be stored in a slot named \Verb|dst|.
\Verb|check-redundancy| checks if the expression being computed by the
instruction is already a key of the \Verb|exprs>vns| table.  If it is, the
instruction is redundant, and we call \Verb|redundant-instruction|;
otherwise, we call \Verb|useful-instruction|.  The former uses
\Verb|set-vn| to map the instruction's \Verb|dst| virtual register to the
same value number as the expression that was a key of \Verb|exprs>vns|.
Since value numbers are actually virtual registers, we may also use these two
integers the source and destination registers in a new \Verb|##copy|
instruction, which is then returned.  On the other hand,
\Verb|useful-instruction| saves the instruction's information in the
expression graph by setting the appropriate values in \Verb|vregs>vns|,
\Verb|exprs>vns|, and \Verb|vns>insns|.  Note that in definitions using the
syntax
%
\factor|:: word-name ( stack -- effect ) ... ;|,
%
the input values in the stack effect are actually named lexical variables, like
in most programming languages\todo{move this to primer?}.  Furthermore, the
first line
%
\factor|insn dst>> :> vn|
%
assigns the input instruction's destination register to the variable
\Verb|vn|, which is used later in the definition of
\Verb|useful-instruction|.

The \Verb|##copy| method of \Verb|process-instruction| cannot do anything
to simplify the instruction, but will set the value number of the destination
register to that of the source.  By calling \Verb|vreg>vn| on the source
register, we make sure to call \Verb|set-vn| between the destination and the
canonical value number of the source.

Finally, the \Verb|array| method is used for the purposes of recursion, in
the case that \Verb|rewrite| returns a sequence of replacement instructions.
The correct action is, of course, to descend into this new sequence of
instructions with \Verb|process-instruction|.

Underlying all of the redundancy elimination is the \Verb|rewrite| generic
word.  It has too many methods to look at the source code in-depth here, but
it's instructive to give an overview of the transformations.  These methods
actually make up the bulk of the \Verb|compiler.cfg.value-numbering| code.
They're spread across various sub-vocabularies.
\Verb|compiler.cfg.value-numbering.rewrite| defines the generic itself, along
with a handful of utilities.  The method for the most general instruction
class, \Verb|insn|, is defined to unconditionally return \factor|f|, meaning no
rewriting is performed by default.  That way, we need only define
\Verb|rewrite| methods for more specific instruction classes to get specialized
behavior.

\begin{sloppypar}
\Verb|compiler.cfg.value-numbering.alien| contains methods that simplify nodes
related to Factor's \gls{FFI}.  Most involve fusing together the results of
intermediate arithmetic.  The instructions that access raw memory (namely
\Verb|##load-memory|, \Verb|##load-memory-imm|, \Verb|##store-memory|, and
\Verb|##store-memory-imm|) tend to have inputs to perform address arithmetic.
Each has slots for a \Verb|base| register containing an address and a literal
\Verb|offset| from it.  But if \Verb|base| is defined by an \Verb|##add-imm|
instruction, we can just update the \Verb|offset|, incrementing it by the
literal operand of the \Verb|##add-imm|.  Then, \Verb|base| will just be
changed to the register operand of the \Verb|##add-imm|.  This removes the
memory instruction's need for the \Verb|##add-imm|, increasing the chances that
the latter will become dead code to be removed later.  Unlike the \Verb|-imm|
variants, \Verb|##load-memory| and \Verb|##store-memory| also take a
\Verb|displacement| register, which works like a non-immediate \Verb|offset|.
Therefore, \Verb|##add|s can be similarly fused into \Verb|##load-memory-imm|
and \Verb|##store-memory-imm| by transforming them into \Verb|##load-memory|
and \Verb|##store-memory| instructions with the \Verb|##add|'s operand as the
\Verb|displacement|.  A few other similar transformations are also done,
including rewrites for \Verb|##box-displaced-alien|s and
\Verb|##unbox-any-c-ptr|s.
\end{sloppypar}

\Verb|compiler.cfg.value-numbering.comparisons| defines methods for the
various branching and comparison instructions (which simply store booleans in
registers, rather than branching upon them).  The major optimizations performed
are as follows:
%
\begin{itemize}
%
\item If possible, instructions are converted to more specific forms.  For
example, non-immediate instructions (e.g., \Verb|##compare|) may be turned
into their \Verb|-imm| counterparts (e.g., \Verb|##compare-imm|)  if one of
their source registers corresponds to a literal value.
\Verb|##compare-integer-imm| is also converted to \Verb|##test| if the
architecture supports it.  This corresponds to a special instruction in x86
that performs a bitwise AND for its side effects on particular flags,
discarding the actual result.  This can be more efficient when using the AND
result as a boolean.
%
\item If both inputs to a comparison or branch are literals, we may
constant-fold the instruction.  In the case of comparisons, this means
converting it into a \Verb|##load-reference| of the proper boolean.  In
branches, this modifies the \gls{CFG} so that the path which isn't taken is
removed completely.
%
\item Like a novice programmer writing
%
\mint{java}|if (some_boolean != false) { ... }|
%
in Java, the compiler may generate redundant boolean comparisons that need
cleaning up.  That is, the intermediate boolean values are eliminated when the
result of a comparison is used by another comparison, collapsing the whole
thing into a single instruction.
%
\end{itemize}

\Verb|compiler.cfg.value-numbering.folding| defines some auxiliary words for
constant-folding arithmetic words.  Mainly, \Verb|unary-constant-fold| and
\Verb|binary-constant-fold| perform the actual operation on the one or two
constant inputs provided.  These words are used in
\Verb|compiler.cfg.value-numbering.math|, which predictably simplifies math
via standard rules.  Arithmetic identities are rewritten---conceptually, $x+0$
becomes just $x$, for instance.  If self-inverting instructions (namely
\Verb|##neg| for numerical negation and \Verb|##not| for boolean negation)
are called on registers that themselves correspond to the same instruction, we
can safely rewrite them into \Verb|##copy| instructions.  Non-immediate
instructions are converted to their \Verb|-imm| forms, if possible, and if
both operands are constant, the expression is folded.  The most interesting
math optimizations use the associative and distributive laws.
\term{Reassociation} conceptually converts $(x \otimes y) \otimes z$ into $x
\otimes (y \otimes z)$ when both $y$ and $z$ are constants and $\otimes$ is
associative.  So, for example,
%
\begin{center}
\Verb|##add-imm 1 X Y|\\
\Verb|##add-imm 2 1 Z|
\end{center}
%
\noindent is converted into just
%
\begin{center}
\Verb|##add-imm 2 X (Y+Z)|
\end{center}
%
\noindent where \Verb|X| is a virtual register, and \Verb|Y| and \Verb|Z|
are constants.  \term{Distribution} converts $(x \oplus y) \otimes z$ into $(x
\otimes z) \oplus (y \otimes z)$, where $y$ and $z$ are constants, $\oplus$
corresponds to addition or subtraction, and $\otimes$ to multiplication or left
bitwise shifts.  Therefore,
%
\begin{center}
\Verb|##add-imm 1 X Y|\\
\Verb|##mul-imm 2 1 Z|
\end{center}
%
\noindent is converted into
%
\begin{center}
  \begin{minipage}{0.2\linewidth}
    \begin{factorcode*}{gobble=6,frame=none}
      ##mul-imm 3 X Y
      ##add-imm 2 3 (Y*Z)
    \end{factorcode*}
  \end{minipage}
\end{center}
\noindent Notice that a new intermediate virtual register, \Verb|3|, had to
be created.  However, if the product of \Verb|Y| and \Verb|Z| can be
computed at compile-time and fits in an immediate operand, then we save cycles
by using \Verb|##mul-imm| on a smaller number.

\begin{sloppypar}
The last few methods of \Verb|rewrite| provide some obvious simplifications.
\Verb|compiler.cfg.value-numbering.simd| performs some limited constant-folding
for vector operations.  \Verb|compiler.cfg.value-numbering.slots| propagates
\Verb|##add-imm| address calculation to \Verb|##slot|, \Verb|##set-slot|, and
\Verb|##write-barrier| instructions in a manner similar to
\Verb|compiler.cfg.value-numbering.alien|.  Finally,
\Verb|compiler.cfg.value-numbering.misc| provides a single method to rewrite
\Verb|##replace| into \Verb|##replace-imm| if possible.
\end{sloppypar}

\inputfig{lvn}

To finish the discussion of local value numbering and Factor's particular
implementation, we'll examine the example from \vref{fig:value-numbering} in
depth.  For convenience, the before/after snapshot of the \gls{CFG} is
reproduced in \vref{fig:lvn}.

\Verb|value-numbering-step| begins at block $1$, where
\Verb|process-instruction| is \factor|map|ped across the instructions.
%
\Verb|##inc-d 3|
%
does not have a \Verb|rewrite| method, so remains untouched; it is also not a
\Verb|foldable-insn|, so it is simply returned.  While
%
\Verb|##load-integer 21 0|
%
doesn't have a \Verb|rewrite| method, it is a \Verb|foldable-insn|, so
\Verb|process-instruction| calls \Verb|check-redundancy|.  At this point,
the expression graph is empty.  Calling \Verb|>expr| converts this
instruction into an \Verb|integer-expr| object representing \Verb|0|.
\Verb|useful-instruction| leaves the tables as follows:
%
  \begin{factorcode}
    ! vregs>vns
    H{ { 21 21 } }

    ! exprs>vns
    H{ { T{ integer-expr { value 0 } } 21 } }

    ! vns>insns
    H{
        { 21 T{ ##load-integer { dst 21 } { val 0 } { insn# 1 } } }
    }
  \end{factorcode}
%
\noindent The next instruction in block $1$,
%
\Verb|##load-integer 22 100|,
%
behaves similarly, leaving:
%
  \begin{factorcode}
    ! vregs>vns
    H{ { 21 21 } { 22 22 } }

    ! exprs>vns
    H{
        { T{ integer-expr { value 0 } } 21 }
        { T{ integer-expr { value 100 } } 22 }
    }

    ! vns>insns
    H{
        { 21 T{ ##load-integer { dst 21 } { val 0 } { insn# 1 } } }
        {
            22
            T{ ##load-integer { dst 22 } { val 100 } { insn# 2 } }
        }
    }
  \end{factorcode}
%
\noindent The following instruction is
%
\Verb|##load-integer 23 0|.
%
In calling \Verb|check-redundancy|, we discover that the integer expression
for \Verb|0| is already in \Verb|exprs>vns|, so this is turned into a
\Verb|##copy|, and the value number is noted.  The remaining instructions in
block $1$ (aside from \Verb|##branch|) are all instances of \Verb|##copy|.
\Verb|process-instruction| thus only sets their value numbers in the
\Verb|vregs>vns| table, leaving them with the following at the end of block
$1$:
%
  \begin{factorcode}
    ! vregs>vns
    H{
        { 21 21 }
        { 22 22 }
        { 23 21 }
        { 24 22 }
        { 25 21 }
        { 26 22 }
        { 27 21 }
    }

    ! exprs>vns
    H{
        { T{ integer-expr { value 0 } } 21 }
        { T{ integer-expr { value 100 } } 22 }
    }

    ! vns>insns
    H{
        { 21 T{ ##load-integer { dst 21 } { val 0 } { insn# 1 } } }
        {
            22
            T{ ##load-integer { dst 22 } { val 100 } { insn# 2 } }
        }
    }
  \end{factorcode}

Next, block $2$ in \vref{fig:lvn} is processed.  The tables are all reset, so
even though block $1$ happens to dominate block $2$, none of its definitions
are known to \Verb|value-numbering|.  The \Verb|##phi|s are ignored, as no
important methods dispatch upon them.  In trying to rewrite the
\Verb|##compare-integer|, we call \Verb|vreg>vn| on the operands.  Since they
aren't in the \Verb|vregs>vns| table yet, they are assumed to be unique values.
This assumption is pessimistic---we'd rather the values be the same, so we can
remove redundancy.  It happens to be correct here, though, as \Verb|26|
corresponds to the integer \Verb|100|, while \Verb|30| is an induction variable
of the loop.  However, \Verb|##compare-integer| cannot be rewritten into an
immediate form, since our focus is local to the basic block, so we don't know
that \Verb|26| has the value \Verb|100|.  The \Verb|##copy| instructions are
processed as usual, and
%
\Verb|##compare-imm-branch 32 f cc/=|
%
is rewritten into a \Verb|##compare-integer-branch|, as the virtual register
\Verb|32| has the same value (through the copies) as the
\Verb|##compare-integer| result.  This is a case of simplifying the 
%
\mint{java}|if (some_boolean != false) { ... }|
%
pattern, and the definition of the register \Verb|31| becomes dead code after
\Verb|rewrite| finishes with this last instruction.  The expression graph is
populated thus by the end:
%
  \begin{factorcode}
    ! vregs>vns
    H{
        { 32 31 }
        { 33 26 }
        { 34 31 }
        { 26 26 }
        { 30 30 }
        { 31 31 }
    }

    ! exprs>vns
    H{ { { ##compare-integer 30 26 cc< } 31 } }

    ! vns>insns
    H{
        {
            31
            T{ ##compare-integer
                { dst 31 }
                { src1 30 }
                { src2 22 }
                { cc cc< }
                { temp 9 }
                { insn# 2 }
            }
        }
    }
  \end{factorcode}

Once again, the tables are reset and we proceed to block $3$.  The first
instruction,
%
\Verb|##load-integer 35 1|,
%
is entered into the expression graph.  Since \Verb|35| is an operand of
%
\Verb|##add 36 29 35|,
%
\Verb|rewrite| changes this instruction into an \Verb|##add-imm|, as we
know the constant value of the operand.  The next \Verb|##load-integer| gets
turned into a \Verb|##copy|, like in block $1$, and the next \Verb|##add|
is similarly changed to \Verb|##add-imm|.  The copies do little but set more
value numbers.  As \Verb|process-instruction| calls \Verb|vreg>vn| on their
sources, we'll insert entries into \Verb|vregs>vns| for those defined outside
of the block, like \Verb|26|.  This leaves us with the following tables:
%
  \begin{factorcode}
    ! vregs>vns
    H{
        { 35 35 }
        { 36 36 }
        { 37 35 }
        { 38 38 }
        { 39 30 }
        { 40 26 }
        { 41 36 }
        { 26 26 }
        { 42 38 }
        { 29 29 }
        { 30 30 }
    }

    ! exprs>vns
    H{
        { { ##add-imm 30 1 } 38 }
        { { ##add-imm 29 1 } 36 }
        { T{ integer-expr { value 1 } } 35 }
    }

    ! vns>insns
    H{
        { 36 T{ ##add-imm { dst 36 } { src1 29 } { src2 1 } } }
        { 38 T{ ##add-imm { dst 38 } { src1 30 } { src2 1 } } }
        { 35 T{ ##load-integer { dst 35 } { val 1 } { insn# 0 } } }
    }
  \end{factorcode}
%
\noindent The fourth invocation of \Verb|value-numbering-step| does not do
anything interesting, as the \Verb|##replace| cannot be changed into a
\Verb|##replace-imm|.

\inputfig{finalize-lvn}

In summary, we managed to replace redundancies within basic blocks online by
maintaining some simple hash tables.  After copy propagation and dead code
elimination, the \gls{CFG} gets finalized to the one shown in
\vref{fig:finalize-lvn}.  Because the value numbering algorithm was local, the
\Verb|##compare-integer-branch| in block $2$ could not be simplified to a
\Verb|##compare-integer-imm-branch|, and we instead have to waste a register
on the integer \Verb|100|.  But it's important to note that even considering
a topological ordering of the \gls{CFG} wouldn't have worked, as we'd have to
ignore back-edges.  The \Verb|##phi|s that used to be in block $2$ had inputs
that flowed along the back-edge, and our pessimistic assumption would have to
classify these values as distinct.  One is for the counter introduced by
\Verb|times|, and the other is from the top value of the stack being
incremented by \Verb|fixnum+fast|.  In this case, however, these induction
variables are actually equal: both start at \Verb|0| and are incremented by
\Verb|1| on each loop.  In terms of the \gls{CFG} in \vref{fig:finalize-lvn},
the \Verb|EAX| and \Verb|EDX| registers are equivalent.  Yet the
combination of the pessimism and locality of the algorithm keep us from
discovering this.

\subsection{Global Value Numbering}\label{sec:vn:global}

Answering the challenges of Cocke\todo{cite-like}, AWZ~\todo{cite-like}
described what would be the de facto value numbering algorithm for several
years, and rightly so.  It was a properly \term{global} value numbering
algorithm, working across an entire \gls{CFG} instead of on single basic
blocks.  Their paper was important in another very relevant way: it is the
first published reference to SSA form\todo{cite VanDrunen}, including an
algorithm for its construction.

Though we could try to extend the scope of Factor's local value numbering, it
is still inherently pessimistic.  The algorithm of AWZ\todo{cite-like}, which
is commonly referred to simply as AWZ\todo{gls?}, uses a modification of
Hopcroft's~\todo{cite-like} minimization algorithm for finite state automata.
It works on an \term{optimistic} assumption by first assuming every value has
the same value number, then trying to prove that values are actually different.
It does this by treating value numbers as \term{congruence classes} that
partition the set of virtual registers.  If two values are in the same class,
then they are congruent, where congruence is defined as in \cref{sec:vn:local}.

Such a partition is not unique, in general.  For instance, a trivial one places
each value in its own congruence class.  So, we look for the \term{maximal
fixed point}---the solution that has the most congruent values and therefore
the fewest congruence classes.  We must start with a congruence class for each
operation so that, say, all values computed by \factor|##add|s are grouped
together, those computed by \factor|##mul|s are in the same class, etc.  We
must then iteratively look at our collection of classes, separating them when
we discover incongruent values.  For an \gls{SSA} variable in class $P$, we
look at its defining expression.  If an operand at position $i$ belongs to
class $Q$, then the $i^\text{th}$ operand of every other value in $P$ should
also be in $Q$.  Otherwise, $P$ must be \term{split} by removing those
variables whose $i^\text{th}$ operands are not in class $Q$ and placing them in
a new congruence class.  We keep splitting classes until the partitioning
stabilizes.

The optimistic assumption may seem dangerous.  Is it possible that we're
``overoptimistic''?  That two values assumed to be congruent and not proven
incongruent might actually be inequivalent when the program is run?  The
AWZ~\todo{gls?} paper dedicates a section to proving that two congruent
variables are equivalent at a point $p$ in the program if their definitions
dominate $p$.  The proof is a bit quirky, but reasonable.  They develop a
dynamic notion of dominance in a running program which implies static dominance
in the code, then show that congruence implies runtime equality (though
equivalence does not imply congruence).

AWZ\todo{gls?}~ made the need for \gls{GVN} algorithms apparent.  However,
finite state automata minimization makes for a more complicated algorithm than
hash-based value numbering.  A na\"{i}ve implementation can be quadratic,
although careful data structure and procedure design can make it $O(n\log n)$.
Furthermore, it's resistant to the same improvements we easily added to the
local value numbering.  To even consider the commutativity of operations
requires changes in operand position tracking and splitting---the heart of the
algorithm.  It is generally limited by what the programmer writes down: deeper
congruences due to, say, algebraic identities can't be discovered.

In fact, by performing an optimization that uses the \gls{GVN} information,
more \gls{GVN} congruences may arise.  If we can somehow perform the two
analyses simultaneously, they'll produce better results.  This generalizes to
interdependent compiler optimizations at large, as elucidated in
Click's\todo{cite-like}~ dissertation, which describes a method for formalizing
and combining separate optimizations that make optimistic assumptions (whatever
they happen to be for each particular analysis).  He uses this to merge
\gls{GVN} with \term{conditional constant propagation}, which itself is a
combination of constant propagation and unreachable code elimination (pretty
much like the \factor|propagate| pass from \cref{sec:compiler:tree}).
Furthermore, \gls{GVN} is extended to handle algebraic identities, propagate
constants, and fold redundant $\phi$s.  Unfortunately, the straightforward
algorithm for this is $O(n^2)$, while the $O(n\log n)$ version presented is not
just complicated, but can also miss some congruences between
$\phi$-functions\todo{cite Click, Simpson}.

Hot on the heels of this work, Simpson's\todo{cite-like}~ dissertation provides
probably the most exhaustive treatment of \gls{GVN} algorithms.  He presents
several extensions, such as incorporating hash-based local value numbering into
\gls{SSA} construction, handling commutativity in AWZ\todo{gls?}~ \gls{GVN},
and performing redundant store elimination.  He builds off of the two classical
algorithms independently, which underlines their inherent differences and
limitations.  In general, hash-based value numbering is easy to extend without
greatly impacting the runtime complexity, as is the case in Factor's
implementation.

Drawing from this experience, Simpson's hallmark algorithm combines the best of
both worlds by taking the hash-based algorithm which is easy to understand,
implement, and extend, and making it global, so it identifies more congruences.
Dubbed the ``\gls{RPO} algorithm'', it simply applies hash-based value
numbering iteratively over the \gls{CFG} until we reach the same fixed point
computed by AWZ\todo{gls?}.  (The fact that it computes the \emph{same} fixed
point is proven fairly straightforwardly in the dissertation.)  It could
technically traverse the \gls{CFG} in any topological order, but Simpson
defaults to reverse postorder.

Because it is based off the hashing algorithm, we get the benefits essentially
for free.  The same simplifications can be performed, but with the added
knowledge of global congruences.  Since the majority of Factor's value
numbering code is dedicated to the \factor|rewrite| generic, it makes sense to
reuse as much of that code as possible.  Therefore, to convert Factor's local
algorithm to a global one, I modified the existing code to use the \gls{RPO}
algorithm.

\inputlst{gvn-graph}

The most fundamental change is to the expression graph.  Referring to
\vref{lst:gvn-graph}, we see most of the same code as in
\vref{lst:value-numbering-graph}, with changes indicated by arrows
($\longrightarrow$).  Two more global variables have been added, namely
\factor|changed?| and \factor|final-iteration?|.  The former is what we use to
guide the fixed-point iteration.  As long as value numbers are changing, we
keep iterating.  An important side effect of this is that we can no longer
perform \factor|rewrite| online, since the transformations we make aren't
guaranteed to be sound on any iteration except the final one.  This makes the
\gls{RPO} algorithm work \term{offline}, first discovering redundancies, then
eliminating them in a separate pass.  When this elimination pass starts, we'll
set \factor|final-iteration?| to \factor|t|.

A key change is in the \factor|vreg>vn| word, which now makes an optimistic
assumption about previously unseen values.  Given a new virtual register that
wasn't in the \factor|vregs>vns| table, the old version would map the register
to itself, making the value its own canonical representative.  However, if this
version tries to look up a key that does not exist in the hash table, it will
simply return \factor|f| (which Factor will do by default with the \factor|at|
word).  Therefore, every value in the \gls{CFG} starts off with the same value
``number'', \factor|f|.  By the end of the \gls{GVN} pass, there should be no
value left that hasn't been put in the \factor|vregs>vns| table, as we'll have
processed every definition.

To keep track of whether \factor|vregs>vns| changes, we simply need to alter
\factor|set-vn|.  Here, we use \factor|maybe-set-at|, a utility from the
\factor|assocs| vocabulary.  This works like \factor|set-at|, establishing a
mapping in the hash table.  In addition, it returns a boolean indicating
change: if a new key has been added to the table, we return \factor|t|.
Otherwise, we return \factor|t| only in the case where an old key is mapped to
a new value.  If an old key is mapped to the same value that's already in the
table, \factor|maybe-set-at| returns \factor|f|.  Therefore, when
\factor|vregs>vns| does change, we set \factor|changed?| to \factor|t| (which
is what the \factor|on| word does).

Finally, we define a new utility word, \factor|clear-exprs|, which resets the
\factor|exprs>vns| and \factor|vns>insns| tables.  Unlike the local value
numbering phase, we don't reset the entire expression graph.  Instead, we make
a pass over the whole \gls{CFG} at a time.  The only reason optimism works is
that we keep trying to disprove our foolhardy assumptions.  Really,
\factor|vregs>vns| establishes congruence classes of value numbers.  At first,
every value belongs in one class, \factor|f|.  We make a pass over the
\gls{CFG} to disprove whatever we can about this.  If we've introduced new
congruence classes (new values in the \factor|vregs>vns| hash), we do another
iteration.  But each time, we use the congruence classes discovered from the
previous iteration.  At the start of each new pass, the expressions and
instructions in \factor|exprs>vns| and \factor|vns>insns| are
invalidated---their results are based on old information.  So, these are erased
on each iteration.  Much like AWZ\todo{gls?}, we keep splitting classes until
they can't be split anymore.

\inputlst{gvn-step}

This logic is captured in \vref{lst:gvn-step}.  Rather than reset the tables
when we start processing each basic block in \factor|value-numbering-step| like
before, we call \factor|clear-exprs| on each iteration over the \gls{CFG} in
\factor|value-numbering-iteration|.  Note that \factor|value-numbering-step| no
longer returns the changed instructions, as we aren't replacing them online.
\factor|value-numbering-iteration| uses \factor|simple-analysis| instead of
\factor|simple-optimization|, which only expects global state to change---no
instructions are updated in the block.  Much to our advantage,
\factor|simple-analysis| already traverses the \gls{CFG} in \acrlong{RPO}, so
we needn't worry about traversal order.  The top-level word
\factor|determine-value-numbers| ties this all together by calling
\factor|value-numbering-iteration| until we can get through it with
\factor|changed?| remaining false.

\inputlst{gvn-simplify}
\inputlst{gvn-value-number}

Note that the work of \factor|value-numbering-step| is further divided into two
words, \factor|simplify| and \factor|value-number|.  These combine to do much
the same work as \factor|process-instruction| in
\vref{lst:process-instruction}.  \factor|simplify| makes the repeated calls to
\factor|rewrite| until the instruction cannot be simplified further.  Its
definition is in \vref{lst:gvn-simplify}.  We then pass the simplified
instruction to \factor|value-number|, which is defined in
\vref{lst:gvn-value-number}.  This also has a similar structure to
\factor|process-instruction|.  The main difference is that instructions are no
longer returned (again, they aren't altered in place).  So, the \factor|array|
method uses \factor|each| instead of \factor|map| to recurse into the results
of \factor|rewrite|.

A subtle change is necessary with the \factor|alien-call-insn| and
\factor|##callback-inputs| methods.  Whereas \factor|process-instruction|
merely skipped over certain instructions that could not be rewritten, here we
don't have that luxury.  We need to be careful to \factor|set-vn| every virtual
register that gets defined by any instruction.  While making a pessimistic
assumption, it didn't matter if we did this: any unseen value would be presumed
important by \factor|vreg>vn|.  However, with the optimistic assumption,
\factor|vreg>vn| will give the impression that unseen values are all the same
by returning \factor|f|.  Therefore, we simply record the virtual registers
defined in instructions that may define one or more of them.  Specifically,
\factor|alien-call-insn| and \factor|##callback-inputs| are classes that
correspond to \gls{FFI} instructions.

The \factor|##copy| method uses \factor|set-vn| the same way as before.
\factor|redundant-instruction|, \factor|useful-instruction|, and
\factor|check-redundancy| are also largely the same.  These have just been
tweaked to not return instructions.

\inputlst{phi-expr}

The \factor|##phi| method in \vref{lst:gvn-value-number} represents a major
change. Before, \factor|##phi|s were left uninterpreted.  Congruences between
induction variables that flowed along back-edges would not be identifiable.
But now, by checking for redundant \factor|##phi|s, we may reduce them to
copies.  Each \factor|##phi| object has an \factor|inputs| slot, which is a
hash table from basic block to the virtual register that flows from that block.
Thus, there is one input for each predecessor.  The \factor|values| of the hash
will be the virtual registers that might be selected for the \factor|dst|
value.  We look up the value numbers of these, removing all instances of
\factor|f| with the \factor|sift| word.  If all of the inputs are congruent, we
can call \factor|redundant-instruction|, setting the value number of the
\factor|##phi|'s \factor|dst| to the value number of its first input (without
loss of generality).  The \factor|all-equal?| word will return \factor|t| if
the sequence is empty (as it's vacuously true), so we must make sure not to
call \factor|first| on the sequence, since this will be a runtime error.  If
the sequence is empty, we needn't note the redundancy, as the \factor|##phi|'s
\factor|dst| will already have the optimistic value number \factor|f| anyway.
Otherwise, we call \factor|check-redundancy|.  The purpose of this is to
identify \factor|##phi|s that are equal to each other.  Even if its inputs are
incongruent, a \factor|##phi| might still represent a copy of another induction
variable.  So that \factor|check-redundancy| works, we also define a
\factor|>expr| method in \factor|compiler.cfg.gvn.expressions|, as seen in
\vref{lst:phi-expr}.  Here, the expression is an array consisting of the
\factor|##phi| class word, the current basic block's number, and the inputs'
value numbers.  We include the basic block number because only \factor|##phi|s
within the same block can be considered equivalent to each other.

The final method in \vref{lst:gvn-value-number} defines the default behavior
for \factor|value-number|, which calls \factor|check-redundancy| on the
simplified instruction if it defines a single virtual register.  Note that we
separate the \factor|alien-call-insn| and \factor|##callback-inputs| logic from
this, since they happen to define a variable number of registers.  If
particular instances define only one register, we still don't want to call
\factor|check-redundancy| on them, since they don't have a \factor|dst| slot.
To avoid calling \factor|dst>>| and triggering an error in
\factor|useful-instruction|, we needed separate methods for the \gls{FFI}
classes.

\inputfig{gvn}

With these changes, we can globally identify value numbers, including
equivalences that arise from simplifying instructions (even though no
replacements are actually done yet).  To illustrate this, consider again the
example
%
\factor|0 100 [ 1 fixnum+fast ] times|,
%
reproduced in \vref{fig:gvn}.  As the expression graph changes frequently in
this new algorithm, instead of showing the literal hash tables we'll use a
shorthand notation.  Virtual registers will be integers, and to avoid confusion
value numbers will be written in brackets, like \vn{n}.  Then, we'll show
\factor|vreg>vn| mappings with the notation $n\to\vn{n}$, where $n$ is the
register and \vn{n} is the value number.  If there is a corresponding
expression in \factor|exprs>vns|, it will be denoted after the mapping, like
$n\to\vn{n}~(\textit{expression})$.  With the expressions, the instructions in
\factor|vns>insns| are a bit redundant for understanding the value numbering
process, so they will be elided.  Any mappings to \factor|f| will be elided, as
they're understood to be implicit when a key is absent.

\todo[inline]{Might make separate figures of each block, for easier reference}

\factor|determine-value-numbers| starts the first iteration, which of course
starts at basic block $1$.  \factor|##inc-d| is a no-op, but the first two
\factor|##load-integer|s are established as useful instructions.
%
\factor|##load-integer 23 0|
%
is recognized as redundant, since at this point we know that \factor|21| has
the value \factor|0|.  The \factor|##copy| instructions all pile on value
number mappings, leaving us with the following:
%
\begin{align*}
  21 &\to \vn{21} \quad (0)  \\
  22 &\to \vn{22} \quad (100)\\
  23 &\to \vn{21}            \\
  24 &\to \vn{22}            \\
  25 &\to \vn{21}            \\
  26 &\to \vn{22}            \\
  27 &\to \vn{21}
\end{align*}

At iteration $1$, basic block $2$, the first \factor|##phi| has inputs
\factor|25| (from block $1$) and \factor|41| (from block $3$).  The former has
the value number \vn{21}, while the latter is still at \factor|f|.  We treat
this value number much like a $\top$ element, unifying it with the other input
to give us the assumption that \factor|29| will be a copy of \factor|25|.
Thus, it gets the same value number.  A similar choice happens for the second
\factor|##phi|.  The instruction 
%
\factor|##compare-integer 31 30 26 cc< 9|
%
is an interesting case.  Due to our optimistic assumptions thus far, we believe
\factor|30| is carrying the value \factor|0|, and that \factor|26| is set to
\factor|100|.  Thus, this instruction gets constant-folded by \factor|simplify|
into
%
\factor|##load-reference 31 t|.
%
The \gls{CFG} isn't changed, but the expression graph reflects this belief.
Later, this assumption will be invalidated.  The following copies are processed
as usual, with the distinct difference here that 
%
\factor|##copy 33 26 any-rep|
%
has the global knowledge of the value number of \factor|26|.  Because the
\factor|##compare-integer| was constant-folded, so is the
\factor|##compare-imm-branch|---and to the same value, no less.  This leaves us
with:
%
\begin{align*}
  21 &\to \vn{21} \quad (0)                 \\
  22 &\to \vn{22} \quad (100)               \\
  23 &\to \vn{21}                           \\
  24 &\to \vn{22}                           \\
  25 &\to \vn{21}                           \\
  26 &\to \vn{22}                           \\
  27 &\to \vn{21}                           \\
  29 &\to \vn{21}                           \\
  30 &\to \vn{21}                           \\
  31 &\to \vn{31} \quad (\text{\factor|t|}) \\
  32 &\to \vn{31}                           \\
  33 &\to \vn{22}                           \\
  34 &\to \vn{31}
\end{align*}

Block $3$ of iteration $1$ gives the \factor|##load-integer|s' destinations the
same value number, corresponding to the integer $1$.  Because optimism makes
the algorithm think that \factor|29| and \factor|30| correspond to the integer
$0$, the \factor|##add|s are constant-folded.  This leaves us with:
%
\begin{align*}
  21 &\to \vn{21} \quad (0)                 \\
  22 &\to \vn{22} \quad (100)               \\
  23 &\to \vn{21}                           \\
  24 &\to \vn{22}                           \\
  25 &\to \vn{21}                           \\
  26 &\to \vn{22}                           \\
  27 &\to \vn{21}                           \\
  29 &\to \vn{21}                           \\
  30 &\to \vn{21}                           \\
  31 &\to \vn{31} \quad (\text{\factor|t|}) \\
  32 &\to \vn{31}                           \\
  33 &\to \vn{22}                           \\
  34 &\to \vn{31}                           \\
  35 &\to \vn{35} \quad (1)                 \\
  36 &\to \vn{35}                           \\
  37 &\to \vn{35}                           \\
  38 &\to \vn{35}                           \\
  39 &\to \vn{21}                           \\
  40 &\to \vn{22}                           \\
  41 &\to \vn{35}                           \\
  42 &\to \vn{35}
\end{align*}

While block $4$ is visited in each iteration, it doesn't define any registers,
so doesn't affect the state of value numbering.  Therefore, the above is the
state left at the end of iteration $1$.

Since \factor|vregs>vns| clearly changed, iteration $2$ commences by clearing
the expressions, though the value numbers remain.  Block $1$ doesn't change
from iteration $1$, giving us:
%
\begin{align*}
  21 &\to \vn{21} \quad (0)                 \\
  22 &\to \vn{22} \quad (100)               \\
  23 &\to \vn{21}                           \\
  24 &\to \vn{22}                           \\
  25 &\to \vn{21}                           \\
  26 &\to \vn{22}                           \\
  27 &\to \vn{21}                           \\
  29 &\to \vn{21}                           \\
  30 &\to \vn{21}                           \\
  31 &\to \vn{31}                           \\
  32 &\to \vn{31}                           \\
  33 &\to \vn{22}                           \\
  34 &\to \vn{31}                           \\
  35 &\to \vn{35}                           \\
  36 &\to \vn{35}                           \\
  37 &\to \vn{35}                           \\
  38 &\to \vn{35}                           \\
  39 &\to \vn{21}                           \\
  40 &\to \vn{22}                           \\
  41 &\to \vn{35}                           \\
  42 &\to \vn{35}
\end{align*}

Now that we're in iteration $2$, the inputs to the \factor|##phi|s of block $2$
have been processed once before.  For instance, we still believe that
\factor|25| corresponds to the integer $0$ (which is incidentally correct), but
now that \factor|41| has the value number \vn{35}, we think it corresponds to
the integer $1$.  While this is incorrect, it does break the congruence between
the inputs, making the first \factor|##phi| a useful instruction.  The second
\factor|##phi|, however, still looks like a copy of the first.  Even so, this
is sufficiently different that the following \factor|##compare-integer| cannot
be constant-folded like before.  However, it can still be converted to a
\factor|##compare-integer-imm|, as one of its operands corresponds to an
integer.  The redundant \factor|##compare-imm-branch| gets rewritten to the
same expression as the \factor|##compare-integer|, so winds up getting the same
value number.  This gives us:
%
\begin{align*}
  21 &\to \vn{21} \quad (0)                                                \\
  22 &\to \vn{22} \quad (100)                                              \\
  23 &\to \vn{21}                                                          \\
  24 &\to \vn{22}                                                          \\
  25 &\to \vn{21}                                                          \\
  26 &\to \vn{22}                                                          \\
  27 &\to \vn{21}                                                          \\
  29 &\to \vn{29} \quad (\text{\factor|##phi 2 21 35|})                    \\
  30 &\to \vn{29}                                                          \\
  31 &\to \vn{31} \quad (\text{\factor|##compare-integer-imm 29 100 cc<|}) \\
  32 &\to \vn{31}                                                          \\
  33 &\to \vn{22}                                                          \\
  34 &\to \vn{31}                                                          \\
  35 &\to \vn{35}                                                          \\
  36 &\to \vn{35}                                                          \\
  37 &\to \vn{35}                                                          \\
  38 &\to \vn{35}                                                          \\
  39 &\to \vn{21}                                                          \\
  40 &\to \vn{22}                                                          \\
  41 &\to \vn{35}                                                          \\
  42 &\to \vn{35}
\end{align*}

Block $3$ of iteration $2$ also changes, since the \factor|##add|s can't be
constant-folded like before due to our new discovery about the \factor|##phi|s.
However, the first one can still be converted to an \factor|##add-imm|, and the
second is marked the same as the first.  This leaves the following value
numbers:
%
\begin{align*}
  21 &\to \vn{21} \quad (0)                                                \\
  22 &\to \vn{22} \quad (100)                                              \\
  23 &\to \vn{21}                                                          \\
  24 &\to \vn{22}                                                          \\
  25 &\to \vn{21}                                                          \\
  26 &\to \vn{22}                                                          \\
  27 &\to \vn{21}                                                          \\
  29 &\to \vn{29} \quad (\text{\factor|##phi 2 21 35|})                    \\
  30 &\to \vn{29}                                                          \\
  31 &\to \vn{31} \quad (\text{\factor|##compare-integer-imm 29 100 cc<|}) \\
  32 &\to \vn{31}                                                          \\
  33 &\to \vn{22}                                                          \\
  34 &\to \vn{31}                                                          \\
  35 &\to \vn{35} \quad (1)                                                \\
  36 &\to \vn{36} \quad (\text{\factor|##add-imm 29 1|})                   \\
  37 &\to \vn{35}                                                          \\
  38 &\to \vn{36}                                                          \\
  39 &\to \vn{29}                                                          \\
  40 &\to \vn{22}                                                          \\
  41 &\to \vn{36}                                                          \\
  42 &\to \vn{36}
\end{align*}

Since the value numbers changed, we start iteration $3$.  The expressions are
cleared, and block $1$ once again does not change anything.  The first
\factor|##phi| in block $2$ still gets classified as useful, so no value
numbers change.  The major difference, though, is that the previous iteration's
value numbers for registers in block $3$ update the expression we have for the
\factor|##phi|.  Whereas before we thought it was choosing between \vn{21} (the
integer $0$) and \vn{35} (the integer $1$), the \factor|##add| wasn't
constant-folded in the previous iteration.  Therefore, the virtual register
\factor|41| now corresponds to the result of the \factor|##add| with the value
number \vn{36}.  We still can't disprove that the second \factor|##phi| is
different (because it, in fact, isn't).  So, we're left with the following
after iteration $3$ finishes with block $2$:
%
\begin{align*}
  21 &\to \vn{21} \quad (0)                                                \\
  22 &\to \vn{22} \quad (100)                                              \\
  23 &\to \vn{21}                                                          \\
  24 &\to \vn{22}                                                          \\
  25 &\to \vn{21}                                                          \\
  26 &\to \vn{22}                                                          \\
  27 &\to \vn{21}                                                          \\
  29 &\to \vn{29} \quad (\text{\factor|##phi 2 21 36|})                    \\
  30 &\to \vn{29}                                                          \\
  31 &\to \vn{31} \quad (\text{\factor|##compare-integer-imm 29 100 cc<|}) \\
  32 &\to \vn{31}                                                          \\
  33 &\to \vn{22}                                                          \\
  34 &\to \vn{31}                                                          \\
  35 &\to \vn{35}                                                          \\
  36 &\to \vn{36}                                                          \\
  37 &\to \vn{35}                                                          \\
  38 &\to \vn{36}                                                          \\
  39 &\to \vn{29}                                                          \\
  40 &\to \vn{22}                                                          \\
  41 &\to \vn{36}                                                          \\
  42 &\to \vn{36}
\end{align*}

Blocks $3$ and $4$ do not produce any more changes, so \gls{GVN} has stabilized
after $3$ iterations, with our final congruence classes being:
%
\begin{align*}
  \vn{21} &= \{21, 23, 25, 27\}     \\
  \vn{22} &= \{22, 24, 26, 33, 40\} \\
  \vn{29} &= \{29, 30, 39\}         \\
  \vn{31} &= \{31, 32, 34\}         \\
  \vn{35} &= \{35, 37\}             \\
  \vn{36} &= \{36, 38, 41, 42\}
\end{align*}

\todo[inline]{teletype the numbers in the align*s, I guess}

\subsection{Redundancy Elimination}\label{sec:vn:avail}

Now that we've identified congruences across the entire \gls{CFG}, we must
eliminate any redundancies found.  Since value numbering is now offline, this
entails another pass.  However, replacing instructions is more subtle with
global value numbers than it is with local ones.  Because values come from all
over the \gls{CFG}, we must consider if a definition is \term{available} at the
point where we want to use it.  

\inputfig{not-avail}
\inputfig{is-avail}

\Vref{fig:not-avail,fig:is-avail} show the difference.  In the former, we can
see the \gls{CFG} before value numbering for the code
%
\factor|[ 10 ] [ 20 ] if 10 20 30|.
%
The two extra integers being pushed at the end are there to avoid branch
splitting (see \vref{sec:compiler:cfg}).  In block $4$, there's a
%
\Verb|##load-integer 27 10|,
%
which loads the value \factor|10|.  In globally numbering values, we associate
the
%
\Verb|##load-integer 22 10|
%
in block $2$ with the value \factor|10| first, making it the canonical
representative.  However, we can't replace the instruction in block $4$ with
%
\Verb|##copy 27 22|,
%
because control flow doesn't necessarily go through block $2$, so the virtual
register \factor|22| might not even be defined.  However, in
\vref{fig:is-avail}, we see the \gls{CFG} for the code
%
\factor|10 swap [ 10 ] [ 20 ] if 10 20 30|.
%
In this case, the first definition of the value \factor|10| comes from block
$1$, which dominates block $4$.  So, the definition is available, and we can
replace the \Verb|##load-integer| in block $4$ with a \Verb|##copy|.

There are several ways to decide if we can use a definition at a certain point.
For instance, we could use dominator information, so that if a definition in a
basic block $B$ can be used by any basic block dominated by $B$\todo{cite
Simpson}.  However, here we'll use a data flow analysis called \term{available
expression analysis}, since it was readily implemented.  Mercifully, Factor has
a vocabulary that automatically defines data flow analyses with little more
than a single line of code.

\inputlst{avail}

\Vref{lst:avail} shows the vocabulary that defines the available expression
analysis.  It is a forward analysis\todo{cite?}~ based on the flow equations
below:
\begin{align*}
  \text{\factor|avail-in|}_i &=
    \begin{cases}
      \varnothing
        & \text{if $i=0$} \\
      \bigcap_{j\in\mathrm{pred}(i)}\text{\factor|avail-out|}_j
        & \text{if $i>0$}
    \end{cases} \\
  \text{\factor|avail-out|}_i &= \text{\factor|avail-in|}_i
                                 \cup 
                                 \text{\factor|defined|}_i
\end{align*}
%
\noindent Here, the subscripts indicate the basic block number.
$\text{\factor|defined|}_i$ denotes the result of the \factor|defined| word
from \vref{lst:avail}.  This returns the set of virtual registers defined in a
basic block.

\section{Results}\label{sec:vn:results}

The goal of improving the optimization in Factor is, of course, to reduce the
average running time of programs, and to do so without changing their
semantics.  Short of formal verification, the latter requirement makes it
necessary to thoroughly test any code that gets compiled with the new pass
enabled.  To this end, we'll employ Factor's extensive unit test coverage.
While some vocabularies will have more tests than others, the total number of
unit tests is quite large.  By compiling each vocabulary and running their
tests, we're indirectly testing the compiler: if tests that used to pass no
longer do, then the new pass is changing the semantics of the code somehow.
Though passing all tests does not guarantee the algorithm is correct, it does
let us know that no known regressions have been introduced.  Happily, with the
new \Verb|value-numbering| phase enabled, all the same tests pass as before in
a call to \factor|test-all| from a freshly bootstrapped image.

The efficacy of the changes, on the other hand, must be measured relative to
old benchmarks.  Again, Factor has its bases covered, with a suite of $80$
benchmarks run by the \Verb|benchmark| vocabulary.  Each benchmark is run $5$
times, and the garbage collector is run before each iteration.  The minimum
time from these runs is then used as the benchmark result.  The informal data
below comes from two separate runs on my own personal computer of the
\factor|benchmarks| word, which invokes all the \Verb|benchmark|
sub-vocabularies.  The ``before'' time used the local value numbering, while
``after'' times had \Verb|value-numbering| replaced with the \gls{GVN} pass.
The ``change'' is measured by the formula
%
$$\frac{\text{before} - \text{after}}{\text{before}} \times 100$$
%
to indicate the relative running times.  Negative values in this column are
good, as that means the running time has decreased.

\begin{longtable}{llll}
\toprule
Benchmark & Before (seconds) & After (seconds) & Change (\%) \\
\midrule
\endhead
\texttt{3d-matrix-scalar}         & 3.705816738       & 3.046126696         & $-17.80$    \\
\texttt{3d-matrix-vector}         & 0.161298778       & 0.089539887         & $-44.49$    \\
\texttt{backtrack}                & 4.280001561       & 2.358672591         & $-44.89$    \\
\texttt{base64}                   & 5.127831493       & 2.853612485         & $-44.35$    \\
\texttt{beust1}                   & 7.531546384       & 4.604929188         & $-38.86$    \\
\texttt{beust2}                   & 20.308680548      & 12.843534349        & $-36.76$    \\
\texttt{binary-search}            & 3.729776895       & 2.349520427         & $-37.01$    \\
\texttt{binary-trees}             & 9.403166818       & 6.518867479         & $-30.67$    \\
\texttt{bootstrap1}               & 32.472196349      & 30.887877896        & $-4.88$     \\
\texttt{chameneos-redux}          & 2.923900422       & 2.041007328         & $-30.20$    \\
\texttt{continuations}            & 0.273525202       & 0.200695972         & $-26.63$    \\
\texttt{crc32}                    & 0.010623653       & 0.005282642         & $-50.27$    \\
\texttt{dawes}                    & 1.588111926       & 1.027176578         & $-35.32$    \\
\texttt{dispatch1}                & 7.640720326       & 5.106558985         & $-33.17$    \\
\texttt{dispatch2}                & 5.221652668       & 3.984754032         & $-23.69$    \\
\texttt{dispatch3}                & 9.710520454       & 6.203527737         & $-36.12$    \\
\texttt{dispatch4}                & 8.224931156       & 4.098265543         & $-50.17$    \\
\texttt{dispatch5}                & 4.74357434        & 3.478219608         & $-26.68$    \\
\texttt{e-decimals}               & 3.903754723       & 2.646958072         & $-32.19$    \\
\texttt{e-ratios}                 & 4.774454589       & 3.658075473         & $-23.38$    \\
\texttt{empty-loop-0}             & 0.251816164       & 0.199189271         & $-20.90$    \\
\texttt{empty-loop-1}             & 1.039242509       & 0.857588545         & $-17.48$    \\
\texttt{empty-loop-2}             & 0.472215346       & 0.387974286         & $-17.84$    \\
\texttt{euler150}                 & 37.785852299      & 27.05450689         & $-28.40$    \\
\texttt{fannkuch}                 & 9.627490235       & 6.8970571           & $-28.36$    \\
\texttt{fasta}                    & 7.25292282        & 5.640517069         & $-22.23$    \\
\texttt{fib1}                     & 0.179389215       & 0.164933805         & $-8.06$     \\
\texttt{fib2}                     & 0.205853157       & 0.138174211         & $-32.88$    \\
\texttt{fib3}                     & 0.785036151       & 0.539739186         & $-31.25$    \\
\texttt{fib4}                     & 0.391805799       & 0.260370111         & $-33.55$    \\
\texttt{fib5}                     & 1.508625224       & 1.002724851         & $-33.53$    \\
\texttt{fib6}                     & 19.202504502      & 13.146010511        & $-31.54$    \\
\texttt{gc0}                      & 7.360087104       & 5.508594031         & $-25.16$    \\
\texttt{gc1}                      & 0.418173431       & 0.281497214         & $-32.68$    \\
\texttt{gc2}                      & 25.611210221      & 19.716168704        & $-23.02$    \\
\texttt{gc3}                      & 2.757943071       & 2.210785891         & $-19.84$    \\
\texttt{hashtables}               & 8.068216942       & 7.997106348         & $-0.88$     \\
\texttt{heaps}                    & 4.360368411       & 4.32169158          & $-0.89$     \\
\texttt{iteration}                & 7.875561986       & 6.277891729         & $-20.29$    \\
\texttt{javascript}               & 17.881224721      & 12.74204052         & $-28.74$    \\
\texttt{knucleotide}              & 5.490420772       & 3.5704101           & $-34.97$    \\
\texttt{mandel}                   & 0.251711276       & 0.198695557         & $-21.06$    \\
\texttt{matrix-exponential-scalar}& 16.451432774      & 12.017000042        & $-26.95$    \\
\texttt{matrix-exponential-simd}  & 0.681684747       & 0.536850343         & $-21.25$    \\
\texttt{md5}                      & 10.40516678       & 9.198666403         & $-11.60$    \\
\texttt{mt}                       & 33.91981743       & 29.961085146        & $-11.67$    \\
\texttt{nbody}                    & 9.203478441       & 6.795154145         & $-26.17$    \\
\texttt{nbody-simd}               & 0.845814208       & 0.854773096         & $+1.06$     \\
\texttt{nested-empty-loop-1}      & 0.097090973       & 0.068475608         & $-29.47$    \\
\texttt{nested-empty-loop-2}      & 0.893126911       & 0.861327078         & $-3.56$     \\
\texttt{nsieve}                   & 1.086110659       & 1.137648699         & $+4.75$     \\
\texttt{nsieve-bits}              & 2.707271763       & 2.815509077         & $+4.00$     \\
\texttt{nsieve-bytes}             & 0.785041878       & 1.211421146         & $+54.31$    \\
\texttt{partial-sums}             & 3.762171661       & 4.130144177         & $+9.78$     \\
\texttt{pidigits}                 & 2.182877913       & 2.195385034         & $+0.57$     \\
\texttt{random}                   & 5.66540782        & 5.71913683          & $+0.95$     \\
\texttt{raytracer}                & 5.047070171       & 4.39514879          & $-12.92$    \\
\texttt{raytracer-simd}           & 1.072588515       & 0.980927338         & $-8.55$     \\
\texttt{recursive}                & 2.703509403       & 2.529087637         & $-6.45$     \\
\texttt{regex-dna}                & 2.208584014       & 1.808859571         & $-18.10$    \\
\texttt{reverse-complement}       & 2.801163847       & 2.353254665         & $-15.99$    \\
\texttt{ring}                     & 1.822206473       & 1.62482491          & $-10.83$    \\
\texttt{sfmt}                     & 2.675838657       & 2.463367198         & $-7.94$     \\
\texttt{sha1}                     & 11.964973943      & 11.142380303        & $-6.88$     \\
\texttt{simd-1}                   & 1.857778672       & 1.703206011         & $-8.32$     \\
\texttt{sockets}                  & 10.636346636      & 10.516448454        & $-1.13$     \\
\texttt{sort}                     & 0.695635429       & 0.581855635         & $-16.36$    \\
\texttt{spectral-norm}            & 3.433630383       & 2.960833789         & $-13.77$    \\
\texttt{spectral-norm-simd}       & 2.743240011       & 3.237017655         & $+18.00$    \\
\texttt{stack}                    & 1.580016742       & 2.004478602         & $+26.86$    \\
\texttt{struct-arrays}            & 2.180774222       & 2.421915609         & $+11.06$    \\
\texttt{sum-file}                 & 0.883097981       & 0.957151577         & $+8.39$     \\
\texttt{terrain-generation}       & 1.611800222       & 1.887047663         & $+17.08$    \\
\texttt{tuple-arrays}             & 0.262747557       & 0.329399609         & $+25.37$    \\
\texttt{typecheck1}               & 1.750223408       & 1.674592158         & $-4.32$     \\
\texttt{typecheck2}               & 1.674738245       & 1.553203741         & $-7.26$     \\
\texttt{typecheck3}               & 1.891206648       & 1.735390184         & $-8.24$     \\
\texttt{ui-panes}                 & 0.305595039       & 0.29985214          & $-1.88$     \\
\texttt{xml}                      & 3.013709363       & 2.722223892         & $-9.67$     \\
\texttt{yuv-to-rgb}               & 0.398174487       & 0.318891664         & $-19.91$    \\
\end{longtable}

\begin{sloppypar}
These informal results are promising: the mean speedup was $-16.35\%$ (median
$-18.97\%$), and of $80$ benchmarks, only $13$ showed any increase in running
time.  The mean speedup among those that ran faster was $-22.24\%$ (median
$-22.23\%$).  Of the $13$ that ran slower, even fewer showed significant
increases in running time.  Duplicated below for convenience are the slower
benchmarks, sorted in decreasing order of the percent change.  We can see the
last five or six benchmarks exhibited negligible differences---not only is the
relative change tiny, but the absolute difference in running times is less than
$0.1$ seconds.  (The \Verb|tuple-arrays| results also show a similar absolute
change, but it is relatively much larger.)
\end{sloppypar}

\begin{longtable}{llll}
\toprule
Benchmark & Before (seconds) & After (seconds) & Change (\%) \\
\midrule
\endhead
\texttt{nsieve-bytes}             & 0.785041878       & 1.211421146         & $+54.31$    \\
\texttt{stack}                    & 1.580016742       & 2.004478602         & $+26.86$    \\
\texttt{tuple-arrays}             & 0.262747557       & 0.329399609         & $+25.37$    \\
\texttt{spectral-norm-simd}       & 2.743240011       & 3.237017655         & $+18.00$    \\
\texttt{terrain-generation}       & 1.611800222       & 1.887047663         & $+17.08$    \\
\texttt{struct-arrays}            & 2.180774222       & 2.421915609         & $+11.06$    \\
\texttt{partial-sums}             & 3.762171661       & 4.130144177         & $+9.78$     \\
\texttt{sum-file}                 & 0.883097981       & 0.957151577         & $+8.39$     \\
\texttt{nsieve}                   & 1.086110659       & 1.137648699         & $+4.75$     \\
\texttt{nsieve-bits}              & 2.707271763       & 2.815509077         & $+4.00$     \\
\texttt{nbody-simd}               & 0.845814208       & 0.854773096         & $+1.06$     \\
\texttt{random}                   & 5.66540782        & 5.71913683          & $+0.95$     \\
\texttt{pidigits}                 & 2.182877913       & 2.195385034         & $+0.57$     \\
\end{longtable}

Overall, even transitioning to a relatively simple \gls{GVN} algorithm amounts
to a positive change in Factor's compiler.  More redundancies are eliminated,
resulting in speedier programs.  Judging by unit tests, the implementation is
at least as sound as the previous local value numbering,  as all the same tests
have passed.

% Future
%   SCC (discussion of potential improvement)
%   Click, Gargi, et al. (future directions)



\renewcommand{\bibname}{References}\newpage\printbibliography

\end{document}

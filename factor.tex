\section{Language Primer}

\todo[inline]{citations for this history are fragmented across the internet;
should consolidate some kernel of citation from it}

Factor is a rather young language created by Slava Pestov in September of 2003.
Its first incarnation targeted the \gls{JVM} as an embedded scripting language
for a game.  As such, its feature set was minimal.  Factor has since evolved
into a general-purpose programming language, gaining new features and
redesigning old ones as necessary for larger programs.  Today's implementation
sports an extensive standard library and has moved away from the \gls{JVM} in
favor of native code generation.  In this section\todo{figure out section /
chapter names}, we cover the basic syntax and semantics of Factor for those
unfamiliar with the language.  This should be just enough to understand the
later material in this thesis\todo{cite chapter?}.  More thorough documentation
can be found via Factor's website\todo{cite factorcode.org}.

\subsection{Stack-Based Programming}

\inputfig{rpn}

Like \gls{RPN} calculators, Factor's essential evaluation model uses a global
\term{stack} upon which operands are pushed before operators are called.  This
lends itself to \term{postfix} notation, in which operators are written after
their operands.  For example, instead of \texttt{1~+~2}, we write
\texttt{1~2~+}.  \Vref{fig:rpn} shows how \texttt{1~2~+} works conceptually:
\begin{itemize}
  \item \texttt{1} is pushed onto the stack
  \item \texttt{2} is pushed onto the stack
  \item \texttt{+} is called, so two values are popped from the stack, added,
        and the result (\texttt{3}) is pushed back onto the stack
\end{itemize}
Other stack-based programming languages include Forth\todo{cite},
Joy\todo{cite}, Cat\todo{cite}, and PostScript\todo{cite}.

The strength of this model is its simplicity.  Parsing becomes very flexible,
since whitespace is essentially the only thing that separates tokens.  In the
Forth tradition, functions (being single tokens delineated by whitespace) are
called \term{words}.  This also lends to the term \term{vocabulary} instead of
``module'' or ``library''.   In Factor, the parser works as follows.
\begin{itemize}
  \item If the current character is a double-quote (\lstinline|"|), try to
        parse ahead for a string literal.
  \item Otherwise, scan ahead for a single token.
        \begin{itemize}
          \item If the token is the name of an \term{ordinary word}, it's added
                to the parse tree.
          \item If the token is the name of a \term{parsing word}, it's invoked
                with the parser's current information.
          \item Otherwise, try to parse the token as a numeric literal.
        \end{itemize}
\end{itemize}
Notice too that evaluation (as in \vref{fig:rpn}) simply goes left-to-right:
literals are pushed to the stack, words are invoked.

Ordinary words correspond to functions.  While infix languages tend to
distinguish syntactically significant operators (e.g., for arithmetic) from
normal user-defined functions, there is no such difference here.  Since
\lstinline|1 2 +| and \lstinline|1 2 foo| are tokenized the same way, other
languages' special operators are simply words in Factor.

\todo[inline]{More ordinary word examples}

Parsing words serve as hooks into the parser, letting Factor users extend its
syntax dynamically.  For instance, instead of having special knowledge of
comments built into the parser, the parsing word \lstinline|!| scans forward
for a newline and discards any characters read (adding nothing to the parse
tree).

Other parsing words act as sided delimiters.  The parsing word for the
left-hand delimiter will scan ahead for the right-hand delimiter, using
whatever was read in between to add objects to the parse tree.  For example,
array literals are created by the parsing word \lstinline|{| parsing ahead
until it finds a \lstinline|}| token, collecting the results into an array
object that's added to the parse tree.  As it parses recursively, parsing words
and literals in between the braces are collected while ordinary words are not
called, since the tokens are collected at parse time.

\begin{itemize}

\item Basic syntax \& semantics
      \begin{itemize}
        \item Example ordinary words: stack shufflers
        \item Example parsing words: quotations
      \end{itemize}

\item Concatenative programming
      \begin{itemize}
        \item ``Pipeline code''
        \item Whitespace = function composition
        \item Point-free style
        \item Origin of name ``Factor''
      \end{itemize}

\item Stack effects
      \begin{itemize}
        \item Basic stack effects: stack shufflers illustrated
        \item Complex stack effects: row polymorphism \& types
        \item Stack checker
      \end{itemize}

\item Control flow
      \begin{itemize}
        \item Combinators
              \begin{itemize}
                \item if
                \item each
                \item while
              \end{itemize}
        \item Dataflow combinators
              \begin{itemize}
                \item Dip/keep
                \item Cleave
                \item Spread
                \item Apply
              \end{itemize}
      \end{itemize}

\item Object system
      \begin{itemize}
        \item tuples
        \item generics \& methods
      \end{itemize}

\item Libraries \& metaprogramming
      \begin{itemize}
        \item Results of evolution
        \item locals?
        \item fry?
        \item macros?
        \item functors?
        \item ffi?
      \end{itemize}

\end{itemize}

\end{document}

\section{Language Primer}\label{sec:primer}

\todo[inline]{citations for this history are fragmented across the internet;
should consolidate some kernel of citation from it}

Factor is a rather young language created by Slava Pestov in September of 2003.
Its first incarnation targeted the \gls{JVM} as an embedded scripting language
for a game.  As such, its feature set was minimal.  Factor has since evolved
into a general-purpose programming language, gaining new features and
redesigning old ones as necessary for larger programs.  Today's implementation
sports an extensive standard library and has moved away from the \gls{JVM} in
favor of native code generation.  In this section\todo{figure out section /
chapter names}, we cover the basic syntax and semantics of Factor for those
unfamiliar with the language.  This should be just enough to understand the
later material in this thesis\todo{cite chapter?}.  More thorough documentation
can be found via Factor's website\todo{cite factorcode.org}.

\subsection{Stack-Based Languages}\label{sec:primer:stack-based}

\inputfig{rpn}

Like \gls{RPN} calculators, Factor's evaluation model uses a global
\term{stack} upon which operands are pushed before operators are called.  This
lends itself to \term{postfix} notation, in which operators are written after
their operands.  For example, instead of \texttt{1~+~2}, we write
\texttt{1~2~+}.  \Vref{fig:rpn} shows how \texttt{1~2~+} works conceptually:
\begin{itemize}
  \item \texttt{1} is pushed onto the stack
  \item \texttt{2} is pushed onto the stack
  \item \texttt{+} is called, so two values are popped from the stack, added,
        and the result (\texttt{3}) is pushed back onto the stack
\end{itemize}
Other stack-based programming languages include Forth\todo{cite},
Joy\todo{cite}, Cat\todo{cite}, and PostScript\todo{cite}.

The strength of this model is its simplicity.  Parsing becomes very flexible,
since whitespace is basically the only thing that separates tokens.  In the
Forth tradition, functions (being single tokens delineated by whitespace) are
called \term{words}.  This also lends to the term \term{vocabulary} instead of
``module'' or ``library''.   In Factor, the parser works as follows.
\begin{itemize}
  \item If the current character is a double-quote (\inlinecode|"|), try to
        parse ahead for a string literal.
  \item Otherwise, scan ahead for a single token.
        \begin{itemize}
          \item If the token is the name of a \term{parsing word}, that word is
                invoked with the parser's current information.
          \item If the token is the name of an ordinary (i.e., non-parsing)
                word, that word is added to the parse tree.
          \item Otherwise, try to parse the token as a numeric literal.
        \end{itemize}
\end{itemize}

Words correspond to \term{first-class} functions from other languages.  That
is, words are themselves objects that just perform instructions whenever
they're told---most commonly, when Factor is executing a piece of code.  While
infix languages tend to distinguish syntactically significant operators (e.g.,
for arithmetic) from normal user-defined functions, there is no such difference
here.  Since \inlinecode|1 2 +| and \inlinecode|1 2 foo| are tokenized the same
way, other languages' special operators are just words in Factor.  This simple
syntax gives rise to a simple evaluation model that essentially goes
left-to-right (as we saw in \vref{fig:rpn}): literals are pushed to the stack,
words are executed.

\todo[inline]{Factor syntax examples; figure out listings (grrr)}

Parsing words serve as hooks into the parser, letting Factor users extend its
syntax dynamically.  For instance, instead of having special knowledge of
comments built into the parser, the parsing word \inlinecode|!| scans forward
for a newline and discards any characters read (adding nothing to the parse
tree).

Other parsing words act as sided delimiters.  The parsing word for the
left-delimiter will scan ahead for the right-delimiter, using whatever was read
in between to add objects to the parse tree.  For example, array literals are
created by the parsing word \inlinecode|{| parsing ahead until it finds a
\inlinecode|}| token, collecting the results into an array that's added to the
parse tree.  Since it parses recursively, array literals can be nested, contain
comments, etc\todo{example}.  That is, anything parsing words do will work
inside of an array.  If an ordinary word is parsed, it is added to the
array---not executed.  Thus, \inlinecode|{ 1 2 + }| denotes an array of three
elements: the numbers \inlinecode|1| and \inlinecode|2|, and the word
\inlinecode|+|\todo{offshore to example listing?}.

\inputfig{quot}

\todo[inline]{Figure out how to typeset stacks in a more consistent way}

A particularly important set of parsing words in Factor are the square
brackets, \inlinecode|[| and \inlinecode|]|.  Any code in between such brackets
is collected up into a single sequence called a \term{quotation}.  Essentially,
it's a snippet of code whose execution is suppressed.  The code inside a
quotation can then be run with the \inlinecode|call| word.  Quotations are like
anonymous functions in other languages, but the stack model makes them
conceptually simpler, since we don't have to worry about variable binding and
the like.  Consider a small example like \inlinecode|1 2 [ + ] call|.  You can
think of \inlinecode|call| working by ``erasing'' the brackets around a
quotation, so this example behaves just like \inlinecode|1 2 +|.
\Vref{fig:quot} shows its evaluation: instead of adding the numbers
immediately, \inlinecode|+| is placed in a quotation, which is pushed to the
stack.  The quotation is then invoked by \inlinecode|call|, so \inlinecode|+|
pops and adds the two numbers and pushes the result onto the stack.  We'll show
how quotations are used in \verb|\ref{sec:primer:control-flow}|\todo{ref}.

%\subsection{Control Flow}\label{sec:primer:control-flow}

\subsection{Concatenative Programming}

The biggest step needed to understand Factor isn't really big: everything else
follows from the stack-based structure outlined in
\ref{sec:primer:stack-based}.  With some input on the stack, consecutive words
transform the data is discrete steps, thereby shaping a result.  This breeds a
fairly stylized way of writing code that resembles a pipeline.  Though postfix
code like \inlinecode|1 2 +| seems to read ``backwards'', when viewed in this
pipelined fashion, it can read more clearly than standard syntax.  Suppose
we're given a value \inlinecode|x| to manipulate by applying three functions,
\inlinecode|f|, \inlinecode|g|, and \inlinecode|h|, in order.  Typical function
application notation looks like \inlinecode|h(g(f(x)))|, which must be read
from the inside out because of the nesting.  In Factor, we'd write
\inlinecode|x f g h|, which is flat and reads left-to-right.

\begin{itemize}

\item Concatenative programming
      \begin{itemize}
        \item Whitespace = function composition
        \item Point-free style
        \item What is concatenative programming?
        \item Origin of name ``Factor''
      \end{itemize}

\item Stack effects
      \begin{itemize}
        \item Basic stack effects: stack shufflers illustrated
        \item Complex stack effects: row polymorphism \& types
        \item Stack checker
      \end{itemize}

\item Control flow
      \begin{itemize}
        \item Combinators
              \begin{itemize}
                \item if
                \item each
                \item while
              \end{itemize}
        \item Dataflow combinators
              \begin{itemize}
                \item Dip/keep
                \item Cleave
                \item Spread
                \item Apply
              \end{itemize}
      \end{itemize}

\item Object system
      \begin{itemize}
        \item tuples
        \item generics \& methods
      \end{itemize}

\item Libraries \& metaprogramming
      \begin{itemize}
        \item Results of evolution
        \item locals?
        \item fry?
        \item macros?
        \item functors?
        \item ffi?
      \end{itemize}

\end{itemize}

\end{document}
